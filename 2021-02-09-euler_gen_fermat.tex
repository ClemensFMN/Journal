\DiaryEntry{Euler's Generalization of Fermat's Theorem}{2021-02-09}{Number Theory}

We start with Euler's ophi function.

\begin{definition}[Euler Phi Function]
For $n \geq 1$, let $\phi(n)$ denote the number of positive integers below $n$ that are relatively prime to $n$.
\end{definition}

As an example, we have $\phi(30) = 8$ as the following $8$ numbers $1, 7, 11, 13, 17, 19, 23, 29$ are coprime with $30$. We can obtain this with Gap as follows

\begin{verbatim}
gap> Phi(30);
8
gap> PrimeResidues(30);
[ 1, 7, 11, 13, 17, 19, 23, 29 ]
\end{verbatim}

If $n$ is prime, then every number below $n$ is coprime with it and we therefore have $\phi(n) = n-1$. If $n$ is composite, then $\phi(n) \leq n-2$.

We seek a formula to calculate $\phi(n)$ for any integer $n$ from the prime factorization of $n$. A first step in this direction is the following theorem.

\begin{theorem}
  If $p$ is prime and $k > 0$, then

  \bee
  \phi(p^k) = p^k - p^{k-1} = p^k \left( 1 - \frac{1}{p} \right)
  \eee

\end{theorem}

Let's write down the integers divisible by $p$. Since $p$ is prime, these are

\bee
p, 2p, 3p, \ldots, p^{k-1} p
\eee

and these are $p^{k-1}$ numbers. From $1$ to $p^k$ we have $p^k$ numbers and therefore $p^k - p^{k-1}$ numbers are relatively prime to $p^k$, which yields $\phi(p^k) = p^k - p^{k-1}$. \qed

As a simple example consider $\phi(5^2) = 20$ (obtained via GAP) and $\phi(5^2) = 5^2 - 5 = 20$ (using the above theorem).

In order to be able to calculate $\phi(n)$ for \emph{any} number from its prime factorization, we need to show that $\phi(n)$ is a multiplicative function. We omit the proof (for now) and directly state the following theorem.

\begin{theorem}
  If the integer $n$ has the prime factorization $n = p_1^{k_1} p_2^{k_2} \cdots p_r^{k_r}$, then

  \bee
  \phi(n) = n \left( 1 - \frac{1}{p_1}\right) \left( 1 - \frac{1}{p_2}\right) \cdots \left( 1 - \frac{1}{p_r}\right)
  \eee
  
\end{theorem}

Since we know that $\phi(n)$ is a multiplicative function, the proof is simple: We have

\bee
\phi \left( p_1^{k_1} p_2^{k_2} \cdots p_r^{k_r} \right) = \phi \left( p_1^{k_1} \right) \phi \left( p_2^{k_2} \cdots p_r^{k_r} \right) = p_1^{k_1} \left( 1 - \frac{1}{p_1}\right) \phi \left( p_2^{k_2} \cdots p_r^{k_r} \right)
\eee

Repeating this procedure, we eventually arrive at

\begin{align*}
\phi \left( p_1^{k_1} p_2^{k_2} \cdots p_r^{k_r} \right) &= p_1^{k_1} \left( 1 - \frac{1}{p_1}\right) p_2^{k_2} \left( 1 - \frac{1}{p_2}\right) \cdots p_r^{r} \left( 1 - \frac{1}{p_r}\right) \\ &= n \left( 1 - \frac{1}{p_1}\right) \left( 1 - \frac{1}{p_2}\right) \cdots \left( 1 - \frac{1}{p_r}\right) \\ &= \left( p_1^{k_1} - p_1^{k_1 - 1} \right)\left( p_2^{k_2} - p_2^{k_2 - 1} \right) \cdots \left( p_r^{k_r} - p_r^{k_r - 1} \right) \qed
\end{align*}

Here the last line shows that the exponents $k_i$ do enter the expression for $\phi(n)$.

For example, we have

\begin{verbatim}
gap> PrintFactorsInt(100);
2^2*5^2
gap> 100*(1-1/2)*(1-1/5);
40
gap> Phi(100);
40
\end{verbatim}


%%% Local Variables:
%%% mode: latex
%%% TeX-master: "journal"
%%% End:
