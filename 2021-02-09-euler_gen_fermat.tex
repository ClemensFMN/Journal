\DiaryEntry{Euler's Generalization of Fermat's Theorem}{2021-02-09}{Number Theory}

\subsection{Euler's Phi Function}

See also \ref{2017-04-26:entry}.

\begin{definition}[Euler Phi Function]
For $n \geq 1$, let $\phi(n)$ denote the number of positive integers below $n$ that are relatively prime to $n$.
\end{definition}

As an example, we have $\phi(30) = 8$ as the following $8$ numbers $1, 7, 11, 13, 17, 19, 23, 29$ are coprime with $30$. We can obtain this with Gap as follows

\begin{verbatim}
gap> Phi(30);
8
gap> PrimeResidues(30);
[ 1, 7, 11, 13, 17, 19, 23, 29 ]
\end{verbatim}

If $n$ is prime, then every number below $n$ is coprime with it and we therefore have $\phi(n) = n-1$. If $n$ is composite, then $\phi(n) \leq n-2$.

We seek a formula to calculate $\phi(n)$ for any integer $n$ from the prime factorization of $n$. A first step in this direction is the following theorem.

\begin{theorem}
  If $p$ is prime and $k > 0$, then

  \bee
  \phi(p^k) = p^k - p^{k-1} = p^k \left( 1 - \frac{1}{p} \right)
  \eee

\end{theorem}

Let's write down the integers divisible by $p$. Since $p$ is prime, these are

\bee
p, 2p, 3p, \ldots, p^{k-1} p
\eee

and these are $p^{k-1}$ numbers. From $1$ to $p^k$ we have $p^k$ numbers and therefore $p^k - p^{k-1}$ numbers are relatively prime to $p^k$, which yields $\phi(p^k) = p^k - p^{k-1}$. \qed

As a simple example consider $\phi(5^2) = 20$ (obtained via GAP) and $\phi(5^2) = 5^2 - 5 = 20$ (using the above theorem).

In order to be able to calculate $\phi(n)$ for \emph{any} number from its prime factorization, we need to show that $\phi(n)$ is a multiplicative function. We omit the proof (for now) and directly state the following theorem.

\begin{theorem}
  If the integer $n$ has the prime factorization $n = p_1^{k_1} p_2^{k_2} \cdots p_r^{k_r}$, then

  \bee
  \phi(n) = n \left( 1 - \frac{1}{p_1}\right) \left( 1 - \frac{1}{p_2}\right) \cdots \left( 1 - \frac{1}{p_r}\right)
  \eee
  
\end{theorem}

Since we know that $\phi(n)$ is a multiplicative function, the proof is simple: The integers $p_i^{k_i}$ are all relatively prime and therefore we have

\bee
\phi \left( p_1^{k_1} p_2^{k_2} \cdots p_r^{k_r} \right) = \phi \left( p_1^{k_1} \right) \phi \left( p_2^{k_2} \cdots p_r^{k_r} \right) = p_1^{k_1} \left( 1 - \frac{1}{p_1}\right) \phi \left( p_2^{k_2} \cdots p_r^{k_r} \right)
\eee

Repeating this procedure, we eventually arrive at

\begin{align*}
\phi \left( p_1^{k_1} p_2^{k_2} \cdots p_r^{k_r} \right) &= p_1^{k_1} \left( 1 - \frac{1}{p_1}\right) p_2^{k_2} \left( 1 - \frac{1}{p_2}\right) \cdots p_r^{r} \left( 1 - \frac{1}{p_r}\right) \\ &= n \left( 1 - \frac{1}{p_1}\right) \left( 1 - \frac{1}{p_2}\right) \cdots \left( 1 - \frac{1}{p_r}\right) \\ &= \left( p_1^{k_1} - p_1^{k_1 - 1} \right)\left( p_2^{k_2} - p_2^{k_2 - 1} \right) \cdots \left( p_r^{k_r} - p_r^{k_r - 1} \right) \qed
\end{align*}

Here the last line shows that the exponents $k_i$ do enter the expression for $\phi(n)$.

For example, we have

\begin{verbatim}
gap> PrintFactorsInt(100);
2^2*5^2
gap> 100*(1-1/2)*(1-1/5);
40
gap> Phi(100);
40
\end{verbatim}

We have the fact that $\phi(n)$ is an even integer for $n > 2$: We first assume that $n$ is a power of two; $n = 2^k$. By the previous theorem, we have

\bee
\phi(2^k) = 2^k \left(1 - \frac{1}{2}\right)
\eee

which is even. If $n$ is not a power of $2$, then it is divisible by an odd prime $p$ and we can write $n = m p^k$ with $k \geq 1$ and $\gcd(p^k, m) = 1$. We can therefore write

\bee
\phi(n) = \phi(m p^k) = \phi(m) \phi(p^k) = \phi(m) p^k \left(1 - \frac{1}{p} \right) = \phi(m) p^{k-1} (p-1)
\eee

Since $p$ is odd, $p-1$ is even and therefore $\phi(n)$ is also even. \qed

What is also interesting is the fact that, for $n$ odd, $\phi(n) = \phi(2n)$. We can see this as follows: We have

\bee
n = p_1^{k_1} p_2^{k_2} \cdots p_r^{k_r}
\eee

and since $n$ is odd, none of the primes $p_1, \ldots, p_r$ can be equal to $2$. Therefore, $2n$ has the factorization

\bee
2n = 2 p_1^{k_1} p_2^{k_2} \cdots p_r^{k_r}
\eee

and therefore

\bee
\phi(2n) = 2 n \left( 1 - \frac{1}{2}\right) \left( 1 - \frac{1}{p_1}\right)  \cdots \left( 1 - \frac{1}{p_r}\right) = n \left( 1 - \frac{1}{p_1}\right)  \cdots \left( 1 - \frac{1}{p_r}\right) = \phi(n) \qed
\eee

If $n$ is even, $2$ must be contained in the factorization of $n$,

\bee
n = 2^{k_1} p_2^{k_2} \cdots p_r^{k_r}
\eee

and therefore

\bee
2n = 2^{k_1+1} p_2^{k_2} \cdots p_r^{k_r}
\eee


Using this, we can calculate $\phi(2n)$ as

\bee
\phi(2n) = \left( 2^{k_1+1} - p_1^{k_1 } \right) \cdots \left( p_r^{k_r} - p_r^{k_r - 1} \right) = 2 \left( 2^{k_1} - p_1^{k_1-1} \right) \cdots \left( p_r^{k_r} - p_r^{k_r - 1} \right) = 2 \phi(n) \qed
\eee

As an example, consider $\phi(9) = 6 = \phi(19)$ and $\phi(8) = 4, \phi(16) = 8$.

\subsection{Euler's Theorem}

\begin{theorem}
  If $n>1$ and $\gcd(a, n) = 1$, then

  \bee
  a^{\phi(n)} \equiv 1 \mod n
  \eee
\end{theorem}

As an example, consider $n=16, a=3$: We have $3^\phi(16) = 3^8 = 6561 \equiv 1 \mod$. The striking part of the theorem is the fact that the modulo-equivalence does not dpeend on $a$ (as long as $\gcd(a,n)=1$), so we also have $5^\phi(16) \equiv 7^\phi(16) \equiv 9^\phi(16) \equiv 1 \mod 16$.

Before we can prove the theorem, we need the following lemma.

\begin{theorem}
  Let $n > 1$ and $\gcd(a,n) = 1$. If $a_1, a_2, \ldots, a_{\phi(n)}$ are the positive integers less than $n$ which are relatively prime to $n$, then

  \bee
  a a_1, a a_2, \ldots a a_{\phi(n)}
  \eee

  are congruent modulo-$n$ to $a_1, a_2, \ldots, a_{\phi(n)}$ in some order.
\end{theorem}

We first observe that no two integers $a a_i, a a_j$ with $i \neq j \leq \phi(n)$ are congruent modulo-$n$. If we had $a a_i \equiv a a_j \mod a$, then we could cancel $a$ (as $\gcd(a.n)=1$) and we would have the contradiction $a_i = a_j$. In addition, because $\gcd(a_i,n) = 1$ and $\gcd(a,n) = 1$, the $a a_i$ are relatively prime to $n$ (no proof given).

For a particular $a a _i$, there exists a unique integer $0 \leq b < n$ for which $a a _i \equiv b \mod n$. Because $\gcd(b,n) = \gcd(a a_i, n) = 1$, $b$ must be one of the integers $a_1, a_2, \ldots, a_{\phi(n)}$. All together, this proves that the numbers $a a_1, a a_2, \ldots, a a_{\phi(n)}$ and $a_1, a_2, \ldots, a_{\phi(n)}$ are equivalent modulo-$n$ in a certain order. \qed

With that in place, we can prove Euler's theorem. We assume $n > 1$ and let $a_1, a_2, \ldots, a_{\phi(n)}$ denote positive integers less than $n$ which are also relatively prime to $n$. Because $\gcd(a,b) = 1$, the integers $a a_1, a a_2, \ldots, a a_{\phi(n)}$ are congruent to $a_1, a_2, \ldots, a_{\phi(n)}$ in some order. Then we have

\begin{align*}
  a a_1 & \equiv a_1' \mod n \\
  \cdots & \cdots \\
  a a_{\phi(n)} & \equiv a_{\phi(n)}' \mod n \\
\end{align*}

where the $a_1', \ldots, a_{\phi(n)}'$ are the integers $a_1, a_2, \ldots, a_{\phi(n)}$ in some order. Multiplying all $\phi(n)$ congruences together, we obtain

\bee
(a a_1) (a_2) \cdots (a a_{\phi(n)}) \equiv a_1' a_2' \cdots a_{\phi(n)}' \mod n \equiv a_1 a_2 \cdots a_{\phi(n)} \mod n
\eee

where we have used the fact that a product does not depend on the order of its factors and therefore $a_1' a_2' \cdots a_{\phi(n)}' = a_1 a_2 \cdots a_{\phi(n)}$. On the left, we have $\phi(n)$-times $a$, and therefore

\bee
a^{\phi(n)} (a_1 a_2 \cdots a_{\phi(n)}) \equiv a_1 a_2 \cdots a_{\phi(n)} \mod n
\eee

Since for each $a_i$ we have $\gcd(a_i,n) = 1$, we can cancel the term and we arrive at

\bee
a^{\phi(n)} \equiv 1 \mod n \qed
\eee

\paragraph{Example.} We can use Euler's theorem to reduce large powers modulo $n$. Let us find the last two digits of $3^{256}$; i.e. $3^{256} \mod 100$. We have $\gcd(3,100) = 1$ and $\phi(100) = \phi(2^2 5^2) = 100 \left(1 - \frac{1}{2})\right) \left(1 - \frac{1}{5})\right) = 40$, therefore

\bee
3^{40} \equiv 1 \mod 100
\eee

We can write $256 = 6 \cdot 40 + 16$ and So

\bee
3^{256} \equiv 3^{6 \cdot 40 + 16} \equiv (3^{40})^6 3^{16} \equiv 3^{16} \mod 100
\eee

We can solve the last one either directly $3^{16} = 43046721$ or via repeated squaring;

\bee
3^2 \equiv 9 \mod 100 \rightarrow 3^4 \equiv 81 \mod 100 \rightarrow 3^8 \equiv 61 \mod 100 \rightarrow 3^{16} \equiv 21 \mod 100
\eee

\subsection{Some (more) properties of Euler's Phi Function}

\begin{theorem}
  For $n \geq 1$, we have

  \bee
    \sum_{d | n} \phi(d) = n
  \eee

\end{theorem}

We can prove this by putting the integers $1, 2, \ldots n$ into equivalence classes which depend on their gcd value with $n$,

\bee
S_d = \{ m | \gcd(m,n) = d, 1 \leq m \leq n \}
\eee

For example, with $n=10$, we have $S_1 = \{1, 3, 7, 9\}$; i.e. all elements $m$ in $S_1$ have $\gcd(m,10)=1$. Continuing, we have $S_2 = \{2,4,6,8\}, S_5=\{5\}$ and $S_{10} = \{10\}$.

Now $\gcd(m,n) = d$ iff $\gcd(m/d, n/d) = 1$. Therefore, the number of integers in a class $S_d$ equals the integers not exceeding $n/d$ which are relatively prime to $n/d$; i.e. equal $\phi(n/d)$. In our running example, we have $|S_1| = \phi(10/1) = \phi(10) = 4$.

We have partitioned the integers into equivalence classes, meaning that each integer is contained in only one class. We therefore obtain

\bee
\sum_{d | n} \phi \left( \frac{n}{d} \right) = n
\eee

But as $d$ runs through all divisors of $n$, so does $n/d$ and we can substitute

\bee
\sum_{d | n} \phi \left( \frac{n}{d} \right) = \sum_{d | n} \phi(d)) = n \qed
\eee

In our runnig example, we therefore have $|S_1| + |S_2| + |S_5| + |S_{10}| = 4 + 4 + 1 + 1 = 10$.

Another proof uses the multiplicity of $\phi(n)$. Define

\bee
F(n) = \sum_{d | n} \phi(d)
\eee

which is also a multiplicative function. If $n = p_1^{k_1} \cdots p_r^{k_r}$, then

\bee
F(n) = F(p_1^{k_1}) \cdots F(p_r^{k_r})
\eee

For each factor we have

\begin{align*}
  F(p_i^{k_i}) &= \sum_{d | p_i^{k_i} } \phi(d) \\
  &= \phi(1) + \phi(p_i) + \phi(p_i^2) + \cdots + \phi(p_i^{k_i}) \\
  &= 1 + (p_1 - 1)+  (p_1^2 - p_1) + \cdots + (p_i^{k_i} - p_i^{k_i-1}) \\
  &= p_i^{k_i}
\end{align*}

Inserting this into $F(n) = F(p_1^{k_1}) \cdots F(p_r^{k_r})$, we obtain

\bee
F(n) = p_1^{k_1} \cdots p_r^{k_r} = n
\eee

and therefore $F(n) = \sum_{d | n} \phi(d) = n$. \qed


\begin{theorem}
  For $n>1$, the sum of all positive integers less than $n$ and relatively prime to $n$ is $\frac{1}{2}n\phi(n)$.
\end{theorem}

For $n = 10$, we have $\frac{1}{2}10 \phi(10) = 5 \phi(10) = 20$ which equals $1 + 3 + 7 + 9 = 20$.

Proof: Let $a_1, a_{\phi(n)}$ the integers smaller than $n$ and relatively prime to $n$. Because $\gcd(a,n) = 1$ iff $\gcd(n-a,a) = 1$, the numbers $n - a_1, n - a_{\phi(n)}$ are equal in some order to the $a_1, a_{\phi(n)}$. Thus we have

\bee
a_1 + a_2 + \cdots + a_{\phi(n)} = (n-a_1) + (n-a_2) + \cdots + (n-a_{\phi(n)}) = \phi(n) - (a_1 + a_2 + \cdots + a_{\phi(n)})
\eee

and therefore

\bee
2(a_1 + a_2 + \cdots + a_{\phi(n)}) = \phi(n) n \qed
\eee

\begin{theorem}
  For any positive integer $n$,

  \bee
    n \sum_{d | n} \frac{\mu(d)}{d} = \phi(n)
  \eee

\end{theorem}

As above, define $F(n) = n = \sum_{d | n} \phi(d)$. If we apply the Möbius inversion formula, we obtain

\bee
\phi(n) = \sum_{d | n} \mu(d) F\left( \frac{n}{d} \right) = \sum_{d | n} \mu(d) \frac{n}{d} \qed
\eee



%%% Local Variables:
%%% mode: latex
%%% TeX-master: "journal"
%%% End:
