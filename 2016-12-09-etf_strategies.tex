\DiaryEntry{ETF Strategien}{2016-12-09}{Economics}


In \href{http://www.finanzwesir.com/blog/etf-index-rendite-vergleich}{diesem Blogeintrag} werden unterschiedliche ETF Strategien verglichen: MSCI World + MSCI Emerging Markets + MSCI North America + \ldots{}

Es zeigt sich, dass Zugabe vom MSCI EM die Rendite verbessert (aber auch die Volatilität erhöht); und weitere Zumischung von small cap ETFs etc die Performance weiter verbessern kann.

Das ist alles ganz nett und schön, aber

\begin{itemize}
\item
  die Performancevergleiche schauen sich historische Daten an. Niemand weiß, was die Zukunft bringen wird und ob die in der Vergangenheit optimale Zusammensetzung auch inZukunft optimal sein wird.
\item
  Am Ende des Tages sind die Renditeunterschiede gering, wenn man sie mit Szenarien vergleicht, wo die Sparquote geringer ist (z.b. monatliche Einzahlung von €50 in einen ETF Sparplan statt €100) oder Szenarien, in denen zur Mitte der Laufzeit Kapital entnommen wird (z.b. 10 Jahre ansparen, dann 50\% Kapitalentnahme dann weiter 10 Jahre ansparen vs durchgehend 20 Jahre ansparen). In beiden Fällen schlägt der Zinseszinseneffekt zu - das am längsten liegende Kapital bringt die meisten Erträge wird dieses reduziert, schlägt das sehr stark auf die Endsumme durch.
\end{itemize}

Buchempfehlung

\begin{itemize}

\item
  The Intelligent Asset Allocator: How to Build Your Portfolio to Maximize Returns and Minimize Risk by William J. Bernstein
\end{itemize}
