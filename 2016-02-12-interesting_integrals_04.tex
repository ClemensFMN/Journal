\DiaryEntry{Interesting Integrals, 4}{2016-02-12}{Integrals}

Consider the integral

\[
I = \int \frac{ax+b}{x^2 + c x + d} dx
\]

There are two cases: The denominator has real roots and it has no real
roots.

\subsubsection{Denominator with Real Roots}

The denominator has two real roots \(r_1, r_2\) and we can make a
partial fraction expansion as

\[
\frac{ax+b}{x^2 + c x + d} = \frac{A}{x-r_1} + \frac{B}{x-r_2}
\]

with \(A, B\) being real numbers. The the integral becomes

\[
I = \int \frac{ax+b}{x^2 + c x + d} = A \ln(x-r_1) + B \ln(x-r_2)
\]

and we are finished.

\subsubsection{No Real Roots}

In this case we can split the integrand into two parts: In the first
part we choose the enumerator so that it becomes the derivative of the denumerator. The second part then ``fixes'' the overall expression.

As example consider

\[
I = \int \frac{x+3}{x^2+2x+5} dx
\]

We note that \((x^2+2x+5)' = 2x+2\). Therefore, we split the integrand
as follows

\[
\frac{x+3}{x^2+2x+5} = \frac{1}{2} \frac{2x+2}{x^2+2x+5} + \frac{2}{x^2+2x+5}
\]

The second part is needed that the RHS equals the LHS.

For the first integral we substitute \(u = x^2+2x+5\) and
\(dx = \frac{du}{2x+2}\). For the second part we complete the square of
the denumerator: \(x^2+2x+5 = (x+1)^2+4\) and substitute \(v = x+1\).
The we have

\[
I = \frac{1}{2} \int \frac{2x+2}{u} \frac{du}{2x+2} + \int \frac{2 dv}{v^2 + 4}
\]

Both integrals can be solved and we obtain

\[
I = \frac{1}{2} \ln(x^2+2x+5) + \arctan \frac{x+1}{2}
\]
