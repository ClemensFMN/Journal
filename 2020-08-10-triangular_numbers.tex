\DiaryEntry{Triangular Numbers}{2020-08-10}{General}

There is an old entry also dealing with the topc; \href{2015-10-08:entry}{here}. For repetition, the $n$-th triangular number $_n$ is

\bee
T_n = \frac{n(n+1)}{2}
\eee

We note that the triangular number $T_n$ can also be expressed as

\bee
T_n = \frac{n(n+1)}{2} = {n+1 \choose 2}
\eee

Triangular numbers have some interesting characteristics shown in the following.

If $n$ is a triangular number, then $8n+1$ is a perfect square. We have

\bee
8T_n+1 = 8 \frac{n(n+1)}{2} + 1 = 4n(n+1)+1 = 4n^2 + 4n + 1 = (2n+1)^2 \qed
\eee

The sum of two consecutive triangular numbers is a perfect square.

\begin{align*}
T_n + T_{n+1} &= \frac{n(n+1)}{2} + \frac{(n+1)(n+2)}{2} = \frac{n^2 + n + n^2 + 3n + 2}{2} \\ &= \frac{2n^2 + 4n + 2}{2} = n^2 + 2n +1 = (n+1)^2 \qed
\end{align*}

If $n$ is a triangular number, so is $9n+1$. We can express $n$ as $n = \frac{m(m+1)}{2}$ for some integer $m$. Then we have

\bee
9n+1 = 9 \frac{m(m+1)}{2} + 1 = \frac{9m^2 + 9m + 2}{2} = \frac{M(M+1)}{2}
\eee

as $9n+1$ itself must be expressible as $\frac{M(M+1)}{2}$ with some $M$. From this we obtain

\bee
9m^2 + 9m + 2 = M^2 + M \rightarrow 9m^2 + 9m + 2 - M^2 - M = 0
\eee

Solving the quadratic equation yields $m = \frac{M-1}{3}$ and

\bee
M = 3m+1 \qed
\eee

As an example, take $m=4$, therefore $n = 10$. Now $9\cdot 10+1 = 91$ is also a triangular number and we have $91 = \frac{13 \cdot 14}{2}$; i.e. $M = 3m+1 = 13$.

We also have the statements if $n$ is a triangular number, so are $25n+3$ and $49n+6$. We can use the previous method to show their validity; however, it seems as if there is a pattern: if $n$ is triangular, so is

\bee
An + B \quad \text{with} \quad A = (2k+1)^2, B = T_k = \frac{k(k+1)}{2}
\eee

For $k=2$, we have $(2 \cdot 2 + 1)^2 n + T_2 = 25n + 3$ is triangular and for $k=3$, we have $(2 \cdot 3 + 1)^2 n + T_3 = 49n + 6$ is triangular. Using above method, we can show that the quadratic equation has a solution which is

\bee
m = \frac{M-k}{2k+1} \rightarrow M = m(2k+1) + k \qed
\eee

\todo{Would be interesting if there is an intuitive explaination for this?}

Also entertaining is the fact that the difference between two consecutive triangular numbers is a cube,

\bee
\left( \frac{k(k+1)}{2} \right)^2 - \left( \frac{(k-1)k}{2} \right)^2 = \cdots = k^3
\eee

Another fun fact is that the sum of the reciprocals of the triangular numbers approaches $2$; i.e.

\bee
\sum_{k=1}^\infty \frac{1}{t_k} \rightarrow 2
\eee

We can expand $\frac{1}{t_k} = \frac{2}{k(k+1)} = 2 \left( \frac{1}{k}- \frac{1}{k+1}\right)$ and insert this into our sum to obtain

\begin{align*}
  \sum_{k=1}^N \frac{1}{t_k} = 2 \sum_k \left( \frac{1}{k}- \frac{1}{k+1}\right) &= 2 \left[ \left( \frac{1}{1} - \frac{1}{2} \right) + \left( \frac{1}{2} - \frac{1}{3} \right) + \left( \frac{1}{3} - \frac{1}{4} \right) + \cdots \right] \\
  &= 2 \left( 1 - \frac{1}{n+1}\right) \qed
\end{align*}


%%% Local Variables:
%%% mode: latex
%%% TeX-master: "journal"
%%% End:
