\DiaryEntry{Rings - I}{2016-06-27}{Algebra}

A ring is a set with operations addition and multiplication which
fulfill the follwing axioms:

\begin{itemize}
\item
  A with addtion alone is an abelian group
\item
  Multiplication alone is associative
\item
  Multiplication is distributive over addtion
\end{itemize}

The group structure of A ensures that there is a neutral element for
addition denoed by 0 and we have \(0 + a = a + 0 = a, a \in A\). In a
similar spirit, there is a negative element denoted by \(-a\) and we
have \(a + (-a) = 0\).

\subsection{Definitions}\label{definitions}

Addition is always commutative (first condition above); if
multiplication is also commutative; i.e. \(ab = ba\), then we speak of a
\textbf{commutative ring}.

A ring need not have a unit element for multiplication; if for every
element of A there is a neutral element for multiplication, this element
is denoted by \(1\), we have \(a1 = 1a = a\) and we speak of a
\textbf{ring with unity}. In a ring with unity, elements may have a
multiplicative inverse \(a^{-1}\) and \(a a^{-1} = a^{-1} a = 1\). These
elements are called invertible; if A is a commutative ring with unity in
which every nonzero element is invertible, we call A a \textbf{field}.

In a ring, the fact that \(ab = 0\) need not imply that either \(a\),
\(b\), or \(a\) and \(b\) are zero (as is the case with e.g.~the
integers). For example \(\mathbb{Z}_6\) is a ring and we have
\(2 \cdot 3 = 0\). Such numbers are called divisors of zero: A nonzero
element a is called divisor of zero, if there is a nonzero element
\(b\), such that \(ab\) or \(ba\) is equal to zero.

There are rings with no divisors of zero; for example,
\(\mathbb{Z}, \mathbb{Q}, \mathbb{R}\) all do not have any divisors of
zero: If the product of two elements is zero, at least one of the
factors is zero.

A ring is to have a cancellation property, if \(ab = ac\) implies
\(b=c\) for any elements \(a,b,c\) in the ring and \(a \neq 0\). Again,
\(\mathbb{Z}_6\) does not have the cancellation property as
\(2 \cdot 5 = 2 \cdot 2 = 4\) but from this it does not follow that
\(2 = 5\).

A ring has the cancelation property iff it has no divisors of zero.

An \textbf{integral domain} is defined to be a commutative ring with
unity having the cancelation property. Every field is also an integral
domain (as every nonzero element of a field is invertible); however, the
converse is \textbf{not} true: Not every integral domain is a field.
Most prominent example is \(mathbb{Z}\) which is an integral domain but
not a field (all other elements but \(-1\) and \(1\) are not
invertible).

\subsection{Ideals}\label{ideals}

If a nonempty set B is closed with respect to addition, multiplication,
and negatives, then B (with the operations of A) is a subring of A.

In group theory, the notion of a normal subgroups played a special role.
In case of rings, it is ideals which play a similar important role. If A
is a ring and B is a set, then B absorbs products in A, if the product
of an element from B with any element of A yields an element of B;
i.e.~for all \(a \in A\) and \(b \in B \rightarrow ab \in B\). A
nonempty subset B of a ring A is an ideal if it is closed with respect
to addition and negatives, and B absorbs products in A.

A simple example are the even integers, \(\{2,4,6,8,\ldots\}\) which
form an ideal of the integers: Sum and product of two even integers are
again even and the product of an even integer with \textbf{any} integer
yields an even number.

In a commutative ring with unity, an ideal is the set \(\{ax\}\) with
\(a\) fixed and \(x\) ranging over all elements of the ring. We have
\(ax + ay = a(x+y), -(ax) = (-x)a, y(xa) = (xy)a\). This ideal is called
the principal ideal generated by \(a\) and is denoted by
\(\langle a \rangle\).

Every ideal in the ring of integers \(\mathbb{Z}\) is a principal ideal.
The zero ideal \(\{0\}\) is a principal ideal because
\(\{0\} = \langle 0 \rangle\). Consider any non-zero ideal I: It must
contain some positive integer m and also a least positive integer n. Let
a be any element of I which we can express using the division algorithm
as

\[
a = nq + r
\]

with \(0 \leq r < n\). Rearranging for \(r\), we obtain \(r = a - nq\).
\(a \in I\) and \(nq\) is also in \(I\) (because it is of the form
\(n \times\) ``something''), an ideal is closed with respect to
subtraction); therefore \(r = a - nq \in I\). Since r by definition is
smaller than n, we conclude that \(r=0\) (I assume that n is \emph{the}
least positive integer; i.e.~1). So we have \(a = nq\) and therefore
\(I = \langle n \rangle\). \(\Box\)

\subsubsection{Example}\label{example}

As an example, consider
$\langle 3 \rangle = \{\ldots, -9, -6, -3, 0, 3, 6, 9, \ldots$.
Generally, the set \(n \mathbb{Z}\) is ideal in the ring of integers: If
\(na \in n \mathbb{Z}\) and \(b \in \mathbb{Z}\), then
\(nab \in \mathbb{Z}\) as required. In fact, the \(n \mathbb{Z}\) are
the \emph{only} ideals of $\mathbb{Z}$.

\subsection{Homomorphisms and Kernel}\label{homomorphisms-and-kernel}

A homomorphism from a ring A to a ring B is a function
\(f: A \rightarrow B\) satisfying the identities
\(f(x_1+x_2) = f(x_1) + f(x_2)\) and \(f(x_1 x_2) = f(x_1) f(x_2)\).

The kernel of a homomorphism is the set of all elements from A which the
homomorphism f maps onto the zero element of B.

\subsection{\texorpdfstring{Rings
\(\mathbb{Z}[x]\)}{Rings \textbackslash{}mathbb\{Z\}{[}x{]}}}\label{rings-mathbbzx}

This notation defines rings with the structure
\(\mathbb{Z}[x] = \{a + b x \}, a,b \in \mathbb{Z}\) and x so that
\(x \notin \mathbb{Z}\). An example can be \(x=\sqrt{3}\) or \(x=j\).
