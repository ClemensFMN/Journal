\DiaryEntry{Rings - Structure of Ideals}{2016-07-26}{Algebra}

A \emph{principal ideal} \(\langle a \rangle\) of \(\mathbb{Z}\) is
defined as follows:

\[
\langle a \rangle = \{ax\}
\] with \(a\) fixed in \(\mathbb{Z}\) and \(x\) ranging over all
\(\mathbb{Z}\). For example,
\(\langle 3 \rangle = \{\ldots, -9, -6, -3, 0, 3, 6, 9, \ldots\}\).

We have the fact that every ideal of \(\mathbb{Z}\) is principal.

\subsection{Divisibility}\label{divisibility}

If \(r, s\) are integers and \(s = rk\) with integer \(k\), then s is a
multiple of \(r\) or \(r\) is a factor of \(s\), or \(r\) divides \(s\).
We can further symbolize this as \(r | s\).

The only invertible elements of \(\mathbb{Z}\) are \(1\) and \(-1\);
i.e.~to have \(rs=1\), \(r=1\) or \(r=-1\) for \(s \in \mathbb{Z}\).

An integer pair \((r,s)\) is called associative, if the integers divide
each other; i.e. \(r|s\) and \(s|r\). If r and s are associates, then
\(r = \pm s\).

If an integer \(t\) divides both \(r\) and \(s\); i.e. \(t|r, t|s\),
then \(t\) is called a common divisor of \(r\) and \(s\). The greatest
common divisor \(\gcd(r,s)\) divides both \(r\) and \(s\) and every
other common divisor of \(r\) and \(s\) divides \(t\):
\(\gcd(r,s) | r, \gcd(r,s) | s\) and for any integer \(u\), if \(u|r\)
and \(u|s\), then \(u|t\).

The term greatest in the gcd means not necessarily ``greater in
magnitude'', but merely ``divides any other gcd''.

Any two non-zero integers r and s have a gcd t. This t is equal to a
linear combination of r and s as

\[
t = kr + ls
\]

for some integers k and l.

Proof: Let J be the set of all linear combinations of r and s;
\(J=\{ur+vs\}\) as u and v range over \(\mathbb{Z}\). J is closed with
respect to addition, and negatives, and absorbs products
(\(w(ur+vs) = (wu)r + (wv)s\)). Therefore J is an ideal of
\(\mathbb{Z}\); but since every ideal is a principal one, we can also
write \(J=\langle t \rangle\) for some t. Since t is in J, it is also a
linear combination of r and s: \(t = kr + ls\). r and s are also linear
combinations of r and s, therefore r and s are also in J. All elements
of J are multiples of t, so r and s are multiples of t; i.e.
\(t|r, t|s\). If u is a common divisor of r and s, we have
\(r=xu, s=yu\) for some integers x and y. Thus

\[
t = kr + ls = kxu + lyu = u(kx + ly)
\]

and from this follows that \(u|t\). From this follows that t is the gcd
of r and s.

\subsection{Primality and Factorization into
Primes}\label{primality-and-factorization-into-primes}

Two numbers r and s are relatively prime, if their gcd is 1. In other
words, we have \(kr + ls = 1\) for some integers k and l. This holds
also in the reverse direction: If \(kr + ls = 1\), then r and s are
relatively prime.

If an integer m has other factors than the trivial factors \(\pm 1\) and
\(\pm m\), these are called proper factors of m and m is called
composite. We have

\[
m = rs, \quad 1 < r < m, 1 < s < m
\]

If m and n are integers and p is a prime. If \(p|(mn)\), then either
\(p|m\) or \(p|n\).

Proof: If \(p|m\) then we are done. If p does not divide m, then we
proceed as follows: The only divisors of p are \(\pm 1\) and \(\pm p\).
Since p does not divide m, \(\pm p\) are ruled out as common divisors of
p and m; their only common divisors are \(\pm 1\).

Therefore \(\gcd(p,m=1\) and we can write \(kp + lm = 1\) (for some
integers k,l) by the previous theorem. Multiplying with n yields
\(kpn + lmn = n\). We assumed that \(p|(mn)\), so \(mn = ph\) (for some
integer h). So we can write \(kpn + lph = n \rightarrow p(kn + lh) = n\)
and therefore \(p|n\).

This can be extended to products with more than two factors: if
\(p|(m_1 \ldots m_t)\), then \(p|m_i\) for one of the factors. If the
factors \(m_1,\ldots,m_t\) are all positive primes, then p must equal
one of them.

Every integer \textgreater{} 1 can be expressed as product of positive
primes.

Proof: Let the set K represent positive integers \textgreater{} 1 which
cannot be written as product of one or more primes. By well-ordering, K
contains a least integer m which is not be prime (if it were, it would
not be in K). So m is composite, and we have \(m = rs\) for
\(1 < r,s < m\). The r and s are not in K because m is the least integer
in K. This means that r and s can be expressed as product of primes and
therefore also \(m=rs\). This is a contradicton (we assumed that m is
the smallest integer which cannot be expressed as product of one or more
primes) and therefore every \(n>1\) can be expressed as product of one
or more primes.

The prime factorization is unique.

Proof: Assume that \(m = p_1 \cdots p_r = q_1 \cdots q_t\) and cancel
common factors from each side until no more cancelling is possible. If
all factors cancel, we are done and factorization is unique. Otherwise,
we are left with an expression of the form

\[
p_i \cdots p_k = q_j \cdots q_m
\]

from which we deduce that \(p_i | q_j \cdots q_m\). The q's are all
prime, therefore \(p_i\) must be equal to one of them (see above). But
this is impossible since we assumed that no more cancelling is possible.
