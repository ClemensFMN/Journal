\DiaryEntry{Summing $1/(n-1)-1/n$}{2020-03-12}{Maths}

We consider the following sum (inspired by a BlackPenRedPen Youtube video),

\bee
S_1 = \sum_{n=2}^\infty \frac{1}{n-1} - \frac{1}{n}
\eee

As a first step, lets rewrite the expression,

\bee
S_1 = \sum_{n=2}^\infty = \frac{1}{n(n-1)}
\eee

From this we see - at least - that the summands asymptotically have the form $1/n^2$ and that therefore the sum should converge. Still, it is not obvious how we can tackle the sum; i.e. get a closed form expression for $S_1$. We use the following trick to express the summands via an integral. We have

\bee
\int_{x=0}^1 x^{n-2} dx = \left. \frac{1}{n-1}  x^{n-1} \right|_{x=0}^1 = \frac{1}{n-1}
\eee

and similarly,

\bee
\int_{x=0}^1 x^{n-1} dx = \left. \frac{1}{n-1} x^{n} \right|_{x=0}^1 = \frac{1}{n}
\eee

So instead of summing the easy expressions, we can sum the integrals instead.

\bee
S_1 = \sum_{n=2}^\infty \left( \int_{x=0}^1 x^{n-2} dx - \int_{x=0}^1 x^{n-1} dx \right)
\eee

This looks more difficult than before, however - when the sum converges - we are allowed to exchange sum with integral. Then we have

\bee
S_1 = \int_{x=0}^1 \left( \sum_{n=2}^\infty x^{n-2} \right) dx - \int_{x=0}^1 \left( \sum_{n=2}^\infty x^{n-1} \right) dx
\eee

Now we can solve the sums via the geometric series sum formula,

\bee
\sum_{n=2}^\infty x^{n-2} = x^0 + x + x^2 + \cdots = \frac{1}{1-x}, \quad \sum_{n=2}^\infty x^{n-1} = x + x^2 + \cdots = \frac{x}{1-x}
\eee

Inserting this back, we arrive at

\bee
S_1 = \int_{x=0}^1 \frac{1}{1-x} dx - \int_{x=0}^1 \frac{x}{1-x} dx
\eee

We can join the integrands, simplify them and finally get

\bee
S_1 = \int_{x=0}^1 \frac{1-x}{1-x}dx = 1
\eee

We can calculate the sum with Julia,

\begin{verbatim}
julia> n=2:10000; sum(1 ./ (n.-1) .- 1 ./ n)
0.9998999999999999
\end{verbatim}

which is reasonable close. \qed

It's so much fun, let's try another one,

\bee
S_2 = \sum_{n=1}^\infty \frac{1}{2n-1} - \frac{1}{2n}
\eee

We can rewrite the summands using integrals as follows,

\bee
\frac{1}{2n-1} = \int_{x=0}^1 x^{2n-2} dx , \quad \frac{1}{2n} = \int_{x=0}^1 x^{2n-1} dx
\eee

Inserting this into our sum epxressions, we obtain

\bee
S_2 = \sum_{n=1}^\infty \left( \int_{x=0}^1 x^{2n-2}dx - \int_{x=0}^1 x^{2n-1} dx \right) = \int_{x=0}^1 \left( \sum_{n=1}^\infty x^{2n-2} dx -  \sum_{n=1}^\infty x^{2n-1} \right) dx
\eee

Solving the sums (geometric series), we obtain

\bee
S_2 = \int_{x=0}^1 \frac{1}{1-x^2} - \frac{x}{1-x^2} dx = \int_{x=0}^1 \frac{1-x}{(1-x)(1+x)} dx = \int_{x=0}^1 \frac{1}{(1+x)} dx
\eee

This is a boring standard integral, we substitute $u = 1+x$ and arrive at

\bee
S_2 = \int \frac{1}{u} du = \ln u = \left. \ln(1+x) \right|_{x=0}^1 = \ln(2)
\eee

Comparing with Julia yields

\begin{verbatim}
julia> log(2)
0.6931471805599453

julia> n=1:100; sum(1 ./ (2*n.-1) .- 1 ./ (2*n))
0.690653430481824

julia> n=1:1000; sum(1 ./ (2*n.-1) .- 1 ./ (2*n))
0.6928972430599374

julia> n=1:10000; sum(1 ./ (2*n.-1) .- 1 ./ (2*n))
0.6931221811849451
\end{verbatim}

It converges slowly, but converges to $\ln 2$. \qed

%%% Local Variables:
%%% mode: latex
%%% TeX-master: "journal"
%%% End:
