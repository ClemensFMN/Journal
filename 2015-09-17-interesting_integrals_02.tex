\DiaryEntry{Interesting Integrals, 2}{2015-09-17}{Integrals}


\subsubsection{Integrals based on arctan}

Consider the integral (based on differentiation of arctan; see blog entry from 2015-08-25).


\[\int \frac{dx}{1+x^2} = \arctan x\]

From this we get for the definite integrals

\[ \int_0^1 \frac{dx}{1+x^2} = \arctan 1 = \frac{\pi}{4}\]

and

\[ \int_1^\infty \frac{dx}{1+x^2} = \arctan \infty - \arctan 1 = \frac{\pi}{4}\]

Funny that these integrals yield the same value.

The case of $\int\frac{x}{1+x^2}$ is simpler; set $u=1+x^2 \rightarrow dx = \frac{du}{2x}$, and we get

\[\int\frac{x}{1+x^2} = \frac{1}{2} \int \frac{du}{u} = \frac{1}{2} \ln(1+x^2)\]

A slightly more tricky integral is

\[\int \frac{1}{a+x^2} dx = \frac{1}{a} \int \frac{dx}{x^2/a+1}\]

With $u^2 = x^2/a$, we get $u = x / \sqrt{a} \rightarrow \frac{du}{dx} = 1 / \sqrt{a}$ and $dx = \sqrt{a} du$. Substitution back into the integral yields

\[ \frac{1}{a} \int \frac{dx}{x^2/a+1} = \frac{1}{a} \int \frac{\sqrt{a} du}{1+u^2} = \frac{\arctan (x/\sqrt{a})}{\sqrt{a}}\]

This is nice and useful and allows integration of

\[\int \frac{dx}{x^2+x+1} = \int \frac{dx}{(x+1/2)^2 + 3/4} \]

by completing the square. Setting $u = x+1/2$, we obtain $\frac{du}{dx} = 1$, and therefore

\[\int \frac{dx}{(x+1/2)^2 + 3/4} = \int \frac{du}{u^2 + 3/4} = \frac{\arctan (u/\sqrt{3/4})}{\sqrt{3/4}} = \frac{\arctan \left( \frac{x+1/2}{\sqrt{3/4}} \right)}{\sqrt{3/4}}\]

which can be further simplified to

\[ \int \frac{dx}{x^2+x+1} = \frac{2 \arctan \left( \frac{2x+1}{\sqrt{3}} \right)}{\sqrt{3}} \]

The final generalization is the integral $\int\frac{dx}{x^2+bx+c}$ which can be calculated by completing the square according to
$x^2+bx+c = (x+b/2)^2 + c - b^2/4$. With $u=x+b/2$ we
obtain $du=dx$, and finally

\[ \int\frac{dx}{x^2+bx+c} = \int \frac{du}{u^2 + c - b^2/4} = \frac{\arctan \frac{x+b/2}{\sqrt{c-\frac{b^2}{4}}}}{\sqrt{c-\frac{b^2}{4}}}\]

\subsubsection{Integrals based on arcsin}

From the blog entry on 2015-08-25 we have

\[\int \frac{dx}{\sqrt{1-x^2}} = \arcsin x\]

Generalizing, we obtain

\[\int \frac{dx}{\sqrt{a^2-x^2}} = \int \frac{dx}{a\sqrt{1-x^2/a^2}} \]

Substituting $u=x/a$, we get $dx = a du$ and finally

\[\int \frac{dx}{a\sqrt{1-x^2/a^2}} = \int \frac{du}{\sqrt{1-u^2}} = \arcsin \frac{x}{a}\]

Now consider the integral

\[ \int \frac{dx}{\sqrt{1-x-x^2}} \]

We can manipulate the expression
$1-x-x^2$ as follows

\[1-x-x^2= 1-(x^2+x) = 1-\left(x+\frac{1}{2}\right)^2+\frac{1}{4} = \frac{5}{4} - \left( x+\frac{1}{2} \right)^2\]

Setting $u=x+\frac{1}{2}$, we obtain for the integral

\[ \int \frac{dx}{\sqrt{1-x-x^2}} = \int \frac{du}{\sqrt{\frac{5}{4} - u^2}} = \arcsin \frac{u}{\sqrt{5/4}} = \arcsin \frac{2x+1}{\sqrt{5}}\]

\subsubsection{Trick: Completing the Square}

Most integrals were solved by a completing the square trick: Consider $x^2+x+1$. We can use an expression of the form $(x+1/2)^2 = x^2 + x + 1/4$ where the coefficients for $x^2$ and
$x$ are correct; only the coefficient for $x^0$ is wrong. We can correct this coefficient by adding a constant to the expression so that it becomes correct:

\[ x^2+x+1 = \left(x+\frac{1}{2}\right)^2 + \frac{3}{4} \]

Now we can e.g.~substitute $u=x+\frac{1}{2}$ which has the
additional advantage that $\frac{du}{dx} = 1$.
