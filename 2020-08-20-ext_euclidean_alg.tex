\DiaryEntry{Extended Euclidean Algorithm}{2020-08-20}{General}

We start with the following fact (coming from \ref{2017-05-01:entry}): If $a \geq b$, then $\gcd(a,b) = \gcd(a-b, b)$. We can repeatedly subtract and arrive at $\gcd(a,b) = \gcd(a \mod b, b)$.

This forms the basis for the Euclidean algorithm: We repeatedly use $\gcd(a,b) = \gcd(a-b, b)$ to reduce $a$ and $b$ until one of them becomes zero. In this case, the non-zero value (this can be either $a$ or $b$) is the gcd of $a, b$.

In Julia, the algorithm looks as follows. It follows above description quite closely, the only tricky thing is to store the value $b$ of the previous iteration; i.e. the iteration before $b$ became zero.

\begin{verbatim}
    function euclidean_gcd(a, b)
    while b!= 0
        t = b
        b = mod(a, b)
        a = t
    end
    return a
end
\end{verbatim}

The extended Euclidean algorithm uses the same loop; in addition it tracks additional variables which allow it to return $x$ and $y$ which fulfill

\bee
ax + by = \gcd(a,b)
\eee

The Julia implementation looks as follows. A proof for the correctness can be found \href{https://en.wikipedia.org/wiki/Extended_Euclidean_algorithm}{here}. Similar to above, the tricky part is to keep track of $a, x, y$ in the iteration \emph{before} the variable $a$ became zero.

\begin{verbatim}
function ext_euclidean_gcd(a, b)
    x = 1
    y = 0
    x1 = 0
    y1 = 1
    a1 = a
    b1 = b
    while(b1 != 0)
        q = div(a1, b1)
        x, x1 = x1, x - q*x1
        y, y1 = y1, y - q*y1
        a1, b1 = b1, a1 - q*b1
    end
    return a1, x, y
end
\end{verbatim}




%%% Local Variables:
%%% mode: latex
%%% TeX-master: "journal"
%%% End:
