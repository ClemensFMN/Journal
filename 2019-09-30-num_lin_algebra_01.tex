\DiaryEntry{Numerical Linear Algebra, 1}{2019-09-30}{Numercial Algebra}

\subsection{Matrix $\times$ Vector}

First we consider a multiplication $m \times n$ matrix $\Abf$ with an $n$-element vector $\abf$. We can rewrite the product as linear combination of the matrix columns with the vector elements as weigths,

\bee
\Abf \vbf = [\abf_1 \abf_2 \cdots \abf_n] \begin{bmatrix} v_1 \\ v_2 \\ \cdots \\ v_n \end{bmatrix} = \abf_1 v_1 + \abf_2 v_2 + \cdots + \abf_n v_n
\eee

Based on this interpretation, we can define the \emph{range} of a matrix as the vector space spanned by the columns of $\Abf$. The \emph{nullspace} of the matrix $\Abf$ is the set of vectors for which $\Abf \xbf = \zerobf$.

If the vectors $\abf_i$ all point in different directions (i.e. are linearly independent), then no linear combination of them can become the zero vector. If one or more vectors point in the same direction, then a vector $\vbf \neq \zerobf$ can cause the linear combination to become $\zerobf$.

As an example, consider

\bee
\Abf = \begin{pmatrix} 1 & 0 \\ 0 & 1 \end{pmatrix} \, , \quad \Abf \vbf = [v_1 \, v_2]^T
\eee

This will not become $\zerobf$ unless $v_1 = v_2 = 0$; i.e. $\vbf = \zerobf$. On the other hand, the two column vectors of the matrix

\bee
\Abf = \begin{pmatrix} 1 & 2 \\ 2 & 4 \end{pmatrix}
\eee

are linearly dependent and we have

\bee
\Abf \vbf = [v_1 + 2 v_2 \, 2 v_1 + 4_2]^T
\eee

This becomes zero whenever $v_1 = -2v_2$; i.e. vectors of the form

\bee
\vbf = [t \, -\frac{1}{2}t]^T \, , \quad t \in \mR
\eee

yield $\Abf \vbf = \zerobf$.

\paragraph{Rank.} The \emph{column rank} of a matrix is the dimension of its column space. The \emph{row rank} of a matrix is the dimension of its row space. Interestingly, the column rank is always equal the column rank.

Therefore, an $m \times n$ matrix can have a maximum rank of $\min(m,n)$. If this is the case, the matrix is said to be \emph{full-rank}. In this case it maps to two distinct vectors to the same vector.

If $\Abf$ is full-rank, then its columns are linearly independent and form a biass fir $\Abf$. This means that every $\bbf \in \text{range}(\Abf)$ has a unique expansion in terms of the columns of $\Abf$ and therefore $\bbf$ has a unique $\xbf$ such that $\bbf = \Abf \xbf$. If $\Abf$ is not full-rank, its columns are not independent and there is a non-trivial combination $\sum_i c_i \abf_i = \zerobf$. The non-zero vector $\cbf$ formed from the coefficients $c_j$ satisifies $\Abf \cbf = \zerobf$. But then $\Abf$ maps distinct vectors to the same vector because we have $\Abf \xbf = \Abf(\xbf + \cbf)$.

\paragraph{Inverse.} 

A \emph{nonsingular} or \emph{inverstible} matrix is a square matrix of full rank. The $m$ columns of a nonsingular $m \times m$ matrix $\Abf$ for a basis for the whole space $\mC^m$ or $\mR^m$. Any vector can therefore be uniquely expressed as a linear combination of its columns. In particular, the $j$-th unit vector $\ebf_j$ can be expressed as linear combination

\bee
\ebf_j = \sum _i z_{ij} \abf_i
\eee

Let $\Zbf$ be the matrix with entries $z_{ij}$ and let $\zbf_j$ denote the $j$-th column. Then we have $\ebf_j = \Abf \zbf_j$ and $\Ibf = \Abf \Zbf$ with identity matrix $\Ibf$ and $\Zbf$ begin the \emph{inverse} of $\Abf$. We write $\Abf^{-1}$ for the inverse and therefore $\Abf \Abf^{-1} = \Ibf$.

The following theorem links several conditions that hold when a square matrix is nonsingular.

\begin{theorem}
  For $\Abf \in \mC^{m \times m}$, the following conditions are equivalent:

  \begin{itemize}
  \item $\Abf$ has an inverse $\Abf^{-1}$,
  \item $\text{rank}(\Abf) = m$,
  \item $\text{range}(\Abf) = m$,
  \item $\text{null}(\Abf) = \{0\}$,
  \item $0$ is not an eigenvalue of $\Abf$,
  \item $0$ is not a singular value of $\Abf$,
  \item $\det(\Abf) \neq 0$.
\end{itemize}
\end{theorem}

\subsection{Orthogonal Vectors and Matrices}

A \emph{hermitian conjugate} or \emph{adjoint} $\Abf^\star$ of an $m \times n$ matrix $\Abf$ is the $n \times m$ matrix whose $i-j$-th entry is the complex conjugate of the $j-i$-th entry of $\Abf$. If $\Abf = \Abf^\star$, then $\Abf$ is \emph{hermitian}.

In case of real $\Abf$, the adjoint interchanges rows and columns and is also denoted as \emph{transpose}, written as $\Abf^T$. 

\paragraph{Innter Product.} The inner product of two vectors $\xbf$ and $\ybf$ is given by $\xbf^T \ybf$ and the Euclidian length of $\abf$ is

\bee
|| \xbf || = \sqrt{\xbf^T \xbf} = \left( \sum_i x_i^2 \right)^{1/2}
\eee

The cosine of the angle $\alpha$ between two vectors is given by

\bee
\cos \alpha = \frac{\xbf^T \ybf}{||\xbf|| ||\ybf||}
\eee

Two vectors are \emph{orthogonal} if their inner product is zero.

The inner product is bilinear which means

\begin{align*}
  (\xbf_1 + \xbf_2)^T \ybf &= \xbf_1^T \ybf + \xbf_2^T \ybf \\
  \xbf^T(\ybf_1 + \ybf_2) &= \xbf^T \ybf_1 + \xbf^T \ybf_2 \\
  (\alpha \xbf)^T (\beta \ybf) &= \alpha \beta \xbf^T \ybf
\end{align*}

In a similar spirit, we also have for matices

\bee
(\Abf \Bbf)^T = \Bbf^T \Abf^T
\eee

and

\bee
(\Abf \Bbf)^{-1} = \Bbf^{-1} \Abf^{-1}
\eee

\paragraph{Orthogonal Vectors.} A set $\Sc$ of non-zero vectors is \emph{orthogonal} if the elements are pairwise orthogonal. The set is \emph{orthonormal} if it is orthogonal and all vectors have unit-length.

Using orthonormal vectors, we can decompose any vector into orthogonal components. Suppose we have a set of orthonormal vectors, $\Sc = \{\qbf_1 , \ldots, \qbf_N\}$, and let $\vbf$ be an arbitrary vector. We can calculate the residual vector

\bee
\rbf = \vbf - (\vbf^T \qbf_1) \qbf_1 - (\vbf^T \qbf_2) \qbf_2 - \cdots - (\vbf^T \qbf_N) \qbf_N
\eee

This residual vector is orthogonal to $\Sc$ as

\bee
\qbf_k^T \rbf = \qbf_k^T \vbf - (\vbf^T \qbf_1) \qbf_k^T \qbf_1 - (\vbf^T \qbf_2) \qbf_k^T \qbf_2 - \cdots - (\vbf^T \qbf_N) \qbf_k^T \qbf_N 
\eee

All terms $\qbf_k^T \qbf_i$ vanish for $k \neq i$ and we have

\bee
\qbf_k^T \rbf = \qbf_k^T \vbf - (\vbf^T \qbf_k) \qbf_k^T \qbf_k = \qbf_k^T \vbf - (\vbf^T \qbf_k) = 0
\eee

Thus we see that we decompose $\vbf$ into $N+1$ orthogonal components

\bee
\vbf = \rbf + \sum_{i=1}^N (\qbf_i^T \vbf) \qbf_i
\eee

In this decomposition, $\rbf$ is the part of $\vbf$ which is orthogonal to the vector set $\Sc$. If the vector set $\Sc$ is a basis for $\mR^m$, then $n = m$ and $\rbf$ is the zero vector; i.e. we can decompose $\vbf$ into the $N$ vectors $\{\qbf_i\}$,

\bee
\vbf = \sum_{i=1}^N (\qbf_i^T \vbf) \qbf_i
\eee

\paragraph{Unitary Matrices.} A square matrix $\Qbf$ is \emph{unitary}, if $\Qbf^T = \Qbf^{-1}$; i.e. if $\Qbf^T \Qbf = \Ibf$. This means that the columns of a unitary matrix are orthonormal and form a basis for $\mC^m$. The inner product is an invariant under multiplication with unitary matrices; we have

\bee
(\Qbf \xbf)^T (\Qbf \xbf) = \xbf^T \Qbf^T \Qbf \xbf = \xbf^T \xbf
\eee

This also implies that the vector length does not change when the the vector is multiplied by a unitary matrix. If a unitary matrix $\Qbf$ is real and $\det \Qbf = 1$, then multiplication corresponds to a rigid rotation; if $\det \Qbf = -1$, then multiplication corresponds to reflection.

\subsection{Basis Expansion.} A basis expansion of a vector $\xbf$ in a basis given by the matrix columns $\qbf_i$ of the matrix $\Qbf$ finds the coefficients of the same vector in the new basis. Collecting these coefficients in a vector $\ybf$, we have

\bee
\Qbf \ybf = \xbf \rightarrow \ybf = \Qbf^{-1} \xbf
\eee

This shows that a basis expansion requires a full-rank matrix $\Qbf$; i.e. the column vector vectors have to be linearly independent. The range of a matrix with not full rank does not provide ``enough dimensions''. In particular, the columns need not be orthogonal. However, orthogonal basis vectors simplify things considerably, as we then have

\bee
\ybf = \Qbf^{-1} \xbf = \Qbf^H \xbf
\eee

This can be interpretated that the basis expansion corresponds to taking the inner products of $\langle \qbf_i, \xbf \rangle$.

\paragraph{Examples.} As a first example, consider the matrix

\bee
\Abf = \begin{pmatrix} 2 & 1 \\ 0 & 1 \end{pmatrix}
\eee

which columns are linearly independent (but not orthonormal). The matrix is invertible with

\bee
\Abf^{-1} = \begin{pmatrix} 1/2 & -1/2 \\ 0 & 1 \end{pmatrix}
\eee

and a vector $\xbf=[2 \; 1]^T$ becomes $\ybf = [1/2 \; 1]^T$. As a contrast, the matrix
\bee
\Abf = \frac{1}{\sqrt{2}} \begin{pmatrix} 1 & -1 \\ 1 1 \end{pmatrix}
\eee

corresponds to a rotation by $\pi/4$. The columns are linearly independent and additionally, orthonormal: $\Abf \Abf^T = \Ibf$. A vector $\xbf=[2 \; 1]^T$ becomes $\ybf = [2.12 \; -0.707]^T$.



%%% Local Variables:
%%% mode: latex
%%% TeX-master: "journal"
%%% End:
