\DiaryEntry{50 Challenging Problems in Probability, Problem 9 (Craps)}{2016-07-15}{Stochastic}


Consider what happens when we have a sum of 4; called point 4 (btw, this
happens with probability \(P_4 = 3/36 = 1/12\) as there are three
different ways to obtain a sum of \(4\): \(1+3, 3+1, 2+2\) and each
summand has a probability of \(1/36\)): We throw again the two dice and
have 11 outcomes (2\ldots{}12):

\begin{itemize}
\item
  With probability of \(3/36\) (as before) we throw a sum of \(4\)
  again, have won immediately and stop.
\item
  With probability of \(6/36 = 1/6\) (\(1+6, 2+5, 3+4, 4+3, 5+2, 6+1\))
  we throw a sum of \(7\), have lost immediately and stop.
\item
  In all other cases we contine with throwing the dice. The probability
  of doing this is \(P_{R4} = 1 - P_4 - 1/6\) which happens to be
  \(P_{R4} = 3/4\).
\end{itemize}

The (conditional) probability that we win after we have thrown a point
4; i.e.~we throw another point 4, P(win \textbar{} point 4), can be
calculated as follows:

\begin{itemize}
\item
  We immediately throw another point 4 with probability \(P_4\).
\item
  We throw ``something else'' and then a point 4 - this has probability
  \(P_{R4}P_4\)
\item
  We throw ``something else'' two times and then a point 4 - this has
  probability \(P_{R4}^2 P_4\)
\end{itemize}

Therefore the conditional probability P(win \textbar{} point 4) becomes
\(P_4 (1 + P_{R4} + P_{R4}^2 + \cdots = P_4 \frac{1}{1 - P_{R4}} = 1/12 \frac{1}{1 - 3/4} = 1/3\).

There is appearently another (simpler) way to obtain this conditional
probability, called ``reduced sample space''. For this we ignore
\emph{all other} outcomes than the one for immediate win and immediate
loss: There are three good cases (\(1+3, 3+1, 2+2\)) and six bad cases
(\(1+6, 2+5, 3+4, 4+3, 5+2, 6+1\)); therefore the conditional
probability P(win \textbar{} point 4) is \(3/(3+6) = 1/3\).

Given the fact that we obtain a point 4 with probability \(P_4\), we
have the overall probability to win with a point 4 of
\(P_4^2 \frac{1}{1 - P_{R4}} = 1/12 \times 1/3\).

\subsection{Simpler Problem}

Maybe the example with the non-uniform probabilities is a bit
complicated and convoluted. Let's try a simpler one: We have one dice;
when we throw a 1 we win immediately, when we throw a 6 we lose
immediately, and in all other cases, we start again (with throwing the
dice). With a probability of \(1/6\) we win immediately, with a
probability of \(1/6\) we lose immediately, and with probability \(4/6\)
we throw again. In this case, the probabilites for losing, winning, and
continuing are the same.

The probability for winning in the continue case is therefore:
\(1/6 \times 4/6 + 1/6 \times (4/6)^2 + \cdots = 1/6 \times \left[ \frac{1}{1-4/6} - 1\right] = 1/3\).

Therefore the overall probability of winning is \(1/6 + 1/3 = 1/2\) (we
have added the probability of throwing a one).

The simpler way is to solve the problem as folllws: At every instant,
there is one good case (i.e.~we win) by throwing a 1 out of a total of
two cases (throw 1 and throw 6); therefore the probability of winning is
\(1/2\). The other 4 cases (which lead to another dice throw) are not
considered as we ``reduce the sample space''.

\subsection{Finite Problem}

Above problems have in common that they are infinite; i.e.~under certain
conditions (throwing a 2\ldots{}5), the players throws the dice again
and this repeats indefinitely. Things are different when the game stops
after a certain amount of throws; as example consider the following
ruleset:

If the first throw gives a 1, the game is won immediately; if a 6 is
thrown, then the game is lost immediately. In case of throwing
2\ldots{}5, throw the dice again. For the second throw, the same rules
apply. In the third throw, everything but a 1 leads to an immediatey
loss, throwing a 1 leads to an immediatey win; i.e.~the game ends in any
case.

The probability of winning is therefore

\begin{itemize}

\item
  probability of throwing a one in the first throw, \(1/6\)
\item
  plus the probility of throwing a one in the second throw,
  \(4/6 \times 1/6\)
\item
  plus the probability of throwing a one in the third throw,
  \(4/6 \times 4/6 \times 1/6\)
\end{itemize}

which sums to
\(1/6 + 4/6 \times 1/6 + 4/6 \times 4/6 \times 1/6 = 0.35...\). It is
the third throw ending the game which causes the probability to become
less than in the games above.
