\DiaryEntry{NTP / Clock Sync}{2020-10-23}{General}

\subsection{Introduction}

We have two computers A and B, each having their own clock. The clocks have the same period (that's one assumption) but are shifted by some time offset $\theta$. The computers are connected via a network (IP) introducing some delay when the computer exchange UDP packets (which ar eused by NTP).

Ad time offset: Assume the operators at the two computers observe the same event (e.g. flash of lightning). When they look at their computer's clock, the difference is $\theta$.

NTP has several operation modes; we concentrate on syncing a client computer's clock (computer A) to a server computer's clock (computer B).

\paragraph{Definition Round-trip Delay.} Assume computer A sends a packet to computer B at time $t_0$; i.e. compter A's clock shows $t_0$ at time of sending. The packet is received when computer B's clock shows time $t_1$. Now the server processes the packet and sends it back at computer B's time $t_2$ and it is received at computer A's time $t_3$. The time the packet spent on the network is defined as round-trip delay $\delta$ and given by

\bee
\delta = t_3 - t_0 - (t_2 - t_1)
\eee

Note that it does not matter that computer A's clock has an offset to computer B's clock; we are only interested in time differences, therefore the time offset $\theta$ is not relevant here.

\paragraph{Time Shift.} Now let's concentrate on what happens when computer A sends the packet: It is sent at computer A's time $t_0$, then it spends $\delta/2$ time on the network (assuming the network delay is the same in both directions, namely $\delta/2$); at time of reception computer B's time shows $t_1$ which is shifted by $\theta$ from computer A's time. We therefore have

\be\label{eq:ntp_1}
t_1 = t_0 + \delta/2 + \theta
\ee

The same happens in the other direction; at time $t_2$ computer B sends the packet, it spends $\delta/2$ on the network and is received at computer B's time $t_3$. We therefore have

\bee
t_3 = t_2 + \delta/2 - \theta
\eee

Note that in this case the time offset works in the "other direction"; i.e. $\theta$ is subtracted to get from computer B's time to computer A's time. We can rewrite and obtain

\be\label{eq:ntp_2}
t_2 = t_3 - \delta/2 + \theta
\ee

Adding the last equation to \eqref{eq:ntp_1} yields

\bee
t_1 + t_2 = t_0 + \theta + t_3 + \theta
\eee

Luckily the network delay has cancelled and we obtain for the time offset

\bee
\theta = \frac{(t_1 - t_0) + (t_2 - t_3)}{2}
\eee


\subsection{Extensions, Further Thoughts}

What happens when the round-trip delay is different in each direction? Then we have $\delta_1 + \delta_2 = \delta$ but $\delta \neq \delta_2$. Adding \eqref{eq:ntp_1} and \eqref{eq:ntp_2} does not cancel the delays. Maybe the NTP interleaved mode helps?

We have assumed that the clocks have the same period and have only a ttime shift. What happens if the clock period is different? As a first consequence the time shift becomes time dependent; i.e. we have $\theta = \theta(t)$ and the magic in above equations does not work either. Not sure, however, whether this is relevant as the times $t_0 \cdots t_3$ are typcially small (less than a second) and the clock period difference is much higher (clocks maybe differ in a couple of seconds per day?)


%%% Local Variables:
%%% mode: latex
%%% TeX-master: "journal"
%%% End:
