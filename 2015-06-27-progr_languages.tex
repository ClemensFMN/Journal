\DiaryEntry{Programming Languages}{2015-06-27}{Programming}

\subsection{Tasks}\label{tasks}

\begin{enumerate}
\def\labelenumi{\arabic{enumi}.}
\item
  Numerical stuff (lin. alg., (statistical)signal progessing,
  plotting\ldots{})
\item
  Scripting language for simple, day-to-day stuff
\item
  Gui development; i.e.~Qucs complexity level
\item
  something non-standard, supporting advanced concepts; e.g.~functional
  programming, macros, actors\ldots{}
\item
  something for web applications(?)
\end{enumerate}

\subsection{Languages}\label{languages}

\subsubsection{Julia}\label{julia}

Advantages: Modern, performant. Vector / matrix / ... support ootb. A language for scientific computing. Quite vibrant ecosystem but not as complete as Python; maybe use PyCall.

\subsubsection{Python}\label{python}

Nice for everyday scripting, but numerical stuff (numpy) is awkward; e.g.

\begin{verbatim}
self.current_x_hat = dot(A, x_hat)
self.current_Sigma = dot(A, dot(Sigma, A.T)) + Q
\end{verbatim}

Vectors are supported, but matrices are not really integrated into the language. A general purpose language with support for scientific computing (see Julia). Julia offers with PyCall a nice way to tap into the Python scientific ecosystem.

\subsubsection{C\# / Mono + Winforms}\label{c-mono-winforms}

The ``classical'' way to go; cross-platform, lots of documentation available (Windows World), sufficiently ``modern'' (more than Java).

\subsubsection{Scheme / Racket}\label{scheme-racket}

Seems to be sufficiently close to ``classical LISP books'' (SICP, The little Schemer\ldots{}) to use it; nice IDE and nice library ecosystem.

\subsubsection{Ruby}\label{ruby}

Perfect for the few web apps I do (e.g.~in contrast to Django). In addition, maybe it can replace Python as everyday scripting language. Need to check ecosystem and compare with Python (main competitor). Nicer/cleaner syntax/structure than Python in any case; e.g.

Python:

\begin{verbatim}
x = [1,4,3,6]
len(x)
but: x.sort() is a method which does in-place modification!
\end{verbatim}

Ruby

\begin{verbatim}
x = [1,4,3,6]
x.size
x.sort returns a sorted list; no in-place modification!
x.sort! sorts in-place
\end{verbatim}

\subsubsection{C++ / Qt}\label{c-qt}

Qt is a quite complete framework; so C++ with Qt is not ``really'' C++. Nevertheless, too cumbersome and replaced by C\# / Mono with WinForms.

\subsubsection{Java / Swing}\label{java-swing}

Java is enterprise, but Swing is a bit older than C\# Mono + WinForms is more modern.

\subsubsection{Scala}\label{scala}

Coming up as something new. Uses the JVM large ecosystem. Sufficiently modern / functional / cool. Not overly hyped and not so weird as e.g.~Haskell.

\subsubsection{Rust}\label{rust}

Nope - I don't think it introduces new concepts, but tries to be a better (safer) C/C++. However, for all microcontroller/low-level system stuff, there is C. No need for another Baustelle.
