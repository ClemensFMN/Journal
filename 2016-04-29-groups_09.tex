\DiaryEntry{Groups - External Direct Product}{2016-04-29}{Algebra}


If we have two groups \((G, \star)\) and \((H, \circ)\) then we can make
the Cartesian product of G and H into a new group. The new group is just
the ordered pairs \((g, h) \in G \times H\) and we define a binary
operation on \(G \times H\) by
\((g_1, h_1)(g_2, h_2) = (g_1 \star g_2, h_1 \circ h_2)\) - i.e.~we
multiply elements in the first coordinate as we do in G and and elements
in the second coordinate as we do in H.

The result of this is a group, as

\begin{itemize}
\item
  the binary operation is closed.
\item
  If \(e_G\) and \(e_H\) are the identities of the groups G and H
  respectively, then \((e_G, e_H)\) is the identity of \(G \times H\).
\item
  The inverse of \((g, h) \in G\times H\) is \((g^{-1}, h^{-1})\).
\end{itemize}

\subsubsection{Example}\label{example}

Let \(\mathbb{R}\) be the group of real numbers under addition. The
Cartesian product of \(\mathbb{R} \times \mathbb{R}\) is also a group,
in which the group operation is just addition in each coordinate; that
is, \((a, b) + (c, d) = (a + c, b + d)\). The identity is (0,0) and the
inverse of \((a, b)\) is \((-a, -b)\).

The external product can be extended to the Cartesian product of \(M\)
groups; for example \(\mathbb{Z}_2^M\) is a group of all binary
M-tuples. The group operation is the \emph{exclusive or} of two binary
M-tuples.

\subsection{Internal Direct Product}\label{internal-direct-product}

The external product makes a larger group out of two (or more) smaller
groups. The internal direct product allows to express a large group as
isomorphic to the direct product of two of its subgroups.

Let G be a group with subgroups H and K satisfying the following
conditions:

\begin{itemize}
\item
  G can be expressed by combining elements of the two subgroups by the
  group operation; i.e. \(G = HK = \{ h \star k : h \in H, k \in K \}\).
\item
  \(H \cap K = \{ e \}\)
\item
  \(hk = kh\) for all \(k \in K\) and \(h \in H\)
\end{itemize}

Then G is the \emph{internal direct product} H and K. The above three
conditions are pretty strong in the sense that it is rather easy to find
subgroups of a given group; however finding two subgroups from which the
original group can be constructed is much harder. I would assume that
only few groups can be expressed as internal direct prduct.

As an example consider the group of unity \(U(8) = \{1,3,5,7\}\). If we
define two subgroups \(H = \{1,3\}\) and \(K = \{1,5\}\), we can express
G as product of the elemts of the subgroups (the only non-obvious case
is \(3 \star 5 = 3 \times 5 \mod 8 = 7\)). The subgroups H and K do not
share any element apart the unit element and finally \(U(8)\) is an
Abelian group. Therefore \(U(8)\) is the internal direct product of H
and K.

Another example is \(G = \mathbb{Z}_4\) with modulo-4 addition. If we
define \(H=\{0,2\}\) and \(K=\{0,1\}\), then we express every element of
\(G\) by combining elements from H and K, and H and K do not share any
element apart from the identity element. Since G is abelian, the last
condition is fulfilled as well.
