\DiaryEntry{Interesting Integrals, 3}{2016-02-11}{Integrals}

Consider the integral

\[
I(A,B) = \int_A^B \frac{dx}{x^2-1}
\]

Make a partial fraction expansion according to

\[
\frac{1}{x^2-1} = \frac{1}{(x-1)(x+1)} = \frac{A}{x-1} + \frac{B}{x+1}
\]

Cross-multiplication yields

\[
1 = A(x+1) + B(x-1) = x(A+B) + (A-B)
\]

and


\begin{align}
A+B & = 0 \\
A-B & = 1
\end{align}


from which we obtain \(A = 1/2\) and \(B = -1/2\). So finally, the
integral can be rewritten as

\[
\int_A^B \frac{1}{x^2-1} = \frac{1}{2} \int_A^B \frac{1}{x-1} - \frac{1}{2} \int_A^B \frac{1}{x+1}
\]

and this form has a closed-form solution according to

\[
\int \frac{1}{x^2-1} \left( \frac{1}{2} \ln \frac{1}{x-1} - \frac{1}{2} \ln \frac{1}{x+1} \right) = \ln \sqrt{\frac{x-1}{x+1}}
\]

Therefore the definite integral becomes

\[
I(A,B) = \ln \sqrt{\frac{(B-1)(A+1)}{(B+1)(A-1)}}
\]

We can verify this with the following Python script

\begin{verbatim}
def f3(x):
    return 1/(x**2 - 1)

A=1.1
B=4
I = integrate.quad(f3, A, B)
print(I)

t1 = (B-1)*(A+1)/(B+1)/(A-1)

print(np.log(np.sqrt(t1)))
\end{verbatim}
