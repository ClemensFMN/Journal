\DiaryEntry{Conjugacy Classes}{2020-09-29}{Algebra}

Conjugacy classes group "similar" group elements together. These elements share many properties and allow further insights into the structure of (non-Abelian) groups.

\begin{definition}
    Two elements $a$ and $b$ of a group $G$ are conjugate, if $b = g^{-1}ag$ for some $g \in G$. This is an equivalence relation and the equivalence classes are called conjugacy classes.    
\end{definition}

The formal definition of a conjugacy class is then as follows.

\begin{definition}
    The conjugacy class $Cl(a)$of $a$ is defined as $CL(a) = \{gag^{-1} | g \in G\}$. The class number of $G$ is the number of distinct conjugacy classes. All elements belonging to the same conjugacy class have the same order.
\end{definition}

If we have an Abelian group (i.e. $ab = ba$), every group element is conjugate to itself only, as $b = g^{-1}ag = g^{-1}ga = a$. Correspondingly, the conjugacy classes contain only one element; i.e. $Cl(a) = \{a\}$. Therefore, conjugacy classes are (mostly) interesting for non-Abelian groups only.

\paragraph{Symmetric Group $S_3$.} As a first example, we calculate the conjugacy classes of $S_3$. We use a brute force approach by choosing a group element $a$ and iterate over all $g \in G$. The resulting elements of $gag^{-1}$ are collected in a set. The elements may be duplicated; however, they are removed when the result is stored in a set; see \href{file/gap_conjugcy_classes.gp}{this script}.

For $a = e$, we obtain $CL(a) = \{\}$, for $a = (1,2)$ we obtain $Cl((1,2)) = \{(1,2), (2,3), (1,3)\}$ and finally for $a = (1,2,3)$ we obtain $Cl((1,2,3)) = \{(1,2,3), (1,3,2)\}$. From this we see that similar permutations (no change, exchange of two elements, exchange of three elements) fall into to the same conjugy class.

We can also use GAP to obtain the conjugcy classes; this produces the following result.

\begin{verbatim}
    gap> c:=ConjugacyClasses(S3);
    [ ()^G, (1,2)^G, (1,2,3)^G ]
\end{verbatim}

The notation may be a bit unusual, but matches the results calculated manually. Finally, we note that we can obtain the elements of a conjugacy class as well.

\begin{verbatim}
    gap> Elements(c[1]);
    [ () ]
    gap> Elements(c[2]);
    [ (2,3), (1,2), (1,3) ]
    gap> Elements(c[3]);
    [ (1,2,3), (1,3,2) ]
\end{verbatim}

We can describe this in the \emph{class equation}; it describes the partitioning of the group by its conjugacy classes: the sum of of the size of the conjugacy classes equals the order of the group. The class equation is therefore $1 + 3 + 2 = 6$ which equals the order $6$ of $S_3$.

\paragraph{Symmetric Group $S_4$.} Things get a bit more interesting with the conjugacy classes of $S_4$. Using GAP we obtain the following classes

\begin{verbatim}
    gap> S4:=SymmetricGroup(4);
    Sym( [ 1 .. 4 ] )
    gap> c:=ConjugacyClasses(S4);
    [ ()^G, (1,2)^G, (1,2)(3,4)^G, (1,2,3)^G, (1,2,3,4)^G ]
\end{verbatim}

We can interpret this as follows:

\begin{itemize}
    \item There is one conjugacy class containing the identity element $()$ which only contains the identity element.

    \item There is a class $(1,2)^G$ containing permutations between two items. It has the following ${4 \choose 2} = 6$ elements

    \begin{verbatim}
        gap> Elements(c[2]);
        [ (3,4), (2,3), (2,4), (1,2), (1,3), (1,4) ]
    \end{verbatim}

    \item There is a class $(1,2)(3,4)^G$ containing two permutations of two elements each; its elements are
    
    \begin{verbatim}
        gap> Elements(c[3]);
        [ (1,2)(3,4), (1,3)(2,4), (1,4)(2,3) ]
    \end{verbatim}

    \item There is class $(1,2,3)^G$ containing permutations of three items; it has elements
    
    \begin{verbatim}
        gap> Elements(c[4]);
        [ (2,3,4), (2,4,3), (1,2,3), (1,2,4), 
          (1,3,2), (1,3,4), (1,4,2), (1,4,3) ]
    \end{verbatim}

    \item And finally we have the conjugacy class $(1,2,3,4)^G$ of 4-item permutations with elements
    
    \begin{verbatim}
        gap> Elements(c[5]);
        [ (1,2,3,4), (1,2,4,3), (1,3,4,2), 
          (1,3,2,4), (1,4,3,2), (1,4,2,3) ]
    \end{verbatim}

\end{itemize}

The corresponding class equation is $1 + 6 + 3 + 8 + 6 = 24$.

%%% Local Variables:
%%% mode: latex
%%% TeX-master: "journal"
%%% End:
