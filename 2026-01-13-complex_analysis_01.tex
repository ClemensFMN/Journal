\DiaryEntry{Complex Analysis, 1}{2026-01-13}{Complex Analysis}

\subsubsection{Introduction}

We consider a complex number $z = x + jy$ and have $\Re z = x, \Im z = y$. The complex conjugate $\bar{z}$ is defined as $\bar{z} = x - jy$ and from this we obtain the following

\begin{equation*}
\Re z = \frac{z + \bar{z}}{2}, \Im z = \frac{z - \bar{z}}{2j}        
\end{equation*}


There is the triangle inequality

\begin{equation*}
|z_1 + z_2| \leq |z_1| + |z_2|    
\end{equation*}

When we apply the inequality to $z_2 - z_1$ and $z_1$, we get another interesting form

\begin{equation*}
|z_2| \leq |z_2 - z_1| + |z_1| \rightarrow |z_2| - |z_1| \leq |z_2 - z_1|    
\end{equation*}

We can use the triangle inequality in the following bound

\begin{align*}
    |z_1 + z_2|^2 &= (z_1 + z_2)(\bar{z_1} + \bar{z_2}) = z_1 \bar{z_1} + z_1 \bar{z_2} + z_2 \bar{z_1} + z_2 \bar{z_2} \\
    &= |z_1|^2 + (z_1 \bar{z_2} + \overline{z_1 \bar{z_2}} )+ |z_2|^2 \\
    &= |z_1|^2 + 2 \Re (z_1 \bar{z_2}) + |z_2|^2 \\
    &\leq |z_1|^2 + 2 |z_1 z_2| + |z_2|^2 = (|z_1| + |z_2|)^2
\end{align*}

\paragraph{Example.} When $|z| = 1$, we can bound

\begin{equation*}
|z^2 + 2z + 6 + 8j| \leq |z|^2 + 2 |z| + |6 + 8j| = 1 + 2 + \sqrt{36 + 64} = 13    
\end{equation*}


We can use $e^{j \theta} = cos \theta + j \sin \theta$ in the following

\begin{equation*}
    (\cos \theta + j \sin \theta)^n = (\left( e^{j \theta} \right)^n = e^{j n \theta} = \cos n \theta+  j \sin n \theta
\end{equation*}

A nice application if this is

\begin{align*}
    \cos 3 \theta &= \Re \left( cos 3\theta+  j \sin 3 \theta\right) \\
&= \Re \left( cos \theta + j \sin \theta\right)^3 \\
&= \Re \left( cos^3 \theta + 3j \cos^2 \theta \sin \theta - 3 \cos \theta\sin^2 \theta - j \sin^3 \theta \right) \\
&= cos^3 \theta  - 3 \cos \theta\sin^2 \theta = cos^3 \theta  - 3 \cos \theta (1 - \cos^2 \theta) = 4 \cos^3 \theta - 3 \cos \theta
\end{align*}


\subsubsection{Exponential Function}

\bee
e^{z} = e^{x + jy} = e^x e^{jy} = e^x (\cos(y) + j \sin(y))
\eee

\subsubsection{Trigonometric Functions}

Start with

\begin{align*}
    e^{j \theta} &= \cos \theta + j \sin \theta \\
    e^{-j \theta} &= \cos \theta - j \sin \theta
\end{align*}

Adding the two equations yields

\bee
\cos \theta = \frac{1}{2} \left( e^{j \theta} + e^{-j \theta} \right)
\eee

Subtracting yields

\bee
\sin \theta = \frac{1}{2j} \left( e^{j \theta} - e^{-j \theta} \right)
\eee

\subsubsection{Complex Differentiation}

The derivative of a complex function $f(z)$ is defined as

\bee
\frac{df(z)}{dz} = \lim_{\delta z \rightarrow 0} \frac{f(z + \delta z)}{\delta z}
\eee

Note that this simple definition disguises the fact, that we can choose $\delta z$ to have arbitrary direction and - in the general case - it is not clear that the limit is independent of the direction.

A complex function is called \emph{analytic} if the derivative is independent of the direction.

We consider two cases: (i) $\delta z = \delta x$ being real and (ii) $\delta z = j \delta y$ being imaginary. For each case, we calculate the derivative. In the following we split the real and imaginary parts; so we have

\bee
f(z) = u(x,y + j(v(x,y)))
\eee

\paragraph{Case (i): $\delta z = \delta x$.} We have

\begin{align*}
\frac{df(z)}{dz} &= \lim_{\delta x \rightarrow 0} \frac{u(x + \delta x, y) + j v(x + \delta x, y) - u(x, y) - j v(x, y) }{\delta x} \\
&= \lim_{\delta x \rightarrow 0} \frac{u(x + \delta x, y) - u(x, y) + j v(x + \delta x, y)  - j v(x, y) }{\delta x} = \frac{\partial u}{\partial x} + j \frac{\partial v}{\partial x}
\end{align*}


\paragraph{Case (ii): $\delta z = j \delta y$.} We have

\begin{align*}
\frac{df(z)}{dz} &= \lim_{\delta y \rightarrow 0} \frac{u(x, y + \delta y) + j v(x, y + \delta y) - u(x, y) - j v(x, y) }{j \delta y} \\
&= \lim_{\delta y \rightarrow 0} \frac{v(x, y + \delta y)  - v(x,y) - j v(u, y + \delta y) + j u(x, y) }{\delta y} = \frac{\partial v}{\partial y} - j \frac{\partial u}{\partial y}
\end{align*}

If we want the derivative to be independent of the direction, the real and imaginary part have to be the same; so we have the \emph{Cauchy-Riemann equations}

\bee
\frac{\partial u}{\partial x} = \frac{\partial v}{\partial y}, \quad \frac{\partial u}{\partial y} = - \frac{\partial v}{\partial x}
\eee


%%% Local Variables:
%%% mode: latex
%%% TeX-master: "journal"
%%% End: