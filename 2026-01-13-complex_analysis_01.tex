\DiaryEntry{Complex Analysis, 1}{2026-01-13}{Complex Analysis}

Some introductory notes

\subsubsection{Exponential Function}

\bee
e^{z} = e^{x + jy} = e^x e^{jy} = e^x (\cos(y) + j \sin(y))
\eee

\subsubsection{Trigonometric Functions}

Start with

\begin{align*}
    e^{j \theta} &= \cos \theta + j \sin \theta \\
    e^{-j \theta} &= \cos \theta - j \sin \theta
\end{align*}

Adding the two equations yields

\bee
\cos \theta = \frac{1}{2} (e^{j \theta} + e^{-j \theta})
\eee

Subtracting yields

\bee
\sin \theta = \frac{1}{2j} (e^{j \theta} - e^{-j \theta})
\eee

\subsubsection{Complex Differentiation}

The derivative of a complex function $f(z)$ is defined as

\bee
\frac{df(z)}{dz} = \lim_{\delta z \rightarrow 0} \frac{f(z + \delta z)}{\delta z}
\eee

Note that this simple definition disguises the fact, that we can choose $\delta z$ to have arbitrary direction and - in the general case - it is not clear that the limit is independent of the direction.

A complex function is called \emph{analytic} if the derivative is independent of the direction.

We consider two cases: (i) $\delta z = \delta x$ being real and (ii) $\delta z = j \delta y$ being imaginary. For each case, we calculate the derivative. In the following we split the real and imaginary parts; so we have

\bee
f(z) = u(x,y + j(v(x,y)))
\eee

\paragraph{Case (i): $\delta z = \delta x$.} We have

\begin{align*}
\frac{df(z)}{dz} &= \lim_{\delta x \rightarrow 0} \frac{u(x + \delta x, y) + j v(x + \delta x, y) - u(x, y) - j v(x, y) }{\delta x} \\
&= \lim_{\delta x \rightarrow 0} \frac{u(x + \delta x, y) - u(x, y) + j v(x + \delta x, y)  - j v(x, y) }{\delta x} = \frac{\partial u}{\partial x} + j \frac{\partial v}{\partial x}
\end{align*}


\paragraph{Case (ii): $\delta z = j \delta y$.} We have

\begin{align*}
\frac{df(z)}{dz} &= \lim_{\delta y \rightarrow 0} \frac{u(x, y + \delta y) + j v(x, y + \delta y) - u(x, y) - j v(x, y) }{j \delta y} \\
&= \lim_{\delta y \rightarrow 0} \frac{v(x, y + \delta y)  - v(x,y) - j v(u, y + \delta y) + j u(x, y) }{\delta y} = \frac{\partial v}{\partial y} - j \frac{\partial u}{\partial y}
\end{align*}

If we want the derivative to be independent of the direction, the real and imaginary part have to be the same; so we have the \emph{Cauchy-Riemann equations}

\bee
\frac{\partial u}{\partial x} = \frac{\partial v}{\partial y}, \quad \frac{\partial u}{\partial y} = - \frac{\partial v}{\partial x}
\eee


%%% Local Variables:
%%% mode: latex
%%% TeX-master: "journal"
%%% End: