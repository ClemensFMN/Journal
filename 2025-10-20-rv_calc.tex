\DiaryEntry{Calculation with Random Variables}{2025-10-20}{Stochastic}

\subsection{Sum of Random Variables}

\paragraph{Using convolution integral.} We start with a simple example of the sum $Z$ of two normal random variables $X, Y$,

\bee
Z = X + Y, \quad X, Y \sim \Nc(0,1)
\eee

The pdfs of $X$ and $Y$ are

\begin{align*}
f_X(x) &= \frac{1}{\sqrt{2\pi}} \exp \left( - \frac{x^2}{2}\right) \\
f_Y(y) &= \frac{1}{\sqrt{2\pi}} \exp \left( - \frac{y^2}{2}\right)
\end{align*}

The pdf $f_Z(z)$ of the sum $Z$ is given by the convolution of the two pdfs,

\bee
f_Z(z) = \int_x f_X(x) f_Y(z-x) dx
\eee

which yields

\begin{align}
    f_Z(z) &= \frac{1}{2\pi} \int_x  \exp \left( - \frac{x^2}{2}\right) \exp \left( - \frac{(z-x)^2}{2}\right) dx \\
    &= \frac{1}{2\pi} \int_x  \exp \left( - \frac{x^2 + (z-x)^2}{2}\right) dx \\
    &= \frac{1}{2\pi} \int_x  \exp \left( - \frac{2x^2 - 2zx + z^2}{2}\right) dx \label{2025-10-20:eq0}
\end{align}

To make progress in solving this integral, we need to complete the square. Let's consider the relevant part only.

We have 

\bee
2z^2 - 2xz + z^2
\eee

and we want to bring that into a form 

\be\label{2025-10-20:eq1}
(\sqrt{2} x - Az)^2 + B z^2 = 2x^2 - 2 \sqrt{2}A xz + A^2 z^2 + B^2 z^2
\ee

Comparing the coefficients of $xz$ we find 

\bee
2 \sqrt{2}A = 2 \rightarrow A = 1 / \sqrt{2}
\eee

Inserting this into \eqref{2025-10-20:eq1} yields

\bee
2x^2 - 2 xz + z^2 / 2 + B^2 z^2
\eee

and for this to match we have $B^2 = 1/2$. So all in all 


\bee
2z^2 - 2xz + z^2 = \left( \sqrt{2} x - \frac{1}{\sqrt{2}}z \right)^2 + \frac{z^2}{2}
\eee

Inserting this back into \eqref{2025-10-20:eq0} yields

\begin{align*}
    f_Z(z) &= \frac{1}{2\pi} \int_x  \exp \left( - \frac{2x^2 - 2zx + z^2}{2}\right) dx \\
    &= \frac{1}{2\pi} \int_x  \exp \left( - \frac{\left( \sqrt{2} x - \frac{1}{\sqrt{2}}z \right)^2}{2}\right) \exp \left( - \frac{z^2}{2 \cdot 2}\right) dx
\end{align*}

The right factor does not depend on $x$, so we can move that out of the integral

\bee
f_Z(z) = \exp \left( - \frac{z^2}{2 \cdot 2}\right) \frac{1}{2\pi} \int_x  \exp \left( - \frac{\left( \sqrt{2} x - \frac{1}{\sqrt{2}}z \right)^2}{2}\right) dx
\eee

The integral corresponds to the area under a normal distribution (with mean $z$ and variance $1$) and therefore evaluates to a constant \emph{independent} of $z$. So we arrive at

\bee
f_Z(z) = C \exp \left( - \frac{z^2}{2 \cdot 2}\right)
\eee

where we have collected all factors into $C$. The factors are not important (their value must be such that the area under $f_Z(z)$ is $1$); what is important is that $z$ has normal distribution with $\Nc(0, 2)$.

\paragraph{Using moment-generating functions.} When we consider the sum of normal random variables with non-zero mean and different variances, above calculations become cumbersome. The simpler solution is to consider the moment generation function $\phi_x(t)$ which can be obtained from the pdf as follows,

\bee
\phi_x(t)
\eee

The MGF has the nice property that the MGF of the sum of random variables is equal the product of the MGFs.

A normal distribution has a MGF

\bee
X \sim \Nc(\mu_x, \sigma_x^2) \leftrightarrow \phi(t) = \exp \left( \mu_x t + \frac{1}{2} \sigma_x^2 t^2\right)
\eee

Then the MGF of the sum of two random variables $X \sim \Nc(\mu_x, \sigma_x^2)$ and $Y \sim \Nc(\mu_y, \sigma_y^2)$ is

\bee
\phi_Z(t) = \exp \left( \mu_x t + \frac{1}{2} \sigma_x^2 t^2\right) \exp \left( \mu_y t + \frac{1}{2} \sigma_y^2 t^2\right) = \exp \left( (\mu_x + \mu_y) t + \frac{1}{2} (\sigma_x^2 + \sigma_y^2) t^2\right)
\eee

and this corresponds to a normal variable distributed as 

\bee
Z \sim \Nc( \mu_x + \mu_y, \sigma_x^2 + \sigma_y^2)
\eee

\paragraph{Sum of $\chi^2$ RVs.} Using the MGF, we can also obtain the pdf of the sum $Z = \sum_N X_i$ of $N$ RVs $X_i$, each having a $\chi^2$ distribution with $k$ degrees of freedom. The MGF of $X_i$ is given by (Wikipedia)

\bee
\phi_X(t) = \frac{1}{(1-2t)^{k/2}}
\eee

and therefore their sum has MGF

\bee
\phi_Z(t) = \frac{1}{(1-2t)^{N k/2}}
\eee

which is a $\chi^2$ distribution having $N$ degrees of freedom.



\subsection{Quotient}

Assume we have two RVs, $X, Y$, having pdfs $f_X(x), f_Y(y)$, respectively. We can obtain the pdf of the quotient, $Z = X / Y$, $f_Z(z)$ from the following integral,

\todo{check if this is really correct}

\bee
f_Z(z) = \int_x f_X(x) f_Y(z / x) dx
\eee

or

\bee
f_Z(z) = \int_x |x| f_X(x) f_Y(x z) dx
\eee

\subsubsection{Uniform Distribution}

We can directly calculate the integral or make a sidestep using the cdf $F_Z(z)$ (as is done \href{https://math.stackexchange.com/questions/113295/pdf-of-a-quotient-of-uniform-random-variables}{here}).

We start with the case of $z = 1$ which corresponds to the case $Y = X$ as shown below.

\begin{figure}[H]
\centering
\includegraphics[scale=0.6]{images/2025-10-20_quot_uni_1.png}
\caption{PDF}
\end{figure}

If $0 \leq Z \leq 1$, the RVs are related as in the following Figure. In particular, the intersection of the line with the right boundary of the square domain will be $(1,z)$.

\begin{figure}[H]
\centering
\includegraphics[scale=0.6]{images/2025-10-20_quot_uni_2.png}
\caption{PDF}
\end{figure}

The area in dark blue will be the probability $P(Y/X \leq z) = P(Y \leq zX )=1/2z$.


If $z>1$, the line will intersect the top margin of the square at $x=1/z$. The area below the line will be cumbersome to calculate, and it is hence much easier to calculate the complementary area (shown as the yellow triangle in the Figure below).


\begin{figure}[H]
\centering
\includegraphics[scale=0.6]{images/2025-10-20_quot_uni_2.png}
\caption{PDF}
\end{figure}

The probability $P(Y / X \leq z) = 1 - \frac{1}{2z}$.

Combining the three cases, we obtain the following for the CDF,

\bee
F_Z(z) = \begin{cases} z/2 \quad 0 \leq z \leq 1 \\
1 - \frac{1}{2z} \quad z > 1 \\
0 \quad \text{otherwise}
\end{cases}
\eee

By differentiation wrt to $z$, we can obtain the pdf

\bee
f_Z(z) = \begin{cases} 1/2 \quad 0 \leq z \leq 1 \\
1 - \frac{1}{2z^2} \quad z > 1 \\
0 \quad \text{otherwise}
\end{cases}
\eee



\subsubsection{Normal Distribution}

This is a case where we can directly calculate the integral,

\begin{align*}
f_Z(z) &= \frac{1}{2\pi} \int_{- \infty}^\infty |x| e^{- \frac{x^2}{2}} e^{- \frac{x^2 z^2}{2}} dx \\
&= \frac{1}{2\pi} \int_{0}^\infty x e^{- \frac{1}{2} x^2 (1 + z^2)} dx
\end{align*}

where we have used the symmetry of the integrand. We can substitute $u = x^2$ and with $du/dx = 2x \rightarrow dx = du / (2x)$ we obtain

\begin{align*}
f_Z(z) &= \frac{1}{2\pi} \int_{0}^\infty x e^{- \frac{1}{2} u (1 + z^2)} \frac{du}{2x} \\
&= \frac{1}{4\pi} \int_{0}^\infty e^{- \frac{1}{2} u (1 + z^2)} du = \cdots = \frac{1}{\pi (z^2 + 1)}
\end{align*}

which is the \emph{Cauchy} distribution.


%%% Local Variables:
%%% mode: latex
%%% TeX-master: "journal"
%%% End:
