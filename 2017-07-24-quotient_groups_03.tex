\DiaryEntry{Quotient Groups, III}{2017-07-24}{Algebra}

Some further theorems / observations about cosets et al:

The (left / right) cosets partition a group. Furthermore, for all $u,v \in G, uN = vN$ iff $v^{-1}u \in N$ and in particular, $uN = vN$ iff $u$ and $v$ are representatives of the same coset. In other words, distinct cosets have an empty intersection.

The operation on the left cosets of $N$ in $G$ given by $uN vN = (uv)N$ is well defined iff $gng^{-1} \in N$ for all $g \in G$ and all $n \in N$. In this case, the left cosets of $N$ in $G$ form a group with the identity element being the coset $1N$ and the inverse element of $gN$ being the coset $g^{-1}N$. 

This gives rise to a couple of definitions:

The element $gng^{-1}$ is called the conjugate of $n \in N$ by $g$. The set $gNg^{-1} = \{ gng^{-1} | n \in N\}$ is called the conjugate of $N$ by $g$. The element $g$ normalizes $N$ if $gng^{-1} = N$. Finally, a subgroup $N$ of a group $G$ is called normal if every element of $G$ normalizes $N$; i.e. $gNG^{-1} = N$ for all $g \in G$. If $N$ is a normal subgroup og $G$, we write $N \trianglelefteq G$.

In the special case of an Abelian group $G$ (i.e. where $ab = ba, a,b \in G$), we have $gNg^{-1} = \{gng^{-1} | n \in N \} = \{n |n \in N\} = N$; i.e. every element of $G$ normalizes $N$. So every subgroup is normal (and Abelian, by the way).

To summarize, we have the following equivalent statements

\begin{itemize}

\item $N \trianglelefteq G$

\item $N_g(N) = G$

\item $gN = Ng$; i.e. left and right cosets are the same

\item The operation $uN vN = (uv)N$ makes the set of left cosets a group.

\item $gNg^{-1}$ is a subset of $N$ for all $g \in G$.

\end{itemize}

Finally, we have a simple condition for a group being normal: A subgroup is normal iff if it is the kernel of some homomorphism.

\subsection{Natural Projection}

Let $N \trianglelefteq G$ and $H = G/N$. Define a mapping $\pi: G \rightarrow G/N$ by $\pi(g) = gN$ for all $g \in G$. This mapping is a homomorphism; check this by

\bee
\pi(g_1 g_2) = (g_1 g_2)N =g_1 N g_2 N = \pi(g_1) \pi(g_2)
\eee

Next calculate the kernel of this mapping:

\bee
\text{ker}(\pi) = \{ g \in G | \pi(g) = 1N \} =  \{ g \in G | gN = 1 N \} = \{g \in G | g \in N \} = N
\eee

This shows that $N$ is the kernel of the homomorphism $\pi$ which is called the \emph{natural projection (homomorphism)} of $G$ onto $G/N$.
