\DiaryEntry{Quadratic Reciprocity Law}{2023-02-13}{Number Theory}

\subsection{Euler’s Criteria}

The quadratic reciprocity law deals with the solvability of quadratic congruences,

\be\label{2023-02-13:eq1}
ax^2 + bx + c \equiv 0 \mod p
\ee

where $p$ is an odd prime and $a \not\equiv 0 \mod p$ and therefore $\gcd(a,p) = 1$. The assumption of $p$ being an odd prime implies that $\gcd(4a, p) = 1$ \todo{I don't get that} and we can therefore write

\bee
4a( ax^2 + bx + c) \equiv 0 \mod p
\eee

We can expand the LHS and obtain

\bee
4a^2x^2 + 4abx + 4ac \equiv 0 \mod p
\eee

and we complete the square as

\bee
4a^2x^2 + 4abx + 4ac = (2ax + b)^2 - b^2 + 4ac
\eee

So our quadratic congruence becomes

\bee
(2ax + b)^2 \equiv b^2 - 4ac \mod p
\eee

By substituting $y = 2ax + b$ and $d = b^2 - 4ac$ we obtain the simple looking quadratic congruence

\be\label{2023-02-13:eq2}
y^2 \equiv d \mod p
\ee

If $x_0$ is a solution to \eqref{2023-02-13:eq1}, then $y_0 = 2ax_0 + b \mod p$ will be a solution to \eqref{2023-02-13:eq2}; conversely, if we have an integer $y_0$ solving \eqref{2023-02-13:eq2}, we can solve $2ax_0 \equiv y_0 - b \mod p$ for $x_0$ which will then be a solution to \eqref{2023-02-13:eq1}.

So in order to solve \eqref{2023-02-13:eq1}, we can first solve \eqref{2023-02-13:eq2}, and then solve the linear congruence $2ax_0 \equiv y_0 - b \mod p$. The quadratic congruence $y^2 \equiv d \mod p$ has either two or no solution; if it has two, they are related via $x_1 \equiv p - x_0 \mod p$ (see also previous entry).

\paragraph{Example.} Let's consider the quadratic congruence

\bee
5x^2 - 6x + 2 \equiv 0 \mod 13
\eee

We calculate $d = 6^2 - 4\cdot 5 \cdot 2 = -4 \equiv 9 \mod 13$ and therefore have to solve

\bee
y^2 \equiv 9 \mod 13
\eee

This has two solutions, $y = 3, 10 \mod 13$ (note that $p-3 \equiv 10 \mod 13$. We next need to solve the linear congruence $y_0 = 2ax_0 + b \mod p$ which becomes

\bee
10x \equiv y_0 + 6 \mod 13
\eee

For $y = 3$, this becomes $10 x \equiv 9 \mod 13$ and therefore $x = 10$ and for $y = 10$, this becomes $10x \equiv 16 \mod 13 \rightarrow 10x \equiv 3 \mod 13$ with solution $x = 12$. So our two solutions are $x = 10, 12 \mod 13$ which we can check by inserting them into the original quadratic congruence. \qed

Main topic of this entry is a test for solutions of $x^2 \equiv a \mod p, \gcd(a,p)=1$ or - to put it differently - we want to identify integers $a$ that are perfect squares modulo-$p$.

We have the following definition: Let $p$ be an odd prime and the integer $a$ fulfill $\gcd(a,p) = 1$. Then $x^2 \equiv a \mod p$ has a solution, then $a$ is said to be a \emph{quadratic residue of $p$}. Otherwise, $a$ is called a \emph{quadratic nonresidue of $p$}.

If $a \equiv b \mod p$, then $a$ is a quadratic residue of $p$ ifff $b$ is a quadratic residue of $p$.

Looking at the table of $x, x^2 \mod 13$ in the last entry we see that $1, 3, 4, 9, 10, 12$ are quadratic residues of $13$; the nonresidues are $2,5,6,7,8,11$. In this case, there are two pairs of consecutive quadratic residues, namely the pair $3,4$ and the pair $9,10$. For any odd prime $p$, there are 

\bee
\frac{1}{4}(p-4-(-1)^{(p-1)/2})
\eee

such consecutive pairs.

Later on, we will show that for $p$ being an odd prime, $(p+1)/2$ integers are quadratic residues. For other values of $p$, there are fewer quadratic residues. This is shown in the following plot which shows the number of quadratic residues for numbers between $1$ and $1000$.


\begin{figure}[H]
    \centering
    \includegraphics[scale=0.7]{images/2023-02-13-num_quad_residues.png}
\end{figure}


There is a simple criterion by Euler for deciding whether an integer $a$ is a quadratic residue of a given prime p.

\begin{theorem}
Let $p$ be an odd prime and $\gcd(a,p)=1$. Then $a$ is a quadratic residue of $p$ iff

\bee
a^{(p-1)/2} \equiv 1 \mod p
\eee
\end{theorem}

Suppose that $a$ is a quadratic residue of $p$ and $x_1$ is a solution, so that $x_1^2 \equiv a \mod p$. Because $\gcd(a,p) = 1$, we also have $\gcd(x_1,p) = 1$. We can therefore use Fermat's theorem (see entry \ref{2020-12-09:entry}) to obtain

\bee
a^{(p-1)/2} \equiv (x_1^2)^{(p-1)/2} \equiv x_1^{p-1} \equiv 1 \mod p
\eee

For the opposite direction, assume that the congruence $a^{(p-1)/2} \equiv 1 \mod p$ holds and let $r$ be a primitive root of $p$. Then $a \equiv r^k \mod p$ for some integer $k$ with $1 \leq k \leq p-1$. It follows that

\bee
r^{k(p-1)/2} \equiv a^{(p-1)/2} \equiv 1 \mod p
\eee

By theorem \ref{2021-03-02:th0}, the order of $r$ (namely, $p-1$) must divide the exponent $k(p-1)/2$ which implies that $k$ is an even integer which we can express as $k = 2j$. Therefore,

\bee
(r^j)^2 = r^{2j} = r^k \equiv  a \mod p
\eee

which makes the integer $r^j$ a solution of $x^2 \equiv a \mod p$. This proves that $a$ is a quadratic residue of the prime $p$. \qed

If $p$ is (as always) an odd prime and $\gcd(a,p) = 1$, Fermat's theorem states that

\bee
a^{p-1} \equiv 1 \mod p \rightarrow a^{p-1} - 1 \equiv 0 \mod p
\eee

We can factor this as

\bee
a^{p-1} - 1 \equiv \left(a^{(p-1)/2} - 1\right)\left(a^{(p-1)/2} + 1\right) \equiv 0 \mod p
\eee

and for this to hold, either

\bee
a^{(p-1)/2} \equiv 1 \mod p \qquad a^{(p-1)/2} \equiv -1 \mod p
\eee

but not both. If both congruences held simultaneously, we would have $1 \equiv -1 \mod p$, and this would imply $p | 2$ (because the difference between $1$ and $-1$ is two - check for e.g. $p=4$), but would conflict the hypothesis. A quadratic nonresidue does not satisfy $a^{(p-1)/2} \equiv 1 \mod p$ (see previous theorem), it must therefore satisfy $a^{(p-1)/2} \equiv -1 \mod p$.

We can collect this in another theorem.

\begin{theorem}\label{2023-02-13:th2}
Let $p$ be an odd prime and $\gcd(a,p)=1$. Then $a$ is a quadratic residue iff

\bee
a^{(p-1)/2} \equiv 1 \mod p
\eee

and a quadratic nonresidue iff

\bee
a^{(p-1)/2} \equiv -1 \mod p
\eee

\end{theorem}


There is another pprof due to Dirichlet but we skip this here.

Euler’s criterion is not a practical test for determining whether a given integer is or is not a quadratic residue; the calculations involved are too cumbersome unless the modulus is small.

\paragraph{Exercise 1.} We are given $6x^2 + 5x + 1 \equiv 0 \mod p$ and shall show that it has a solution for every prime (even though $6x^2 + 5x + 1$ has no solution over the integers). We have $d = 25 - 4 \cdot 6 \cdot 1 = 1$ and therefore

\bee
y^2 \equiv 1 \mod p \rightarrow y = 1, p-1
\eee

We now need to solve the linear congruence $12x \equiv y-5 \mod p$. For $y = 1$ this becomes $12x \equiv p-4 \mod p$. Using theorem \ref{2020-11-25:th1}, we have $d = \gcd(12, p) = 1$ and the linear congruence has a solution iff $d | b \rightarrow 1 | p-4$ which trivially holds. For $y = p-1$ we have $12x \equiv p-6 \mod p$ and this also has always a solution by the same reasoning. \qed.


\subsection{Legendre Symbols}

We start with the definition of the \emph{Legendre symbol $(a/p)$}. If $p$ is an odd prime and $\gcd(a,p)=1$, the Legendre symbol $(a/p)$ is defined as
 
\bee
(a/p) = \begin{cases} 1 \quad &\text{if $a$ is a quadratic residue of $p$} \\
-1 \quad &\text{if $a$ is a quadratic nonresidue of $p$} \end{cases}
\eee 
 
We call $a$ the numerator and $p$ the denominator of $(a/p)$.

\paragraph{Example.} Looking at our running example with $p=13$, we have $(1/13) = (3/13) =(4/13) = (9/13) = (10/13) = (12/13) = 1$ as these are all the quadratic residues, and $(2/13) = (5/13) = (6/13) = (7/13) = (8/13) = (11/13) = -1$ as these are all the quadratic nonresidues. \qed

The following theorem summarizes some facts about Legendre symbols.

\begin{theorem}
Let $p$ be an odd prime and $a, b$ integers that are relatively prime to $p$. Then the Legendre symbol has the following properties:

\begin{enumerate}
	\item If $a \equiv b \mod p$, then $(a/p) = (b/p)$.
	\item $(a^2/p) = 1$
	\item $(a/p) \equiv a^{(p-1)/2} \mod p$
	\item $(ab/p) = (a/p)(b/p)$
	\item $(1/p) = 1$ and $(-1/p) = (-1)^{(p-1)/2}$
	\item $(ab^2/p) = (a/p)(b^2/p) = (a/p)$
\end{enumerate}
\end{theorem}

Proofs: For 1), we observe that if $a \equiv b \mod p$, then the congruences $x^2 \equiv a \mod p$ and $x^2 \equiv b \mod p$ have exactely the same solutions if any. Thus, $x^2 \equiv a \mod p$ and $x^2 \equiv b \mod p$ are both solvable or neither one has a solution. This is reflected in the statement $(a/p) = (b/p)$.

For 2), we observe that the integer $a$ satisfies $x^2 \equiv a^2 \mod p$ and therefore $(a/p) = 1$.

Part 3) is just a restatement of \ref{2023-02-13:th2}.

For part 4) we use the definition of 3): $(ab/p) = (ab)^{(p-1)/2} = a^{(p-1)/2} b^{(p-1)/2} = (a/p)(b/p)$.

Part 5) can be obtained from 3) by setting $a=1$ and $a=-1$, respectively.

And finally 6) can be obtained by combining 2) and 4). \qed

The second part of 5) can be used to obtain a result for primes having special forms. If $p$ has the form $p = 4k+1$, then $(p-1)/2 = 2k$ which is even and therefore $(-1/p) = (-1)^{2k} = 1$. In a similar spirit, for $p = 4k+3$, we have $(p-1)/2 = 2k+1$ which is odd and therefore $(-1/p) = -1$. We can combine this as

\bee
(-1/p) = \begin{cases} 1 \quad & p \equiv 1 \mod 4 \\
-1 \quad & p \equiv 3 \mod 4 \end{cases} \qed
\eee

The next theorem to tackle is about the number of quadratic residues and nonresidues.

\begin{theorem}
If $p$p is an odd prime, then

\bee
\sum_{a=1}^{p-1} (a/p) = 0	
\eee	

Therefore, there are $(p-1)/2$ quadratic residues and $(p-1)/2$ quadratic nonresidues.
\end{theorem}

Let $r$ be a primitive root of $p$ and from previous entries we know that we can express $a$ in terms of this primitive root: $r^k \equiv a \mod p$. Let's calculate $(a/p)$ as

\bee
(a/p) = \equiv a^{(p-1)/2} \mod p \equiv (r^k)^{(p-1)/2} \mod p = (r^{(p-1)/2})^k \mod p \equiv (-1)^k \mod p
\eee

where, because $r$ is a primitive root of $p$, $r^{(p-1)/2} \equiv -1 \mod p$. Both $(a/p)$ and $(-1)^k$ can attain only $-1$ and $1$; therefore, equality holds in the above equation (instead of "only" $\equiv$). Adding up the Legendre symbols yields,

\bee
\sum_{a=1}^{p-1} (a/p) = \sum_{k=1}^{p-1} (-1)^k = 0 \qed
\eee

We have a corollary: The quadratic residues of an odd prime $p$ are congruent modulo $p$ to the even powers of a primitive root $r$ of $p$; the quadratic nonresidues are congruent to the odd powers of $r$.

\paragraph{Example.} The odd prime $p=13$ has a primitive root $r=2$. According to the corollary, even powers of $r$ yield the quadratic residues, and odd powers yields the quadratic nonresidues.

The quadratic residues are $1, 3, 4, 9, 10, 12$ and the even powers of $r=2$ are

\begin{align*}
& 2^0 \equiv 1 \mod 13, 2^2 \equiv 4 \mod 13, 2^4 \equiv 3 \mod 13 \\
& 2^6 \equiv 12 \mod 13 2^8 \equiv 9 \mod 13, 2^{10} \equiv 12 \mod 13
\end{align*}

The quadratic nonresidues are therefore $2, 5, 6, 7, 8, 11$ and the odd powers of $r=2$ are

\begin{align*}
& 2^1 \equiv 1 \mod 13, 2^3 \equiv 8 \mod 13, 2^5 \equiv 6 \mod 13 \\
& 2^7 \equiv 11 \mod 13 2^9 \equiv 5 \mod 13, 2^{11} \equiv 7 \mod 13
\end{align*}

\qed

We next have Gauss' Lemma.

\begin{theorem}
Let $p$ be an odd prime and $\gcd(a,p)=1$. If $n$ denotes the number of integers in the set $S = \{a, 2a, 3a, \cdots, \frac{p-1}{2}a \}$ whose remainders upon division by $p$ exceeds $p/2$, then

\bee
(a/p) = (-1)^n
\eee

\end{theorem}

\paragraph{Example.} For $p=13$ and $a=5$, $(p-1)/2 = 6$ and

\bee
S = \{5, 10, 15, 20, 25, 30\}
\eee

If we take the elements $\mod 13$, we have $\{5, 10, 2, 7, 12, 4\}$ and three of them are greater than $p/2 = 13/2$, therefore $n=3$. According to the theorem $(5/13) = (-1)^3 = -1$ and this corresponds to the fact that $5$ is a quadratic nonresidual. \qed

\subsection{Quadratic Reciprocity}


%%% Local Variables:
%%% mode: latex
%%% TeX-master: "journal"
%%% End:
