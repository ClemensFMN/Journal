\DiaryEntry{Permutations, 2}{2015-12-26}{Combinatorics}

As stated before, a permutation is a bijection (one-to-one mapping) from \(\mathcal{S}\) onto \(\mathcal{S}\).

A permutation can have fixed points; i.e.~elements \(x^\star\) which
stay constant under the mapping: \(\sigma(x^\star) = x^\star\).

The opposite to fixed points are derangements, which are permutations of the elements, such that no element appears in its original position.

If we take \(\mathcal{S} = \{1,2,3,4\}\), then there are \(4!\)
permutations. Out of these, the derangements are as follows


\begin{align*}
    & (2,1,4,3), (2,3,4,1), (2,4,1,3) \\
    & (3,1,4,2), (3,4,1,2), (3,4,2,1) \\
    & (4,1,2,3), (4,3,1,2), (4,3,2,1) \\ 
\end{align*}


We will denote with \(!n\) the number of derangements of \(n\) elements;
from above example it follows that \(!3 = 9\).

\subsubsection{Recurrence Relation}

Let's consider the permutation of the first position \(\sigma(1) = i\)
with \(i \neq 1\). There are \(n-1\) way to select such an \(i\). Next
we need to select a value for \(\sigma(i)\). There are two options:

\begin{itemize}
\item
  We choose any \(i \neq 1\) (and \(\sigma(i) \neq i\)). In this case we
  have a derangement with \(n-1\) elements.
\item
  We choose \(\sigma(i)=1\), in this case a derangement with \(n-2\)
  elements remains.
\end{itemize}

Combining these two options (and considering that we can select the
original \(i\) in \(n-1\) way), we obtain for the number of derangements

\begin{equation}
\label{recur}
!n = (n-1) \left[ !(n-1) +  !(n-2)\right]
\end{equation}

The initial conditions are \(!0=1\) and \(!1=0\). The next values are
then (starting with \(n = 2\))

\[
1, 2, 9, 44, 265, 1854, 14833, 133496, 1334961, 14684570, 176214841, 2290792932, \cdots
\]

which is \href{https://oeis.org/A000166}{OEIS sequence}. We have
\(!2 = 1\); i.e.~there is only one derangement (which is \((2,1)\)),
\(!3 = 2\), the two derangements are \((2,3,1)\) and \((3,1,2)\).

\subsubsection{Relation with Factorial}

As the factorial counts the number of \emph{all} permutations and not
all permutations are derangements, we have \(!n < n!\).

First observe that we have the same recurrence relation as
\(\eqref{recur}\) for the factorials:

\begin{equation}
\label{recur2}
n!= (n-1) \left[ (n-1)! +  (n-2)!\right]
\end{equation}

because

\[
n(n-1)! - (n-1)! + (n-2)! (n-1) = n! - (n-1)! + (n-1)! = n!
\]

Dividing \(\eqref{recur}\) by \(\eqref{recur2}\), we have

\[
\frac{!n}{n!} = \frac{ (n-1) \left[ !(n-1) +  !(n-2)\right] }{ (n-1) \left[ (n-1)! +  (n-2)!\right] }
\]

\subsection{Add-on}

Some more stuff can be found
\href{http://math.ucr.edu/home/baez/qg-winter2004/derangement.pdf}{here}.
