\DiaryEntry{Legendre Polynomials - Polynomial Ansatz}{2019-01-11}{Maths}

The Legendre polynomials $P_n(x)$ solve the Legendre ODE

\bee
\frac{d}{dx} \left[ (1-x^2) \frac{dP_n(x)}{dx}  \right] + n(n+1)P_n(x) = 0
\eee

In order to solve the ODE, we can make a Polynomial Ansatz

\bee
P_n(x) = \sum_{i=0}^n a_i x^i
\eee

insert it into the ODE and solve for the coefficients $a_i$. The above Ansatz uses the assumption that $P_n(x)$ is a polynomial of degree $n$. This assumption is correct; however, I am not sure how to justify it (with looking at the ODE only). If we drop the assumption, the Ansatz becomes $P_n(x) = \sum_{i=0}^\infty a_n x^n$. We can still insert into the ODE and obtain coefficients (see below).

I am really too lazy to work this out manually and used Maxima instead. As an example, for $n=3$ we have

\begin{verbatim}

(%i2)	n:2;
	P:a0+a1*x+a2*x^2;
(n)	2
(P)	a2*x^2+a1*x+a0

(%i3)	LHS:diff((1-x^2)*diff(P,x,1), x,1) + n*(n+1)*P;
(LHS)	6*(a2*x^2+a1*x+a0)+2*a2*(1-x^2)-2*x*(2*a2*x+a1)

(%i4)	ratsimp(LHS);
(%o4)	4*a1*x+2*a2+6*a0

(%i11)	eq0:ratcoeff(LHS,x,0);
	eq1:ratcoeff(LHS,x,1);
	eq2:ratcoeff(LHS,x,2);

(eq0)	2*a2+6*a0
(eq1)	4*a1
(eq2)	0

(%i12)	solve([eq0=0, eq1=0, eq2=0], [a0,a1,a2]);
solve: dependent equations eliminated: (3 7 6 5 4)

(%o12)	[[a0=-%r5/3,a1=0,a2=%r5]]

\end{verbatim}

From this we obtain $P_2(x) = -\frac{C}{3} + Cx^2$ with the unknown value $C$. This can be obtained from the boundary condition $P_n(1) = 1$ as $C = 3/2$ and we obtain

\bee
P_2(x) = \frac{1}{2}(3x^2-1) \qed
\eee


%%% Local Variables:
%%% mode: latex
%%% TeX-master: "journal"
%%% End:
