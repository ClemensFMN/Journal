\DiaryEntry{Transformation of RVs}{2025-10-09}{Stochastic}

Some additions to the continuous case covered in \ref{2019-01-31:entry}.

To repeat, assume we have a continuous random variable $X$ which is mapped to another random variable $Y = g(X)$ by a function $g$. Then $Y$ has the pdf $f_Y(y)$ which depends on the distribution of $X$ and the function $g$,

\be
\label{2025-10-09:eq1}
f_Y(y) = \sum_{y = g(x)} \left| \frac{dg^{-1}(y)}{dy} \right| f_X(g^{-1}(y))
\ee

\subsection{Transformation by $g(x) = x^2$}

\paragraph{} With $y = g(x) = x^2$, we have $g^{-1}(y) = \sqrt{y}$ and therefore 

\bee
\frac{dg^{-1}(y)}{dy} = \frac{1}{2 \sqrt{y}}
\eee

According to \eqref{2025-10-09:eq1}, the pdf of $Y$ then becomes

\bee
f_Y(y) = \frac{1}{2 \sqrt{y}} \left[ f_X(\sqrt{y}) + f_X(-\sqrt{y})\right]
\eee

We can further simplify this when the distribution of $X$ is symmetric; ie $f_X(x) = f_X(-x)$,

\be
\label{2025-10-09:eq2}
f_Y(y) = \frac{1}{\sqrt{y}} f_X(\sqrt{y})
\ee

\paragraph{Uniform Distribution (1).} When we transform $X$ with a uniform distribution in $[-1,1]$ with $g(x) = x^2$, then we have

\bee
f_Y(y) = \begin{cases} \frac{1}{2 \sqrt{y}} \quad 0 \leq y \leq 1 \\ 0 \qquad \text{otherwise} \end{cases}
\eee

\begin{figure}[H]
\centering
\includegraphics[scale=0.7]{images/2025-10-09-plt_1.png}
\caption{PDF}
\end{figure}


The "interesting" observation is that the pdf goes to infinity for $y \rightarrow 0$. But this is actually not a problem, as we still get a finite probability for $Y$ being small,

\bee
P(Y \leq \epsilon) = \int_0^\epsilon \frac{1}{2 \sqrt{y}} dy = \frac{1}{2} \int_0^\epsilon y^{-1/2} dy = \left. y^{1/2} \right|_0^\epsilon = \sqrt{\epsilon}
\eee

This is also verified in the Python simulation in \verb+ /home/cnovak/sry/python/Notebooks/+ 
\verb+ maths/RV_Transform_square.ipynb + and shown below.

\begin{verbatim}
    # generate a uniform distribution (-1,1)
    n = stats.uniform(-1,2)
    # and obtain RVs
    x = n.rvs(N)
    y = x**2
    e = 0.02
    print("analytical result = ", np.sqrt(e), "measured = ", len(y[y < e]) / N)
    analytical result =  0.1414213562373095 measured =  0.141875
\end{verbatim}


\paragraph{Uniform Distribution (2).} If we take a uniform distribution between $[0,1]$, then we cannot use the simplified formula \eqref{2025-10-09:eq2}for symmetric distributions, but we need \eqref{2025-10-09:eq1} instead.

In the interval $[0,1]$ we have

\bee
f_Y(y) = \frac{1}{2 \sqrt{y}} \left[ f_X(\sqrt{y}) + f_X(-\sqrt{y})\right] = \frac{1}{2 \sqrt{y}}
\eee

\paragraph{Normal Distribution.} The distribution of $X$ is given by

\bee
f_X(x) = \frac{1}{\sqrt{2 \pi}} e^{-x^2 / 2}
\eee

With \eqref{2025-10-09:eq2} we get

\bee
f_Y(y) = \frac{1}{\sqrt{2 \pi y}} e^{-y / 2}
\eee

which is a $\chi^2$ distribution with one degree of freedom.

\begin{figure}[H]
\centering
\includegraphics[scale=0.7]{images/2025-10-09-plt_1.png}
\caption{PDF}
\end{figure}


\subsection{Transformation by $g(x) = 1/x$}

Starting from \eqref{2025-10-09:eq1}, we note that $g^{-1}(y) = y^{-1}$ and 

\bee
\frac{dg^{-1}(y)}{dy} = - \frac{1}{y^2}
\eee

Since $g(x)$ is a one-to-one function, we do not need the summation in \eqref{2025-10-09:eq1} and arrive at

\bee
f_Y(y) =  \left| \frac{dg^{-1}(y)}{dy} \right| f_X(g^{-1}(y)) = \frac{1}{y^2} f_X(g^{-1}(y))
\eee

\paragraph{Uniform Distribution.} Let's start again with a uniform distribution in $[0,1]$. The expression $f_X(g^{-1}(y))$ can be simplified to

\bee
f_X(g^{-1}(y)) = f_x\left(\frac{1}{y}\right) = \begin{cases} 1 \quad y \geq 1 \\ 0 \quad y < 1 \end{cases}
\eee

We finally get the desired pdf

\bee
f_Y(y) =  \begin{cases} \frac{1}{y^2} \quad y \geq 1 \\ 0 \quad y < 1 \end{cases}
\eee

We can see that the pdf is "thinned out" to the right as the $1/x$ moves small values far to the right.

\begin{figure}[H]
\centering
\includegraphics[scale=0.7]{images/2025-10-09-plt_3.png}
\caption{PDF}
\end{figure}


\paragraph{Normal Distribution.} We can repeat above with a normal distribution

\bee
f_X(x) = \frac{1}{\sqrt{2 \pi}} e^{-x^2 / 2}
\eee

and obtain

\bee
f_Y(y) = \frac{1}{y^2\sqrt{2 \pi}} e^{-1 / (2 y^2)}
\eee



%%% Local Variables:
%%% mode: latex
%%% TeX-master: "journal"
%%% End:
