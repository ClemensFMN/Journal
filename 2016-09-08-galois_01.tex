\DiaryEntry{Galois Theory, High-level}{2016-09-08}{Algebra}

A polynomial of degree \(n\) has \(n\) roots over \(\mathbb{C}\).

Polynomials of degree 2, 3, and 4 can be solved using expressions
involving addition/subtraction, multiplication/division, and roots (of
arbitrary) order. Such expressions are called radicals. However, finding
general expressions for polynomials of order 5 is not possible.

This does not mean that these roots do not exist (e.g.~because the field
does not contain the roots) but only that they are not expressible via
radicals. If the roots of such a polynomial are calculated by numerical
means, they do not look any different; they are just real or complex
numbers.

\subsection{Basic Idea}\label{basic-idea}

The roots of a specific polynomial can be connected via algebraic
equations. Galois Theory considers permutations of the roots; The Galois
group is the permutation group containing all permutations that leave
any algebraic equation (with rational coefficients) of the roots intact.

Iff the Galois group has the property of ``solvability'', then the
corresponding polynomial can be solved by means of radicals (otherwise
it is not possible).

As an example consider the equation

\[
x^2-4x+1
\]

with roots \(A = 2 + \sqrt{3}, B = 2 - \sqrt{3}\). Examples of algebraic
equations (with rational coefficients) the roots fulfill are


\begin{align}
A+B= & 4 \\
A \times B= &1
\end{align}


Since there are only two roots, there is only one permutation possible;
i.e. \(A \rightarrow B, B \rightarrow A\). Permuting the roots, the two
equations above are still fulfilled. It is not obvious, but the
permutation will leave \emph{any} algebraic equation with rational
coefficients intact.

The constraint of algebraic equation with rational coefficients is
important; as e.g. \(A - B - 2\sqrt{3}=0\) but which does not hold when
A and B are exchanged. However, this is not an equation with rational
coefficients, so this is ok.
