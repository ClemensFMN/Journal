\DiaryEntry{Permutations - Counting Cycle Classes}{2022-04-25}{Combinatorics}

This is based on \cite{Dominus2007}. We consider permutations and their cycles. A permutation is a mapping from a set $\Sc$ to itself. A cycle of a permutation is a subset of the set for which the elements fall into a single orbit. We can assign permutations into cycle classes by saying that two permutations are in the same cycle class if the lengths of the cycles are all the same.

Consider $n = 3$ elements. We have $3! = 6$ different permutations and we classify them according to their cycles as follows,

\vspace{2mm}

\begin{tabular}{|c|c|c|}
  Cycle Structure & Permutations & \# Permutations \\ \hline
  three $1$-cycles $[1,1,1] $ & $(1)(2)(3)$ & $1$ \\
  one $1$-cycle and one $2$-cycle $[1,2]$ & $(1)(2,3) \;\; (2)(1,3) \;\; (3)(1,2)$ & $3$ \\
  one $3$-cycle $[3]$ & $(1,2,3) \;\; (2,1,3)$ & $2$
\end{tabular}

\vspace{2mm}

Note that the total counts of all items in the cycle classes for permutations of $n$ items adds up to $n!$.

We want to count the number of permutations in a cycle class: A cycle class $\Cc=(p_1, p_2, \ldots)$ contains a series of $p_i$ permutations of length $i$. The number of permutations in cycle class $\Cc$ is then given by

\bee
|\Cc| = \frac{n!}{\prod_i i^{p_i} p_i!}
\eee

In our example above, the number permutations having three $1$-cycles ($p_1 = 3, p_2=p_3=0$) is $|\Cc| = 3! / ((1^3 3!) \cdot 1 \cdot 1) = 1$, the number of permutations having one $1$-cycle and one $2$-cycle ($p_1 = p_2 = 1, p_3=0$) is $|\Cc| = 3! / ((1^1 1!) \cdot (2^1 1!) \cdot 1) = 3$, and finally the number of permutations having one $3$-cycle ($p_1=p_2=0, p_3 = 1$) is $|\Cc| = 3! / ((3^1 1!) \cdot 1 \cdot 1) = 2$. \qed

We next consider $n=5$ items. As an example, we consider $[3,1,1]$ which is the permutation class having one $3$-cycle and two $1$-cycles. This implies $p_1=2, p_3=1$ and therefore $|\Cc| = 5! / ((1^2 2!) \cdot (3^1 1!)) = 10$. 

The following table shows the cycle structure and cardinality.

\vspace{2mm}

\begin{tabular}{cc}
    Cycle Structure & Number \\ \hline
    $[5]$ &  24 \\
    $[4, 1]$ & 30 \\
    $[3, 2]$ & 20 \\
    $[3, 1, 1]$ & 20 \\
    $[2, 2, 1]$ & 15 \\
    $[2, 1, 1, 1]$ & 10 \\
    $[1, 1, 1, 1, 1]$ & 1
\end{tabular}

\vspace{2mm}

The first row contains the permutation class having one $n$-cycle. We have $p_n=1$ and therefore $|\Cc| = n!/(n^1 1!) = (n-1)!$.The last line denotes the permutation class having only $1$-cycles and there is only one such class (We have $p_1=n$ and therefore $|\Cc| = n!/(1^n n!) = 1$). We also see that most permutations have one fixed point and one permutation of cycle length $n-1=4$. Therefore $p_1=1, p_{4}=1$ and $|\Cc| = 5!/(1^1 1! \cdot 4^1 1! ) = 30$.

The case of $n=7$ has the same behaviour with a maximum of $840$ permutations having one fixed point and one permutation of cycle length $6$ (We have $p_1=1, p_6=1$ and therefore $|\Cc| = 7!/(1^1 1! \cdot 6^1 1!) = 840$).

Note: This may not be so easy as we need to find the max of $\Cc$ from the epxression above. There is another article \href{https://math.ucr.edu/home/baez/permutations/permutations_1.html}{here}, but this asks for the number of permutations having $k$ fixed points.

We choose $k$ points and create a derangements of the remaining $n-k$ points, so we have

\bee
{n \choose k} !(n-k)
\eee

However, this does not care about the structure of the derangement. In case of $n=6$ and $k=2$ fixed points, we could have $(1)(2)(3,4,5,6)$ but also $(1)(2)(3,4)(5,6)$.


%%% Local Variables:
%%% mode: latex
%%% TeX-master: "journal"
%%% End:
