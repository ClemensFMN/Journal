\DiaryEntry{Permutations-  Counting Cycle Classes}{2022-04-25}{Combinatorics}

We consider permutations and their cycles. We can assign permutations into cycle classes by saying that two permutations are in the same cycle class if the lengths of the cycles are all the same. 

Consider $n = 3$ elements. We have $3! = 6$ different permutations and we classify them according to their cycles as follows,

\vspace{2mm}

\begin{tabular}{|c|c|c|}
  Cycle Structure & Permutations & Number \\ \hline
  3 $1$-cycles & $(1)(2)(3)$ & $1$ \\
  1 $1$-cycle, 1 $2$-cycle & $(1)(2,3) \; (2)(1,3) \; (3)(1,2)$ & $3$ \\
  1 $3$-cycle & $(1,2,3) \; (2,1,3)$ & $2$
\end{tabular}

\vspace{2mm}

Note that the total counts of all the cycle classes for permutations of $n$ items adds up to $n!$.

We want to count the number of permutations in a cycle class: We have $\Sc = 1,2,3,\ldots n$ and a cycle class $\Cc$ which denotes how many cycles of length $i$ we have. So $\Cc = (p_1, p_2, \ldots )$, where $p_1$ denotes the number of $1$-cycles,  $p_2$ denotes the number of $2$-cycles etc. Then the number of permutations in cycle class C is given by

\bee
|\Cc| = \frac{n!}{\prod_i i^{p_i} p_i!}
\eee

In our example above, the number permutations having three $1$-cycles ($p_1 = 3$) is $|\Cc| = 3! / ((1^3 3!) \cdot 1 \cdot 1) = 1$, the number of permutations having one $1$-cycle and one $2$-cycle ($p_1 = p_2 = 1$) is $|\Cc| = 3! / ((1^1 1!) \cdot (2^1 1!) \cdot 1) = 3$, and finally the number of permutations having one $3$-cycle ($p_3 = 1$) is $|\Cc| = 3! / ((3^1 1!) \cdot 1 \cdot 1) = 2$.

In the next table we show the number of permutations per cycle class for $n=5$ items. Notation is slightly different, we list the cycles directly and so $[3,1,1]$ denotes one $3$-cylce and two $1$-cycles.

\vspace{2mm}

\begin{tabular}{cc}
    Cycle Structure & Number \\ \hline
    $[5]$ &  24 \\
    $[4, 1]$ & 30 \\
    $[3, 2]$ & 20 \\
    $[3, 1, 1]$ & 20 \\
    $[2, 2, 1]$ & 15 \\
    $[2, 1, 1, 1]$ & 10 \\
    $[1, 1, 1, 1, 1]$ & 1
\end{tabular}

\vspace{2mm}

There is only one permutation having only $1$-cycles. The number of permutations having one $n$-cycle ($p_n=1$) is $|\Cs| = n!/(n^1 1!) = (n-1)!$. Finally, most permutations have one fixed point and one permutation of cycle length $4$ ($n-1$). The case of $n=7$ has the same behaviour with a maximum of $840$ permutations having one fixed point and one permutation of cycle length $6$. This is shown \href{https://math.ucr.edu/home/baez/permutations/permutations_1.html}{here}.

%%% Local Variables:
%%% mode: latex
%%% TeX-master: "journal"
%%% End:
