\DiaryEntry{Permutations, 1}{2015-12-20}{Combinatorics}

There is a rather good and comprehensive Wikipedia
\href{https://en.wikipedia.org/wiki/Permutation}{article}.

Short summary:

\begin{itemize}
\item
  There is a set of items \(\mathcal{S}\).
\item
  We consider ``arrangements'' of the element's set. E.g.
  \(\mathcal{S} = \\{1,2,3\\}\). Arrangement in this context means, that
  the elements are reshuffled.
\item
  No duplicates are allowed i.e.~no \((1,1,2)\) and order is important;
  i.e. \((1,2,3) \neq (1,3,2)\).
\end{itemize}

If the set has size \(k\), i.e. \(\|\mathcal{S}\| = k\), there are
\(k!\) permutations possible: The first position can be filled with
\(k\) different values, the second position can be filled with \(k-1\)
and so on, till only one element is left for the \(k\)-th position.

\subsubsection{Bijection Interpretation}

Seen another way, a permutation is a bijection from \(\mathcal{S}\) onto
\(\mathcal{S}\). A bijection is a one-to-one function. This view allows
to interpret a permutation as a function (mapping from \(\mathcal{S}\)
to \(\mathcal{S}\)) and can be composed.

The two-line notation lists the elements of \(\mathcal{S}\) in the first
row and the image (the ``permutation's output'') on the second row.

For example, we have

\[
\sigma=\begin{pmatrix}
1 & 2 & 3 & 4 & 5 \\
2 & 5 & 4 & 3 & 1\end{pmatrix}
\]

which means that 1 is mapped onto 2 (written as \(\sigma(1) = 2\)), 2 is
mapped onto 5 (\(\sigma(2) = 5\)) and so on.

The permutation does not change when the columns are rearranged;
e.g.~above permuation can also be rewritten as

\[
\sigma=
\begin{pmatrix}
2 & 5 & 4 & 3 & 1 \\
5 & 1 & 3 & 4 & 2
\end{pmatrix}
\]

\subsubsection{Permutation Cycles}

If we repeatedly apply the permutation to an element \(x\) we obtain a
sequence \(x, \sigma(x), \sigma(\sigma(x)), \cdots\). After some length,
the sequence will return to the initial element \(x\) and the
corresponding sequence is called an orbit / cycle of the permutation.
Choosing any other value from \(\mathcal{S}\) which is not element of
the orbit, one obtains another orbit and so on until all element of
\(\mathcal{S}\) are covered. A permutation can also be represented by
listing it cycles.

The cycles of the permutation above are \((1,2,5), (3,4)\). Note that
the representation is not unique; e.g. \((2,5,1), (4,3)\) represents the
same permutation.

A cycle of length k is called a \(k\)-cycle. An element in a \(1\)-cycle
is a fixed point of the permutation \(\sigma\): The element does not
change position under the permutation.

A permutation without a \(1\)-cycle is called a derangement; in this
case, all element are placed into a new position and no element stays on
its position.

\subsubsection{Permutation Composition}

Composition of two permutations \(\sigma_1, \sigma_2\) means that first
the permutation \(\sigma_1\) is applied to a sequence \(x\) and then the
second permutation \(\sigma_2\) is applied to the result; i.e.
\(\sigma_2(\sigma_1(x))\).

The composition can be obtained by rearranging the columns of the second
(leftmost) permutation so that its first row is identical with the
second row of the first (rightmost) permutation. The product can then be
written as the first row of the first permutation over the second row of
the modified second permutation. For the example permutation above we
obtain

\[
\sigma(\sigma) = 
\begin{pmatrix}
2 & 5 & 4 & 3 & 1 \\
5 & 1 & 3 & 4 & 2
\end{pmatrix}
\begin{pmatrix}
1 & 2 & 3 & 4 & 5 \\
2 & 5 & 4 & 3 & 1
\end{pmatrix}
=
\begin{pmatrix}
1 & 2 & 3 & 4 & 5 \\
5 & 1 & 3 & 4 & 2
\end{pmatrix}
\]

From the result the effect of the \(2\)-cycle \((3,4)\) can be seen
clearly.

Multyplying the permutation in a third time, we obtain

\begin{align*}
\sigma(\sigma(\sigma)) &= 
\begin{pmatrix}
1 & 2 & 3 & 4 & 5 \\
2 & 5 & 4 & 3 & 1
\end{pmatrix}
\begin{pmatrix}
1 & 2 & 3 & 4 & 5 \\
5 & 1 & 3 & 4 & 2
\end{pmatrix}
=
\begin{pmatrix}
5 & 1 & 3 & 4 & 2 \\
1 & 2 & 4 & 3 & 5
\end{pmatrix}
\begin{pmatrix}
1 & 2 & 3 & 4 & 5 \\
5 & 1 & 3 & 4 & 2
\end{pmatrix}
 \\ & =
\begin{pmatrix}
1 & 2 & 3 & 4 & 5 \\
1 & 2 & 4 & 3 & 5
\end{pmatrix}
\end{align*}

Now the effect of the \(3\)-cycle becomes visible; the \(2\)-cycle
exchanges elements \(3\) and \(4\).

In general the composition of two permutations is not commutative, but
it is associative.

The identity permutation, which maps every element of the set to itself,
is the neutral element for this product and has the following form:

\[
\begin{pmatrix}
1 & 2 & 3 & \ldots & n \\
1 & 2 & 4 & \ldots & n
\end{pmatrix}
\]

Since a permutation is a bijection, it has an inverse \(\sigma^{-1}\)
which can be obtain by exchanging the first and second line (followed by an optional rearrangement of columns):

\[
\sigma^{-1} = 
\begin{pmatrix}
1 & 2 & 3 & 4 & 5 \\
2 & 5 & 4 & 3 & 1
\end{pmatrix}^{-1} = 
\begin{pmatrix}
2 & 5 & 4 & 3 & 1 \\
1 & 2 & 3 & 4 & 5
\end{pmatrix}
=
\begin{pmatrix}
1 & 2 & 3 & 4 & 5\\
5 & 1 & 4 & 3 & 2
\end{pmatrix}
\]

In order to check the result, we compose \(\sigma\) with \(\sigma^{-1}\)
and obtain

\[
\begin{pmatrix}
1 & 2 & 3 & 4 & 5 \\
2 & 5 & 4 & 3 & 1
\end{pmatrix}
\begin{pmatrix}
1 & 2 & 3 & 4 & 5\\
5 & 1 & 4 & 3 & 2
\end{pmatrix}
=
\begin{pmatrix}
5 & 1 & 4 & 3 & 2 \\
1 & 2 & 3 & 4 & 5
\end{pmatrix}
\begin{pmatrix}
1 & 2 & 3 & 4 & 5\\
5 & 1 & 4 & 3 & 2
\end{pmatrix}
=
\begin{pmatrix}
1 & 2 & 3 & 4 & 5\\
1 & 2 & 3 & 4 & 5
\end{pmatrix}
\]

which is the identity permutation again.

\subsubsection{Matrix Representation}

We can also represent a permutation in matrix notation; the
\(n \times n\) matrix associated with the permutation \(\sigma\) has one
\(1\) in every row and every column and is zero otherwise. The entry
\(M_{i,j}\) of the matrix is one if \(\sigma(j) = i\). From this it can
be clearly seen that permutation composition is not commutative.
Furthermore, the identity permutation becomes the \(n \times n\)
identity matrix.

For our example, we have the following matrix

\[
M = 
\begin{pmatrix}
0 & 1 & 0 & 0 & 0\\
0 & 0 & 0 & 0 & 1\\
0 & 0 & 0 & 1 & 0\\
0 & 0 & 1 & 0 & 0\\
1 & 0 & 0 & 0 & 0
\end{pmatrix}
\]

In the matrix \(M_{3,4} = 1\) because \(\sigma(4 = 3\).
