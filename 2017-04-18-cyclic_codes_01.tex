\DiaryEntry{Cyclic Codes, I}{2017-04-18}{Coding}

\subsection{Repetition - Rings}

\begin{definition}
  A ring $(R,+,\cdot)$ is a set $R$ with two binary opeations, $+$ (addition) and $\cdot$ (multiplication), defined on $R$ such that

  \begin{itemize}
    \item $(R,+)$ forms an abelian group with additive identity typically denoted as $0$.
    \item The multiplication operation $\cdot$ is associative: $(a \cdot b) \cdot c = a \cdot (b \cdot c)$ for $a,b,c \in R$.
    \item Left and right distributive laws hold: $a(b+c) = ab + ac, (a+b)c = ac + bc$
  \end{itemize}
%
  A ring is commutative if $a \cdot b = b \cdot a$, for all $a,b \in R$. A ring is said to be a ring with identity if $\cdot$ has an identitiy element which is typically denoted as $1$.

\end{definition}

As a simple example, consider $(\mZ_6, +, \cdot)$ with addition and multiplication tables as follows (addition and multiplication are taken modulo-6):

\begin{figure}[H]
  \includegraphics[scale=0.75]{images/cyclic_codes_01.png}
\end{figure}

Note that the set does not form a group under multiplication because (i) of the element $0$ and (ii) because elements $2,3,4$ do not have an inverse. The set forms a group under addition as required by the definition. It is therefore a valid ring.


\subsection{Repetition - Rings of Polynomials}

If $R$ is a ring, then the set of all polynomials with coefficients in $R$ form a ring under the usual operations for polynomial addition and multiplication. This ring is denoted as $R[x]$, where $x$ is the polynomial variable:

\bee
f(x) = \sum_{i=0}^n a_i x^i, a_i \in R
\eee

Addition (adding coefficients with the same degree according to the ring operation) of polynomials yields another polynomial in the ring, the same holds for multiplication. The inverse of a polynomial will in general not be another polynomial (e.g. $(1 + 4x)^{-1}$), but note that a ring need not have multiplicative inverses (it's not a field after all).

\paragraph{Example.} Taking $R$ to be $(\mZ_6, +, \cdot)$, we obtain a polynomial ring as $R[x]$ with some example elements $f_1(x) = 3 + x, f_2(x) = 5 + 3x^2$. For example, we have $f_1 + f_2 = 2 + x + 3x^2$ and $f_1 f_2 = 3 + 5x + 3x^2 + 3x^3$.

\subsection{Repetition - Quotient Rings}

In Group Theory, a set of cosets was created by ``translating'' a subgroup; i.e. if $H$ is a subgroup of a group $G$, then we formed cosets by $g + H, g \in G$.

In a similar spirit, we can collect polynomials over a ring into equivalence classes by their remainder after division by a fixed polynomial.

As an example, consider the ring of polynomials $GF(2)[x]$ and a fixed polynomial $x^3+1$. Now collect all polynomials with remainder $0$ after division modulo $x^3+1$:

\bee
S_0 = \{0, x^3+1, x^4+x, x^5+x^2, \ldots\} = \langle x^3 + 1 \rangle
\eee

where $S_0 = \langle x^3 + 1 \rangle$ is the set of polynomials generated by $x^3+1$. Note that $S_0$ is actually a subring: Denote $p_1(x), p_2(x)$ two polynomials with remainder $0$ after division by $x^3+1$. The sum $p_1(x) + p_2(x)$ will also have a remainder $0$ after division by $x^3+1$

\bee
\frac{p_1(x) + p_2(x)}{x^3+1} = \frac{p_1(x)}{x^3+1} + \frac{p_2(x)}{x^3+1} \equiv 0 \bmod (x^3+1)
\eee

and in a similar spirit, the multiplication as well

\bee
\frac{p_1(x)p_2(x)}{x^3+1} = \frac{p_1(x)}{x^3+1} \frac{p_2(x)}{x^3+1} \equiv 0 \bmod (x^3+1)
\eee

The set $S_1$ contains all polynomials with remainder $1$ after division modulo $x^3+1$:

\bee
S_1 = \{1, x^3, x^4+x+1, x^5+x^2+1, \ldots\} = 1 + \langle x^3 + 1 \rangle
\eee

Noe that his is not a (sub)ring as it does not contain the identity element. The other equivalence classes are obtained in the same manner.

\begin{figure}[H]
  \includegraphics[scale=0.65]{images/cyclic_codes_02.png}
\end{figure}

By dividing through $x^3+1$, only 7 different remainders are possible: $0, 1, x, 1+x, 1+x^2, x+x^2, 1+x+x^2$. Therefore, all polynomials of $GF(2)[x]$ fall into of these equivalence classes.

In a similar spirit to defining an induced operation of cosets, we can define induced operations $+$ and $\cdot$ for the equivalence classes of polynomials modulo $x^3+1$ by operation on representative elements. We obtain the following tables:

\begin{figure}[H]
  \includegraphics[scale=0.65]{images/cyclic_codes_03.png}
\end{figure}

As an example, note that $S_3 + S_5 = S_6$. Adding one of the corresponding elements from the cosets yields $(x+1) + (x^2+1) \equiv x^2+x$, which corresponds to $S_6$. For multiplication, let us take $S_3 S_7 = S_0$, corresponding to the following polynomial identitiy: $(x+1)(x^2+x+1) \equiv x^3+1$ which corresponds to $S_0$.

If we define $R=\{S_0, S_1,\ldots, S_7\}$, then $(R,+)$ is an Abelian group with $S_0$ as identity and for multiplication $S_1$ acts as identity. Not every element has a multiplicative inverse, so $R - S_0$ is not a group. However, the whole thing $(R,+,\cdot)$ is a ring denoted as $GF[x]/\langle x^3+1 \rangle$.

In the general case, denote a ring $GF(2)[x]/\langle x^n-1 \rangle$ as $R_n$ and a ring $\mF_q[x]/\langle x^n-1\rangle$ as $R_{n,q}$.

\paragraph{General Case.} For a field $\mF$ (with $q$ elements), the associated ring of polynomials $\mF[x]$ can be partitioned by a polynomial $f(x)$ of degree $m$ into $q^m$ different equivalence classes with one equivalence class for each remainder modulo $f(x)$. This ring is denoted as $\mF[x]/\langle f(x)\rangle$ or $\mF[x]/f(x)$.

As a side note (which will be proven later on), we note that the ring $\mF[x]/f(x)$ is a field iff the polynomial $f(x)$ cannot be factored over $\mF[x]$. In the example above, we have $x^3+1 = (x+1)(x^2+x+1)$ so $\mF[x]/(x^3+1)$ is not a field.



\subsection{Ideals in Rings}

\begin{definition}
  Let $R$ be a ring. A nonempty subset $I \subseteq R$ is an ideal if

  \begin{itemize}
    \item $I$ forms a group under addition in $R$.
    \item For any $a \in I$ and $r \in R$, $ar \in I$.
  \end{itemize}
\end{definition}

The first point ensures that the ideal has a structure; i.e. adding two ideal elements yields another ideal element. The second point is interesting in that the product of an ideal and a ``non-ideal'' element is still an ideal.

As a simple example, take $R$ as the set of integers. An ideal is the set of even numbers; the sum of any two even numbers is even and the product of any number with an even one gives an even number.

\begin{definition}
  An ideal $I$ in a ring $R$ is said to be principal if there exists some $g \in I$ such that every element $a \in I$ can be expressed as a product $a = mg$ for some $m \in R$. For a principal ideal, such an element $g$ is called the generator element. The ideal generated by $g$ is denoted as $\langle g \rangle$:

  \bee
    \langle g \rangle = \{hg : h \in R\}
  \eee
  
\end{definition}

The ideal in the example above is principal with a generator element $g = 2$. Every element of the ideal can be created by $2m$ with $m \in R$.

\begin{theorem}
  Let $I$ be an ideal in $\mF_q[x] / \langle x^n-1 \rangle$. Then
  \begin{itemize}
    \item There is a unique monic polynomial $g(x) \in I$ of minimal degree.
    \item $I$ is principal with generator $g(x)$.
    \item $g(x)$ divides $(x^n-1)$ in $\mF_q[x]$.
  \end{itemize}

\end{theorem}

\paragraph{Example.}Consider the polynomial $x^7+1$ over $GF(2)[x]$ which can be factored as

\bee
x^7 + 1 = (x+1)(x^3 + x + 1)(x^3 + x^2 + 1)
\eee

This factorization can be obtained in GAP via the following commands

\begin{verbatim}
a:=GF(2);
x:=X(GF(2));
Factors(p);
   [ x_1+Z(2)^0, x_1^3+x_1+Z(2)^0, x_1^3+x_1^2+Z(2)^0 ]
\end{verbatim}

Any combination of these 3 factors can be used as generator for a principal ideal.

\paragraph{Example.} A simpler example uses the polynomial $p(x) = x^3+1 = (x+1)(x^2+x+1)$. Choosing as generator $g_1(x) = x+1$, we can calculate the ideal by multiplying $g_1(x)$ with the elements of $GF(2)[x]/x^3+1 = \{0,1,x,1+x, x^2, x^2+1, x^2+x, x^2+x+1\}$ (and taking modulo-$x^3+1$). We can do this in GAP with the following script

\begin{verbatim}
a:=GF(2);
x:=X(GF(2));

# let's take the first factor, x+1 as generator for an ideal

GElements := [0,1,x,1+x,x^2,x^2+1,x^2+x, x^2+x+1];
p1 := x+1;

for e in GElements do
   Print(e,"...", e*p1 mod p, "\n");
od;
\end{verbatim}

There are some duplicates; after removal of them we obtain the ideal $I_1 = \{0, x+1, x^2+1, x^2+x\}$. The sum of two ideal elements is again an ideal element (``proof'' by going over all combinations) and the product of an ideal element and any element of $GF(2)[x]/x^3+1$ is again in the ideal $I_1$ (``proof'' by going over all combinations).

\subsection{Cyclic Codes}

A cyclic shift of a binary code word looks like this: We have a vector $\cbf = (c_0, c_1, \ldots,c_{n-2}, c_{n-1})$ and shift it cyclically to the right to obtain $\cbf' = (c_{n-1}, c_0, c_1, \ldots, c_{n-2})$.

\begin{definition}
  An $(n,k)$ block code is cyclic, it it is linear and if for every codeword, its right cyclic shift is also a codeword.
\end{definition}


This cyclic shifting can be expressed in terms of polynomial manipulations; If we have the codeword

\bee
\cbf = (c_0, c_1, \ldots,c_{n-2}, c_{n-1})
\eee
%
we can associate the polynomial

\bee
c(x) = \cbf = c_0 + c_1 x + c_2 x^2 + \cdots c_{n-1} x^{n-1}
\eee
%
A non-cyclic right-shift is then represented as $xc(x)$ and a cyclic right-shift is represented by $xc(x) \mod (c^n-1)$.

%%% Local Variables:
%%% mode: latex
%%% TeX-master: "journal"
%%% End:
