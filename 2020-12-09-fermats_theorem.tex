\DiaryEntry{Fermat's Theorem}{2020-12-09}{Number Theory}

\subsection{Fermat's Theorem}

\begin{theorem}
    let $p$ be a prime and suppose $p \nmid a$. Then

    \bee
        a^{p-1} \equiv 1 \mod p
    \eee
\end{theorem}

The proof starts by considering the first $p-1$ positive multiples of $a$,

\bee
a, 2a, 3a, \ldots, (p-1)a
\eee

None of these numbers is congruent modulo-$p$ to any other, nor is any congruent to zero. If it were, we would have

\bee
ra \equiv sa \mod p, \quad r \neq s
\eee

but then we could cancel $a$ from both sides and this would contradict the assumption $r \neq s$. Therefore, the integer sequence must be congruent modulo-$p$ to $1, 2, \ldots p-1$, taken in some order. When we multiply the integers together, we obtain

\bee
a \cdot 2a \cdot 3a \cdots (p-1)a \equiv 1 \cdot 2 \cdot 3 \cdots (p-1) \mod p
\eee

which can be rewritten as

\bee
a^{p-1} (p-1)! \equiv (p-1)! \mod p
\eee

Since $p \nmid (p-1)!$, we are allowed to cancel $(p-1)!$ from both sides and arrive at

\bee
a^{p-1} \equiv 1 \mod p \qed
\eee

Let's run through above proof with some concrete numbers. We choose $a = 4, p = 5$ and in the table below, we have written down the first multiples and their remainder when divided by $p = 5$.

\vspace{3mm}

\begin{tabular}{|cccc|c|} \hline
    a & 2a & 3a & 4a & 5a \\ \hline
    4 & 8 & 12 & 16 & 20 \\
    $\mod(4, 5)$ & $\mod(8, 5)$ & $\mod(12, 5)$ & $\mod(16, 5)$ & $\mod(20, 5)$ \\ \hline
    4 & 3 & 2 & 1 & 4 \\ \hline
\end{tabular}

\vspace{3mm}

It can be seen that the first $p-1$ terms all have a different remainders and no remainder is equal to zero.


The result can be slightly extended; we have the following theorem.

\begin{theorem}
    If $p$ is prime, then 
    \bee
    a^p \equiv a \mod p
    \eee
    for any integer $a$.
\end{theorem}

When $p \mid a$, then we have $a^p \equiv 0 \equiv a \mod p$. When $p \nmid a$, then Fermat's theorem states that $a^{p-1} \equiv 1 \mod p$. Multiplying both sides with $a$ yields $a^p \equiv a \mod p$. \qed

As an example, consider $p = 5$ and we have

\vspace{3mm}

\begin{tabular}{|ccc|} \hline
    $a$ & $a^5$ & $a^m \rem 5$ \\ \hline
    1 & 1 & 1     \\
    2 & 32 & 2    \\
    3 & 243 & 3    \\
    4 & 1024 & 4  \\
    5 & 3125 & 0  \\ \hline
\end{tabular}

\vspace{3mm}

Please note that in the last line, the RHS of Fermat's theorem yields $5 \equiv 0 \mod 5$ (and this corresponds to the third column in the last row).

\subsection{Usage}

We can use Fermat's theorem to simplify certain calculations; e.g. calculate the remainder of $5^{38}$ when divided by $11$. We can expand this as $5^{38} = (5^{10})^3 (5^2)^4$ and observe $5^{10} \equiv 1 \mod 11$. Since $3$ is prime we can apply Ferma's theorem to

\bee
(5^{10})^3 \equiv 5^{10} \equiv 1 \mod 11
\eee

We cannot use Fermat's theorem for the $(5^2)^4$ as $4$ is not a prime; we need to manually calculate things,

\bee
(5^2)^4 \equiv 25^4 \equiv 3^4 \equiv 81 \equiv 4 \mod 11
\eee

So we finally get

\bee
5^{38} \equiv 1 \cdot 4 \equiv 4 \mod 11 \qed
\eee

A second usage of Fermat's theorem is as a primality test for an integer $n$. If

\bee
a^n \equiv a \mod n
\eee

fails to hold for some $a$, then $n$ is necessarily composite. As an example consider $n = 177$. We chose $a = 2$ (note that $2 \nmid 177$)  and have

\bee
2^{117} = 2^{7 \cdot 16 + 5}
\eee

With $2^7 \equiv 11 \mod 117$ we further obtain

\bee
2^{117} \equiv 11^{16} \cdot 2^5 \equiv 121^8 \cdot 2^5 \equiv 4^8 \cdot 2^5 \equiv 2^{16} 2^5 \equiv 2^{21} \mod 117
\eee

We can rewrite $2^{21} = (2^7)^3$ which leads to

\bee
2^{21]} \equiv 11^3 \equiv  121 \cdot 11 \equiv 4 \cdot 11 \equiv 44 \mod 117
\eee

and we finally obtain

\bee
2^{117} \equiv 44 \neq 2 \mod 117 
\eee

which shows that $117$ is not a prime. Indeed, we have $117 = 3 \cdot 3 \cdot 13$.

Please note that the inverse is not necessarily true; i.e. if $a^{n} \equiv a \mod n$ holds for some $a$, then $n$ need not necessarily be a prime. As an example, consider $n = 341$ which is not a prime ($341 = 11 \cdot 31$). Nevertheless, we have $2^{341} \equiv 2 \mod 341$; however, $3^{341} \equiv 168 \neq 3 \mod 341$ and this shows that $341$ is not a prime (note that we have $2 \nmid 341$ and $3 \nmid 341$).

We call composite numbers $n$ which yield

\bee
a^{n-1} \equiv 1 \mod n
\eee

\emph{pseudoprimes in base $a$}. Equivalent conditions for pseudoprimes in base $a$ are also

\bee
a^n \equiv a \mod n \leftrightarrow n \mid 2^n - 2
\eee

For every base number $a$, there is a smallest pseudprime $n$; e.g. for base $2$ the smallest pseudprime is $341$. The OEIS wiki has a \href{https://oeis.org/wiki/Table_of_Fermat_pseudoprimes}{table} listing the pseudoprimes for various bases. A small excerpt is as follows,

\vspace{3mm}

\begin{tabular}{|c|c|} \hline
  a & n  \\ \hline
  2 & 341 \\
  3 & 91 \\
  4 & 15 \\ \hline
\end{tabular}

\vspace{3mm}

For example, we have

\bee
3^{91} \equiv 1 \mod 91
\eee

although $91 = 7 \cdot 13$ is not a prime.

The so called \emph{Carmichael numbers} are numbers $n$ which are pseudoprime to \emph{every} base $a$; i.e.

\bee
a^{n-1} \equiv 1 \mod n, \forall a \,\, \text{with} \,\, \gcd(a,n) = 1
\eee

The smallest Carmichael number is $561$. We can use the following GAP function to test whether a number is a Carmichael one or not

\begin{verbatim}
TestCarmichael := function(N)
   local coprimes, tst, a;
   coprimes:=PrimeResidues(N);

   for a in coprimes do
       tst:=a^N mod N;
          if(tst <> a) then
             Print("FAILURE with base a = ", a, "\n");
          fi;
   od;
   Print("SUCCESS");
end;
\end{verbatim}



%%% Local Variables:
%%% mode: latex
%%% TeX-master: "journal"
%%% End:
