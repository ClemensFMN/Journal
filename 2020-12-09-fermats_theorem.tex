\DiaryEntry{Fermat's Theorem}{2020-12-09}{Number Theory}

\begin{theorem}
    let $p$ be a prime and suppose $p \nmid a$. Then

    \bee
        a^{p-1} \equiv 1 \mod p
    \eee
\end{theorem}

The proof starts by considering the first $p-1$ positive multiples of $a$,

\bee
a, 2a, 3a, \ldots, (p-1)a
\eee

None of these numbers is congruent modulo-$p$ to any other, nor is any congruent to zero. If it were, we would have

\bee
ra \equiv sa \mod p, \quad r \neq s
\eee

but then we could cancel $a$ from both sides and this wold contradict the assumption $r \neq s$. Therefore, the integer sequence must be congruent modulo-$p$ to $1, 2, \ldots p-1$, taken in some order. When we multiply the integers together, we obtain

\bee
a \cdot 2a \cdot 3a \cdots (p-1)a \equiv 1 \cdot 2 \cdot 3 \cdots (p-1) \mod p
\eee

which can be rewritten as

\bee
a^{p-1} (p-1)! \equiv (p-1)! \mod p
\eee

Since $p \nmid (p-1)!$, we are allowed to cancel $(p-1)!$ from both sides and arrive at

\bee
a^{p-1} \equiv 1 \mod p \qed
\eee

The result can be slightly extended; we have the following theorem.

\begin{theorem}
    If $p$ is prime, then 
    \bee
    a^p \equiv a \mod p
    \eee
    for any integer $a$.
\end{theorem}

When $p \mid a$, then we have $a^p \equiv 0 \equiv a \mod p$. When $p \nmid a$, then Fermat's theorem states that $a^{p-1} \equiv 1 \mod p$. Multiplying both sides with $a$ yields $a^p \equiv a \mod p$. \qed

%%% Local Variables:
%%% mode: latex
%%% TeX-master: "journal"
%%% End:
