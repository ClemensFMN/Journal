\DiaryEntry{Upcoming Ideas...}{}{}

\subsection{Making a denominator square-root free.}

\bee
\frac{1}{1 + \sqrt{2}} = \frac{1 - \sqrt{2}}{(1 + \sqrt{2})(1 - \sqrt{2})} = \frac{1 - \sqrt{2}}{1 - 2} = \sqrt{2} - 1
\eee

Based on $(a-1)(a+1) = a^2 - 1$.

A similar idea is to make the denominator real,

\bee
\frac{1}{1-j} = \frac{1+j}{(1-j)(1+j)} = \frac{1+j}{(1-j)(1+j)} = \frac{1+j}{2}
\eee


\subsection{Graph / Tree traversal}

Actually, two things. Numerb one: The DFS algorithm \ref{2020-02-27:entry} is recursive. There is, however, also an iterative version available which uses a stack. Implement this algorithm.

Number two: Both BFS and DFS simplify in case of a tree as it is a graph without cycles. Show how the algs simplify in this case.


\subsection{Inequalities and Squareroots}

Prove that $0 < 11 - 6 \sqrt{3} < 1$. We have the following reasoning:

\begin{align*}
  &121 > 108 > 100 \\
  &\sqrt{121} > \sqrt{108} > \sqrt{100} \\
  &11 > 6 \sqrt{3} > 10 \\
  &-11 < -6 \sqrt{3}  < -10
\end{align*}

The square root is a convex function, therefore from $A > B$ follows $\sqrt{A} > \sqrt{B}$ which explains the step from the first to second line.

Adding $11$ to the last line yields

\bee
0 < 11 - 6 \sqrt{3} < 1 \qed
\eee

Also interesting is to bound a square root between integers which differ by one; e.g. $A < \sqrt{37} < B$. We can bound this as follows

\begin{align*}
  &\sqrt{36} < \sqrt{37} < \sqrt{49} \\
  & 6 < \sqrt{37} < 7
\end{align*}

For the lower / upper bound we use the ``closest'' square number. The root then gives integers and the bounds differ by $1$ as requested. This holds for all numbers in between; i.e. we have

\bee
6 < \sqrt{37} \cdots \sqrt{48} < 7
\eee


\subsection{Distance Point - Random Point}

Here we have one fixed point $P(x_P,0)$ and ask for the distance $d$ to another point $Q$ located randomly inside the unit circle / box. The distance $d$ is random and given by

\bee
d = \sqrt{(x_p - x)^2 + y^2}
\eee

where $(x,y)$ are the coordinates of $Q$. In case of a unit box, they are distributed according to

\bee
x,y \sim \Uc(-1,1)
\eee

in case of a unit circle, they are constrained by $x^2 + y^2 \leq 1$. According to \href{https://stats.stackexchange.com/questions/481543/generating-random-points-uniformly-on-a-disk}{this} Stackoverflow answer, the distributions of $x$ and $y$ are a bit more tricky...

The expected distance $E(d)$ is then given by

\bee
E(d) = \int_{x,y} d f_x(x) f_y(y) dx dy
\eee

where $f_x(x), f_y(y)$ are the distributions of $x$ and $y$, respectively. The integrals seem to be rather tricky; even in case of the unit box (where the distribution is simple), I'm not sure if there is a closed-form solution for the double-integral. In case of the unit disc, already the distributions are complex (polar coordinates may help as $\phi$ is uniform in $[0, 2\pi]$ and $r$ is $\sim  r$ in $[0,1]$.

\subsection{Predator-Prey}

See entry \ref{2018-07-30:entry} and extend it to general coefficients,

\begin{align}\label{2018-07-30-eq1}
R' = \frac{dR}{dt} &= \alpha R - \beta RF \nonumber \\
F' = \frac{dF}{dt} &= -\gamma F + \delta RF
\end{align}

We can rewrite the first equation as $R(\alpha - \beta F) = 0$ and obtain two fixed points from it: (i) $R = $, (ii) $F = \alpha/\beta$. We can use the second equation similarly, $F(- \gamma+  \delta R)$ and obtain (i) $F = 0$, (ii) $R = \gamma/\delta$. Therefore, we have two fixed points,

\begin{align}
  &R = F = 0\\
  &R = \gamma / \delta, F = \alpha / \beta
\end{align}

To assess the stability of the fixed points, we calculate the Jacobian $\Jbf$ of the differential equation system \todo{Check!!!},

\bee
\Jbf = \begin{pmatrix} \frac{dR'}{dR} & \frac{dR'}{dF} \\ \frac{dF'}{dR} & \frac{dF'}{dF} \end{pmatrix} = \begin{pmatrix} \alpha - \beta F & - \beta R \\ \delta F & \delta R - \gamma \end{pmatrix}
\eee

Let's consider the fixed point $R = F = 0$ first. For these values, the Jacobian becomes

\bee
\Jbf_1 = \begin{pmatrix} \alpha & 0 \\ 0 & - \gamma \end{pmatrix}
\eee

The eigenvalues are the solution to $|\Jbf_1 - \lambda \Ibf | = 0$. We have

\bee
\begin{vmatrix} \alpha - \lambda & 0 \\ 0 & - \gamma - \lambda \end{vmatrix} = (\alpha - \lambda)(- \gamma - \lambda) = 0
\eee

and therefore the two eigenvalues become $\lambda_1 = \alpha, \lambda_2 = - \gamma$. In our model, the parameters $\alpha$ and $\gamma$ are always greater than zero, and therefore such the sign of the eigenvalues will be different. Hence the fixed point at the origin is a saddle point which is an instable fixed point \todo{is a saddle point always instable?}.

For the other fixed point ($R = \gamma / \delta, F = \alpha / \beta$), the Jacobian becomes

\bee
\Jbf_2 = \begin{pmatrix} \alpha - \beta \alpha / \beta & - \beta \gamma / \delta \\ \delta \alpha / \beta & \delta \gamma / \delta - \gamma \end{pmatrix} = \begin{pmatrix} 0 & - \beta \gamma / \delta \\ \delta \alpha / \beta & 0 \end{pmatrix}
\eee

The eigenvalues of this matrix are given by

\bee
\begin{vmatrix} -\lambda & - \beta \gamma / \delta \\ \delta \alpha / \beta & - \lambda \end{vmatrix} = \lambda^2 - (- \beta \gamma / \delta)(\delta \alpha / \beta) = 0
\eee

From this follows

\bee
\lambda^2 + \alpha \gamma = 0 \rightarrow \lambda_{1,2} = \pm j \sqrt{\alpha \gamma}
\eee

As the eigenvalues are both purely imaginary and conjugate to each other, this fixed point must either be a center for closed orbits in the local vicinity or an attractive or repulsive spiral. In conservative systems, there must be closed orbits in the local vicinity of fixed points that exist at the minima and maxima of the conserved quantity. \todo{Copied from Wikipedia - do not understand completely!}

\begin{figure}[H]
    \centering
    \includegraphics[scale=0.6]{images/predator_prey.png}
\end{figure}


\subsection{Walli's Integral}

Defined as (Nahin, Inside Interesting Integrals, p. 122),

\bee
I(k) = \int_0^1 (x-x^2)^k dx = \int_0^1 x^k (1-x)^k dx = B(k+1, k+1) = \frac{\Gamma(k+1) \Gamma(k+1)}{\Gamma(2k+2)}
\eee

which becomes (for integer-valued $k$)

\bee
I(k) = \int_0^1 (x-x^2)^k dx = \frac{(k!)^2}{(2k+1)!}
\eee

Especially interesting is $k =1/2$. TODO


\subsection{Integral $\log(x)/x^k$}

Start with $k = 2$ and consider the following integral

\bee
I = \int_1^\infty \frac{\log(x)}{x^2}dx
\eee

which we can solve via partial integration: We chose $u=\log(x), v'=1/x^2$ and obtain $u' = 1/x, v = -1/x$ and therefore the integral becomes

\bee
I = - \left. \frac{\log(x)}{x} \right|_1^\infty + \int_1^\infty \frac{1}{x^2} dx = 0 + \left.\left( - \frac{1}{x} \right)\right|_1^\infty = 1
\eee

Interestingly, this can be extended to

\bee
I = \int_1^\infty \frac{\log(x)}{x^k}dx
\eee

Same procedure, $u=\log(x), v'=1/x^k$ and obtain $u' = 1/x, v = \frac{x^{-k+1}}{-k+1}$ and therefore we have

\bee
I = - \left. \frac{x^{-k+1} \log(x)}{-k+1} \right|_1^\infty - \int_1^\infty x^{-1} \frac{x^{-k+1}}{-k+1} dx = 0 - \frac{1}{1-k} \int_1^\infty x^{-k} dx = \frac{1}{(k-1)^2}
\eee


\subsection{TIKZ}

Some text...

\begin{figure}[H]
\centering
\begin{tikzpicture}[transform shape]
  \graph [nodes={circle,draw}] {
    %A -> ["1"] B -> ["2"] C;
    %D -> ["4"] C;
    A -> ["1"] B -> ["2"] C;
    D -> ["4"] C;
    A ->["3", bend left] C;
  };
\end{tikzpicture}
\caption{Example Graph, I.}
\end{figure}



%%% Local Variables:
%%% mode: latex
%%% TeX-master: "journal"
%%% End:
