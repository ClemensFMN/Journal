\DiaryEntry{Non-Abelian Groups}{2016-10-19}{Algebra}

Based on
\href{http://math.stackexchange.com/questions/1971166/show-that-a-nonabelian-group-must-have-at-least-five-distinct-elements}{this}.

A group has an identity element, \(e\). Then we need two more elements
\(a, b\) with \(ab \neq ba\). Furthermore, \(a\) must not be the inverse
of \(b\), as this would prevent \(ab \neq ba\). Based on this argument,
a non-abelian group could have 5 elements which are \(1, a, b, ab, ba\).

However, 5 is a prime; and groups of prime order are cyclic, and cyclic
groups are abelian: In a cyclic group, all elements have the form
\(a^n\), with \(n \in \mathbb{Z}\). Take 2 elements
\(x = a^m, y = a^n\); then \(xy = a^{m+n} = a^{n+m} = yx\) qed.
