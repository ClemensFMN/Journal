\DiaryEntry{Wilson's Theorem}{Number Theory}{2020-12-17}

\begin{theorem}
    If $p$ is prime, then
    \bee
    (p-1)! \equiv -1 \mod p \equiv p-1 \mod p
    \eee
\end{theorem}

As an example, consider $p = 7$. We then have

\bee
(p-1)! = 6! = 720 \equiv 6 = p-1 \mod p
\eee

The cases $p =2, 3$ can be evaluated directly; let's concentrate on $p > 3$. Suppose that $a$ is one of the $p-1$ integers $1,2,3,\ldots, p-1$ and consider the linear congruence $ax \equiv 1 \mod p$. Since $p$ is prime, $\gcd(a,p) = 1$ and therefore the linear congruence has a unique solution (modulo-$p$) which we denote $a'$. We therefore have $a a' \equiv 1 \mod p$ with $1 \leq a' \leq p-1$.

We next show that $a = a'$ iff $a = 1$ or $a = p-1$ because $p$ is prime. The congruence relation $a^2 \equiv 1 \mod p$ is equivalent to $(a-1)(a+1) \equiv 0 \mod p$. The LHS becomes zero when either $a = 1$ or when $a = -1 \equiv p-1 \mod p$.

If we omit the values $2, p-1$ from the list of values for $a$, we can group the remaining integers ($3,4, p-2$) into pairs $a, a'$ with $a \neq a'$ and $a a' \equiv 1 \mod p$. We have $p-3$ numbers in the list and therefore $(p-3)/2$ pairs. We can therefore multiply all pair expressions together and obtain

\bee
2 \cdot 3 \cdots p-3 \equiv 1 \mod p
\eee

which is equivalent together

\bee
(p-2)! \equiv 1 \mod p
\eee

If we multiply both sides with $p-1$ we obtain

\bee
(p-1)! \equiv p-1 \equiv -1 \mod p \qed
\eee

Let's illustrate the reasoning with the example from above, $p=7$. The list of integers is $2,3,4,5 = p-2$ and we have the following two groups

\bee
2 \cdot 4 = 8 \equiv 1 \mod 7, \quad 3 \cdot 5 = 15 \equiv 1 \mod 7
\eee

We can multiply the two expressions together and obtain

\bee
2 \cdot 3 \cdot 4 \cdot 5 = 5! \equiv 1 \mod 7
\eee

and by multiplying both sides with $p-1=6$ we finally obtain

\bee
6! \equiv 1 \mod 7 \qed
\eee




%%% Local Variables:
%%% mode: latex
%%% TeX-master: "journal"
%%% End:
