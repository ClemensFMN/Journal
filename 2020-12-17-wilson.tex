\DiaryEntry{Wilson's Theorem}{Number Theory}{2020-12-17}

\begin{theorem}
    If $p$ is prime, then
    \bee
    (p-1)! \equiv -1 \mod p \equiv p-1 \mod p
    \eee
\end{theorem}

As an example, consider $p = 7$. We then have

\bee
(p-1)! = 6! = 720 \equiv 6 = p-1 \mod p
\eee

The cases $p =2, 3$ can be evaluated directly; let's concentrate on $p > 3$. Suppose that $a$ is one of the $p-1$ integers $1,2,3,\ldots, p-1$ and consider the linear congruence $ax \equiv 1 \mod p$. Since $p$ is prime, $\gcd(a,p) = 1$ and therefore the linear congruence has a unique solution (modulo-$p$) which we denote $a'$. We therefore have $a a' \equiv 1 \mod p$ with $1 \leq a' \leq p-1$.

We next show that $a = a'$ iff $a = 1$ or $a = p-1$ because $p$ is prime. The congruence relation $a^2 \equiv 1 \mod p$ is equivalent to $(a-1)(a+1) \equiv 0 \mod p$. The LHS becomes zero when either $a = 1$ or when $a = -1 \equiv p-1 \mod p$.

If we omit the values $2, p-1$ from the list of values for $a$, we can group the remaining integers ($3,4, p-2$) into pairs $a, a'$ with $a \neq a'$ and $a a' \equiv 1 \mod p$. We have $p-3$ numbers in the list and therefore $(p-3)/2$ pairs. We can therefore multiply all pair expressions together and obtain

\bee
2 \cdot 3 \cdots p-3 \equiv 1 \mod p
\eee

which is equivalent together

\bee
(p-2)! \equiv 1 \mod p
\eee

If we multiply both sides with $p-1$ we obtain

\bee
(p-1)! \equiv p-1 \equiv -1 \mod p \qed
\eee

Let's illustrate the reasoning with the example from above, $p=7$. The list of integers is $2,3,4,5 = p-2$ and we have the following two groups

\bee
2 \cdot 4 = 8 \equiv 1 \mod 7, \quad 3 \cdot 5 = 15 \equiv 1 \mod 7
\eee

We can multiply the two expressions together and obtain

\bee
2 \cdot 3 \cdot 4 \cdot 5 = 5! \equiv 1 \mod 7
\eee

and by multiplying both sides with $p-1=6$ we finally obtain

\bee
6! \equiv 1 \mod 7 \qed
\eee

Unlike with Fermat's theorem, Wilson's theorem holds also in the reverse direction; i.e. If $(n-1)! \equiv -1 \mod n$, then $n$ must be prime.

We can use Wilson's theorem to solve certain \emph{quadratic congruences} via the following theorem.

\begin{theorem}
    The quadratic congruence $x^2+1 \equiv 0 \mod p$, where $p$ is an odd prime, has a solution iff $p \equiv 1 \mod 4$.
\end{theorem}

Proof: TBD

\subsection{The Fermat-Kraitchik Factorization Method}

The search for factors of an odd integer $n$ is equivalent to obtaining integral solutions $x$ yand $y$ of the equation

\bee
n = x^2 - y^2
\eee

If $n$ is the difference of two squares, then we can factor $n$ according together

\bee
n = x^2 - y^2 = (x-y)(x+y)
\eee

If $n$ can be factored as $n = ab$ with $a \geq b \geq 1$, then we may write

\bee
n = \left( \frac{a+b}{2})^2 - \frac{a-b}{2})^2 \right)^2
\eee

As $n$ is assumed to be odd, so are $a$ and $b$, and $(a+b)/2$ and $(a-b)/2$ will be nonnegative integers. The factorization method works as follows: We search for $x, y$ satisfying the equation $n = x^2 - y^2$, or

\bee
x^2 - n = y^2
\eee

by first finding the smallest integer $k$ for which $k^2 \geq n$. Now we consider successively the numbers

\bee
k^2 - n, (k+1)^2 - n, (k+2)^2 - n, \ldots
\eee

until a value $m \geq \sqrt{n}$ is found which makes $m^2 - n$ a square. This process cannot go on forever; eventually we reach $m = (n+1)/2$ with

\bee
\left( \frac{n+1}{2} \right)^2 - n = \left( \frac{n-1}{2} \right)^2
\eee

which represents the trivial factorization $n = n \cdot 1$. If this number is reached without a square difference discovered before, then $n$ has no factors other than $n$ and $1$ and is therefore prime.


\paragraph{Example.} We consider the number $n = 119143$. The next biggest square number is $\lfloor \sqrt{119143} \rfloor = 346$. We consider now the sequence of integers described above; i.e. $346 \leq m \leq (119143+1)/2 = 59572$ as follows

\begin{align*}
    &346^2 - 119143 = 573 \\
    &347^2 - 119143 = 1266 \\
    & \cdots \\
    &352^2 - 119413 = 4761 = 69^2
\end{align*}

In the last line $m^2 - n$ is a sqaure and we stop. We deduce $119431 = 352^2 - 69^2 = (352+69)(352-69) = 421 \cdot 283$ which shows that $119431$ is \emph{not} prime.

As another example, consider $n= 5783$. We consider the sequence $\lfloor \sqrt{5783} \rfloor = 77 leq m \leq (5783+1)/2 = 2892$ as follows

\begin{align*}
    &77^2 - 5783 = 146 \\
    &78^2 - 5783 = 301 \\
    &79^2 - 5783 = 458 \\
    & \cdots
    & 2892^2 - 5783 = 8357881 = 2891^2
\end{align*}

The last line is the first square; however $5783 = 2892^2 - 2891^2 = 5783 \cdot 1$ which shows that $5783$ \emph{is prime}.

In Julia, we can express the algorithm as

\begin{verbatim}
    function fermat_kraitchik(n)
        rn = ceil(sqrt(n)) : (n+1)/2 - 1
        for vl in rn
            b = sqrt(vl^2 - n)
            if(isinteger(b))
                print("is composite")
                return((vl+b), (vl-b))
            end
        end
        print("prime")
        return((n, 1))
    end
\end{verbatim}


%%% Local Variables:
%%% mode: latex
%%% TeX-master: "journal"
%%% End:
