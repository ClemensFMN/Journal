\DiaryEntry{Primitive Roots and Indices}{2021-03-02}{Number Theory}

\subsection{Order}

For $\gcd(a,n)=1$ and $n > 1$, Euclid's theorem states that $a^{\phi(n)} \equiv 1 \mod n$. However, $a^k$ could be $\equiv 1 \mod n$ for smaller $k < \phi(n)$. The smallest $k$ for which this is the case is the \emph{order} of the integer $a$ modulo-$n$ and we therefore can bound the order $k$ as $k \leq \phi(n)$.

As an example, consider $a = 2, n = 7 (\gcd(2,7)=1)$, and write down successive powers of $2$ modulo-$7$. We have

\bee
2^1 \equiv 1 \mod 7, 2^2 \equiv 4 \mod 7, 2^3 \equiv 1 \mod 7, 2^4 \equiv 2 \mod 7, \ldots, 2^{\phi(7)} = 2^6 \equiv 1 \mod 7
\eee

From this we can see that the order of $2$ modulo-$7$ is $k=3$ which is lower than $\phi(7) = 6$ provided by Euler's theorem.

If two integers $a, b$ are congruent modulo-$n$, then they have the same order modulo-$n$. For if $a \equiv b \mod n$ and $a^k \equiv \mod n$, we have $a^k \equiv b^k \mod n$ (see entry \ref{2020-11-17:entry}) and therefore $b^k \equiv 1 \mod n$.

Note that the condition $\gcd(a,n)=1$ is a crucial one; if this condition is violated, then $a^k \equiv 1 \mod n$ would have no solution and the order is not defined. Therefore, we assume $\gcd(a,n)=1$ in the following.

In the previous example, we had $2^k \equiv 1 \mod 7$ whenever $k$ was a multiple of $3$, the order of $2$ modulo-$7$. This is no coincidence, as the following theorem shows.

\begin{theorem}
  Let the integer $a$ have order $k$ modulo-$n$. Then $a^h \equiv 1 \mod n$ iff $k | h$; in particular, $k | \phi(n)$.
\end{theorem}

To prove, we start with $h | k$ which implies $h = jk$ for some integer $j$. Because $a^k \equiv 1 \mod n$, we have

\bee
a^h \equiv a^{jk} \equiv \left( a^j \right)^k \equiv 1^k \equiv 1 \mod n
\eee

The other way round starts with an integer $h$ satisfying $a^h \equiv 1 \mod n$. By the division algorithm, there exist integers $q$ and $r$ such that $h = qk+r$ where $0 \leq r < k$. Therefore, we have

\bee
a^h \equiv a^{qk+r} \equiv \left(a^k\right)^q a^r \mod n
\eee

Our hypothesis states that $a^h \equiv a^k \equiv 1 \mod n$ and therefore $a^r \equiv 1 \mod n$. Because $0 \leq r < k$, we arrive at $r=0$. Therefore, $h = qk$ and $k | h$. \qed

With Gap, we can calculate the order as follows.

\begin{verbatim}
gap> OrderMod(2, 7);
3
\end{verbatim}


\begin{theorem}
  \label{2021-03-02:th1}
  If the integer $a$ has order $k$ modulo-$n$, then $a^i \equiv a^j \mod n$ iff $i \equiv j \mod n$.
\end{theorem}

First, suppose $a^i \equiv a^j \mod n$ with $i \geq j$. Because $a$ is relatively prime to $n$, we may cancel a power of $a$ and obtain $a^{i-j} \equiv 1 \mod n$. According to the previous theorem, this holds iff $k | i-j$ which is equivalent to $i \equiv j \mod k$ (definition of congruence). 

Conversely, let $i \equiv j \mod k$. Tnen we have $i = j+  qk$ for some integer $q$. By the definition of $k$, we have $a^k \equiv 1 \mod n$ so that

\bee
a^i \equiv a^{j+qk} \equiv a^j \left( a^k \right)^q \equiv a^j \mod n \qed
\eee

As a corollary, if $a$ has order $k$ modulo-$n$, then the integers $a, a^2, \ldots, a^k$ are incongruent modulo-$n$.

If $a^i \equiv a^j \mod n$, for $1 \leq i \leq j \leq k$, then the previous theorem ensures that $i \equiv j \mod n$. But this is impossible unless $i=j$. \qed

In our running example with $a=2, n=7$, $2$ has order $3$. Therefore, $2, 2^2, 2^3$ are incongruent modulo-$7$.

We next tackle the question whether we can express the order of any integral power of $a$ in terms of the order of $a$? This is answered in the following theorem.

\begin{theorem}
  If the integer $a$ has order $k$ modulo-$n$ and $h > 0$, then $a^h$ has order $k | \gcd(h,k)$ modulo-$n$.
\end{theorem}

Proof is omitted.

As a corollary, we note that if $a$ has order $k$ modulo-$n$, then $a^h$ has also order $k$ iff $\gcd(h,k)=1$.

As an example, consider $n=13$ and obtain the orders of several numbers in a Gap session.

\begin{verbatim}
gap> OrderMod(2, 13);
12
gap> OrderMod(2^2, 13);
6
gap> OrderMod(2^3, 13);
4
\end{verbatim}

We have

\bee
\frac{12}{\gcd(2,12)} = 6, \quad \frac{12}{\gcd(3, 12)} = 4
\eee

which matches the result of the theorem. The order of $2$ equals $12$ modulo-$13$; since $\gcd(5, 12) = 1$, the order of $2^5 \equiv 6$ is also equal to $12$ which Gap shows to be correct

\begin{verbatim}
gap> OrderMod(6, 13);
12
\end{verbatim}

\subsection{Primitive Roots}\label{2021-03-02:primroots}

If an integer $a$ has the largest order possible, then we call it a primitive root of $n$. More exactely, if $\gcd(a,n)=1$ and $a$ is of order $\phi(n)$ modulo-$n$, then $a$ is a \emph{primitive root} of the integer $n$. In other words, $n$ has $a$ as primitive root if $a^{\phi(n)} \equiv 1 \mod n$ but $a^k \not \equiv 1 \mod n$ for all integers $k < \phi(n)$.

Continuing with our example $n=13$, we write down the order of integers from $1$ to $12$,

\vspace*{2mm}

\begin{tabular}{lllllllllllll}
  a     & 1 & 2  & 3 & 4 & 5 & 6  & 7  & 8 & 9 & 10 & 11 & 12 \\ \hline
  Order & 1 & 12 & 3 & 6 & 4 & 12 & 12 & 4 & 3 & 6  & 12 & 2 
\end{tabular}

\vspace*{2mm}

With $\phi(13) = 12$, the maximum order is $12$. From the table, we see that the numbers $2, 6, 7, 11$ have order $12$ and are therefore the primitive roots of $13$. We can do this in Gap via the \verb+PrimitiveRootMod+ function (which takes an optional parameter from which value to start searching for the next primitive root) as follows:

\begin{verbatim}
gap> Phi(13);
12
gap> PrimitiveRootMod(13);
2
gap> OrderMod(2, 13);
12
gap> PrimitiveRootMod(13, 2);
6
gap> OrderMod(6, 13);
12
gap> PrimitiveRootMod(13, 6);
7
gap> OrderMod(7, 13);
12
gap> PrimitiveRootMod(13, 7);
11
gap> OrderMod(11, 13);
12
\end{verbatim}

The whole topic is closely related to Group theory; see e.g. \ref{2017-05-05:entry}.

We have some more theorems.

\begin{theorem}
  Let $\gcd(a,n)=1$ and let $a_1, a_2, \ldots, a_{\phi(n)}$ be the positive integers less than $n$ and relatively prime to $n$. If $a$ is a primitive root of $n$, then

  \bee
  a, a^2, \ldots, a^{\phi(n)}
  \eee

  are congruent modulo $n$ to $a_1, a_2, \ldots, a_{\phi(n)}$ in some order.
\end{theorem}

Because $a$ is relatively prime to $n$, the same holds for all the powers of $a$. Therefore, each $a^k$ is congruent modulo $n$ to one of the $a_i$. The $\phi(n)$ numbers in the set $a, a^2, \ldots, a^{\phi(n)}$ are incongruent by theorem %\ref{2021-03-02:th1} 
and thus, these powers must represent (in different order) the integers $a_1, a_2, \ldots, a_{\phi(n)}$. \qed

We illustrate this with our running example of $n=13$. Before we have shown that $a=2$ is a primitive root and $\phi(13) = 12$ (as $13$ is a prime). All numbers between $1$ and $12$ are relatively prime to $13$ and therefore the numbers $2^1, 2^2, \ldots 2^{12}$ are congruent modulo-$13$ to $1,2,\ldots, 12$.

\vspace*{2mm}

\begin{tabular}{lllllllllllll}
  k     & 1 & 2  & 3 & 4 & 5 & 6  & 7  & 8 & 9 & 10 & 11 & 12 \\ \hline
  $a^k$ & 2 & 4 & 8 & 3 & 6 & 12 & 11 & 9 & 5 & 10  & 7 & 1 
\end{tabular}

\vspace*{2mm}

However, the theorem is not restricted to primes; let's consider $n=9$. With $\phi(9) = 6$, the integers less than $9$ and relatively prime to it are $1,2,4,5,7,8$. A primitive root is $2$ and we have

\vspace*{2mm}

\begin{tabular}{lllllll}
  k     & 1 & 2  & 3 & 4 & 5 & 6  \\ \hline
  $a^k$ & 2 & 4 & 8 & 7 & 5 & 1
\end{tabular}

\vspace*{2mm}

\qed

As a corollary, we have the following: If $a$ has order $k$ modulo-$n$, then $a^h$ has also order $k$ (modulo-$n$) iff $\gcd(h,k) = 1$.

If a primitive root exists, we can state how many there are using the following theorem.

\begin{theorem}
  If $n$ has a primitive root , then it has $\phi(\phi(n))$ of them.
\end{theorem}

If $a$ is a primitive root of $n$, then any other primitive root of $n$ is found among the members of the set $a, a^2, a^{\phi(n)}$. But the number of powers $a^k, 1 \leq k \leq \phi(n)$, that have order $\phi(n)$ is equal to the number of integers $k$ for which $\gcd(k, \phi(n)) = 1$; there are $\phi(\phi(n))$ such integers and $\phi(\phi(n))$ primitive roots of $n$. \qed

In our running example of $n=13$, we have $\phi(\phi(13)) = 4$ primitive roots.

What remains open is the question whether a number has a primitive root and how it can be obtained. What we observe is that non-primes can have primitive roots (e.g. $9$ has a primitive root of $2$); however, the existence of primitive roots is more of an exception than the rule.

\subsubsection{Primitive Roots of Primes}

We start with a slight detour, namely the number of solutions of a polynomial congruence.

\begin{theorem}
If $p$ is prime and

\bee
f(x) = a_n x^n + \cdots + a_1 x + a_0 (a_n \not\equiv 0 \mod p)
\eee

is a polynomial of degree $n \geq 1$ with integral coefficients, then the congruence $f(x) \equiv 0 \mod p$ has at most $n$ incongruent solutions modulo-$p$.
\end{theorem}

Proof is omitted.

We have the following Corollary: If $p$ is prime and $d | p-1$, then

\bee
x^d - 1 \equiv 0 \mod p
\eee

has exactely $d$ solutions.

Proof is omitted.

Next we can show that, for any prime $p$, there exist integers with order corresponding to each divisor of $p-1$.

\begin{theorem}
  If $p$ is a prime number and $d | p-1$, then there are exactly $\phi(d)$ incongruent integers having order $d$ modulo-$p$.
\end{theorem}

Proof is omitted.\todo{add proof}

In the running example with $n=13$, $p-1 = 12$ and the allowed values of $d$ are $1,2,3,4,6, 12$ (as they all divide $p-1=12$). According to the theorem, we have the following numbers with their corresponding order. We can compare with the table %\ref{2021-03-02:primroots} 
at the beginnig of the Section to ensure that this is correct.

\vspace*{2mm}

\begin{tabular}{lll}
  Order & Number & Integers \\ \hline
  1     & 1      & 1        \\
  2     & 1      & 12       \\
  3     & 2      & 3, 9     \\
  4     & 2      & 5, 8     \\
  6     & 2      & 4, 10    \\
  12    & 4      & 2, 6, 7, 11
\end{tabular}

\vspace*{2mm}

For a prime $p$, a primitive root $a$ has to have order $\phi(p) = p-1$. So in the above theorem we set $d=p-1$ and there are $\phi(p-1)$ incongruent integers having order $p-1$; i.e. there are $\phi(p-1)$ primitive roots.


Apart from this, there are tables listing whether and which primitive roots exist. For example, see \href{http://oeis.org/A046147}{OEIS sequence A046147}, which lists the primitive roots for every integer (if it / they exist). For example, the page shows that $n=5$ has two primitive roots ($2$ and $3$) and $n=9$ has also two primitive roots ($2$ and $5$). Just be careful, in the case of $n$ not prime, only the numbers relatively prime to $n$ are relevant.

Looking at such tables, we see that the smallest primitive root, denoted as $\chi(n)$, is less or equal to $19$ for all integers up to $200$. The value $\chi(n)$ becomes arbitrarily large as $n$ increases without bound. However, in most cases $\chi(n)$ is quite small: Among the $78498$ odd primes up to $10^6$, $\chi(n) \leq 6$ holds for about $80\%$ of all primes, $\chi(n) = 2$ holds for about $37\%$ of all primes and $\chi(n) = 3$ for about $22\%$ of all primes.

\subsubsection{Primitive Roots of Composite Numbers}

There are some composite numbers having primitive roots; main result is that in case of $n$ being (an odd) prime, the following theorem shows when primitive roots exist.

\begin{theorem}
  There are primitive roots modulo-$n$ iff $n=1,2,4, p^k, 2p^k$ with integer $k$ and $p$ an odd prime.
\end{theorem}



\subsection{Indices}

Let $n$ be an integer with a primitive root $r$. The first $\phi(n)$ powers of $r$,

\bee
r, r^2, \ldots, r^{\phi(n)}
\eee

are congruent modulo-$n$, in some order, to those integers less than $n$ and relatively prime to it. 

% \todo{n=0:12,mod.(2 .^ n,13) }

Therefore, an arbitrary integer $a$ can be expressed as

\bee
a \equiv r^k \mod n
\eee

for a suitable $k$ with $1 \leq k \leq \phi(k)$. This number $k$ is then called the \emph{index of $a$ relative to $r$} and denoted as $\ind_r a$. We have $1 \leq \ind_r a \leq \phi(n)$ and

\bee
r^{\ind_r a} \equiv a \mod n
\eee

Another name for the index is \emph{discrete logarithm}.

Let's calculate a discrete logarithm $\ind_5 2$ with $n=7$ ($5$ is a primitive root of $7$); i.e. we seek $k$ so that $5^k \equiv 2 \mod 7$. In Gap, we can calculate it as follows,

\bee
  \ind_r a = \verb+ LogMod(a, r, n) +
\eee

For example, we have

\begin{verbatim}
gap> LogMod( 2, 5, 7 );
4
gap> 5^4 mod 7;
2
\end{verbatim}

As a fun side-note, we observe that $7$ has another primitive root, namely $3$. We therefore can also solve e.g. $3^k \equiv 2 \mod 7$ via

\begin{verbatim}
gap> LogMod( 2, 3, 7 );
2
gap> 3^2 mod 7;
2
\end{verbatim}

\qed

The discrete logarithm is a "one-way" function in that it is expensive to compute; the inverse (modular exponentation) is a (comparably) cheap operation.

If $r$ is clear from the context, we can leave it away; i.e. we then have

\bee
r^{\ind a} \equiv a \mod n
\eee

For some other values of the discrete logarithm we obtain

\begin{verbatim}
gap> LogMod(1, 5, 7);
0
gap> LogMod(2, 5, 7);
4
gap> LogMod(3, 5, 7);
5
gap> LogMod(4, 5, 7);
2
gap> LogMod(5, 5, 7);
1
gap> LogMod(6, 5, 7);
3
gap> LogMod(8, 5, 7);
0
gap> LogMod(9, 5, 7);
4
gap> LogMod(10, 5, 7);
5
\end{verbatim}

From above list  we see that indices of integers that are congruent modulo $n$ are equal. For $a \equiv b \mod p$ to hold, we can use $r^{\ind a} \equiv a \mod n$ and $r^{\ind b} \equiv b \mod n$ and therefore $r^{\ind a} \equiv r^{\ind b} \mod p$ from which follows $\ind a = \ind b$.


The discrete logarithm fulfills some properties which are related to those of "normal" logarthms.

\begin{theorem}
  If $n$ has a primitive root $r$ and $\ind a $ denotes the index of $a$ relative to $r$, then the following properties hold.

  \begin{enumerate}
    \item $\ind ab \equiv \ind a + \ind b \mod \phi(n)$
    \item $\ind a^k \equiv k \ind a \mod \phi(n)$
    \item $\ind 1 \equiv 0 \mod \phi(n)$, $\ind r \equiv 1 \mod \phi(n)$
  \end{enumerate}

\end{theorem}

Let's check this in Gap. For part $1$ we have

\begin{verbatim}
gap> a:=LogMod(2, 5, 7);
4
gap> b:=LogMod(3, 5, 7);
5
gap> LogMod(6, 5, 7);
3
gap> (4+5) mod(Phi(7));
3
\end{verbatim}

For part $2$ we have

\begin{verbatim}
gap> LogMod(2^4, 5, 7);
4
gap> 4*LogMod(2, 5, 7);
16
gap> 16 mod(Phi(7));
4
\end{verbatim}

For part 3 we have

\begin{verbatim}
gap> LogMod(1,5,7);
0
gap> LogMod(5,5,7);
1
\end{verbatim}

Let's prove the statements: We have $r^{\ind a} \equiv a \mod n$ and $r^{\ind b} \equiv b \mod n$ by the index definition. If we multiply things together, we obtain

\bee
r^{\ind a + \ind b} \equiv ab \mod n
\eee

We also have $r^{\ind ab} \equiv ab \mod n$ and if we insert this into the RHS, we obtain

\bee
r^{\ind a + \ind b} \equiv r^{\ind ab} \mod n
\eee


Theorem \ref{2021-03-02:th1} states the following: If $a$ has order $k \mod n$, then $a^i \equiv a^j \mod n$ ifff $i \equiv j \mod k$. In our case, the order of $r \mod n$ is $\phi(n)$ - that's the definition of a primitive root $r$. Therefore, we know that the last equation holds iff the exponents are congruent modulo-$\phi(n)$; i.e.

\bee
\ind a + \ind b \equiv \ind ab \mod(\phi(n))
\eee

In a similar spirit, we have $r^{\ind a^k} \equiv a^k \mod n$ and $r^{k \ind a} = (r^{\ind a})^k \equiv a^k \mod n$. Combining this, we obtain

\bee
r^{\ind a^k} \equiv r^{k \ind a}
\eee

As above, using theorem \ref{2021-03-02-th1} yields

\bee
\ind a^k \equiv k \ind a \mod(phi(n)) \qed
\eee

Indices can be used to solve certain types of congruences, namely $x^k \equiv a \mod n$. This is covered in entry \ref{2023-01-18:entry}.

%%% Local Variables:
%%% mode: latex
%%% TeX-master: "journal"
%%% End:
