\DiaryEntry{Primitive Roots and Indices}{2021-03-02}{Number Theory}

The order of an integer $a$ modulo-$n$ (with $n>1$ and $\gcd(a,n)=1$) is the smallest positive integer $k$ such that $a^k \equiv 1 \mod n$.

As an example, consider $a = 2, n = 7$ and write down successive powers of $2$ modulo-$7$. We have

\bee
2^1 \equiv 1 \mod 7, 2^2 \equiv 4 \mod 7, 2^3 \equiv 1 \mod 7, 2^4 \equiv 2 \mod 7, \ldots
\eee

From this we can see that the order of $2$ modulo-$7$ is $k=3$.

We can bound the order $k$ from above by using Euclid's theorem $a^{\phi(n)} \equiv 1 \mod n$ as $k \leq \phi(n)$. The order can be smaller than $\phi(n)$ but this need not be the case.

If two integers $a, b$ are congruent modulo-$n$, then they have the same order modulo-$n$. For if $a \equiv b \mod n$ and $a^k \equiv \mod n$, we have $a^k \equiv b^k \mod n$ (see entry \ref{2020-11-17:entry}) and therefore $b^k \equiv 1 \mod n$.

Note that the condition $\gcd(a,n)=1$ is a crucial one; if this condition is violated, then $a^k \equiv 1 \mod n$ would have no solution and the order is not defined. Therefore, we assume $\gcd(a,n)=1$ in the following.

In the previous example, we had $2^k \equiv 1 \mod 7$ whenever $k$ was a multiple of $3$, the order of $2$ modulo-$7$. This is no coincidence, as the following theorem shows.

\begin{theorem}
  Let the integer $a$ have order $k$ modulo-$n$. Then $a^h \equiv 1 \mod n$ iff $k | h$; in particular, $k | \phi(n)$.
\end{theorem}

To prove, we start with $h | k$ which implies $h = jk$ for some integer $j$. Because $a^k \equiv 1 \mod n$, we have

\bee
a^h \equiv a^{jk} \equiv \left( a^j \right)^k \equiv 1^k \equiv 1 \mod n
\eee

The other way round starts with an integer $h$ satisfying $a^h \equiv 1 \mod n$. By the division algorithm, there exist integers $q$ and $r$ such that $h = qk+r$ where $0 \leq r < k$. Therefore, we have

\bee
a^h \equiv a^{qk+r} \equiv \left(a^k\right)^q a^r \mod n
\eee

Our hypothesis states that $a^h \equiv a^k \equiv 1 \mod n$ and therefore $a^r \equiv 1 \mod n$. Because $0 \leq r < k$, we arrive at $r=0$. Therefore, $h = qk$ and $k | h$. \qed

With Gap, we can calculate the order as follows.

\begin{verbatim}
gap> OrderMod(2, 7);
3
\end{verbatim}


\begin{theorem}\label{2021-03-02-th1}
  If the integer $a$ has order $k$ modulo-$n$, then $a^i \equiv a^j \mod n$ iff $i \equiv j \mod n$.
\end{theorem}

First, suppose $a^i \equiv a^j \mod n$ with $i \geq j$. Because $a$ is relatively prime to $n$, we may cancel a power of $a$ and obtain $a^{i-j} \equiv 1 \mod n$. According to the previous theorem, this holds iff $k | i-j$ which is equivalent to $i \equiv j \mod k$ (definition of congruence). 

Conversely, let $i \equiv j \mod k$. Tnen we have $i = j+  qk$ for some integer $q$. By the definition of $k$, we have $a^k \equiv 1 \mod n$ so that

\bee
a^i \equiv a^{j+qk} \equiv a^j \left( a^k \right)^q \equiv a^j \mod n \qed
\eee

As a corollary, if $a$ has order $k$ modulo-$n$, then the integers $a, a^2, \ldots, a^k$ are incongruent modulo-$n$.

If $a^i \equiv a^j \mod n$, for $1 \leq i \leq j \leq k$, then the previous theorem ensures that $i \equiv j \mod n$. But this is impossible unless $i=j$. \qed

In our running example with $a=2, n=7$, $2$ has order $3$. Therefore, $2, 2^2, 2^3$ are incongruent modulo-$7$.

We next tackle the question whether we can express the order of any integral power of $a$ in terms of the order of $a$? This is answered in the following theorem.

\begin{theorem}
  If the integer $a$ has order $k$ modulo-$n$ and $h > 0$, then $a^h$ has order $k | \gcd(h,k)$ modulo-$n$.
\end{theorem}

Proof is omitted.

As a corollary, we note that if $a$ has order $k$ modulo-$n$, then $a^h$ has also order $k$ iff $\gcd(h,k)=1$.

As an example, consider $n=13$ and obtain the orders of several numbers in a Gap session.

\begin{verbatim}
gap> OrderMod(2, 13);
12
gap> OrderMod(2^2, 13);
6
gap> OrderMod(2^3, 13);
4
\end{verbatim}

We have

\bee
\frac{12}{\gcd(2,12)} = 6, \quad \frac{12}{\gcd(3, 12)} = 4
\eee

which matches the result of the theorem. The order of $2$ equals $12$ modulo-$13$; since $\gcd(5, 12) = 1$, the order of $2^5 \equiv 6$ is also equal to $12$ which Gap shows to be correct

\begin{verbatim}
gap> OrderMod(6, 13);
12
\end{verbatim}

If an integer $a$ has the largest order possible, then we call it a primitive root of $n$. More exactely, if $\gcd(a,n)=1$ and $a$ is of order $\phi(n)$ modulo-$n$, then a is a \emph{primitive root} of the integer $n$.

In other words, $n$ has $a$ as primitive root if $a^{\phi(n)} \equiv 1 \mod n$ but $a^k \not \equiv 1 \mod n$ for all integers $k < \phi(n)$.

Continue with $n=13$, we have $\phi(13) = 12$ and the primitive roots are $2, 6, 7, 11$ because they have order $12$. We can do this in Gap via the \verb+PrimitiveRootMod+ function (which takes an optional parameter from which value to start searching for the next primitive root) as follows:

\begin{verbatim}
gap> Phi(13);
12
gap> PrimitiveRootMod(13);
2
gap> OrderMod(2,13);
12
gap> PrimitiveRootMod(13,2);
6
gap> OrderMod(6,13);
12
gap> PrimitiveRootMod(13,6);
7
gap> OrderMod(7,13);
12
gap> PrimitiveRootMod(13,7);
11
gap> OrderMod(11,13);
12
\end{verbatim}

The whole topic is closely related to Group theory; see e.g. \ref{2017-05-05:entry}.

We have some more theorems.

\begin{theorem}
  Let $\gcd(a,n)=1$ and let $a_1, a_2, \ldots, a_{\phi(n)}$ be the positive integers less than $n$ and relatively prime to $n$. If $a$ is a primitive root of $n$, then

  \bee
  a, a^2, \ldots, a^{\phi(n)}
  \eee

  are congruent modulo $n$ to $a_1, a_2, \ldots, a_{\phi(n)}$.
\end{theorem}

Because $a$ is relatively prime to $n$, the same holds for all the powers of $a$. Therefore, each $a^k$ is congruent modulo $n$ to one of the $a_i$. The $\phi(n)$ numbers in the set $a, a^2, \ldots, a^{\phi(n)}$ are incongruent by theorem \ref{2021-03-02-th1} and thus, these powers must represent (in different order) the integers $a_1, a_2, \ldots, a_{\phi(n)}$. \qed

If a primitive root exists, we can state how many there are using the following theorem.

\begin{theorem}
  If $n$ has a primitive root , then it has $\phi(\phi(n))$ of them.
\end{theorem}

In our running example of $n=13$, we have $\phi(\phi(13)) = 4$ primitive roots.

If $a$ is a primitive root of $n$, then any other primitive root of $n$ is found among the members of the set $a, a^2, a^{\phi(n)}$. But the number of powers $a^k, 1 \leq k \leq \phi(n)$, that have order $\phi(n)$ is equal to the number of integers $k$ for which $\gcd(k, \phi(n)) = 1$; there are $\phi(\phi(n))$ such integers and $\phi(\phi(n))$ primitive roots of $n$. \qed




%%% Local Variables:
%%% mode: latex
%%% TeX-master: "journal"
%%% End:
