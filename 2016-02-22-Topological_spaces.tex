\DiaryEntry{Topological Spaces}{2016-02-22}{Topology}

A topological space is a set, \(\mathcal{X}\), together with a
collection, \(\mathcal{T}\), of subsets of \(\mathcal{X}\). These
subsets are called ``open'' sets and satisfy the following rules:

\begin{itemize}
\item
  The set \(\mathcal{X}\) is ``open''
\item
  The empty set \(\mathcal{O}\) is open
\item
  Arbitrary unions of sets are ``open''
\item
  Finite intersections of ``open'' sets are ``open''
\end{itemize}

This collection \(\mathcal{T}\) is called the topology on
\(\mathcal{X}\). Note that here, the term ``open'' is a definition - any
set can be defined to be open. The 4 conditions are modeled similar to
the definitions of open sets of the real line \(\mathcal{R}\); the idea
is that the conditions ensure that open sets in a topological space
behave similar to open sets in \(\mathcal{T}\).

Therefore, the real line with the ``standard'' definition of openness is
a topological space.

\subsubsection{Example (indiscrete topology)}

We have the set \(\mathcal{B} = \{0,1\}\) and we can define a
topological space as follows: We define the empty set \(\mathcal{O}\)
and the set \(\mathcal{B}\) as open. This satisfies the first two axioms
from above. Because \(\{0,1\} \cup \mathcal{O} = \{0,1\}\) we have
satisifed the third axiom, and since
\(\{0,1\} \cap \mathcal{O} = \mathcal{O}\) which is open again, we have
also satisfied the fourth axiom. This topology is called an
\emph{indiscrete} topolgy.

\subsubsection{Example (discrete topology)}

However, we can also define a different topological space by defining
the sets \(\mathcal{O}, \{0\}, \{1\}, \{0,1\}\) as open. The empty set
and whole set are included, so axioms 1 and 2 are satisfied. Unions and
intersections always yield a set defined to be open; therefore, axioms 3
and 4 are also satisfied. This topology is called a \emph{discrete}
topolgy.

A subset \(\mathcal{X}\) is closed, when the complement
\(\mathcal{T} - \mathcal{X}\) is open.

In the indiscrete topology example above, both \(\mathcal{O}\) and
\(\{0,1\}\) are closed - that is, sets can be open and closed at the
same time! Since neither \(\{0\}\) and \(\{1\}\) are defined as open,
they are also not closed - sets can also be neither closed nor open at
the same time!

\subsection{Definition of Continuity}

A function (map) \(f: \mathcal{S} \rightarrow \mathcal{T}\) between two
topological spaces is continuous, if the preimage
\(f^{-1}(\mathcal{Q})\) of every open set
\(\mathcal{Q} \in \mathcal{T}\) is an open set in \(\mathcal{S}\).

This generalizes the definition of continuity for functions from
\(\mathcal{R}\) to \(\mathcal{R}\).

\subsubsection{Example}

Considering the example with the discrete topology from above. Define
the function \(f\) as \(f(0)=-1\) and \(f(1)=1\). Let \(\mathcal{U}\) be
any open set in \(\mathcal{R}\). The preimage of \(\mathcal{U}\) is then
given as

\[
f^{-1}(\mathcal{U}) = \begin{cases}
{0} & \mbox{if }  -1 \in \mathcal{U} \mbox{ and } 1 \notin \mathcal{U} \\
{1} & \mbox{if }  -1 \notin \mathcal{U} \mbox{ and } 1 \in \mathcal{U} \\
{0,1} & \mbox{if }  -1 \in \mathcal{U} \mbox{ and } 1 \in \mathcal{U} \\
\mathcal{O} & \mbox{if }  -1 \notin \mathcal{U} \mbox{ and } 1 \notin \mathcal{U} \\
\end{cases}
\]

In all cases, the preimage of \(\mathcal{U}\) is open, since every
subset of \(\mathcal{B}\) is open in the discrete topology. Therefore,
the function \(f\) is continuous.
