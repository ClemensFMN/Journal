\DiaryEntry{Linear Congruences and the Chinese Remainder Theorem }{2020-11-25}{Number Theory}

\subsection{Linear Congruences}

We consider the \emph{linear congruence},

\bee
ax \equiv b \mod n
\eee

and seek solution(s) $x_0$ so that $ax_0 \equiv b \mod n$. 

A simple example is given by

\bee
4x \equiv 8 \mod 6
\eee

By trying out values (I also used \href{https://www.a-calculator.com/congruence/}{this}), we obtain two solutions $x_0 = 2$ and $x_0 = 5$. Note, however, that there are infinitely many solutions given by

\bee
x = 2 + 3k, \quad k \in \Zc_0
\eee

The solutions $2$ and $5$ are called \emph{mutually incongruent} because they yield a different remainder when divided by $n = 3$. Adding $n$ to these two solutions yields the series $2, 8, 14, \ldots$ and $5, 11, 17, \ldots$; elements of each series all yield the same remainder and are "equal".

\begin{theorem}
    The linear congruence $ax \equiv b \mod n$ has a solution iff $d = \gcd(a, n) \mid b$. If $d \mid b$, then it has $d$ mutually incongruent solutions module $n$. If $x_0$ is such a solution, then the $d$ incongruent solutions are given by

    \bee
        x_0, x_0 + \frac{n}{d}, x_0 + 2\frac{n}{d}, \ldots, x_0 + (d-1)\frac{n}{d}
    \eee
\end{theorem}

We do not give the proof here, but note that the linear congruence is equivalent to the linear Diophantine equation $ax - ny = b$: The congruence $ax \equiv b \mod n$ is equivalent to $n \mid (ax_0 - b)$ which we can also write as $a x_0 - b = n y_0$ for some integer $y_0$. Thus solving a linear congruence is the same as solving the linear Diophantine equation $a x - b = n y$. We have treated the linear Diophantine equation in \ref{2020-08-11:entry}.

Checking our example, we have $d = \gcd(4, 6) = 2$ and therefore $2$ solutions. If we know that one of them is $x_0 = 2$, we obtain the two incongruent solutions as $2, 2 + \frac{6}{2} = 2 + 3 = 5$. \qed

\paragraph{Multiplicative Inverse module $n$.} As a corollary, we have.

\begin{theorem}
    If $\gcd(a,n) = 1$, then $a x \equiv b \mod n$ has a unique solution modulo $n$. In particular, $a x \equiv 1 \mod n$ has a unique solution called the \emph{(multiplicative) inverse} of $a$ modulo $n$. Note that the condition $\gcd(a,n) = 1$ is trivially fulfilled when $n$ is prime.
\end{theorem}

We can calculate the multiplicative inverse module $n$ via a slight variation of the extended Euclidean algorithm. From above we know that $ax \equiv 1 \mod n$ is equivalent to the linear Diophantine equation $ax - ny = 1$. The extended Euclidean algorithm (entries \ref{2020-08-11:entry} and \ref{2020-08-20:entry}) solves $ar + bs = \gcd(a,b)$. Choosing $a = a, b = -n$, $\gcd(a,b) = \gcd(a, n) = 1$ is fulfilled and the algorithm yields a solution for $r,s$. We are only interested in $r$ and there is a slightly more efficient algorithm available (more efficient as it only calculates $r$) shown \href{https://en.wikipedia.org/wiki/Extended_Euclidean_algorithm}{here}. In Julia, the implementation looks as follows (note the close resemblence to the Julia code in entry \ref{2020-08-20:entry})

\begin{verbatim}
    function inverse_mod(a, n)
        t = 0
        newt = 1
        r = n
        newr = a

        while(newr != 0)
            quotient = div(r, newr)
            (t, newt) = (newt, t - quotient * newt) 
            (r, newr) = (newr, r - quotient * newr)
        end

        if(r > 1)
            error("a is not invertible")
        end

        if(t < 0)
            t = t + n
            return t
        end
    end
\end{verbatim}

As an example, consider $4x \equiv 1 \mod 17$. Calling above function yields $x = 13$ and indeed we have $4 \cdot 13 \equiv 1 \mod 17$.





%%% Local Variables:
%%% mode: latex
%%% TeX-master: "journal"
%%% End:
