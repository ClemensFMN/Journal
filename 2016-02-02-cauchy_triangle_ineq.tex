\DiaryEntry{Cauchy Inequality}{2016-02-02}{Maths}

Any norm \(d(x,y)\) must fulfill four properties:

\begin{itemize}
\item
  It must be positive for any \(x,y\): \(d(x,y) \leq 0\).
\item
  It must be only zero if and only if \(x = y\)
\item
  It must be commutative; i.e. \(d(x,y) = d(y,x)\)
\item
  It must fulfill the triangle inequality:
  \(d(x,y) \leq d(x,z) + d(z, y)\)
\end{itemize}

One norm function is the Euclidian norm; for a d-dimensional space it
has the following form:

\[
d(x,y) = \sqrt{(y_1 - x_1)^2 + \cdots + (y_d - x_d)^2}
\]

The first three norm properties are quite straightforward; the triangle
inequality can be shown using the Cauchy inequality as follows.

Without loss of generality, we can set \(x = 0\), \(y = u+v\), and
\(z = v\) in the triangle inequality. Note that the Euclidian norm is
translation invariant; i.e. \(d(x+w, y+w) = d(x,y)\) for all \(w\). This
allows to express the triangle inequality as follows:

\[
d(0, u+v) \leq d(0, v) + d(v, u+v) = d(0, v) + d(0, u)
\]

where the last equality follows from translation invariance. Inserting
the definition of the norm into the last expression, we obtain:

\[
\sqrt{\sum(u_j + v_j)^2} \leq \sqrt{\sum u_j^2} + \sqrt{\sum v_j^2}
\]

Squaring both sides, we obtain

\[
\sum(u_j + v_j)^2 = \sum u_j^2 + 2 \sum u_j v_j + \sum v_j^2 \leq \sum u_j^2 + 2 \sqrt{\sum u_j^2} \sqrt{\sum v_j^2} + \sum v_j^2
\]

and cancelling terms on both sides yields

\[
\sum u_j v_j  \leq \sqrt{\sum u_j^2} \sqrt{\sum v_j^2}
\]

which is exactely the Cauchy inequality.
