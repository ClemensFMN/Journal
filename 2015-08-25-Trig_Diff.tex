\DiaryEntry{Differentiate Trigonometric Functions}{2015-08-25}{Maths}


\subsection{Prerequisites}

\subsubsection{Expansion}

We want to expand $\sin(x+y)$ and $\cos(x+y$. Use complex functions as follows:

\[\cos(x+y) + j \sin(x+y) = e^{j(x+y)} = e^{jx} e^{jy} = \left( \cos x + j\sin x \right) \left( \cos y + j\sin y \right) \]

Expanding the last expression yields

\[ \cos x \cos y - \sin x \sin y + j \left( \cos x \sin y + \sin x \cos y \right) \]

and by comparing real and imaginary parts we obtain

\[\cos(x+y) = \cos x \cos y - \sin x \sin y \]

and

\[ \sin(x+y) = \cos x \sin y + \sin x \cos y \]

\subsubsection{Differentiate Inverse (Trigonometric) Functions}

The trick is to treat the derivative $\frac{dy}{dx}$ like a fraction; e.g.~if $x$ is the independent variable $y=g(x)$ and $\frac{dy}{dx} = g'(x)$, we can obtain the derivative of the inverse function (where $y$ is the independent variable) as $\frac{dx}{dy} =1 /g(x)$. This expression depends on $x$, though; by substituting $x=g^{-1}(y)$, we obtain the derivative $\frac{dx}{dy}$ depending
on $y$. Writing these steps into one, we get

\[ \left( g^{-1}(y) \right)' = \frac{1}{g'(g^{-1}(y))}\]

\subsection{Basic Trigonometric Functions}

We want to calculate $\frac{d \cos x}{dx}$. Use the definition

\[\frac{d \cos x}{dx} = \lim_{\delta \rightarrow 0} \frac{\cos(x+\delta) - \cos x}{\delta}\]

We can expand the expression in the limit as follows

\[\frac{\cos(x+\delta) - \cos x}{\delta} = \frac{\cos x \cos \delta - \sin x \sin \delta - \cos x}{\delta}\]

To obtain the limit, we use the trigonometric series expansions for dmall x $\cos x = 1 - x +
\mathcal{O}(2)$ and $\sin x = x + \mathcal{O}(2)$,

\[\frac{\cos(x+\delta) - \cos x}{\delta} = \frac{ (1 - \delta) \cos x - \delta \sin x - \cos x}{\delta}\]

Now we can calculate the limit

\[\lim_{\delta \rightarrow 0} \frac{ (1 - \delta) \cos x - \delta \sin x - \cos x}{\delta} = - \sin x\]

and therefore we have $\frac{d \cos x}{dx} = -\sin x$. The same approach works for $\frac{d \sin x}{dx} = \cos x$.

A bit more interesting is $\frac{d \tan x}{dx}$ which can be expressed as

\[ \frac{d}{dx} \frac{\sin x}{\cos x} = \frac{\cos^2 x + \sin^2 x}{\cos^2 x} = 1 + \tan^2 x\]

\subsubsection{arctan}

Consider $\frac{d \arctan x}{dx}$ which we can differentiate as follows: We have $y=\tan x$ and know that $\frac{dy}{dx}= 1 + \tan^2 x$ (see Section above). Using the trick from above, we obtain

\[ \frac{dx}{dy} = \frac{1}{1 + \tan^2 x} \]

But we are interested in calculating $\frac{dx}{dy}$  as a function of $y$, so we have to substitute $
x=\arctan y$

\[ \frac{dx}{dy} = \frac{1}{1 + \tan^2 (\arctan y)} = \frac{1}{1 + y^2} \]

which is the desired result.

\subsubsection{arcsin}

The same trick works with $\frac{d \arcsin x}{dx}$. We have $y=\arcsin x$ and $x = sin y =
\sqrt{1 - \cos^2 y}$. Basic manipulations yield $\cos y = \sqrt{1 - x^2}$. Then $\frac{dx}{dy} = \cos y$  and we obtain

\[ \frac{dy}{dx} = \frac{1}{\cos y} = \frac{1}{\sqrt{1 - x^2}} \]
