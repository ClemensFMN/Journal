\section{Probability of a Triangle in a Random Graph, 2016-11-30}
\label{2016-11-30:entry}

Based on this
\href{http://math.stackexchange.com/questions/2023681/probability-of-a-random-graph-being-triangle-free}{link}.

\subsection{N = 3}\label{n-3}

Consider a graph with 3 vertices: 1,2,3. Every vertex can potentially be
connected with another vertex, so in total there are three possibilities
for edges: 1-2, 1-3, 2-3. We can now write down all graphs with 3
vertices; a zero denotes no edge, a one denotes an edge:

\[
\begin{array}{ccc}
1-2 & 1-3 & 1-3 \\
\hline
0 & 0 & 0 \\
0 & 0 & 1 \\
0 & 1 & 0 \\
... & ... & ... \\
1 & 1 & 1
\end{array}
\]

In total there are \(2^3 = 8\) different graphs. Only one of them
contains (is) a triangle; namely the last graph.

We consider random graphs having an edge between two vertices with
probability \(p\). So the probability for a triangle is \(p^3\) and the
probability for a graph with \textbf{no} triangle is \(1-p^3\). For
\(p=1/2\), the probability of no triangle is therefore \(7/8\).

\subsection{General Case - Bound}\label{general-case---bound}

Next consider larger graphs; i.e.~ones with more than 3 vertices. Let
\(N\) denote the number of vertices. We have \emph{at least} \((N-2)/3\)
pairwise disjoint sets of vertices with size 3 each. E.g. for \(N = 6\),
there is \(\{1,2,3\},\{4,5,6\}\), for \(N = 10\), there is
\(\{1,2,3\},\{4,5,6\},\{7,8,9\},\{10\}\).

Each of these sets fails to form a triangle with probability \(1-p^3\).
Since we split all vertices into disjoint subsets, the probabilities for
failing to form a triangle are independent and can be multiplied. So the
probability \(P\) that none of these sets form a triangle can be bounded
as follows

\[
P < (1-p^3)^{(N-2)/3}
\]

The bound follows because the number of disjoint subsets \((N-2)/3\) is
a lower bound. It can be seen that \(P \rightarrow 0\) for
\(N \rightarrow \infty\); i.e.~a large enough graph will always contain
(at least) a triangle for \(p>0\).

\subsection{Expected Number of
Triangles}\label{expected-number-of-triangles}

From
\href{http://math.stackexchange.com/questions/730294/expected-number-of-triangles-in-a-random-graph-of-size-n?rq=1}{here}.

There are \({N \choose 3}\) ways to select 3 nodes out of N and define
\(X_j\) as indicator that the j-th set defines a triangle. We want to
calculate \(\mathbf{E} \sum_{j=1}^t X_j\) with the expectation taken
over all 8 possible graphs with 3 vertices:

\[
\mathbf{E} \sum_{j=1}^t X_j = \sum_{j=1}^t \mathbf{E} X_j = t \mathbf{E}X_1 = t p^3
\]
