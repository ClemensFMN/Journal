\DiaryEntry{Public Key Cryptography and RSA}{2017-08-30}{Crypto}

\emph{strongly based on Christof Paar and Jan Pelzl, Understanding Cryptography}

\subsection{Symmetric vs Asymmetric Cryptography}

\paragraph{Symmetric Cryptography:} There is a secret key $k$ used both for encryption and decryption. The sender Alice encrypts a message $x$ using $k$ to obtain an encrypted message $y$ which is transported over a (potentially insecure) channel. The receiver Bob uses $k$ to decrypt $y$ and obtain the secret message $x$. Typically, encryption and decryption messages are very similar.

Problems are that every communication pair (i.e. sender - receiver) requires a different key and these keys need to be stored / distibuted between the partners across the network. This is time-consuming and difficult to achieve. When the key leaks, the communication is no longer secure!

\paragraph{Asymmetric Cryptography:} Here the key consists of two parts, a public key $k_{\text{pub}}$ and a private part $k_{\text{pr}}$. Every person in the network generates a public / private key pair and publishes the public part on the network. The encryption operation requires the public part of the recipient's key; however, the decryption requires the private part of the recipient's key. All asymmetric cryptography schemes are based on so called one-way functions $f()$, where calculating $y = f(x)$ is computationally easy but the reverse operation $x = f^{-1}(y)$ is computationally hard / infeasible. In particular, it must be impossible for an attacker to obtain a private key part from the public part.

RSA is based on the fact that multiplication of two large primes $p, q$ is an easy problem whereas factorization of the product into the two primes is computationally intensive (depending on the size of the numbers involved, computationally infeasible).

Another scheme is based on the discrete logarithm problem, used in the Diffie-Hellman key exchange, Elgamal encryption or the Digital Signature Algorithm (DSA).

Elliptic Curve (EC) Schemes are a generalization of the discrete logarithm algorithm. The most popular examaples include Elliptic Curve Diffie–Hellman key exchange (ECDH) and the Elliptic Curve Digital Signature Algorithm (ECDSA).


\subsection{RSA}


RSA stands Rivest–Shamir–Adleman algorithm and is currently the most widely used asymmetric cryptographic scheme.

RSA is several times slower than symmetric ciphers such as AES. This is because of the many computations involved in performing RSA (or any other public-key algorithm). Therefore, there are two main applications:

\begin{enumerate}
	\item Encryption of small pieces of data, especially for key transport. In practice, RSA is often used together with a symmetric cipher such as AES, where the symmetric (and faster) AES cipher does the actual bulk data encryption.

	\item digital signatures
\end{enumerate}


\subsubsection{Principle of Operation}

RSA encrypts a plaintext $x$. This plaintext is converted into a number which must be an element of $\mZ_n = \{0,1,\ldots,n-1\}$. With slight abuse of notation, we use the same symbol $x$ for both the plaintext and this number.

\paragraph{Encryption.} Given the public key $k_{\text{pub}} = (n,e)$ and the message $x$, we obtain the encrypted message $y$ according to

\bee
y = e_{\text{pub}}(x) = x^e \bmod n
\eee

with $x,y \in \mZ_n$. The $\bmod n$ operation explains the restriction $x \in \mZ_n$.

\paragraph{Decryption.} Given the private key $k_{\text{pr}} = d$ and the encrypted message $y$, we recover the message as

\bee
x = d_{\text{pr}}(y) = y^d \bmod n
\eee

In practice $x, y, n, d$ are very long numbers, usually $1024$ bit long or more. The value $e$ is sometimes referred to as encryption exponent or public exponent, and the private key $d$ is sometimes called decryption exponent or private exponent. If Alice wants to send an encrypted message to Bob, Alice needs to have his public key $(n, e)$, and Bob decrypts with his private key $d$. Key generation is discussed in the following.

Some general observations

\begin{itemize}

\item Since an attacker has access to the public key, it must be computationally infeasible to determine the private-key $d$ given the public-key values $e$ and $n$. 

\item Since $x$ is only unique up to the size of the modulus $n$, we cannot encrypt more than $l$ bits with one RSA encryption, where $l$ is the bit length of $n$.

\item It should be relatively easy to calculate $x^ e \bmod n$, i.e., to encrypt, and $y^d \bmod n$, i.e., to decrypt. This means we need a method for fast exponentiation with very long numbers.

\item For a given $n$, there should be many private-key/public-key pairs, otherwise an attacker might be able to perform a brute-force attack.

\end{itemize}


\subsubsection{Key Generation}

Goal of the key generation is to create both private and public key parts. The process is as follows

\begin{itemize}
	\item Choose two large primes $p,q$.

	\item Compute $n = pq$.

	\item Compute $\phi(n) =(p-1)(q-1)$.

	\item Select the public exponent $e \in \{1,2,\ldots,\phi(n)-1\}$ such that $\gcd(e,\phi(n))=1$.

	\item Compute the private key $d$ such that $de \equiv 1 \bmod \phi(n)$ (i.e. $d$ is the inverse of $e$). Above selection criteria for $e$ ensures that this inverse exists.

\end{itemize}

\subsubsection{Proof}

What we want to show is that the decryption operation yields the original message $x$,

\be\label{pub_key_crypt_rsa_1:eq}
d_{\text{pr}}(y) = d_{\text{pr}} \left( e_{\text{pub}}(x) \right) \equiv (x^e)^d = x^{de} = x \bmod n
\ee

From the key generation procedure, we have the fact that $de \equiv 1 \bmod \phi(n)$ which we can rewrite as

\bee
de = 1 + t \phi(n)
\eee

with $t$ being some integer. Inserting this into \eqref{pub_key_crypt_rsa_1:eq}, we obtain

\be\label{pub_key_crypt_rsa_2:eq}
d_{\text{pr}}(y) \equiv x^{de} \equiv x^{1 + t \phi(n)} \equiv x \left( x^{\phi(n)} \right)^t \bmod n
\ee

and we need to prove that above expression equals $x \bmod n$. We can distinguish two cases

\paragraph{Case 1} $\gcd(x,n=1)$: We can use Euler's theorem (see \hyperref[2017-05-08:entry]{here}) which says that $x^{\phi(n)} \equiv 1 \bmod n$ in this case. Inserting into \eqref{pub_key_crypt_rsa_2:eq} yields

\bee
d_{\text{pr}}(y) \equiv x \left( x^{\phi(n)} \right)^t \bmod n = x \left( 1 \right)^t \bmod n = x \bmod n \qed
\eee

\paragraph{Case 2} $\gcd(x,n) = \gcd(x, pq) \neq 1$: Since $p,q$ are primes, $x$ must have one of them as factor, so either $x = rp$ or $x = sq$ with $r,s$ being integers. Let's assume (without loss of generality) that $x = rp$ from which follows $\gcd(x,q) = 1$. Euler's theorem then states that

\bee
x^{\phi(q)} \equiv 1 \bmod q
\eee

Let us take a closer look at $\left( x^{\phi(n)} \right)^t$:

\bee
\left( x^{\phi(n)} \right)^t \equiv \left( x^{(q-1)(p-1)} \right)^t \equiv \left( (x^{\phi(q)})^t \right)^{(p-1)} \bmod q
\eee

where we have used the fact that $\phi(q) = q-1$ if $q$ is being prime. Using Euler's theorem yields

\bee
\left( x^{\phi(n)} \right)^t \equiv \left( (x^{\phi(q)})^t \right)^{(p-1)} \equiv \left(1^t \right)^{(p-1)} \equiv 1 \bmod q
\eee

We can therefore write

\bee
\left( x^{\phi(n)} \right)^t = 1 + uq
\eee

with some integer $u$. Multiplying with $x$ yields

\bee
x \left( x^{\phi(n)} \right)^t = x + xuq = x + (rp)uq = x + rupq = x + run
\eee

and therefore we have

\bee
x \left( x^{\phi(n)} \right)^t \equiv x \bmod n
\eee

Inserting into \eqref{pub_key_crypt_rsa_2:eq} yields

\bee
d_{\text{pr}}(y) \equiv x \left( x^{\phi(n)} \right)^t \equiv x \bmod n \qed
\eee


\paragraph{Example.} Choose the two primes $p=3, q=5$, giving $n=15$ and $\phi(n) = 2 \cdot 4 = 8$. Now select an $e$ such that it is relatively prime to $8$; choose $e = 3$. Compute $d$ as inverse modulo-8 to $e$ which gives $d = 11$. Now everything is in place, the private key becomes $k_{\text{pr}} = d = 11$ and the public one becomes $k_{\text{pub}} = (15,3)$.

Continue with case 1 from above and choose a message $x=2$. We have $x^{\phi(n)} = 2^8 \equiv 1 \bmod 15$. Encryption yields

\bee
y = 2^3 \bmod 15 = 8
\eee

and decryption yields

\bee
x = 8^11 \bmod 15 = 2
\eee

Considering case 2 from above we choose a message $x = 3$ (which is a factor of $n=15$). We have the encrypted value

\bee
y = 3^3 \bmod 15 = 12
\eee

and decryption yields

\bee
x = 12^{11} \bmod 15 = 3
\eee