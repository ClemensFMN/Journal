\DiaryEntry{Quotient Rings}{2016-07-21}{Algebra}

Let A be a ring with an ideal J; then a coset of the ring is deined as
follows:

\[
J + a = \{j + a: j \in J \} \forall a \in A
\]

I.e. it is the set of all sums \(j+a\) as a remains fixed and \(j\)
ranges over all elements of the ideal. This is very similar to the coset
definition of groups, therefore most of the results can be reused. The
family of cosets \(J+a\) as \(a\) ranges over \(A\) is a partition of
the ring \(A\).

Adding and multiplying cosets works as follows:

\((J + a) + (J + b) = J + (a+b)\)

and

\((J + a) (J + b) = J + (ab)\)

That this works is due to the properties of ideals (in a similar manner
as in the case of cosets and goups).

If we think of the set of all cosets of \(J\) in \(A\) denoted by
\(A/J\). If \(a+J, b+J, c+J,\ldots\) are cosets, then

\[
A/J = \{a+J, b+J, c+J, \ldots \}
\]

and \(A/J\) with coset addition and multiplication as defined above is a
ring and \(A/J\) is a homomorphic image of \(A\).

Finally, we also have a fundamental homomorphism theorem for rings: If
\(f: A \rightarrow B\) is a homomorphism with kernel \(K\), then \(B\)
is isomorphic with \(A/K\).
