\DiaryEntry{Integration - Weierstrass Substitution}{2016-06-15}{Integrals}

The task is to solve integrals which contain rational expressions of the
trigonometric functions. This is based on
\href{http://planetmath.org/sites/default/files/texpdf/39380.pdf}{this}
document and
\href{https://en.wikipedia.org/wiki/Tangent_half-angle_substitution}{this
article}.

To this end, we make the substitution \(t = \tan x/2\) and observe the
following identities:

\[
\sin(x) = \frac{2t}{1+t^2}
\]

and

\[
\cos(x) = \frac{1-t^2}{1+t^2}
\]

We have \(x = 2 \arctan(t)\) and differentiating this with respect to t,
we obtain \(\frac{dx}{dt} = 2 \frac{1}{1+t^2}\) and finally
\(dx = \frac{2 dt}{1+t^2}\).

\subsubsection{Example 1}

Consider the integral

\[
I_1 = \int \frac{1+\sin(x)}{\cos(x)} dx = \int\frac{1+\frac{2t}{1+t^2}}{\frac{1-t^2}{1+t^2}} \frac{2}{1+t^2}dt
\]

which can be further simplified to

\[
I_1 = 2 \int \frac{(1+t)^2}{(1-t^2)(1+t^2)} dt = 2 \int \frac{t}{1+t^2} - \frac{1}{t-1} dt
\]

where the partial fraction expansion in the last expression was done via
SymPy. Making the substitution \(u=1+t^2\) for the first integral, we
obtain

\begin{align*}
I_1 = 2 \int\frac{t}{u} \frac{du}{2t} - \ln(t-1) & = \ln(u) - 2\ln(t-1) = \ln(1+t^2) - 2\ln(t-1) \\ &= \ln(1+t^2) - \ln(t-1)^2 = \ln\frac{1+t^2}{(t-1)^2} = \ln \frac{1+\tan^2 x/2}{(\tan x/2 - 1)^2}
\end{align*}

\subsection{Simplifications}

Above substitution works always, however the epxression become rather
complex. There are a couple of simpler substitutions which work in some
special cases:

\subsubsection{R(sin(x)) cos(x)}

In this case, \(t=\sin(x)\) is sufficient: \(\cos(x) = \sqrt{1-t^2}\)
and \(\frac{dt}{dx}=\cos(x) = \sqrt{1-t^2}\). But then we have

\[
I = \int R(\sin(x)) \cos(x) dx = \int R(t) \sqrt{1-t^2} \frac{dt}{\sqrt{1-t^2}}  = \int R(t) dt
\]

As an example consider \(\int \frac{1}{1+\sin(x)} \cos(x) dx\) which we
can transform into \(\int \frac{1}{1+t}dt = \ln(1+t) = \ln(1+\sin(x))\).

\subsubsection{R(cos(x)) sin(x)}

In a similar spirit, we substitute \(t=\cos(x)\):
\(\sin(x) = \sqrt{1 - t^2}\) and
\(\frac{dt}{dx} = -\sin(x) = -\sqrt{1-t^2}\). Then the integral becomes

\[
\int R(\cos(x)) \sin(x) dx = - \int R(t) dt
\]

\subsubsection{R(tan(x))}

In this case, the substitution \(t=\tan(x)\) yields \(x=\arctan(t)\) and
\(\frac{dx}{dt} = \frac{1}{1+t^2}\). Therefore we have

\[
\int R(\tan(x))dx = \int R(t) \frac{dt}{1+t^2}
\]

which again is a rational expression in \(t\).

\subsubsection{R(sin\^{}2 x, cos\^{}2 x)}

In this case use the substitution \(t=\tan(x)\) from which we obtain

\[
\cos^2 x = \frac{1}{1+\tan^2 x} = \frac{1}{1+t^2}, \quad, \sin^2 x = \frac{t^2}{1+t^2}, \quad, dx = \frac{dt}{1+t^2}
\]

this brings the expression \(R(\sin^2 x, \cos^2 x)\) into a rational
form in the variable \(t\) (or \(t^2\)).

\subsection{Examples}

\subsubsection{Example 2}

In order to solve \(I_2 = \int \frac{dx}{1+\cos x}\), we use the ``big''
substitution \(t=tan x/2\) and obtain

\[
I_2 = \int \frac{1}{1+\frac{1-t^2}{1+t^2}} \frac{dt}{1+t^2} = \int \frac{dt}{1+t^2+1-t^2} = \int \frac{dt}{2} = \frac{1}{2} \tan \frac{x}{2}
\]

\subsubsection{Example 3}

The integral

\[
I_3 = \int \frac{dx}{1+  \sin^2 x}
\]

becomes

\[
I_3 = \int \frac{1}{1 + \frac{t^2}{1+t^2}} \frac{dt}{1+t^2} = \int \frac{dt}{1+2t^2}
\]

We can substitute \(2t^2=u^2\) and obtain

\[
I_3 = \frac{1}{\sqrt{2}} \int \frac{du}{1+u^2} = \frac{1}{\sqrt{2}} \arctan(\sqrt{2}t) = \frac{1}{\sqrt{2}} \arctan \left( \sqrt{2} \tan x \right)
\]

Maybe the \(\arctan (\sqrt{2} \tan x)\) can be further
simplified\ldots{}
