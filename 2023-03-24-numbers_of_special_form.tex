\DiaryEntry{Numbers of Special Form}{2023-03-24}{Number Theory}

\subsection{Perfect Numbers}

A positive integer $n$ is said to be perfect if $n$ is equal to the sum of all its positive divisors, excluding $n$ itself.

Example is $n = 28 = 1 + 2 + 4 + 7 + 14$. The divisors are

\bee
1 | 28,\, 2 | 28,\, 4 | 28,\, 7 | 28,\, 14 | 28
\eee

and their sum is equal $n=28$. Some entries before we defined $\sigma(n)$ as the sum of positive divisors of $n$; since this includes $n$, $\sigma(n) - n$ is the sum of divisors of $n$ which are less than $n$. Thus, the condition that $n$ is perfect is equivalent to

\bee
\sigma(n) - n = n \rightarrow \sigma(n) = 2n
\eee

In our example of $n = 28$, we have $\sigma(28) = 56 = 2 \cdot 28$ and therefore $28$ is a perfect number.

OEIS sequence \href{A000386}{https://oeis.org/A000396} is the sequence of perfect numbers and the first few are

\bee
6, 28, 496, 8128, 33550336, 8589869056, 137438691328, 2305843008139952128
\eee

from which we can deduce that there are not many of them. It is not yet known whether there are finitely many or infinitely many of them.

We can use Gap to check for perfectness of a number. Today's computers seem to be quite powerful; below calculation was done almost instantly.

\begin{verbatim}
    gap> n := 2305843008139952128;
    2305843008139952128
    gap> Sigma(n) - 2*n;
    0
\end{verbatim}


The following theorem provides information about the format of perfect numbers.

\begin{theorem}
    If $2^k - 1$ with $k > 1$ is prime, then $n = 2^{k-1}(2^k-1)$ is perfect and every perfect number is of this form.
\end{theorem}

For $k=2$, $2^k-1 = 3$ is prime and this shows that $n=2^2*(2^3-1) = 28$ is a perfect number as stated above.

For the proof, we assume that $p=2^k-1$ is a prime and consider $n = 2^{k-1} p$. With $\gcd(2^{k-1},p) = 1$ and the multiplicativity of $\sigma(\cdot)$, we have

\bee
\sigma(n) = \sigma(2^{k-1} p) = \sigma(2^{k-1}) \sigma(p) = (2^{k-1} -1)(p+1) = (2^{k-1} -1) 2^k = 2n
\eee

showing that $n$ is a perfect number. For the converse, assume that $n$ is an even perfect number. We can write $n = 2^{k-1}m$ with $m$ being an odd integer and $k \geq 2$. We have $\gcd(2^{k-1}m) = 1$ and therefore

\bee
\sigma(n) = \sigma(2^{k-1}m) = \sigma(2^{k-1}) \sigma(m) = 2^{k-1} \sigma(m)
\eee

The requirement for $n$ to be a perfect number gives

\bee
\sigma(n) = 2n = 2^k m
\eee

Combining this, we obtain

\be\label{2023-03-24:eq1}
2^k m = 2^{k-1} \sigma(m)
\ee

which implies $2^{k-1} | 2^k m$. But $2^k$ and $2^{k-1}$ are relatively prime, therefore $2^{k-1} | m$ and we can write $m = 2^{k-1} M$. Inserting this for $m$ in \eqref{2023-03-24:eq1}, we obtain

\bee
2^k 2^{k-1} M = 2^{k-1} \sigma(m)
\eee

from which we obtain

\bee
\sigma(m) = 2^k M
\eee

Both $m$ and $M$ are divisors of $m$, so the sum of $m$'s divisors must be greater or eqal $m + M$, $\sigma(m) > m + M$. Therefore we have

\bee
2^k M = \sigma(m) \geq m + M = 2^{k-1} M + M = 2^k M
\eee

We have "sandwiched" $\sigma(m)$ and obtained that $\sigma(m) = m + M$. The implication is that $m$ has only two positive divisors, $M$ and $m$. It must therefore be that $m$ is prime and $M=1$. With $m = (2^k-1)M = 2^k-1$ is a prime. \qed

With this result, we turn the attention to primes of the form $2^k-1$ and we have the following theorem.

\begin{theorem}
    If $a^k-1$ is prime  ($a > 0, k \geq 2$), then $a=2$ and $k$ is also prime.
\end{theorem}

We can factor $a^k-1$ as follows,

\be\label{2023-03-24:eq2}
a^k - 1 = (a-1)(a^{k-1}+a^{k+2} + \cdots + a + 1)
\ee

and with the assumptions from above, we have

\bee
a^{k-1}+a^{k+2} + \cdots + a + 1 \geq a+1 > 1
\eee

We assumed that $a^k-1$ is prime, therefore the factor $a-1$ must be one and we have $a=2$. If $k$ were composite we could write $k=rs$ with $r, s > 1$, and we can factor this as (with $a^r$ playing the role of $a$ in \eqref{2023-03-24:eq2})

\bee
a^k-1 = (a^r)^s - 1 = ((a^r)-1)(a^{r(s-1)} + a^{r(s-2)} + \cdots + a^{r} + 1)
\eee

and with $a=2$ and $r, s > 1$ each factor is larger than $1$. But this violates the primality of $a^k-1$ and so by contradiction $k$ must be prime. \qed

Be careful, however, with above theorem as it works only "one-way": If $a^k-1$ is prime  then $k$ is prime. The other direction does not work. As an example, we have $k=11$ which is prime, but $2^{11}-1 = 2047 = 23 \cdot 89$.

For the next prime $p=13$, $2^p-1 = 8191$ is also prime. From this we can generate another perfect number as

\bee
n = 2^{k-1}(2^k-1) = 2^{12}(2^{13}-1) = 33550336
\eee

A question that immediately springs to mind is whether there are infinitely many primes of the type $2^p-1$, with $p$ a prime. If the answer were in the affirmative, then there would exist an infinitude of (even) perfect numbers. Unfortunately, this is another famous unresolved problem.

As a last point, we note (without proof) that even perfect numbers end with $6$ or $28$.

\subsection{Mersenne Primes and Amicable Numbers}

\emph{Mersenne numbers} are numbers of the form

\bee
M_n = 2^n - 1
\eee

and \emph{Mersenne primes} are those Mersenne numbers which are prime. As seen before, if $n$ is a composite number then so is $2^n - 1$. Therefore, an equivalent definition of the Mersenne primes is that they are the prime numbers of the form $M_p = 2^p - 1$ for some prime $p$ (but keep in mind that not all $2^p-1$ are prime!).

The book shows several methods whether certain special types of Mersenne numbers are prime or composite.

Two numbers are \emph{amicable}, if each number is equal to the sum of all the positive divisors of the other, not counting the number itself. For example, $220$ and $284$ are amicable. The positive divisors of $220$ are

\bee
1, 2, 4, 5, 10, 11, 20, 22, 44, 55, 110, 220
\eee

and their sum (without $220$) is $1+2+4+5+10+11+20+22+44+55+110 = 284$, whereas the divisors of $284$ are

\bee
1, 2, 4, 71, 142, 284
\eee

and their sum (without $284$) is $1+2+4+71+142 = 220$. In terms of the $\sigma$ function, amicability between the integers $m$ and $n$ can be written as

\bee
\sigma(m) - m = n \quad \sigma(n) - n = m
\eee

or - by inserting the first into the second equation - the following,

\bee
\sigma(n) - (sigma(m) - m)) = n \rightarrow \sigma(n) = \sigma(m)
\eee

The amicable numbers are OEIS sequence A259180, found \href{https://oeis.org/A259180}{here}. It has not yet been established whether the number of amicable pairs is finite or infinite, nor has a pair been produced in which the numbers are relatively prime. Another inaccessible question, already considered by Euler, is whether there are amicable pairs of opposite parity — that is, with one integer even and the other odd.

The first few pairs are

\bee
(220, 284), (1184, 1210), (2620, 2924), (5020, 5564), (6232, 6368), \ldots
\eee

\subsection{Fermat Number}

Finally, we consider the Fermat numbers $F_n$, which are of the form

\bee
F_n = (2^{2^n}) + 1, \quad n \geq 0
\eee

If $F_n$ is prime, it is called a \emph{Fermat prime}. They form sequence \href{https://oeis.org/A019434}{A019434} in the OEIS.

The first few are

\bee
3, 5, 17, 257, 65537, 4294967297, 18446744073709551617, \ldots
\eee



%%% Local Variables:
%%% mode: latex
%%% TeX-master: "journal"
%%% End:
