\DiaryEntry{Cyclic (finite) Groups}{2017-09-01}{Algebra}

This topic keeps coming up again and again and this entry will try to collect all information once and forever.


\subsection{Definitions}

The order of a group $G$ is denoted as $|G|$ and is the number of group elements. The order $|x|$ of a group element $x$ is the smallest integer $n$ for which $x^n = e$ holds. Note that the order of each group element can be different, but is in the range of $1$ (only the identity element) and $|G|$ (there simply aren't more group elements).

Let us write down the sequence of $x^i$:

\be\label{cyclic_groups_1:eq}
x^0 = e, x^1 = x, x^2, x^3, \ldots, x^{n-1}, x^n = e
\ee

For a finite group ($|G| = n < \infty$), the elements of this sequence are all different: Assume that $0 \leq m < a < b < n$ and assume $x^a = x^b$, then $x^{b-a}=1$. With the before assumption, $b-a < n$ and this contradicts the definition of the order of $x$.

If a group element $g$ has the same order as the group order, it is called a generator for the group and we write $G = \langle g \rangle$. As can be seen in the sequence \eqref{cyclic_groups_1:eq}, in this case $g^i$ will run through all group elements; i.e. generate the group. If the order of an element is less than $n$, then above sequence will not cover all elements of the group.


\subsection{Example}

Consider the group $\mZ_5^\star$. Its order is $|G| = 4$ and the group elements are all those numbers being coprime with $5$; i.e. $\mZ_5^\star= \{1,2,3,4\}$ (see also \hyperref[2017-05-05:entry]{here}). Let's write down the sequence \eqref{cyclic_groups_1:eq} for each element:

\be\label{cyclic_groups_2:eq}
\begin{array}{c|ccccc}
	     i      & 0 & 1 & 2 & 3 & 4 \\ \hline
   	1^i \bmod 5 & 1 & 1 & 1 & 1 & 1 \\
	2^i \bmod 5 & 1 & 2 & 4 & 3 & 1 \\
	3^i \bmod 5 & 1 & 3 & 4 & 2 & 1 \\
	4^i \bmod 5 & 1 & 4 & 1 & 4 & 1
\end{array}
\ee

In Julia, this can be achieved via

\begin{verbatim}
[mod(2^i,5) for i in 1:4]
\end{verbatim}

We can see that both $2, 3$ have order $4$, their powers run through all group elements and are therefore generators for $\mZ_5^\star$. The element $4$ has order $2$, its powers do not run through all group elements and therefore the element is not a generator for the group.

\begin{theorem}
	If $G$ is a group, let $x \in G$ and let $a$ be an integer unequal zero.

	We have for $|x| = n < \infty$: $|x^a| = n / \gcd(n,a)$.

	Assume $G = \langle g \rangle$ and $|x| = n < \infty$. Then $G = \langle x^a \rangle$ iff $\gcd(a,n) = 1$. In particular, the number of generators of $G$ is $\phi(n)$.

\end{theorem}

Note that this theorem can only generate new (additional) generators ($x^a$) when one generator ($x$) is known.

In our example, we have $|G| = 4$ and $\phi(4) = 2$. This is in line with the observations above, where we identified two generators, namely $2$ and $3$. Let's see if we can obtain other generators out of these two: The smallest number being coprime with $n$ is $3$ ($\gcd(3,4) = 1$). Using $2^3 = 8 \equiv 3 \bmod 5$ which gives nothing new. Using $3^3 = 27 \equiv 2 \bmod 5$ which is also nothing new.

Elements of the form $x^2$ are not generators, as $\gcd(2,4) \neq 1$. For example, $2^2 = 4$ is not a generator.

Finally, we can provide the subgroup structure of a cyclic subgroup as follows:

\begin{theorem}
	Let $G = \langle g \rangle$ with $|G| = n < \infty$.
	\begin{itemize}
		\item Every subgroup of $G$ is cyclic. If $K \leq G$, then either $K = \{1\}$ or $K = \langle x^d \rangle$, where $d$ is the smallest positive integer such that $x^d \in K$.
		\item For each positive integer $a$ dividing $n$, there is a unique subgroup of $G$ with order $a$. This subgroup is the cyclic subgroup $\langle x^d \rangle$, where $d = n / a$.
	\end{itemize}
\end{theorem}

In the example, $2$ is a divisor of $|G| = 4$, so there must exist a cyclic subgroup of order $2$. Let us try the generator $2$ first. We have $H_1 = \langle 2^2 \rangle =  \langle 4 \rangle = \{1,4\}$. It is straightforward to check that this is a subgroup of $\mZ_5^\star$. Using the other generator, $3$, we have $H_2 = \langle 3^2 = 9 \equiv 4 \bmod 5$ which is the same subgroup as $H_1$. This is no coincidence; the theorem above says that there is a \emph{unique} subgroup of order $a$.


\subsection{Example $\mZ_7^\star$}

The group contains all integers which are coprime with $7$ and with $7$ being a prime this are the following $\mZ_7^\star =\{1,2,3,4,5,6\}$. The group order is therefore $|\mZ_7^\star| = 6$ and the group contains $\phi(6) = 2$ generators. Let's see what they are by writing down the sequence for each group element

\bee
\begin{array}{c|cccccccc}
	     i      & 0 & 1 & 2 & 3 & 4 & 5 & 6 \\ \hline
	1^i \bmod 7 & 1 & 1 & 1 & 1 & 1 & 1 & 1 \\
	2^i \bmod 7 & 1 & 2 & 4 & 1 & 2 & 4 & 1 \\
	3^i \bmod 7 & 1 & 3 & 2 & 6 & 4 & 5 & 1 \\
	4^i \bmod 7 & 1 & 4 & 2 & 1 & 4 & 2 & 1 \\
	5^i \bmod 7 & 1 & 5 & 4 & 6 & 2 & 3 & 1 \\
	6^i \bmod 7 & 1 & 6 & 1 & 6 & 1 & 6 & 1 \\
\end{array}
\eee

The two generators are $3$ and $5$. With group order $6$, the group contains subgroups of order $2$ and $3$, respectively. The order-$2$ subgroup is $\langle 6 \rangle = \{1, 6\}$, the order-$3$ subgroup is $\langle 2 \rangle =\{1,2,4\}$. Another order-$3$ subgroup is  $\langle 4 \rangle =\{1,2,4\}$ which is isomorphic to $\langle 2 \rangle$.
