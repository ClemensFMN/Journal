\DiaryEntry{Knapsack Problem}{2022-01-17}{}

We have a set of $N$ different item types, each type has a weight $w_i$ and a value $v_i$. We want to pack $x_i$ items of each type into a knapsack with maximum weight capacity $W$ so that the overall value $m[W]$ is maximized,

\bee
m[W] = \max_{x_i} \sum_{i=1}^N v_i w_i
\eee

with the constraint

\bee
\sum_{i=1}^N w_i x_i \leq W
\eee

The weights must be strictly positive ($w_i < 0$), otherwise any solution could be improved (unboundedly) by adding an infinite number of items with negative weight.

There are several possibilites for the allowed value of the $x_i$; the \emph{0-1 Knapsack Problem} allows at most one item of each type; i.e. $x_i \in \{0,1\}$. In a \emph{bounded Knapsack problem}, the $x_i$ are restricted to a maximum number of each item type denoted by the integer $c$. Finally, the \emph{unbounded Knapsack problem} has no bounds on the $x_i$.


\subsection{Unbounded Knapsack Problem}

We solve the Knapsack problem in a recursive manner (using a dynamic programming approach); i.e. we solve it for a knapsack with capacity $W = 0$, then for a knapsack with capacity $W=1$ and so on. We define the function

\bee
m[w'] = \max_{x_i} \sum_{i=1}^N v_i w_i
\eee

with the constraint 

\bee
\sum_{i=1}^N w_i x_i \leq w'
\eee

This function has the following characteristics,

\bee
m[0] = 0
\eee

as we can not fit any items in a knapsack having empty capacity, $W = 0$. A second characteristic is

\bee
m[w] = \max(v_1 + m[w - w_1], v_2 + m[w - w_2], cdots, v_N + m[w - w_N]), \quad w_i \leq w
\eee

This can be explained as follows: Assume we want to fill a knapsack with capacity $w$ and think which item we put in next. We can choose any item $i$ out of the $N$ types and this will add a value of $v_i$ to the knapsack. However, this item has weight $w_i$ so the value $v_i$ is added to the knapsack with weight $w - w_i$ in order not to exceed capacity. We chose the item which creates the knapsack with maximum value. In the end, we arrive at $m[W]$ which holds the solution to the problem. If we additionally keep track of which item $i$ we add in each step (i.e. the $\arg \min$), then we also know which items the knapsack contains.

\paragraph{Example.}  We have $N = 2$ different items with weights $w_1 = 2, w_2 = 3$ and values $v_1 = 3, v_2 = 5$, respectively. As outlined above, we start with $m[0] = 0$. For $m[1]$ we observe that the item with the smallest weight has weight $2$, therefore we cannot add anything to the knapsack and $m[1] = 0$. In our formula, we see that there are no terms to take the maximum over and therefore

\bee
m[1] = \max_{w_i \leq 1}(v_i + m[w - w_i]) = 0
\eee

The next one is more interesting, we have

\bee
m[2] = \max_{w_i \leq 2} (v_i + m[2-w_i]) = \max(v_1 + m[2-w_1]) = \max(3 + m[0]) = \max(3) = 3
\eee

which was realized by adding item $1$ to the knapsack. Since the second item has weight $w_2 = 3 > 2$, it was not considered in the maximization. Next we consider $w=3$ and obtain

\bee
m[3] = \max_{w_i \leq 3} (v_i + m[2-w_i]) = \max(3 + m[1], 5 + m[0]) = \max(3, 5) = 5
\eee

which was realized by adding item $2$ to the knapsack. We continue with $m=4$ and get

\bee
m[4] = \max_{w_i \leq 4} (v_i + m[2-w_i]) = \max(v_1 + m[2], v_2 + m[1]) = \max(3 + 3, 5 + 0) = 6
\eee

We have added one item $1$ to the knapsack; however, the ``starting point'' was the knapsack $m[2]$ which already held one item $1$. Therefore $m[4]$ is realized by a knapsack with \emph{two items $1$}. This is typical for such dynamic programming solutions that a small increase in the function argument (going from $m[3]$ to $m[4]$) causes a different solution (similar with the Viterbi algorithm).


\subsection{Binary Knapsack Problem}

A similar dynamic programming solution for the 0-1 Knapsack Problem also runs in pseudo-polynomial time. We define $m[i,w]$  to be the maximum value that can be attained with weight $\leq w$ using only the first $i$ item types. We can define $m[i,w]$ recursively with starting point

\bee
m[0,w]  = 0
\eee

For values $i > 0$, we have two cases: (i) The new item is heavier than the current weight limit and we therefore cannot add it. In this case we have $m[i,w] = m[i-1,w]$. (ii) The new item is allowed from the weight limit. In this case we need to check whether replacing item $i-1$ by item $i$ increases value, $m[i,w] = \max(m[i-1,w], m[i-1,w-w_i]+v_i)$. Combining this, we have

\bee
m[i,w] = \begin{cases} m[i-1,w] \quad w_i > w \\
  \max(m[i-1,w], m[i-1,w-w_i]+v_i) \quad w_i \leq w \end{cases}
\eee

The solution can then be found by calculating $m[n,W]$. To do this efficiently, we can use a table to store previous computations and knapsack content. Be aware, that there is an upper value limit $\sum w_i$ we can get at most into a knapsack (as we are allowed to take at most one of each item type). So for the algorithm to make sense, we need $\sum w_i \geq W$

\paragraph{Example.} We have three different items with weights $w_1 = 2, w_2 = 3, w_3 = 5$ and values $v_1 = 3, v_2 = 5, v_3 = 7$. The weight limit is $W = 5$. We start with $i = 1$ and calculate $m[1,1]$. Since $w_i = 2 > w = 1$, the first item does not fit into the knapsack with capacity $w = 1$. This is reflected by the first condition and therefore $m[1,1] = m[0,1] = 0$. We continue with $m[1,2]$. Now the first item fits into the knapsack with capacity $w = 2 \leq w_1 = 2$ and therefore we use the second condition,

\bee
m[1,2] = \max(m[0,2], m[0,0]+  v_1) = 3
\eee

Here we choose to add the item one to the knapsack. The further values $m[1,3], m[1,4]$ and $m[1,5]$ do not change anything: We are only allowed to use item one (as $i = 1$) and of course this item will fit into the knapsack with higher capacity.

More interesting are the cases for $i = 2$. First consider $m[2,1]$ and since $w_2 > w$, we have $m[2,1] = m[1,1] = 0$. The same holds for $m[2,2] = m[1,2] = 3$ with the knapsack holding item one. Next is

\bee
m[2,3] = max( m[1,2], m[1,0] + 5) = \max(3, 0 + 5) = 5
\eee

and we have decided to add the second item to the knapsack. For $m[2,4]$ we have

\bee
m[2,4] = max( m[1,4], m[1, 0] + 5) = max(3,5) = 5
\eee

where we again have chosen the second item. Finally, we calculate $m[2,5]$ as

\bee
m[2,5] = max( m(1,4), m[1, 2] + 5) = \max( 3, 3 + 5) = 8
\eee

which we achieve by putting item one (cf $m[1,2]$) and item two into the knapsack.

Now to $i = 3$ and let's speed things up a bit. With the same arguments as before $m[3,1] = 0, m[3,2] = 3, m[3,3] = 5, m[3,4] = 5$; i.e. when the knapsack capacity allows it, we replace item one with item two. When calculating $m[3,5]$ we have

\bee
m[3,5] = \max( m[2,5], m[2,0] + 7) = \max(8, 7) = 8
\eee

where we keep the content of $m[2,5]$; i.e. item one and item two. If the value of the third item were higher; e.g. $v_3 = 10$, then we would choose $m[2,0]$ (which is empty) and add item three to it.

\paragraph{Other interesting things.} The available capacity cannot always be filled up; there will be some slack between $\sum w_i x_i$ and $w$. How is this slack distributed for e.g. random weights and values?

Dominance relations; e.g. two items with same weight, but different value -> item with lower value can be ignored.


\subsubsection{Suboptimal Solutions}

\paragraph{Greedy Algorithm.}

\paragraph{Linear Programming Relaxation.}


%%% Local Variables:
%%% mode: latex
%%% TeX-master: "journal"
%%% End:
