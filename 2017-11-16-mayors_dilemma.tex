\DiaryEntry{The Mayor's Dilemma}{2017-11-16}{Stochastic}

Based on Steutel, "The Mayor's Dilemma" (Math. Scientist, 34) also  \href{files/34_2_1.pdf}{here}.

\subsection{Mean}

$K(n)$ is the random number of hundred euro bills; $\mE\{K(n)\}$ is its expectation. We will derive a relation between $\mE\{K(n+1)\}$ and $\mE\{K(n)\}$ as follows: When an $n+1$-th inhabitant is added, it is the largest with probability $1/(n+1)$ and one further bill is needed. It is not the largest with probability $1 - 1/(n+1) = n/(n+1)$ and in this case no further bill is needed. Therefore, we have 

\bee
\mE\{K(n+1)\} = \frac{n}{n+1}\mE\{K(n)\} + \frac{1}{n+1}\left[ \mE\{K(n)\}+1 \right] = \mE\{K(n)\} + \frac{1}{n+1}
\eee

With $\mE\{K(1)\} = 1$ we have

\bee
\mE\{K(n)\} = \sum_{k=1}^n \frac{1}{k}
\eee

This is the well-known harmonic series which behaves for large $n$ like

\bee
\mE\{K(n)\} \approx \log n + \gamma \approx \log n + 0.5772
\eee

\todo{plot this thing}


\subsection{Interlude - Variance Calculations}

Consider a normal RV $X \sim \Nc(\mu, \sigma^2)$. It has $\mE(X^2) = \mu^2 + \sigma^2$.

Next, consider a deterministic transformation according to $Y = X+1$. The mean of $Y$ is $\mE(Y) = \mu + 1$ and we have

\bee
\mE(Y^2) = \mE((X+1)^2) = \mE(X^2 + 2X + 1) = \mu^2 + \sigma^2 + 2 \mu + 1 = \sigma^2 + (\mu+1)^2
\eee

From this we can calculate the variance of $Y$, $\sigma_Y^2$ as follows

\bee
\sigma_Y^2 = \mE(Y^2) - (\mE(Y))^2 = \sigma^2 + (\mu+1)^2 - (\mu+1)^2 = \sigma^2
\eee

This makes intuitively sense, as adding a deterministic value to a RV does not change its variance.

As a more advanced example, we next consider a RV $Z$ defined as follows:

\bee
Z = \begin{cases}
	X & \text{with probability}\, 1-p \\
	X+1 &\text{with probability}\, p
\end{cases}
\eee

with $X \sim \Nc(\mu, \sigma^2)$. The expectation of $Z$ is

\bee
\mE(Z) = (1-p)\mu + p(\mu+1) = p+\mu
\eee

and correspondingly, we have 

\bee
\mE(Z^2) = (1-p)(\mu^2 + \sigma^2) + p(\sigma^2 + (\mu+1)^2) = \cdots =  \mu^2+\sigma^2+p(2\mu+1)
\eee

where I have omitted the tedious simplification steps. We can now write down the variance of $Z$, $\sigma_Z^2$ as

\bee
\sigma_Z^2 = \mu^2+\sigma^2+p(2\mu+1) - (\mu+p)^2 = \cdots = \sigma^2 + p(1-p)
\eee

The correctness of the expressions can be shown via simulations; see \href{https://github.com/ClemensFMN/JuliaStuff/blob/master/articles/mayors_dilemma.jl}{here}.

\subsection{Variance}

Calculating the variance $\mE\{K^2(n)\}$ can be done as follows: The variance of $K(n+1)$ is the same as the variance of $K(n)$ with probability $n/(n+1)$ and the variance increases by one with the remaining probability $1 - n/(n+1) = 1/(n+1)$. This yields the following expression

\be
\label{2017-11-16:eq1}
\mE\{K^2(n+1)\} = \frac{n}{n+1} \mE\{K^2(n)\} + \frac{1}{n+1} \mE\{(K(n) + 1)^2\}
\ee

Given the previous subsection, I think the last summand shouldbe ok; the notation in the paper is a nightmare and really unclear.

Let's assume that this is correct and continue

\begin{align*}
\mE\{K^2(n+1)\} &= \frac{n}{n+1} \mE\{K^2(n)\} + \frac{1}{n+1} \mE\{K^2(n) + 2 K(n) + 1 \} \\ &= \mE\{K^2(n)\} + \frac{2}{n+1} \mE\{K(n)\} + \frac{1}{n+1}
\end{align*}

Using the result for $\mE\{K(n)\}$ from above, we further obtain

\bee
\mE\{K^2(n+1)\} = \mE\{K^2(n)\} + \frac{2}{n+1} \sum_{k=1}^n \frac{1}{k} + \frac{1}{n+1}
\eee

In order to get a closed-form expression for $\mE\{K^2(n+1)\}$, let us consider $n=1$. We have $\mE\{K(1)\} = 1$ and since this term's variance is zero, we have $\mE\{K^2(1)\} = 1$. Inserting this into the previous expression yields

\bee
\mE\{K^2(2)\} = \mE\{K^2(1)\} + \frac{2}{1+1} \sum_{k=1}^1 \frac{1}{k} + \frac{1}{1+1} = 1 + 1 \times 1 + \frac{1}{2} = \frac{5}{2}
\eee

Let's try something (hopefully) easier: Write the variance $\sigma^2_{n+1}$ of $K(n+1)$ as follows:

\bee
\sigma^2_{n+1} = \mE\{K^2(n+1)\} - \mE\{K(n+1)\}^2 = \mE\{K^2(n)\} + \frac{2}{n+1} \sum_{k=1}^n \frac{1}{k} + \frac{1}{n+1} - \left( \mE\{K(n)\} + \frac{1}{n+1} \right)^2
\eee

which can be simplified to

\bee
\sigma^2_{n+1} = \sigma_n^2 + \frac{2}{n+1} \sum_{k=1}^n \frac{1}{k} + \frac{1}{n+1} - 2 \frac{1}{n+1} \mE\{K(n)\} - \frac{1}{(n+1)^2}
\eee

Based on the definition of $\mE$, the second and fourth term cancel and we arrive at

\bee
\sigma^2_{n+1} = \sigma_n^2 + \frac{1}{n+1} - \frac{1}{(n+1)^2} = \sigma_n^2 + \frac{n}{(n+1)^2}
\eee

This looks nice, and can be converted into the expression from the paper. Write down the case $n=1$:

\bee
\sigma_2^2 = \sigma_1^2 + \frac{1}{2} - \frac{1}{2^2}
\eee

and $n=2$:

\bee
\sigma_3^2 = \sigma_2^2 + \frac{1}{3} - \frac{1}{3^2} = \sigma_1^2  + \frac{1}{2} - \frac{1}{2^2} + \frac{1}{3} - \frac{1}{3^2}
\eee

Using the fact that $\sigma_1^2 = 0$ (we need to hand a bill to the very first person for sure), we can rewrite this as

\bee
\sigma_n^2 = \sum_{k=1}^n \frac{1}{k} - \sum_{k=1}^n \frac{1}{k^2}
\eee

which is the expression from the paper. \qed
%%% Local Variables:
%%% mode: latex
%%% TeX-master: "journal"
%%% End:
