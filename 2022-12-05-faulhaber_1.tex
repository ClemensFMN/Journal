\DiaryEntry{Faulhaber's Formula}{2022-12-05}{Misc}

Baed on the Wikipedia \href{https://en.wikipedia.org/wiki/Faulhaber%27s_formula}{entry}, Faulhaber's formula provides a closed-form expression for the sum of the $p$-th power of the first $n$ integers,

\bee
\sum_{k=1}^n k^p = 1^p + 2^p + \cdots + n^p = \frac{n^{p1}}{p+1} + \frac{1}{2}n^p + \sum_{k=2}^p \frac{B_k}{k!} p^{\underline{k-1}} n^{p-k+1}
\eee

Here $x^{\underline{n}} = x(x-1)\cdots(x-n+1)$ denotes the falling factorial and $B_j$ is the Bernoulli number (maybe these require a separate article :-)).

\subsection{$p$ is odd}

If $p$ is odd, then $1^p + 2^p + \cdots + n^p$ is a polynomial function of 

\bee
a = 1 + 2 + 3 + \cdots n = \frac{n(n+1)}{2}
\eee

For example, we have

\bee
1^3 + 2^3 + \cdots n^3 = a^2
\eee

and

\bee
1^5 + 2^5 + \cdots n^5 = \frac{4a^3-a^2}{3}
\eee

\subsection{Matrix Theorem}

On Wikipedia, they write down the first sum expressions for $p=1 \ldots 6$ and see that the sum expression is a polynomial in $n$ with order $p+1$. That's something we have also used in entry \ref{2022-09-01:entry} as well. Next, they write the sum expressions in matrix-vector notation,

\bee
\begin{pmatrix}
    &\sum i^0 \\
    &\sum i^1 \\
    &\sum i^2 \\
    &\sum i^3 \\
    &\sum i^4 \\
    &\sum i^5 \\
    &\sum i^6
\end{pmatrix} = G_7 \begin{pmatrix}
    &n^1 \\
    &n^2 \\
    &n^3 \\
    &n^4 \\
    &n^5 \\
    &n^6 \\
    &n^7
\end{pmatrix}
\eee

with the matrix $G_7$ given as

\bee
G_7 = \begin{pmatrix}
    1 & 0 & 0 & 0 & 0 & 0 & 0 \\
    1/2 & 1/2 & 0 & 0 & 0 & 0 & 0 \\
    1/6 & 1/2 & 1/3 & 0 & 0 & 0 & 0 \\
    0 & 1/4 & 1/2 & 1/4 & 0 & 0 & 0 \\
    -1/30 & 0 & 1/3 & 1/2 & 1/5 & 0 & 0 \\
    0 & -1/12 & 0 & 5/12 & 1/2 & 1/6 & 0 \\
    1/42 & 0 & -1/6 & 0 & 1/2 & 1/2 & 1/7
\end{pmatrix}
\eee

From this we can e.g. read off that $\sum i^2 = 1/6n + 1/2 n^2 + 1/3n^3$. It would be helpful, to b able to generate the matrix $G_p$ somehow, unfortunately, it does not seem to have any structure; however, the inverse $G_7^{-1}$ does have some structure,

\bee
G_7^{-1} = \begin{pmatrix}
    1 & 0 & 0 & 0 & 0 & 0 & 0 \\
    -1 & 2 & 0 & 0 & 0 & 0 & 0 \\
    1 & -3 & 3 & 0 & 0 & 0 & 0 \\
    -1 & 4 & -6 & 4 & 0 & 0 & 0 \\
    1 & -5 & 10 & -10 & 5 & 0 & 0 \\
    -1 & 6 & -15 & 20 & -15 & 6 & 0 \\
    1 & -7 & 21 & -35 & 35 & -21 & 7
\end{pmatrix}
\eee

It somehow resembles a Pascal triangle; i.e. the value of a cell equals the sum of the cell above and the cell above one to the left. In addition, the signs are alternating along a row. This is shown in the following equation.

\bee
G_7^{-1} = \begin{pmatrix}
    1 & 0 & 0 & 0 & 0 & 0 & 0 \\
    - \color{red}{1} & \color{red}{2} & 0 & 0 & 0 & 0 & 0 \\
    1 & -\color{red}{3} & 3 & 0 & 0 & 0 & 0 \\
    -1 & 4 & -6 & 4 & 0 & 0 & 0 \\
    1 & -5 & \color{blue}{10} & - \color{blue}{10} & 5 & 0 & 0 \\
    -1 & 6 & -15 & \color{blue}{20} & -15 & 6 & 0 \\
    1 & -7 & 21 & -35 & 35 & -21 & 7
\end{pmatrix}
\eee

We can construct such a matrix in Julia, for example for $p=10$,

\begin{verbatim}
    N = 10
    g = zeros(N,N)

    # main diagonal
    [g[i,i]=i for i=1:N]
    # first column
    g[:,1].=1

    for j=3:N
    [g[j,i] = g[j-1, i-1] + g[j-1,i] for i=2:j-1]
    end

    sign_matrix = [(-1)^(i+j) for i=1:N, j=1:N]

    g = sign_matrix .* g
    inv(g)
\end{verbatim}

The $10$-th row in the matrix is (rounded values)

\begin{verbatim}
    0  -0.15  0   0.5   0   -0.7   0  0.75   0.5   0.1
\end{verbatim}

and from this we read off

\bee
\sum_{k=1}^n k^{9} = - \frac{3}{20} n^2 + \frac{1}{2} n^4 - \frac{7}{10} n^6 + \frac{3}{4} n^8 + \frac{1}{2} n^9 + \frac{1}{10} n^{10}
\eee

This is correct as

\begin{verbatim}
    julia> k=1:5
    julia> sum(k.^9)
    2235465

    julia> -3/20*5^2 + 1/2*5^4 - 7/10*5^6 + 3/4*5^8 + 1/2*5^9 + 1/10*5^10
    2.235465e6
\end{verbatim}

As another check (or maybe simpler way to obtain the formula?), we can use the method used in entry \ref{2022-09-01:entry} to obtain the sum formula. The good news is that the equations match :-) 

\todo{Consider also https://arxiv.org/abs/math/9207222}

%%% Local Variables:
%%% mode: latex
%%% TeX-master: "journal"
%%% End:
