\DiaryEntry{1000 Problems in Probability - 2}{2018-12-04}{Stochastic}

\subsection{Section 3.1, Problem 1}

\paragraph{Geometric Distribution.} We consider the geometric distribution with pdf

\bee
f(k) = C (1-p)^k, \quad k=0,\ldots,\infty
\eee

Note, that this is the ``Julia'' definition and is different than in the book (there it is $f(k) = Cp^k$ and $k$ starts with $1$). Let us first calculate the constant $C$,

\bee
C \sum_{k=0}^\infty (1-p)^k = C \frac{1}{1-(1-p)} = \frac{C}{p} \doteq 1 \rightarrow C = p
\eee

and the pdf therefore becomes

\bee
f(k) =  p (1-p)^k, \quad k=0,\ldots,\infty
\eee

Next we calculate the probability $P(X>0) = 1 - P(X=0) = 1-p$. The probability that $X$ is even is given by

\begin{align*}
P(X \text{even}) &= p \sum_{k=0,2,4,\ldots}^\infty (1-p)^k = p \sum_{k=0,1,\ldots}^\infty (1-p)^{2k} = p \sum_{k=0,1,\ldots}^\infty \left((1-p)^{2}\right)^k \\ &= \frac{p}{1-(1-p)^2} = \frac{1}{2-p}
\end{align*}

We can simulate a large number of RVs in Julia, ``measure'' the probabilities and compare with the analytical results: \href{https://github.com/ClemensFMN/JuliaStuff/blob/master/stochastic/geom_rv_2.jl}{see here}. Exemplary results for $p=0.73$:

\begin{verbatim}
P(X>0): measured = 0.269944; analytical = 0.27
P(X even): measured = 0.787224; analytical = 0.7874015748031495
\end{verbatim}

As an add-on, let us calculate the mean $\mu$ according to

\bee
\mu = \sum_{k=0}^\infty k p (1-p)^k = p(1-p) \sum_{k=0}^\infty k (1-p)^{k-1} = -p(1-p) \sum_{k=0}^\infty \frac{d}{dp} (1-p)^{k}
\eee

This trick of rewriting the sum terms as derivative (and that we can exchange summation and differentiation) allows a closed-form expression for the sum. 

\bee
\mu = p(p-1) \frac{d}{dp} \sum_{k=0}^\infty (1-p)^k = p(p-1) \frac{d}{dp} \frac{1}{1-(1-p)} = \frac{p(p-1)}{p^2} = \frac{1-p}{p}
\eee

Comparison with numerical results shows the correctness of the result. \qed

\paragraph{Poisson Distribution.} Now we consider the pdf

\bee
f(k) = C \frac{a^k}{k!}, \quad k=0,\ldots,\infty
\eee

Summing over $k$ allows calculation of $C$:

\bee
C \sum_{k=0}^\infty \frac{a^k}{k!} = C \exp(a) \doteq 1 \rightarrow C = \exp(-a)
\eee

and the pdf becomes

\bee
f(k) = e^{-a} \frac{a^k}{k!}
\eee

The probability $P(X > 0)$ is given by $P(X>0) = 1 - P(X=0) = 1-\exp(-a)$. The probability that $X$ is even is given by

\bee
P(X \text{even}) = e^{-a} \sum_{k=0,2,4,\ldots}^\infty \frac{a^k}{k!} = e^{-a} \sum_{k=0,1,\ldots}^\infty \frac{a^{2k}}{(2k)!} = e^{-a} \left( \frac{a^0}{0!} + \frac{a^2}{2!} + \frac{a^4}{4!} + \cdots \right)
\eee

It seems like we are stuck, but we can make the following Ansatz:

\begin{align*}
\sum_k \frac{a^{2k}}{(2k)!} &= \frac{1}{2} \left(\frac{a^0}{0!} + \frac{a^1}{1!} + \frac{a^2}{2!} + \cdots \right) + \frac{1}{2} \left(\frac{a^0}{0!} - \frac{a^1}{1!} + \frac{a^2}{2!} \mp \cdots \right) \\ &= \frac{1}{2} \left( \sum_k \frac{a^k}{k!} + \sum_k \frac{(-a)^k}{k!} \right) = \frac{1}{2} \left(e^a + e^{-a} \right)
\end{align*}

using the fact that the $a^k/k!$ with odd $k$ cancel in the two expressions. Therefore, the probability becomes

\bee
P(X \text{even}) = e^{-a} \sum_{k=0,1,\ldots}^\infty \frac{a^{2k}}{(2k)!} = \frac{1}{2} e^{-a} \left(e^a + e^{-a} \right) = \frac{1}{2} (1 + e^{-2a})
\eee

Julia simulation (\href{https://github.com/ClemensFMN/JuliaStuff/blob/master/stochastic/poisson_rv.jl}{see here}) yields exemplary results for $a=0.4$:

\begin{verbatim}
P(X>0): measured = 0.32944; analytical = 0.3296799539643607
P(X even): measured = 0.7257; analytical = 0.7246644820586108
\end{verbatim}


\paragraph{Logarithmic Distribution.} It has the following form

\bee
f(k) = C \frac{a^k}{k}, \quad k=1,\ldots,\infty
\eee

We cannot directly sum the expression; instead we will use a series expansion ``in the backward'' direction. Consider the series expansion

\bee
-\ln(1-a) = a + \frac{a^2}{2} + \frac{a^3}{3} + \cdots = \sum_{k>0} \frac{a^k}{k}
\eee

we obtain for the pdf

\bee
f(k) = -\frac{1}{\ln(1-a)} \frac{a^k}{k}, \quad k=1,\ldots,\infty
\eee

The probability for drawing an even number can be obtained as follows,

\bee
P(X \text{even}) = -\frac{1}{\ln(1-a)} \sum_{k=2,4,\ldots}^\infty \frac{a^k}{k} = -\frac{1}{\ln(1-a)} \sum_{k>0} \frac{a^{2k}}{2k} = -\frac{\ln(1-a^2)}{2\ln(1-a)} = \frac{1}{2} \ln(1+a)
\eee

This time we just verify the summation via Julia

\begin{verbatim}
a=0.2
k=2:2:20
-1/log(1-a)*sum(a.^k ./ k)
>> 0.0914702537443563
1/2*log(1+a)
>> 0.0911607783969773
\end{verbatim}

%%% Local Variables:
%%% mode: latex
%%% TeX-master: "journal"
%%% End:
