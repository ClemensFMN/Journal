\DiaryEntry{Conjugacy}{2016-11-21}{Algebra}

\subsection{Definition}\label{definition}

Two elements \(x,y\) of a group G are conjugate if

\[
gxg^{-1} = y
\]

for some \(g \in G\). We first note that the definition is symmetric;
i.e when x is conjugate to y then y is conjugate to x (simply by
left-multiplying with \(g^{-1}\), right-multiplying with g, and
substituting \(g \rightarrow g^{-1}\):

\[
gxg^{-1} = y \leftrightarrow x = gyg^{-1}
\]

If we vary g, then we obtain a set of y's which are collected in the
conjugacy class of x:

\[
C(x) = \{y = gxg^{-1} | g \in G\}
\]

If the group G is abelian, the conjugcy classes are boring as we can
rewrite the definition to obtain

\[
C(x) = \{y = xgg^{-1} | g \in G\} = \{x\}
\]

Let the subset \(\mathbb{G}\) of \(G \times G\) consist of all paris
\((x,y)\) of which are conjugate to each other. Each element is
conjugate to itself because \(exe^{-1} = x\). Furthermore, if x and y
are conjugate and y and z are conjugate

\[
g_1 x g_1^{-1} = y, g_2 y g_2^{-1} = z
\]

then x and z are also conjugate. Consider \(g_2 g_1 x (g_2 g_1)^{-1}\)
which can be rewritten as

\[
g_2 g_1 x g_1^{-1} g_2^{-1} = g_2 y g_2^{-1} = z
\]

These characterisitcs define the set \(\mathbb{G}\) as an equivalence
relation and therefore the conjugcy classes partition G.

\subsubsection{Conjugacy Classes of the Dihedral
Group}\label{conjugacy-classes-of-the-dihedral-group}

The dihedral group \(D_n\) is defined in
\href{\%7Bfilename\%7D/2016-03-15-groups_04.markdown}{this post} by the
following equations:


\begin{align*}
r^n &= 1 \\
s^2 &= 1 \\
srs &= r^{-1}
\end{align*}


Let us first calculate the conjugacy class of \(r^\alpha\): We have
\(gr^\alpha g^{-1}\) and if we choose \(g = r^\beta\), we obtain
\(g^{-1} = r^{n-\beta}\) and therefore
\(y = gxg^{-1} = r^\beta r^\alpha r^{n-\beta} = r\alpha\). Choosing
\(g = s\), we have \(g^{-1} = s\) and
\(y = sg^\alpha s = r^{n - \alpha}\). Finally choosing
\(g = r^\beta s\), we have
\(y = r^\beta s r^\alpha (r^\beta s)^{-1} = r^\beta (s r^\alpha s) r^{-\beta} = r^\beta r^{n - \alpha} r^{n-\beta} = r^{n - \alpha}\).
So, the conjugacy class of \(r^\alpha\) is \(r^{n-\alpha}\).

Continuing in this fashion yields the following conjugacy classes


\begin{align*}
& \{e\}, \{r, r^{n-1}\}, \{r^2, r^{n-2}\}, \ldots \\
& \{s, r^2s, r^4s,\ldots\}, \{rs, r^3s, \ldots\}
\end{align*}


\subsubsection{Conjugacy Classes of the Symmetric
Group}\label{conjugacy-classes-of-the-symmetric-group}

Two elements of \(S_n\) have the same cycle structure if when they are
composed as products of disjoint cyclic permutations, they have the same
number of 2-cycles, 3-cycles etc.

Permutations which have the same cycle structure are conjugate in
\(S_n\). We can show this as follows: Assume that \(a,b \in S_n\) have
the same cycle structure and write the cycle decomposition of \(b\)
below the one of \(a\) (in order of decreasing cycle length and
including cycles of length 1). Now define \(g\) to be a permutation
which sends an element of \(a\) to an element in \(b\) directly below.
Then \(gbg^{-1} = a\) because moving down from an element \(a\) to an
element \(b\) (that's the effect of \(g\)), moving one to the right
(that's the effect of \(b\)), and moving back up (that's the effect of
\(g^{-1}\)) is the same as moving one to the right (that's the effect of
\(a\)).

As an example in \(S_9\) take \(a = (2539)(67)(14)\) and
\(b = (5467)(12)(38)\). Both contain a length-4 cycle and two length-2
cycles. Writing them down with the rules as above, we have


\begin{align*}
&(2539)(67)(14)(8) \\
&(5467)(12)(38)(9)
\end{align*}


Therefore, g becomes \(g=(254897)(361)\). The element g is not unique;
e.g.~rewriting \(a\) as \(a=(2539)(14)(67)(8)\) yields


\begin{align*}
&(2539)(14)(67)(8) \\
&(5467)(12)(38)(9)
\end{align*}


and \(g=(254)(36)(789)\).

It is not only that permutations with the same cycle structure are
conjugate in \(S_n\), but also that conjugate permutations have the same
cycle structure.

\paragraph{Using GAP}\label{using-gap}

We can use GAP to list conjugacy classes. In
\href{\%7Bfilename\%7D/files/conjugacy.g}{this script} we use GAP to
list all conjugacy classes of the symmetric group \(S_4\).

\begin{verbatim}
g := SymmetricGroup(4);
cg := ConjugacyClasses(g);

for item in cg do
    Print(item, "->", AsList(item), "\n\n");
od;
\end{verbatim}

which yields (formatting done manually)

\begin{verbatim}
(1,2) -> [ (1,2), (1,3), (1,4), (2,3), (2,4), (3,4)]

(1,2)(3,4) -> [ (1,2)(3,4), (1,3)(2,4), (1,4)(2,3) ]

(1,2,3) -> [ (1,2,3), (1,3,2), (2,3,4), (2,4,3), (1,3,4), (1,4,3), (1,4,2), (1,2,4) ]

(1,2,3,4) -> [ (1,2,3,4), (1,2,4,3), (1,3,2,4), (1,3,4,2), (1,4,2,3), (1,4,3,2) ]
\end{verbatim}

There are conjugacy classese having (i) one 2-cycle, (ii) two 2-cycles,
(iii) one 3-cycle, and (iv) one 4-cycle.
