\DiaryEntry{Groups - Subgroups}{2016-10-27}{Algebra}

For the definition of a subgroup, see
\href{\%7Bfilename\%7D2016-03-02_-groups_01.markdown}{this post}.

\subsection{Words}\label{words}

Let X be a subset of a group G. An expression of the form (the product
being the group operation)

\[
g = x_1^{m_1} x_2^{m_2} \cdots x_n^{m_n}
\]

with the \(m_i\) being integers and \(x_iu \in X\) is called a word in
the elements of X. The collection of all words is a subgroup of G:

\begin{itemize}
\item
  the product of two words is again a word in X.
\item
  the identity element can be expressed as word with all \(m_i = 0\).
\end{itemize}

If the collection of all words fills out G, then X is a set of
generators for G.

\paragraph{Example.}

For the Klein-4 Group, \(X = \{a,b\}\) is a set of generators as all 4
elements of the Klein-4 Group can be expressed by means of \(a\) and
\(b\): \(a^2 = e\), \(ab = c\).

\subsection{Subgroup Theorems}\label{subgroup-theorems}

A non-empty subset H of a group G is a subgroup of G if and only if
\(xy^{-1}\) belongs to H whenever \(x,y \in H\).

Proof: If \(x, y \in H \rightarrow y^{-1} \in H\) and therefore
\(x y^{-1} \in H\). If H is non-empty, \(x y^{-1} \in H\), and
\(x,y \in H\), then
\(e = x x^{-1} \in H \rightarrow x^{-1} = ex^{-1} \in H\). Finally, if
\(y \in H\), then \(y^{-1} \in H\) and therefore
\(xy = x (y^{-1})^{-1} \in H\). Therefore, H is a subgroup of G.

The intersection of two subgroups H and K of a group G is again a
subgroup.

Proof: Assume \(x,y \in H \cap K\). Then \(xy^{-1} \in H \cap K\) and
with the theorem above, $H \cap K$ is also a subgroup.

Every subgroup of \(\mathbb{Z}\) is cyclic. More generally, every
subgroup of a cyclic group is also cyclic.
