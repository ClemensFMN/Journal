\DiaryEntry{Using Indices to solve Congruences}{2023-01-18}{Number Theory}

Coming from \cite{Burton2011}. We want to solve the following equation for $x$,

\bee
x^k \equiv a \mod n
\eee

with the positive integer $n$ having a prime root and $\gcd(a,n) =1$. We can convert this (by "taking logs") to

\bee
k \ind x \equiv \ind a \mod \phi(n)
\eee

with the unknown $\ind x$. Let's start by analyzing when this equation has solutions: To this end define $d = \gcd(k, \phi(n))$. If $d \nmid \ind a$, then there is no solution.

\todo{explain in more detail / give example}

On the other hand, if $d \mid \ind a$, then there are exactly $d$ values of $\ind x$ that will satisfy the congruence; and therefore, there are $d$ incongruent solutions of $x^k \equiv a \mod n$.

The case $k = 2$ corresponds to a quadratic congruence, $x^2 \equiv a \mod n$. With $n=p$ being an odd prime, we have $d = \gcd(2, \phi(p)) = \gcd(2, p-1) = 2$ because $p-1$ is an even number. By the previous arguments, if $d \mid \ind a$, then the equation has exactely $2$ solutions; otherwise it has no solutions.

If $r$ is a primitive root of $p$, then $r^k$ runs $\mod p$ through the integers $1,2,\ldots,p-1$ in some order for $k=1,2,\ldots,p-1$. The even powers of $r$ produce the values of $a$ for which the congruence $x^2 \equiv a \mod n$ is solvable; there are $(p-1)/2$ choices for $a$.

As a reminder, index calculation is done in GAP via

\bee
  \ind_r a = \verb+ LogMod(a, r, n) +
\eee


\paragraph{Example 1.} Let's try this out with

\bee
x^2 \equiv 3 \mod 13
\eee

A primitive root of $13$ is $r = 2$ and $\ind_2 3 = 4$ is even, so we should have $2$ solutions. Let's start with searching for the solution via brute-force. We write down all values of $x$ and $x^2$ ($\mod 13$) for $x=1,\ldots,12$.

\vspace{2mm}

\begin{tabular}{lcccccccccccc}
    $x$   & 1 & 2 & 3 & 4 & 5  & 6  & 7  & 8  & 9 & 10 & 11 & 12 \\ \hline
    $x^2$ & 1 & 4 & 9 & 3 & 12 & 10 & 10 & 12 & 3 & 9  & 4  & 1 
\end{tabular}

\vspace{2mm}

Looking at the values, we see that $x = 4$ and $x = 9$ solve our equation $x^2 \equiv 3 \mod 13$. So far, so good - there are two solutions as predicated. Now, let's use indices to speed up finding the solution.

A primitive root of $p=13$ is $2$, so we can take the index on both sides and have

\bee
2 \ind_2 x \equiv \ind_2 3 \mod 12
\eee

Be careful that the index still uses $p=13$, but the mod on the right side uses $\mod \phi(13) = \mod 12$! With \verb+ LogMod(3, 2, 13) = 4+, we have

\bee
2 \ind_2 x \equiv 4 \mod 12
\eee

This has two solutions (below $12$): One solution is $\ind_2 x = 2$, the other is $\ind_2 x = 8$. To solve for $x$, we now go "backwards"; i.e. using $r^{\ind a} \equiv a \mod p$. Therefore $\ind_2 x = 2 \rightarrow x = 2^2 \mod 13 \equiv 4$ and $\ind_2 x = 8 \rightarrow x = 2^8 \mod 13 \equiv 9$ which match the results obtained by the brute-force search. \qed

Another root of $p=13$ is $6$. Let's repeat the calculation with this value. Since $\ind_6 3 = 8$, we should also have $2$ solutions. Taking indices on both sides of the equation yields

\bee
2 \ind_6 x \equiv \ind_6 3 \mod 12
\eee

With \verb+ LogMod(3, 6, 13) = 8+, we have

\bee
2 \ind_6 x \equiv 8 \mod 12
\eee

The two solutions are $\ind_6 x = 4$ and $\ind_6 x = 10$. These correspond to $x = 6^4 \mod 13 \equiv 9 \mod 13$ and $x = 6^{10} \equiv 4 \mod 13$ which match the results obtained with another root. \qed

One thing to note is that the index values depend on the chosen primitive root and are in general different: So $\ind_2 3 = 4$ and $\ind_6 3 = 8$, but $\ind_2 8 = \ind_6 8 = 3$.

As with "normal" quadratic equations, the two solutions are related with each other; if $x_1^2 \equiv 3 \mod 13$ then the other solution is $x_2 = 13-x_1$. We have

\bee
(13-x_1)^2 = 13^2 - 2 \cdot 13 \cdot x_1 + x_1^2 \equiv 0 - 0 + x_1^2 \equiv 3 \mod 13 \qed
\eee

Indeed, we have $x_1 = 4$ and $x_2 = 13 - 4 = 9$.


\paragraph{Example 2.} Let's try another more complex example,

\bee
4x^9 \equiv 7 \mod 13
\eee

We chose $2$ as primitive root and take indices on both sides to obtain (again, be careful, that our $\mod 13$ becomes $\mod \phi(13) = 12$!)

\bee
\ind_2 4 + 9 \ind_2 x \equiv \ind_2 7 \mod 12
\eee

We can use the index table from below or use GAP (or whatever) to calculate the indices and get

\bee
2 + 9 \ind_2 x \equiv 11 \mod 12 \rightarrow 9 \ind_2 x \equiv 9 \mod 12
\eee

We can now divide both sides of the last equation by $9$ (this is covered in depth in theorem \ref{2020-11-17:th3}). The short story is that we need to divide the $\mod$ value by $d = \gcd(9, 12) = 4$ and we therefore arrive at the following equation,

\begin{align*}
\ind_2 x & \equiv 1 \mod 12/3 \\
& \equiv 1 \mod 4
\end{align*}

\todo{what happens if we cannot divide; e.g. how do we solve $5 x \equiv 9 \mod 13$} 

The original congruence was $\mod 12$, so we seek solutions below $12$ and these are $\ind_2 x = 1, 5, 9$. Looking at our table (or elsewhere) we obtain the corresponding values of $x$ as

\bee
x = 2, 5, 6 \mod 13
\eee

\paragraph{Index Table.} The following table gives the index table for $13$ and primitive root $2$.

\vspace{2mm}

\begin{tabular}{lllllllllllll}
    a     & 1  & 2 & 3 & 4 & 5 & 6 & 7  & 8 & 9 & 10 & 11 & 12 \\ \hline
    ind a & 12 & 1 & 4 & 2 & 9 & 5 & 11 & 3 & 8 & 10 & 7  & 6 
\end{tabular}

\vspace{2mm}

If we chose a different primitive root, the calculation is different (since index values will change) but the results will be the same.

The following theorem gives a condition for solvability of $x^k \equiv a \mod n$.

\begin{theorem}
Let $n$ be an integer with a primitive root and let $\gcd(a,n) = 1$. Then the congruence $x^k \equiv a \mod n$ has a solution iff

\bee
a^{\phi(n) / d} \equiv 1 \mod n
\eee

where $d = \gcd(k, \phi(n))$; if the congruence is solvable, then it has exactely $d$ solutions $\mod n$.
\end{theorem}

We can prove this by taking indices on both sides,

\bee
\frac{ \phi(n)}{d} \ind a \equiv 0 \mod \phi(n)
\eee

and this holds iff $d |\ind a$. Before we have seen that this is a necessary and sufficient condition for the congruence to be solvable. \qed

In the special case of $n$ being prime and $\gcd(a,p) = 1$, the condition simplifies to

\bee
a^{(n-1) / d} \equiv 1 \mod n, \quad d = \gcd(k, n-1) \qed
\eee

Checking our first example $x^2 \equiv 3 \mod 13$, we have $d = \gcd(2, 12) = 2$ and $2^{12/2} = 2^6 \equiv 1 \mod 13$: the congruence is solvable and has $d = 2$ solutions. The congruence $x^2 \equiv 5 \mod 13$ has no solutions: The value $d$ is the same as before, $d = 2$, but we have $5^6 \equiv 12 \neq 1 \mod 13$.

In Python SymPy, we can use \verb+ sqrt_mod + to solve such quadratic congruences. Redoing our previous examples, we have

\begin{verbatim}
print(sqrt_mod(3, 13, all_roots=(True)))
[4, 9]
print(sqrt_mod(5, 13, all_roots=(True)))
[]
\end{verbatim}

\paragraph{Interpretation.} The mod operator causes numbers to "wrap around"; e.g. $3x \mod 13$ attains the following values,

\vspace{2mm}
\begin{tabular}{lllllllllllllll}
  3x        & 0 & 3 & 6 & 9 & 12 & 15 & 18 & 21 & 24 & 27 & 30 & 33 & 36 & 39 \\ \hline
  3x mod 13 & 0 & 3 & 6 & 9 & 12 & 2  & 5  & 8  & 11 & 1  & 4  & 7  & 10 & 0 
\end{tabular}
\vspace{2mm}

Since $\gcd(3,13) = 1$, the expression $3x \mod 13$ will attain all values in the interval $[0, 12]$ and then start anew. If we had a different relation $kx \mod n$ with $\gcd(k, n) != 1$, then certain values would be left out and the congruence could not be solved for all combination of $k$ and $n$.

In case of a quadratic congruence things seem to be similar - see the table below; because $x^2$ is nonlinear, the values are increasing faster. However, the mod operation "wraps" the values back into the interval $[0, 12]$. Every value appears two times; therefore, not all values in $[0, 12]$ appear and this causes not all quadratic congruences to be solvable.

\vspace{2mm}
\begin{tabular}{lllllllllllllll}
  $x$          & 0 & 1 & 2 & 3 & 4  & 5  & 6  & 7  & 8  & 9  & 10  & 11  & 12  & 13  \\
  $x^2$        & 0 & 1 & 4 & 9 & 16 & 25 & 36 & 49 & 64 & 81 & 100 & 121 & 144 & 169 \\ \hline
  $x^2 \mod 13$& 0 & 1 & 4 & 9 & 3  & 12 & 10 & 10 & 12 & 3  & 9   & 4   & 1   & 0  
\end{tabular}
\vspace{2mm}

From the table it can also be seen that the values of $x^2 \mod 13$ are symmetric around the middle; this is due to the fact that both $x_1$ and $13 - x_1$ are solutions.

Interestingly (or unintuitivley), the values of $x^2 \mod 13$ are repeating with $x=13$. This can be seen by considering

\bee
(x+13)^2 \equiv x^2 + 2 \cot 13 \cdot x + 13^2 \equiv x^2 \mod 13 \qed
\eee




%%% Local Variables:
%%% mode: latex
%%% TeX-master: "journal"
%%% End:
