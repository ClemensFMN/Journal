\DiaryEntry{Gamma Function}{2015-07-27}{Maths}

\subsection{Partial Integration}

First thing is the derivation of the partial integration: We differentiate the product of two functions

\bee
\frac{d \left( u(x)v(x) \right)}{dx} = u(x) \frac{d v(x)}{dx} + v(x) \frac{d u(x)}{dx}
\eee

integrate both sides

\bee
u(x)v(x) = \int u(x) \frac{d v(x)}{dx} dx + \int v(x) \frac{d u(x)}{dx} dx
\eee

and rearrange the result so that we obtain

\bee
\int u(x) \frac{d v(x)}{dx} dx = u(x)v(x) - \int v(x) \frac{d u(x)}{dx} dx
\eee

or - in a more sloppy notation - we have

\bee
\int u v' dx = uv - \int u' v dx
\eee

\subsection{Definition and Properties}

The Gamma function $\Gamma(x)$ is defined according to

\be\label{2015-07-27:eq1}
\Gamma(x) = \int_0^\infty t^{x-1}e^{-t} dt
\ee

The following Figure shows a plot of the integrand for $x=1, x=2, x=3$, respectively. Note that with increasing $x$, the maximum moves to the right and the area under the curve increases.

\begin{figure}[H]
\centering
\includegraphics[scale=0.7]{images/2015-07-27-gamma_integrand_plot.png}
\end{figure}

Note, that the Gamma function is also defined for negative values of $x$; the following Figure shows plots of the integrand for $x=-1, x=-2, x=-3$, respectively. It diverges at $t=0$, and falls off for increasing $t$.

\begin{figure}[H]
\centering
\includegraphics[scale=0.7]{images/2015-07-27-gamma_integrand_neg_plot.png}
\end{figure}


The Gamma function fulfills the following recurrence relation

\be\label{2015-07-27:eq2}
\Gamma(x+1) = x \Gamma(x)
\ee

which we can prove as follows: We start with

\bee
\Gamma(x+1) = \int_0^\infty t^{x}e^{-t} dt
\eee

By partial integration (choosing $v' = e^{-t} \rightarrow v = -e^{-t}, u=t^x \rightarrow u'=x t^{x-1}$) we obtain

\begin{align*}
\Gamma(x+1) &= \left. -t^x e^{-t} \right|_0^\infty - \int_0^\infty x t^{x-1}(-1)e^{-t} dt = 0 + \int_0^\infty x t^{x-1}e^{-t} dt = x \int_0^\infty t^{x-1}e^{-t} dt \\
&= x \Gamma(x) \qed
\end{align*}

Next we calculate the value of $\Gamma(1)$ as

\bee
\Gamma(1) = \int_0^\infty t^{0}e^{-t} dt = \int_0^\infty e^{-t} dt = \left. e^{-t} \right|_0^\infty = 1
\eee

From this and \eqref{2015-07-27:eq2} we read off that

\bee
\Gamma(n) = (n-1)!
\eee

The values for the next integer values of $\Gamma(n)$ can be obtained from \eqref{2015-07-27:eq2} as follows

\begin{align*}
    \Gamma(2) &= 1 \Gamma(1) = 1 = 1! \\
    \Gamma(3) &= 2 \Gamma(2) = 2 = 2! \\
    \Gamma(4) &= 3 \Gamma(3) = 6 = 3! \\
    \Gamma(5) &= 4 \Gamma(4) = 24 = 4! \\
    &\cdots
\end{align*}

The function plot $\Gamma(x)$ is shown in the following Figure.

\begin{figure}[H]
\centering
\includegraphics[scale=0.5]{images/2015-07-27-gamma_plot.png}
\end{figure}

\subsection{Incomplete Gamma Functions}

If we choose the integral bounds as parameters, we arrive at the incomplete Gamma functions;

\bee
\gamma(a,z) = \int_0^z t^{a-1} e^{-t} dt
\eee

and

\bee
\Gamma(a,z) = \int_z^\infty t^{a-1} e^{-t} dt
\eee

These become the "normal" Gamma function in the limit;

\bee
\lim_{z \rightarrow \infty} \gamma(a, z) = \Gamma(a), \qquad \Gamma(a, 0) = \Gamma(a)
\eee

We also have the relation $\gamma(a,z) + \Gamma(a,z) = \Gamma(a)$.

By partial integration we obtain the relation

\bee
\gamma(a+1,z) = a \gamma(a,z) - z^a e^{-z}
\eee

tbc...

