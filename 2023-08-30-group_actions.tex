\DiaryEntry{Group Actions}{2023-08-30}{Algebra}

Let $G$ be a group acting on an non-empty set $A$. Then for each $g \in G$, the map

\bee
\sigma_g: A \rightarrow A, \quad \sigma_g: a \rightarrow g \cdot a
\eee

is a permutation of $A$. As a simple example, choose $A = \{1,2,3\}$ and $g = (1,2)$. With $a = 1, g \cdot a = 2$; in Gap

\begin{verbatim}
gap> OnPoints(1, (1,2));
2
\end{verbatim}

In addition, there is also a map $\phi$ from $G$ to $S_A$ defined by $g \rightarrow \sigma_g: \phi(g) = \sigma_g$. This map can be shown to be a homomorphism and is called the permutation representation associated to the given action.

There are a couple of definitions associated with the permutation representation.

\begin{definition}
The \emph{kernel} of the action is the set of elements of $G$ that act trivially on every element of $A$: $\{g \in G | g \cdot a = a \forall a \in A\}$. For each $a \in A$, the \emph{stabilizer} of $a$ in $G$ is the set of elements of $G$ that fix $a$: $G_a = \{g \in G | g \cdot a = a\}$. An action is \emph{faithful} if its kernel is the identity.
\end{definition}

\begin{theorem}
Let $G$ be a group acting on the non-empty set $A$. The relation on $A$ defined by $a \sim b$ iff $a = g \cdot b$ for some $g \in G$ is an equivalence relation.
\end{theorem}

Proof is omitted. Based on this theorem, we see that a group $G$ acting on the set $A$ partitions $A$ into disjoint equivalence classes. This leads to the following definitions.

\begin{definition}
An orbit is defined as the set $\{ g \cdot a | g \in G\}$ for fixed $a$ and all $g \in G$.
\end{definition}

Intuitively, the orbit of an element $a \in A$ is the set of elements in $A$ to which $a$ can be moved by all elements of $G$.

\begin{definition}
The action of $G$ on $A$ is called \emph{transitive} if there is only one orbit; ie given any two elements $a, b \in A$, there is some $g \in G$ such that $a = g \cdot b$.
\end{definition}

For example, the symmetric group $S_n$ is transitive on $A = \{1,2,\ldots, n\}$ as any element of $A$ can be mapped to any other element of $A$ by $S_n$. See in Gap,

\begin{verbatim}
gap> Orbit(SymmetricGroup(3),1);
[ 1, 2, 3 ]
gap> Orbit(SymmetricGroup(3),2);
[ 2, 3, 1 ]
gap> Orbit(SymmetricGroup(3),3);
[ 3, 1, 2 ]
\end{verbatim}

Things are a bit more interesting when we consider a subgroup of $S_3$, eg $G = \{(), (1,2)\}$. The orbit of $1$ is 

\bee
() \cdot 1 = 1, \quad (1,2) \cdot 1 = 2
\eee

The orbit of $2$ is

\bee
() \cdot 2 = 2, \quad (1,2) \cdot 2 = 1
\eee

The orbit of $3$ is

\bee
() \cdot 3 = 3, \quad (1,2) \cdot 3 = 3
\eee

We can also do this in Gap as follows,

\begin{verbatim}
gap> g:=Group((),(1,2));
Group([ (), (1,2) ])
gap> Orbit(g,1);
[ 1, 2 ]
gap> Orbit(g,2);
[ 2, 1 ]
gap> Orbit(g,3);
[ 3 ]
gap> Orbits(g,[1,2,3]);
[ [ 1, 2 ], [ 3 ] ]
\end{verbatim}

We see that the orbits partition the group $S_3$. This is in line with the following theorem.

\begin{theorem}
The orbits of $G$ form a partition of $S$. In particular, $S$ is the disjoint union of the orbits.
\end{theorem}

Proof is omitted.

We can ask for the group elements $g$ which do not change an element $a$; ie ($g \cdot a = a$).

\begin{definition}
A \emph{stabilizer} $G_a$ is a set of group elements which map $a$ onto itself,

\bee
G_a = \{g \in G: g \cdot a = a\}
\eee
\end{definition}

In our running example of the subgroup group $(), (1,2)$, the stabilizers can be obtained from Gap as follows,

\begin{verbatim}
gap> g:=Group((),(1,2));
Group([ (), (1,2) ])
gap> Stabilizer(g,1);
Group(())
gap> Stabilizer(g,2);
Group(())
gap> Stabilizer(g,3);
Group([ (), (1,2) ])
\end{verbatim}

After some length discussions omitted here, we arrive at \emph{Cayley's Theorem}.

\begin{theorem}
Every group is ismorphic to  a subgroup of some symmetric group. If $G$ is a group of order $n$, then $G$ is isomorphic to a subgroup of $S_n$.
\end{theorem}

\subsection{Groups acting on Themselves by Conjugation}

We now consider a group $G$ acting on itself; ie $A = G$ by \emph{conjugation}. Conjugation $\cdot$ is defined according to

\bee
g \cdot a = g a g^{-1}, \quad \forall g \in G, a \in G
\eee

The conjugation operation satisfies the two axioms for a group action because

\bee
g_1 \cdot (g_2 \cdot a) = g_1 \cdot (g_2 a g_2^{-1}) = g_1(g_2 a g_2^{-1})g_1^{-1} = (g_1 g_2)a(g_1 g_2)^{-1} = (g_1 g_2) \cdot a
\eee

and

\bee
1 \cdot a = 1 a 1^{-1} = a
\eee

for all $g_1, g_2 \in G$ and all $a \in G$. Note that we need to distinguish between the group action $\cdot$ (which is conjugation) and the group operation $g_1 g_2$.

Let's calculate the conjugates of $(1,2)$ in $S_3$ in Gap.

\begin{verbatim}
gap> g:=SymmetricGroup(3);
Sym( [ 1 .. 3 ] )
gap> e:=Elements(g);
[ (), (2,3), (1,2), (1,2,3), (1,3,2), (1,3) ]
gap> e[1]*(1,2)*e[1]^(-1);
(1,2)
gap> e[2]*(1,2)*e[2]^(-1);
(1,3)
gap> e[3]*(1,2)*e[3]^(-1);
(1,2)
gap> e[4]*(1,2)*e[4]^(-1);
(1,3)
gap> e[5]*(1,2)*e[5]^(-1);
(2,3)
gap> e[6]*(1,2)*e[6]^(-1);
(2,3)
\end{verbatim}

We see that conjugation changes the elements of the permutation but not the $2$-cycle.

\begin{definition}
Two elements $a, b \in G$ are conjugate in $G$ if there is some $g \in G$ such that $b = gag^{-1}$. The orbits of $G$ acting on itself by conjugation are called the conjugacy classes of $G$.
\end{definition}

\paragraph{Example.} We want to show that the two permutations $(1,2)$ and $(1,3)$ are conjugate. According to the definition, there must be a $g \in S_3$ such that $(1,2) = g(2,3)g^{-1}$. 

Let's write the two permutations below each other to see the differences more clearly

\begin{align*}
&(1,2)(3) \\
&(2,3)(1)
\end{align*}

So we need the following mappings $1 \rightarrow 2, 2 \rightarrow 3, 3 \rightarrow 1$ and this is equivalent to $(1,2,3)$. We can check this in gap as follows.

\begin{verbatim}
    gap> (1,2,3)*(2,3)*(1,2,3)^(-1);
    (1,2)
    gap> IsConjugate(SymmetricGroup(3), (1,2), (1,3));
    true
\end{verbatim}

The \emph{cycle type} is defined as the number of $1-$cycles, $2-$cycles, and so on. From the above example, we can conclude that we can change the elements in the permutation cycles (eg replace a $2$ with a $3$), but we cannot change the cycle structure itself (eg convert a $2$-cycle into a $3$-cycle). This is made precise in the following theorem.

\begin{theorem}
Two permutations $\sigma$ and $\tau$ are conjugate iff they have the same cycle structure.
\end{theorem}

Proof is omitted. Based on this, we can write down the conjugacy classes of $S_n3$.

\paragraph{Example.} In $S_3$, the conjugacy classes are $\{(1)\}$, $\{(1,2)\}$, and $\{(1,2,3)\}$.

We can check in Gap:

\begin{verbatim}
    gap> ConjugacyClasses(SymmetricGroup(3));
    [ ()^G, (1,2)^G, (1,2,3)^G ]
\end{verbatim}

Intuitively, the conjugacy classes contain "related" permutations: In the above example, one class contains the permutations with only fixed-points (represented by $\{(1)\}$), another class contains the permutations with one 2-cycle (represented by $\{(1,2)\}$), and finally one class contains the permutations with one 3-cycle (represented by $\{(1,2,3)\}$).

The number of elements in each conjugacy class is further analysed in entry \ref{2022-04-25:entry}.

%%% Local Variables:
%%% mode: latex
%%% TeX-master: "journal"
%%% End:
