\DiaryEntry{Group Actions}{2023-08-30}{Algebra}

Let $G$ be a group acting on an non-empty set $A$. Then for each $g \in G$, the map

\bee
\sigma_g: A \rightarrow A, \quad \sigma_g: a \rightarrow g \cdot a
\eee

is a permutation of $A$. In addition, there is also a map $\phi$ from $G$ to $S_A$ defined by $g \rightarrow \sigma_g: \phi(g) = \sigma_g$. This map can be shown to be a homomorphism and is called the permutation representation associated to the given action.

There are a couple of definitions associated with the permutation representation.

\begin{definition}
The \emph{kernel} of the action is the set of elements of $G$ that act trivially on every element of $A$: $\{g \in G | g \cdot a = a \forall a \in A\}$. For each $a \in A$, the \emph{stabilizer} of $a$ in $G$ is the set of elements of $G$ that fix $a$: $G_a = \{g \in G | g \cdot a = a\}$. An action is \emph{faithful} if its kernel is the identity.
\end{definition}

\begin{theorem}
Let $G$ be a group acting on the non-empty set $A$. The relation on $A$ defined by $a \sim b$ iff $a = g \cdot b$ for some $g \in G$ is an equivalence relation.
\end{theorem}

Proof is omitted. Based on this theorem, we see that a group $G$ acting on the set $A$ partitions $A$ into disjoint equivalence classes. This leads to the following definitions.

\begin{definition}
An orbit is defined as the set $\{ g \cdot a | g \in G\}$ for fixed $a$ and all $g \in G$.
\end{definition}

Intuitively, the orbit of an element $a \in A$ is the set of elements in $A$ to which $a$ can me moved by all elements of $G$.

\begin{definition}
The action of $G$ on $A$ is called \emph{transitive} if there is only one orbit; ie given any two elements $a, b \in A$, there is some $g \in G$ such that $a = g \cdot b$.
\end{definition}

For example, the symmetric group $S_n$ is transitive on $A = \{1,2,\ldots, n\}$ as any element of $A$ can be mapped to any other element of $A$ by $S_n$. Things are a bit more interesting when we consider a subgroup of $S_3$, eg $G = \{(), (1,2)\}$. The orbit of $1$ is 

\bee
() \cdot 1 = 1, \quad (1,2) \cdot 1 = 2
\eee

The orbit of $2$ is

\bee
() \cdot 2 = 2, \quad (1,2) \cdot 2 = 1
\eee

The orbit of $3$ is

\bee
() \cdot 3 = 3, \quad (1,2) \cdot 3 = 3
\eee

We see that the orbits partition the group $S_3$. This is in line with the following theorem.

\begin{theorem}
The orbits of $G$ form a partitionof $S$. In particular, $S$ is the disjoint union of the orbits.
\end{theorem}

Proof is omitted.

After some length discussions omitted here, we arrive at \emph{Cayley's Theorem}.

\begin{theorem}
Every group is ismorphic to  a subgroup of some symmetric group. If $G$ is a group of order $n$, then $G$ is isomorphic to a subgroup of $S_n$.
\end{theorem}

\subsection{Groups acting on Themselves by Conjugation}

We now consider a group $G$ acting on itself; ie $A = G$ by \emph{conjugation}. Conjugation $\cdot$ is defined according to

\bee
g \cdot a = g a g^{-1}, \quad \forall g \in G, a \in G
\eee

The conjugation operation satisfies the two axioms for a group action because

\bee
g_1 \cdot (g_2 \cdot a) = g_1 \cdot (g_2 a g_2^{-1}) = g_1(g_2 a g_2^{-1})g_1^{-1} = (g_1 g_2)a(g_1 g_2)^{-1} = (g_1 g_2) \cdot a
\eee

and

\bee
1 \cdot a = 1 a 1^{-1} = a
\eee

for all $g_1, g_2 \in G$ and all $a \in G$. Note that we need to distinguish between the group action $\cdot$ (which is conjugation) and the group operation $g_1 g_2$.

\begin{definition}
Two elements $a, b \in G$ are conjugate in $G$ if there is some $g \in G$ such that $b = gag^{-1}$. The orbits of $G$ acting on itself by conjugation are called the conjugacy classes of $G$.
\end{definition}

\paragraph{Example.} We want to show that the two permutations $(1,2)$ and $(1,3)$ are conjugate. According to the definition, there must be a $g \in S_3$ such that $(1,2) = g(2,3)g^{-1}$. 

Let's write the two permutations below each other to see the differences more clearly

\begin{align*}
&(1,2)(3) \\
&(2,3)(1)
\end{align*}

So we need the following mappings $1 \rightarrow 2, 2 \rightarrow 3, 3 \rightarrow 1$ and this is equivalent to $(1,2,3)$. We can check this in gap as follows \todo{check in Gap}

The \emph{cycle type} is defined as the number of $1-$cycles, $2-$cycles, and so on. From the above example, we can conclude that we can change the elements in the permutation cycles, but not the cycle structure. This is made precise in the following theorem.

\begin{theorem}
Two permutations $\sigma$ and $\tau$ are conjugate iff they have the same cycle structure.
\end{theorem}

Proof is omitted. Base on this, we can write down the conjugacy classes of $S_n3$.

\paragraph{Example.} In $S_3$, the conjugacy classes are $\{(1)\}$, $\{(1,2)\}$, and $\{(1,2,3)\}$.

Intuitively, the conjugacy classes contain "related" permutations: In the above example, one class contains the permutations with only fixed-points (represented by $\{(1)\}$), another class contains the permutations with one 2-cycle (represented by $\{(1,2)\}$), and finally one class contains the permutations with one 3-cycle (represented by $\{(1,2,3)\}$).

The number of elements in each conjugacy class is further analysed in entry \ref{2022-04-25:entry}.

%%% Local Variables:
%%% mode: latex
%%% TeX-master: "journal"
%%% End:
