\DiaryEntry{Rings - Integral Domains, 2}{2016-07-26}{Algebra}

The integers are an integral domain, but this definition is not
sufficient, as there are many integral domains which behave rather
differently than the integers. In order to define the integers (and only
the integers), we need to add some more axioms to the integral domain
definitions.

\subsection{Ordered Integral Domain}\label{ordered-integral-domain}

An ordered integral domain is an integral domain with a relation \(<\)
having the following properties:

\begin{itemize}
\item
  For any \(a, b \in A\), one of the following is true: \(a=b\),
  \(a<b\), \(a>b\)
\item
  If \(a<b\) and \(b<c\), then a\textless{}c\$
\item
  If \(a<b\), then \(a+c<b+c\)
\item
  If \(a<b\), then \(ac<bc\) if \(0<c\)
\end{itemize}

The relation allows a definition of positive and negative elements of A.
Furthermore, the square of every non-zero element is positive: If
\(0<c\), then \(0c < c^2 \rightarrow 0<c^2\); if \(c<0\), then
\(0<-c \rightarrow 0(-c) < (-c)^2 = c^2 \rightarrow 0 < c^2\).

The set of all positive elements in \(A\) is denoted as \(A^+\), and an
integral domain is called an integral system if every non-empty subset
of \(A^+\) has a least (smallest) element. This is called the
well-ordering property of \(A^+\).

\(\mathbb{Z}\) is an integral system; e.g.~the least element of all even
numbers is 2. On the contrary, \(\mathbb{Q}\) is not an integral system;
e.g.~the subset \(0<x<1\) has no least element (for every fraction
\(m/n\), there always exists a smaller fraction).

In any integral system there is no element between 0 and 1. Proof by
contradiction - assume that \(0<x<1\). Then the set
\(\{x \in A, 0<x<1\}\) is a non-empty positive set so it has a least
element \(c\) by the well-ordering property. So \(0<c<1\) and if we
multiply with \(c\), we obtain \(0<c^2<c\). We can now combine the 2
inequalities and obtain \(0<c^2<c<1\). This does not seem to be possible
ands we have a contradiction\ldots{}??

In any integral system, all elements are multiplies of 1 and ordered
exactely as in \(\mathbb{Z}\). Therefore, every integral system is
isomorphic to \(\mathbb{Z}\). From this we conclude that any two
integral systems are isomorphic to eah other and therefore
\(\mathbb{Z}\) is the *only\$ integral system (the others are all
isomorphisms).

\subsection{Induction}\label{induction}

The proof by induction method is based on a simple fact of the positive
integers. If \(\mathbb{K}\) is a set of positive integers, and

\begin{itemize}
\item
  if \(1 \in \mathbb{K}\), and
\item
  for any positive \(k \in \mathbb{K}\), then \(k+1 \in \mathbb{K}\)
\end{itemize}

If \(\mathbb{K}\) satisfies these conditions, then \(\mathbb{K}\)
consists of all the positive integers.

Assume we have a statement \(S_n\) about a positive integer \(n\).
Induction then works by starting with \(S_1\) being true, and deducing
that \(S_{n+1}\) is true when \(S_n\) is true. If these conditions hold,
then \(S_n\) is true for every positive integer.

\subsection{Division Algorithm}\label{division-algorithm}

If \(m,n\) are integers and \(n\) is positive, there exist unique
integers \(p,q\) such that

\[
m=nq + r , \quad 0 \leq r < n
\]
