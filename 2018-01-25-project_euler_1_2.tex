\DiaryEntry{Project Euler Problem 1, 2 - Analytic Result}{2018-01-25}{Project Euler}

\subsection{Problem 1}

The task is to find the sum of all the multiples of $3$ and $5$ below $1000$.

\paragraph{Solution.} The multiples of $3$ are $3,6,9,\cdots,999$ and the corresponding sum $S_3$ can be written as follows

\bee
S_3 = \sum_{k=1}^{333} 3k = 3 \sum_{k=1}^{333} k = 3 \frac{1}{2} 333 \cdot 334 = 166833
\eee

where we have used the sum formula for arithmetic series $\sum_{k=1}^N k = 1/2 N (N+1)$. We can do the same for the multiples of $5$ with their sum $S_5$ given as

\bee
S_5 = \sum_{k=1}^{199} 5k = 5 \sum_{k=1}^{199} k = 5 \frac{1}{2} 199 \cdot 200 = 99500
\eee

Be careful about the wording; we shall sum multiples \emph{below} $1000$; i.e.\ not including $1000$.

Summing $S_3$ and $S_5$ does not yield the solution, as both sums also contain multiples of the smallest common multiple of $3$ and $5$ which is $15$. So just adding up the two sums includes the multiples of $15$ two times. We can correct this by calculating the sum of the multiples of $15$,

\bee
S_{15} = \sum_{k=1}^{66} 15k = 15 \sum_{k=1}^{66} k = 15 \frac{1}{2} 66 \cdot 67 = 33165
\eee

The solution then becomes $S_3 + S_5 - S_{15} = 233168$.

\subsection{Problem 2}

This problem deals with the sum of the even Fibonacci numbers.

\paragraph{Solution.} Let us write down the Fibonacci numbers $F_n = F_{n-1} + F_{n-2}$:

\bee
1,1,2,3,5,8,13,21,34,\cdots
\eee

From this we see that every fourth Fibonacci number is even. This is explained, because only the sum of two odd numbers is even. Starting with $1,1$, the third Fibonacci number is therefore even, the fourth Fibonacci number is odd (sum of an even and an odd number), as is the fifth. For the sixth Fibonacci number we add two odd numbers and it is therefore even again. This argument can be repeated for higher numbers.

The sum of the first $N$ even Fibonacci numbers, $S_N$, is therefore

\bee
S_N = F_3 + F_6 + F_9 + \cdots + F_{3N}
\eee

We can calculate this sum using a closed-form expression for the Fibonacci numbers which is

\bee
F_n = \frac{1}{\sqrt{5}} \left( A^n  - B^n\right)
\eee

with $A = (1+\sqrt{5})/2, \, B = (1-\sqrt{5})/2$. Therefore, 

\bee
S_N = F_3 + F_6 + F_9 + \cdots + F_{3N} = \frac{1}{\sqrt{5}} \sum_{n=1}^{N} \left( A^{3n}  - B^{3n} \right) = \frac{1}{\sqrt{5}} \sum_{n=1}^{N} \left( (A^3)^n  - (B^3)^n \right)
\eee

where in the second equation we have brought this into a form $\sum_k a^k$ which can be summed with the geometric sum series. Because $B<1$, we can approximate the closed-form Fibonacci expression for large $n$ to

\be\label{eq2018-01-25:approx_fib}
F_n \approx \frac{1}{\sqrt{5}} A^n
\ee

and

\be\label{eq2018-01-25:approx_fib_sum}
S_N \approx \frac{1}{\sqrt{5}} \sum_{n=1}^{N} (A^3)^n = \frac{1}{\sqrt{5}} \frac{A^3-A^{3(N+1)}}{1-A^3}
\ee

where we used the sum formula

\bee
\sum_{n=1}^N q^n = \frac{q-q^{N+1}}{1-q}
\eee

Sidenote: We can (of course) also use \eqref{eq2018-01-25:approx_fib} also use to approximately calculate the sum of the first $N$ Fibonacci numbers, $s_N$:

\bee
s_N = \sum_{n=1}^N F_n \approx \frac{1}{\sqrt{5}} \sum_{n=1}^N A^n = \frac{1}{\sqrt{5}} \frac{A - A^{N+1}}{1-A}
\eee

We can compare the approximation quality in the following table:

\bee
\begin{array}{c||cc|cc|ccc}
	n  & F_n & F_n (approx) & S_N (exact) & S_N (approx) & s_N (exact) & s_N (approx) &  \\
	1  &  1  &   0.723607   & 2           & 1.89443      & 1           & 0.723607     &  \\ \hline
	2  &  1  &   1.17082    & 10          & 9.91935      & 2           & 1.89443      &  \\ \hline
	3  &  2  &   1.89443    & 44          & 43.9135      & 4           & 3.78885      &  \\ \hline
	4  &  3  &   3.06525    & 188         & 187.915      & 7           & 6.8541       &  \\ \hline
	5  &  5  &   4.95967    & 798         & 797.915      & 12          & 11.8138      &  \\ \hline
	6  &  8  &   8.02492    & 3382        & 3381.91      & 20          & 19.8387      &  \\ \hline
	7  & 13  &   12.9846    & 14328       & 14327.9      & 33          & 32.8233      &  \\ \hline
	8  & 21  &   21.0095    & 60696       & 60695.9      & 54          & 53.8328      &  \\ \hline
	9  & 34  &   33.9941    & 257114      & 2.57114e5    & 88          & 87.8269      &  \\ \hline
	10 & 55  &   88.9978    & 1089154     & 1.08915e6    & 143         & 142.831      &
\end{array} 
\eee

The approximations become really good, even for small values of $n$.