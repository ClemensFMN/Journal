\DiaryEntry{Project Euler 6 - Analytical Result}{2018-02-07}{Maths}

The problem asks for a comparison of the squared sum of the first $N$ natural numbers and the first $N$ squared natural numbers.

We have

\bee
\sum_{k=1}^N k = \frac{N(N+1)}{2}, \quad \sum_{k=1}^N k^2 = \frac{N(N+1)(2N+1)}{6}
\eee

We want to compare

\bee
\left( \sum_{k=1}^N k \right)^2 = \frac{N^2(N+1)^2}{4} \sim N^4
\eee

and

\bee
\sum_{k=1}^N k^2 \sim N^3
\eee

Comparison in Julia shows us that

\bee
\left( \sum_{k=1}^N k \right)^2 > \sum_{k=1}^N k^2
\eee

for all values $N$ as shown in the table below

\bee
\begin{array}{cccccc}
	\hline
	           k            & 1 & 2 & 3  &  4  & 5   \\ \hline
	        \sum k          & 1 & 3 & 6  & 10  & 15  \\ \hline
	\left( \sum k \right)^2 & 1 & 9 & 36 & 100 & 225 \\ \hline
	          k^2           & 1 & 4 & 9  & 16  & 25  \\ \hline
	       \sum k^2         & 1 & 5 & 14 & 30  & 55  \\ \hline
\end{array}
\eee

Intuitively, $\sum k$ grows quadratically in $N$, therefore the square grows with $N^4$. On the other hand, the summands $k^2$ grow quadratically, therefore their sum grows only with $N^3$.

By the way, we have the following bound

\bee
\sum_{k=1}^N k^n \leq \sum_{k=1}^N N^n = N N^n = N^{n+1}
\eee

i.e. the sum of $k^n$ scales according to a power law with exponent $n+1$. A more general result is obtained by using results from \ref{2015-10-29:entry}; in particular

\bee
\sum_{0 \leq x < N} x^{\underline{n}} = \frac{x^{\underline{n+1}}}{n+1} \bigg|_0^N = \frac{N^{\underline{n+1}}}{n+1}
\eee

Here, $x^{\underline{n}}$ has terms up to $x^n$ and the sum expression contains terms up to $N^{n+1}$. \qed