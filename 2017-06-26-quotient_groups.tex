\DiaryEntry{Quotient Groups, I}{2017-06-26}{Algebra}

This (series of) posts is based on Abstract Algebra by Dummit and Foote and digs a little bit deeper and is more formal than previous posts.

\begin{definition}[Fibers]
Assume we have a function $f$ which maps elements from a set $A$ to a set $B$. For each element $b \in B$, the set $f^{-1}(b) = \{a \in A | f(a) = b\}$ is denoted the fiber of $f$ over $b$.
\end{definition}


\begin{definition}[Homomorphism]
Let $G$ and $H$ be groups. A map $\phi: G \rightarrow H$ such that
\bee
\phi(x y) = \phi(x) \phi(y) \quad \forall x,y \in G
\eee
is called a homomorphism. Note that the operation $xy$ is defined on the group $G$, whereas the operation $\phi(x) \phi(y)$ is defined on $H$.
\end{definition}

Stated informally, a homomorphism preserves the group structure of $G$ while it is mapped to $H$.

\begin{definition}[Kernel]
The kernel of a homomorphism $\phi: G \rightarrow H$ is defined as the set

\bee
\text{ker}(\phi) = \{g \in G | \phi(g) = 1\}
\eee
\end{definition}

In the following, let $G$ and $H$ be groups and let $\phi$ be a homomorphism. We have the following properties:

\begin{enumerate}

\item $\phi(1_G) = 1_H$, where $1_G, 1_H$ are the identity elements of $G$ and $H$, respectively.

\item $\phi(g^{-1}) = \phi(g)^{-1}$

\item $\phi(g^n) = \phi(g)^n$, for all $n \in mZ$.

\item $\text{ker}(\phi)$ is a subgroup of $G$.

\item The image of $\phi$ under $G$ is a subgroup of $H$.

\end{enumerate}

\paragraph{Proof.} For the first one, note that $\phi(1_G) =\phi(1_G 1_G) = \phi(1_G) \phi(1_G)$. We can sneak in a $1_H$ and get $1_H \phi(1_G) =  \phi(1_G) \phi(1_G)$, cancel $\phi(1_G)$ and get $1_H = \phi(1_G)$. \qed

For the second one, start with $\phi(1_G) = \phi(gg^{-1}) = \phi(g) \phi(g^{-1})$ and use $\phi(1_G) = 1_H$, so we have $1_H = \phi(g) \phi(g^{-1})$. Multiplying both sides on the left with $\phi(g)^{-1}$ we obtain $\phi(g)^{-1} = \phi(g^{-1})$. \qed

The proof for the third part is not shown here (can be done via induction).

For (4), note that $1_G \in \text{ker}(\phi)$, so the kernel is not empty. Let $x,y \in \text{ker}$, so $\phi(x) = \phi(y) = 1_H$. We need to show that $\phi(xy^{-1}) = 1_H$: $:\phi(x)\phi(y^{-1}) = 1_H \phi(y)^{-1} = 1_H$ wich proves that $\text{ker}(\phi) \leq G$. \qed

The proof of (5) is omitted. \qed

It is interesting to see that such a "simple" condition like $\phi(x y) = \phi(x) \phi(y)$ has such big consequences: Having a homomorphism (and finding the kernel) allows to construct a subgroup. Also note that there can (and probably will) be more than one homomorphism exist for a group. Each homomorphism will have its own kernel (and subgroup).

Based on what we have, we can make the following definition. 

\begin{definition}[Quotient Group]
Let $\phi$ be a homomorphism with kernel $K$. The quotient group or factor group $G/K$ is the group whose elements are the fibers of $\phi$ with group operation defined as follows: If $X$ is the fiber above $a$ and $Y$ is the fiber above $b$, then the product $XY$ is the fiber above the product $ab$.
\end{definition}

I'm not sure whether we have plready proven "enough" that this is clear.

We can interpret the kernel $K$ as a single element in the group $G/K$ and the other elements of $G/K$ are just translates of $K$. This will be made more exact in the following by defining cosets.


If $\phi: G \rightarrow K$ is a homomorphism with kernel $K$ and let $X \in G/K$ be the fiber above $a$; i.e. $X = \phi^{-1}(a)$. The we have
\begin{enumerate}
	\item For any $u \in X, X = \{uk | k \in K\}$

	\item For any $u \in X, X = \{ku | k \in K\}$
\end{enumerate}

\paragraph{Proof.} If $u \in X$, then $\phi(u) = a$. Then $\phi(uk) = \phi(u) \phi(k) = \phi(u) 1 = a$ because $k \in \text{ker}(\phi)$. This shows that $uk \in X$ and therefore $uK \subseteq X$. Next the reverse inclusion: Suppose $g \in X$ and $k = u^{-1}g$. Then $\phi(k) = \phi(u^{-1}g) = \phi(u^{-1}) \phi(g) = \phi(u)^{-1} \phi(g) = a^{-1}a = 1$. Therefore $k \in \text{ker}(\phi)$. Since $k=u^{-1}g, g = uk \in uK$, we have that $X \subseteq uK$. This shows that $X = uK$ for any $u \in X$. \qed

These sets are called left and right cosets. Note that they are defined for any subgroup $K$ of a group $G$ and not only for kernels of homomorphisms.

\begin{definition}[Cosets]
For any subgroup $N \leq G$ and any $g \in G$, the sets

\bee
gN = \{gn | n \in N \}, \quad Ng = \{ng | n  \in N\}
\eee

are called the left and right cosets of $N$ in $G$, respectively. Any coset element is called a representative of the coset.
\end{definition}

With that in place, we can state the following theorem: 

\begin{theorem}
Let $G$ be a group and let $K$ be the kernel of a homomorphism from $G$ to another group. Then the set whose elements are the left cosets of $K$ in $G$ with operation defined by

\bee
uK vK = (uv)K
\eee

forms a group called $G/K$. The group operation is well-defined in the sense that for any coset member $u_1 \in uK$ and any coset member $v_1 \in vK$, then $u_1 v_1 \in uvK$. In other words, the group operation does \emph{not} depend on the coset representative. The whole theorem holds true for right cosets as well.
\end{theorem}

It is important to note that the group operation does not depend on the coset representative. The multiplication of two cosets $X$ and $Y$ is the coset $uvK$ containing the the product $uv$ where $u,v$ are any representatives for the cosets $X$ and $Y$, respectively.

We can interpret the members of the cosets as being equivalent in the sense that the group operation does not depend on the coset representative. In this light, the term quotient group makes sense, as the structure of the (normal) subgroup is divided out. What remains is the structure between the cosets.

\paragraph{Proof.} Let $X,Y \in G/K$ and let $Z = XY$ in $G/K$, so $X,Y,Z$ are left cosets of $K$. Since $K$ is a kernel, $X = \phi^{-1}(a)$ and $Y = \phi^{-1}(b)$ for some $a,b \in H$. Then $Z = \phi^{-1}(ab)$. Let $u$ and $v$ be arbitrary representatives of $X$ and $Y$, respectively. The $\phi(u) = a, \phi(v) = b, X = uK, Y = vK$. We must show that $uv \in Z: uv \in \phi^{-1}(ab) \rightarrow \phi(uv) = ab \rightarrow \phi(u)\phi(v) = ab$. Therefore, $uv \in Z$, therefore $Z$ is the left coset $uvK$. This proves that the product of $X$ and $Y$ is the coset $uvK$ for any choice of representatives $u \in X, v \in Y$.

We denote with $\bar{u}$ the coset $uK$ with representative $u$. Then the quotient group $G/K$ is denoted by $\bar{G}$ and the product of elements $\bar{u}$ and $\bar{v}$ is the coset containing $uv$ which we denote by $\bar{uv}$.


By the theorem above, if we have a kernel $K$ of some homomorphism, we may define the quotient $G/K$ without using the homomorphism by the multiplication $uKvK = uvK$. We can ask whether this works for \emph{any} subgroup (and not only for kernels of homomorphisms). The answer is no: We will show that a group structure on a subgroup $N$ can be defined \emph{if and only if} $N$ is the kernel of a homomorphism. An equivalent statement is that the subgroup $N$ must be a normal subgroup.


