\DiaryEntry{ARMA Models, 1}{2021-09-28}{Coding}

ARMA stands for \emph{autoregressive moving average} and describes a class of random processes. An LTI system with $p$ poles and $q$ zeros is excited by a random time-discrete signal $u[n]$ and outputs a time-discrete signal $s[n]$ according to

\bee
s[n] = \sum_{k=1}^p a_k s[n-k] + \sum_{k=0}^q b_k u[n-k]
\eee

The corresponding transmission function is given by

\bee
H(z) = \frac{B(z)}{A(z)} = \frac{ \sum_{k=0}^q b_k z^{-k} }{ 1 + \sum_{k=1}^p a_k z^{-1} }
\eee

If the input signal is white noise with variance $\sigma_u^2$, the output power spectrum is given by

\bee
P(e^{j \omega}) = \sigma_u^2 \frac{|B(e^{j \omega})|^2}{|A(e^{j \omega})|^2}
\eee

The process is known as ARMA process of order $(p,q)$.


Interesting question(s) are how to obtain the autocorrelation function of the output signal $s_n$.


\subsection{Moving Average Process}

We first consider the moving average process. It is given by the following relation,

\bee
s[n] = \sum_{k=0}^q b_k u[n-k]
\eee


which is a weighted average of white noise. The autocorrelation function $r[k]$ is defined as 

\bee
r[k] = E\left\{ s[n] s[n+k] \right\}
\eee

where the expectation is taken over $n$. For the MA model, the expression inside the expectation is calculated as follows

\begin{align*}
&\left(\sum_{i=0}^q b_i u[n-i] \right) \left( \sum_{j=0}^q b_j u[n+k-j] \right) \\
= &\left( b_0 u[n] + b_1 u[n-1] + \cdots + b_q u[n-q] \right) \\
  &\left( b_0 u[n+k] + b_1 u[n+k-1] + \cdots + b_q u[n+k-q] \right)
\end{align*}

The noise is white; i.e. $E(u[n] u[n-k]) = \sigma_u^2 \delta[n-k]$. This allows us to simplify above double sum, as $E(u[n-i]u[n+k-j])$ will only yield a non-zero value for $n-i = n+k-j$. Simplifying this yields $j = k+i$ and the double sum collapses into a single sum as

\bee
r[k] = \sigma_u^2 \sum_i b_i b_{k+i}
\eee

The signal is multiplied with a shifted version of itself and then summed over. It is symmetric (that's what we expect from an autocorrelation function anyway) and zero outside $[-q;q]$. We can also express $r[k]$ as convolution of $b[n]$ with a time-reversed version of itself,

\bee
r[k] = \sigma_u^2 b_n \star b_{-n}
\eee

where we use $\star$ to denote convolution.


\subsection{Autoregressive Process}

This process is defined as


\begin{equation}\label{2021-09-29:eq1}
s[n] = \sum_{k=1}^p a_k s[n-k] + u[n]
\end{equation}

It is a weighted sum of $p$ past signals $s[n-k]$ and an additive noise term. Calculating the autocorrelation function is not as easy as before, we will obtain the \emph{Yule-Walker equations} from which we can obtain the autocorelation function.

We first note the relation

\begin{align*}
E\{ u[n] s[n] \} &= E\left\{ u[n] \left( a_1 s[n-1] + a_2 s[n-2] + \cdots + a_p s[n-p] + u[n] \right) \right\} \\
&= E\left\{ u[n] u[n] \right\} = \sigma_u^2
\end{align*}

as all previous signals $s[n-k], k>0$ are independent from the noise at time $n$. We therefore have

\bee
E\{ u[n] s[n-k] \} = 0, \qquad k > 0
\eee

Let's multiply \eqref{2021-09-29:eq1} with $s[n-k]$ for different values of $k=0,1, \ldots$ and take expectations. For $k=0$ we obtain

\begin{align*}
r[0] = E\left\{ s[n] s[n] \right\} &= E\left\{ s[n] \left( \sum_{k=1}^p a_k s[n-k] + u[n] \right) \right\} \\
&= E\left\{ s[n] \left( a_1 s[n-1]) + a_2 s[n-2] + \cdots + a_p s[n-p] + u[n] \right) \right\} \\
&= a_1 r[1] + a_2 r[2] + \cdots a_p r[p] + \sigma_u^2
\end{align*}

where we have used the fact that the autocorrelation is symmetric; i.e. $r[k] = r[-k]$. For $k = 1$ we obtain

\begin{align*}
r[1] = E\left\{ s[n] s[n+1] \right\} &= E\left\{ s[n] \left( \sum_{k=1}^p a_k s[n+1-k] + u[n+1] \right) \right\} \\
&= E\left\{ s[n] \left( a_1 s[n]) + a_2 s[n-1] + \cdots + a_p s[n+1-p] + u[n+1]) \right) \right\} \\
&= a_1 r[0] + a_2 r[1] + \cdots a_p r[p-1]
\end{align*}

There is a pattern and we can write the equations together,

\begin{align*}
r[0] &= a_1 r[1] + a_2 r[2] + \cdots a_p r[p] + \sigma_u^2 \\
r[1] &= a_1 r[0] + a_2 r[1] + \cdots a_p r[p-1] \\
r[2] &= a_1 r[-1] + a_2 r[0] + \cdots a_p r[p-2] \\
\cdots & \cdots
\end{align*}

These are the Yule-Walker equations and they can be used to calculate the autocorrelation function.

\paragraph{Order-1 AR Model.} In this case we have

\bee
s[n] = a_1 s[n-1] + u[n]
\eee

and the Yule-Walker equations become

\begin{align*}
r[0] &= a_1 r[1] + \sigma_u^2 \\
r[1] &= a_1 r[0] \\
r[2] &= a_1 r[-1] \\
\cdots & \cdots
\end{align*}

Using symmetry of the autocorrelation, we obtain

\begin{align*}
r[0] &= a_1 r[1] + \sigma_u^2 \\
r[1] &= a_1 r[0] \\
r[2] &= a_1 r[1] \\
\cdots & \cdots
\end{align*}

The value of $r[0]$, which is also the energy of the signal, can be calculated as follows

\bee
r[0] = E(y^2[n]) = a^2_1 E(y^2[n-1]) + \sigma_u^2
\eee

and since the process is stationary, the energy stays constant. We therefore have

\bee
r[0] = a^2_1 r[0] + \sigma_u^2 \rightarrow r[0] = \frac{\sigma_u^2}{1 - a^2_1}
\eee

From this, we can use the Yule-Walker equations to calculate the autocorrelation function and obtain

\bee
r[k] = \frac{a^k_1 \sigma_u^2}{1 - a^2_1}, \qquad k \geq 0
\eee


\paragraph{Order-2 AR Model.} In this case we have

\bee
s[n] = a_1 s[n-1] + a_2 s[n-2] + u[n]
\eee


and the Yule-Walker equations become

\begin{align*}
r[0] &= a_1 r[1] + a_2 r[2] + \sigma_u^2 \\
r[1] &= a_1 r[0] + a_2 r[1] \\
r[2] &= a_1 r[-1] + a_2 r[0] \\
\cdots & \cdots
\end{align*}

Using symmetry, we arrive at the following three equations

\begin{align*}
r[0] &= a_1 r[1] + a_2 r[2] + \sigma_u^2 \\
r[1] &= a_1 r[0] + a_2 r[1] \\
r[2] &= a_1 r[1] + a_2 r[0]
\end{align*}

Let's rewrite the system in terms of the unknown $r[n]$ values,

\begin{align*}
r[0] - a_1 r[1] - a_2 r[2] &= \sigma_u^2 \\
-a_1 r[0] + r[1] (1- a_2) &= 0 \\
- a_2 r[0] - a_1 r[1] + r[2] &= 0
\end{align*}

We can solve this system and obtain values for $r[0], r[1], r[2]$. To calculate the autocorrelation function for higher lags, we use the relation $r[2] = a_1 r[-1] + a_2 r[0] = r[2] = a_1 r[1] + a_2 r[0]$ and deduce the following relation from it

\bee
r[n] = a_1 r[n-1] + a_2 r[n-2], \qquad n > 1
\eee

\qed

Two things are interesting to note: First, the autocorrelation function of AR models does not become zero as it does outside the filter length for MA models. Second, the AR filter coefficients control the autocorrelation - but this works in the other direction as well: The $p$ filter coefficients control the first $p$ values of the autocorrelation function; afterwards, the autocorrelation function is determined by the expression $r[n] = a_1 r[n-1] + \cdots + a_p r[n-p], n>p$. If we want to have a signal with a defined autocorrelation (at least $p$ values), we can use the Yule-Walker equations to determine the filter coefficients.


\subsection{ARMA Process}

If we combine autoregressive and moving average, we obtain an ARMA model as stated at the beginning,

\be\label{2021-09-29:eq2}
s[n] = \sum_{k=1}^p a_k s[n-k] + \sum_{k=0}^q b_k u[n-k]
\ee

Calculating the autocorrelation follows the same principle as with the AR model; we multiply \eqref{2021-09-29:eq2} with $y[n-k]$ for various values of $k$ and take the expectation.

\paragraph{ARMA(2,2) Model.} We have $s[n] = a_1 s[n-1] + b_0 u[n] + b_1 u[n-1]$ and start by multiplying with $s[n]$to obtain

\bee
r[0] = a_1 E(s[n-1] s[n]) + b_0 E(u[n] s[n]) + b_1 E(u[n-1] s[n]) = a_1 r[1] + (b_0 + a_1 b_1 + b_1) b_1 \sigma_u^2
\eee

where we have used $E(u[n-1] s[n]) = (a_1 +1) b_1 \sigma_u^2$. In a similar spirit, we obtain

\bee
r[1] = a_1 r[0] + b_1^2 \sigma_u^2 \qed
\eee

We can use the first $p+q+1$ equations to obtain $r[0],\ldots,r[p+q]$. The next equation defines $r[p+q+2]$ in terms of autocorrelations with smaller lag and we can extend this equation to calculate the "remaining" functional values.

