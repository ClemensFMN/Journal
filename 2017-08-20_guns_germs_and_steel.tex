\DiaryEntry{Guns, Germs, and Steel}{2017-08-20}{Misc.}

\paragraph{Intro.} I read the book; nevertheless, most part of the summary is stolen from \href{https://en.wikipedia.org/wiki/Guns,_Germs,_and_Steel}{Wikipedia}.

\paragraph{Summary.} The book attempts to explain why Eurasian and North African civilizations have been "successful" compared to others. An interesting example he brings is the Spanish conquest of the Aztecs, where a hundred Spaniards stood (and won) against ten thousands of Aztecs.

He argues that this success is not due to any form of Eurasian intellectual, moral, or inherent genetic superiority, but the gaps in power and technology between human societies originate primarily in environmental differences, which are amplified by various positive feedback loops.

Civilization is not created out of superior intelligence, but is the result of a chain of developments, each made possible by certain preconditions.

\subsection{Agriculture}

The first step towards civilization is the move from nomadic hunter-gatherer to rooted agrarian society. Several conditions are necessary for this transition to occur: access to high-carbohydrate vegetation that endures storage; a climate dry enough to allow storage; and access to animals docile enough for domestication and versatile enough to survive captivity. Control of crops and livestock leads to food surpluses. Surpluses free people to specialize in activities other than sustenance and support population growth. The combination of specialization and population growth leads to the accumulation of social and technological innovations which build on each other. Large societies develop ruling classes and supporting bureaucracies, which in turn lead to the organization of nation-states and empires.

Although agriculture arose in several parts of the world, Eurasia gained an early advantage due to the greater availability of suitable plant and animal species for domestication. In particular, Eurasia has barley, two varieties of wheat, and three protein-rich pulses for food; flax for textiles; and goats, sheep, and cattle. Eurasian grains were richer in protein, easier to sow, and easier to store than American maize or tropical bananas.

\subsection{Trade and Animals}

As early Western Asian civilizations began to trade, they found additional useful animals in adjacent territories, most notably horses and donkeys for use in transport. Diamond identifies 13 species of large animals over 100 pounds (45 kg) domesticated in Eurasia, compared with just one in South America (counting the llama and alpaca as breeds within the same species) and none at all in the rest of the world. Australia and North America suffered from a lack of useful animals due to extinction, probably by human hunting, shortly after the end of the Pleistocene, whilst the only domesticated animals in New Guinea came from the East Asian mainland during the Austronesian settlement some 4,000–5,000 years ago. Sub-Saharan biological relatives of the horse including zebras and onagers proved untameable; and although African elephants can be tamed, it is very difficult to breed them in captivity; Diamond describes the small number of domesticated species (14 out of 148 "candidates") as an instance of the Anna Karenina principle: many promising species have just one of several significant difficulties that prevent domestication.

Eurasians domesticated goats and sheep for hides, clothing, and cheese; cows for milk; bullocks for tillage of fields and transport; and benign animals such as pigs and chickens. Large domestic animals such as horses and camels offered the considerable military and economic advantages of mobile transport.

\subsection{Geography}

Eurasia's large landmass and long east-west distance increased these advantages. Its large area provided it with more plant and animal species suitable for domestication, and allowed its people to exchange both innovations and diseases. Its east-west orientation allowed breeds domesticated in one part of the continent to be used elsewhere through similarities in climate and the cycle of seasons. The Americas had difficulty adapting crops domesticated at one latitude for use at other latitudes (and, in North America, adapting crops from one side of the Rocky Mountains to the other). Similarly, Africa was fragmented by its extreme variations in climate from north to south: crops and animals that flourished in one area never reached other areas where they could have flourished, because they could not survive the intervening environment. Europe was the ultimate beneficiary of Eurasia's east-west orientation: in the first millennium BCE, the Mediterranean areas of Europe adopted Southwestern Asia's animals, plants, and agricultural techniques; in the first millennium CE, the rest of Europe followed suit.

\subsection{Consequences}

The plentiful supply of food and the dense populations that it supported made division of labor possible. The rise of nonfarming specialists such as craftsmen and scribes accelerated economic growth and technological progress. These economic and technological advantages eventually enabled Europeans to conquer the peoples of the other continents in recent centuries by using the guns and steel of the book's title.

Eurasia's dense populations, high levels of trade, and living in close proximity to livestock resulted in widespread transmission of diseases, including from animals to humans. Smallpox, measles, and influenza were the result of close proximity between dense populations of animals and humans. Natural selection forced Eurasians to develop immunity to a wide range of pathogens. When Europeans made contact with the Americas, European diseases (to which Americans had no immunity) ravaged the indigenous American population, rather than the other way around (the "trade" in diseases was a little more balanced in Africa and southern Asia: endemic malaria and yellow fever made these regions notorious as the "white man's grave";[4] and syphilis may have originated in the Americas). The European diseases – the germs of the book's title – decimated indigenous populations so that relatively small numbers of Europeans could maintain their dominance.

Diamond also proposes geographical explanations for why western European societies, rather than other Eurasian powers such as China, have been the dominant colonizers, claiming Europe's geography favored balkanization into smaller, closer, nation-states, bordered by natural barriers of mountains, rivers, and coastline. Threats posed by immediate neighbours ensured governments that suppressed economic and technological progress soon corrected their mistakes or were outcompeted relatively quickly, whilst the region's leading powers changed over time. Other advanced cultures developed in areas whose geography was conducive to large, monolithic, isolated empires, without competitors that might have forced the nation to reverse mistaken policies such as China banning the building of ocean-going ships. Western Europe also benefited from a more temperate climate than Southwestern Asia where intense agriculture ultimately damaged the environment, encouraged desertification, and hurt soil fertility.