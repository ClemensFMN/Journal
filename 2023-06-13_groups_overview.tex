\DiaryEntry{Group Theory}{2023-06-13}{Algebra}

The existing entries about group theory date back from 2016 and have various issues: (i) They are a bit too informal (as they are based on a introductory textbook), the topics repeat themselves and are spread across several entries (which themsevles are rather short in general). On a technical level, the entries were written in Pelican and have been converted to LaTex; some formatting issues arise from this conversion.

The idea is to rewrite and restructure the entries; I'm not sure whether that much is missing; the whole exercise is more about restructuring things.

I want to write new entries instead of rewriting the existing ones. The main reason is to preserve the history of the Journal.

\subsection{Introduction}

Groups are binary operations defined on a set. In addition, the set is closed under the binary operation. Let's start with some definitions.

\begin{definition}
A binary operation $\star$ on a set $G$ is a function $G \times G \rightarrow G$. For any $a, b \in G$ we write $a \star b$ when we combine them via this operation. 
\end{definition}


\begin{definition}
A group is an ordered pair $(G, \star)$ of a set $G$ and a binary operation $\star$. It satisfies the following axoims:

\begin{itemize}

\item The operation is associatvie; ie $(a \star b) \star c = a \star (b \star c)$ for all $a, b, c \in G$.
\item There exists an identity element $e \in G$ such that $a \star e = e \star a = a$ for all $a \in G$.
\item For each $a \in G$, there exists an inverse element $a^{-1}$ such that $a \star a^{-1} = e$.

\end{itemize}

The group is called abelian or commutative, if $a \star b = b \star a$ for all $a, b \in G$.

\end{definition}


Examples for groups are the numbers $\mZ$ with addition $+$ as binary operation. The addition is known / defined to be associative, the identity element is $0$, and the inverse for an element $a \in \mZ$ is given by $-a$. Another example are the rational numbers $\mQ$ with multiplication $\times$: The identity element is $1$ (as $1 \cdot a = a \cdot 1 = a$) and the inverse of $a$ is $\frac{1}{a}$.

A bit more tricky things happen when we consider addition modulo-$n$. Addition $\mod-n$ on a set of integers $G = \{0, 1, \cdots, n-1\}$ yields a group denoted as $\mZ / n \mZ$ with identitiy element $0$ and inverse element $n-a \equiv -a \mod n$. The multiplication table for $n = 5$ has the following structure.


\bee
\begin{array}{c|ccccc}
\star & 0 & 1 & 2 & 3 & 4 \\ \hline
0     & 0 & 1 & 2 & 3 & 4 \\
1     & 1 & 2 & 3 & 4 & 0 \\
2     & 2 & 3 & 4 & 0 & 1 \\
3     & 3 & 4 & 0 & 1 & 2 \\
4     & 4 & 0 & 1 & 2 & 3 \\
\end{array}
\eee

The first line repeats (in a shifted manner) in the lines below; this is a chracteristic of a \emph{cyclic group}; in addition, such a cyclic group can be created by repeatedly applying the group operation on the \emph{generator element}. The element $1$ is such a generator element and indeed we have

\bee
0 \star 1 = 1, 1 \star 1 = 2, 2 \star 1 = 3, 3 \star 1 = 4, 4 \star 1 = 0
\eee


As a last example, we again consider a set of integers $G = \{0, 1, \cdots, n-1\}$ but now with multiplication $\mod-n$ instead. The element $1$ acts as identitiy element $a \star 1 = 1 \star a = a$, but there is no inverse element for the element $0$: $a \star 0 = 0$ for all $a$. The way out of this is to remove the element $0$; ie we consider $G = \{1, 2, \cdots, n-1\}$ instead.

In the table below, the multiplication table of $G = \{1, 2, 3\}$ under multiplication $\mod-4$ is shown.

\bee
\begin{array}{c|ccc}
\star & 1 & 2 & 3 \\ \hline
1     & 1 & 2 & 3 \\
2     & 2 & 0 & 2 \\
3     & 3 & 2 & 1 \\
\end{array}
\eee

We can see that $0 \notin G$ appears in the second row and $2$ does not have an inverse - this is not a group! From the number theory entries and in particular theorm \ref{2020-11-25:th2}, we know that $a x \equiv 1 \mod n$ has a solution iff $\gcd(a,n) = 1$. So there are two ways to continue: (i) We remove all those elements $a$ from $G$ for which $\gcd(a,n) \neq 1$ or (ii) we consider only sets $G$ with prime cardinality (thereby trivially fulfilling the $\gcd(a,n) = 1$ condition).

Continuing way (i) in above example, we see that $2$ has $\gcd(2,4) = 2$ and we therefore must remove it from $G$. This results in the following multiplication table.

\bee
\begin{array}{c|cc}
\star & 1 & 3 \\ \hline
1     & 1 & 3 \\
3     & 3 & 1 \\
\end{array}
\eee

The alternative is to consider a set with a prime number of elements; "closest" is $n=5$ and for $G = \{1,2,3,4\}$ we have

\bee
\begin{array}{c|cccc}
\star & 1 & 2 & 3 & 4 \\ \hline
1     & 1 & 2 & 3 & 4 \\
2     & 2 & 4 & 1 & 3 \\
3     & 3 & 1 & 4 & 2 \\
4     & 4 & 3 & 2 & 1
\end{array}
\eee

No elements other than $G$ occur and every element has an inverse, so we have a group. We also see that the first row is not repeated (neither in a shifted manner), so this does not appear to be a cyclic group.



%%% Local Variables:
%%% mode: latex
%%% TeX-master: "journal"
%%% End:
