\DiaryEntry{Field Extensions, I}{2016-08-05}{Algebra}

If F is a field, then a subfield K is a non-empty subset of F which is
closed with respect to addition, subtraction, multiplication, and
division. In other words, K is a field in its own right.

If we reverse our point of view, we can also say that F is a field
extension of K.

If F is an arbitrary field, there are, in general, polynomials over F
which have no roots in F. More exactely, every polynomial of degree n
has n roots, but they \textbf{need not} necesarily be in F. However, one
can show that the n roots are contained in a suitable extension E of F.

We can group the elements of E into two groups: \emph{Algebraic}
elements of E are elements which are roots of non-zero polynomials in
F{[}x{]}. \emph{Transcendental} elements of E are all the others;
i.e.~elements which are not roots of any polynomials in F{[}x{]}.

If we consider \(\mathbb{Q}\), then an extension field is
\(\mathbb{R}\): The polynomial \(x^2-2\) has no roots in \(\mathbb{Q}\);
however \(\pm \sqrt{2} in \mathbb{R}\). So for this polnomial,
\(\mathbb{R}\) is a suitable extension field. However, it is not the
``minimum'' extension field; i.e.~it contains elements which are not
roots of \(x^2-2\), e.g. \(\sqrt{3}\). The ``minimum'' extension field
would be \(\mathbb{Q}(\sqrt{2})\).

Transcendental elements are \(\pi\); there is no polynomial in
\(\mathbb{Q}[x]\) which has \(\pi\) as its root (non-trivial statement).

\subsubsection{Substitution Function / Evaluation
Homomorphism}\label{substitution-function-evaluation-homomorphism}

For now consider a field F with extension field E.

We define a substitution function (also called evaluation homomorphism)
\(\sigma_c\) for a polynomial s\(a(x) \in F[x]\) as
\(\sigma_c(a(x)) = a(c)\) with \(c \in E\); i.e.~it replaces the free
variable \(x\) with the constant value \(c\). This function maps
elements from \(F[x]\) onto E and is a homomorphism:
\(\sigma_c(a(x) + b(x)) = \sigma_c(a(x)) + \sigma_c(b(x))\) and
\(\sigma_c(a(x)b(x)) = \sigma_c(a(x)) \sigma_c(b(x))\).

The kernel of \(\sigma_c\) is the set of all polynomials \(a(x)\) so
that \(a(c) = \sigma_c(a(x)) = 0\). In other words, the kernel contains
all polynomials \(a(x)\) having a root \(c\). Let us denote the kernel
of \(\sigma_c\) by \(J_c\). \(J_c\) is an ideal; moreover in \(F[x]\)
every ideal is a principal ideal and so \(J_c\) is also a principal
ideal. Therefore \(J_c = \langle p(x) \rangle\) is the set of all
multiples of \(p(x)\). \(p(x)\) is the polynomial with lowest degree of
all nonzero polynomials in \(J_c\) and \(p(x)\) is irreducible. We can
further constrict \(p(x)\) to be a monic polynomial and all together
\(p(x)\) is an irreducible monic polynomial of lowest degree in \(J_c\)
with root \(c\). It is called the minimum polynomial.

As an example consider the substitution function \(\sigma_{\sqrt{2}}\);
i.e.~the function which substitutes \(\sqrt{2}\) into \(x\). E.g.
\(\sigma_{\sqrt{2}}(x^2-1) = \sqrt{2}^2-1 = 1\). By the previous
discussion, the kenel of the homomorphism \(\sigma_{\sqrt{2}}\) contains
all polynomials with one of the roots being \(\sqrt{2}\). The minimum
polynomial is \(p(x) = x^2-1\): It is monic, nonzero, and is has lowest
degree of all polynomials which have \(\sqrt{2}\) as one of their roots.
The ideal \(J_c = \langle x^2-1 \rangle\) is the set of all polynomials
\(J_c = \{(x^2-1) a(x), a(x) \in F[x]\}\). From this definition we can
see that all elements from this ideal have a root at \(x=c\) (the first
factor is zero).

The range of \(\sigma_c\) is closed with respect to addition,
multiplication, and negatives (this follows from \(\sigma_c\) being a
homomorphism. However, it is also closed to multiplicative inverses
(this is evident and needs to be proven).

Therefore, the range of \(\sigma_c\) is a subfield of E and is given as
the set of all elements \(a(c)\) for all \(a(x) \in F[x]\). It is also
the \emph{smallest field containing both F and c}. By smallest we mean
the field which contains F and c and is contained in any other field
containing F and c. It is called the \emph{field generated by F and c}
and is denoted by \(F(c)\). By the fundamental homomorphism theorem,
\(F(c)\) is isomorphic to \(F[x] / \langle p(x) \rangle\).

Summarizing, if \(p(x)\) is an irreducible polynomial in \(F[x]\), then
\(F(c)\) is a suitable field extension to F; i.e.~it contains a root of
\(p(x)\) and \(F(c)\) is isomorphic to \(F[x] / \langle p(x) \rangle\).

\subsection{Vector Spaces and Field
Extensions}\label{vector-spaces-and-field-extensions}

Let F and K be fields and K an extension field over F. We may consider
the elements of K as vectors, whereas the elements of F are scalars.
Adding elements of K amounts to vector addition, adding and multiplying
elements of F can be thought in terms of scalar addition and
multiplication. Multiplying an element of F with one of K can be thought
of multiplying a scalar with a vector.

Especially interesting is the case of vector spaces with finite
dimension. If K - as a vector space- is of finite dimension, we call K a
finite extension of F. If the vector space dimension of K is n, then K
is an extension of degree n over F. We write this as

\[
[K:F] = n
\]

If c is algebraic over F and let \(p(x)\) be the minimum polynomial of c
over F. Let the degree of \(p(x)\) be n. Then, the elemnts

\[
1, c, c^2, \ldots, c^{n-1}
\]

are linearly independent and span \(F(c)\).

Proof: Any \(a(c)\) is an element of \(F(c)\), we can divide \(a(x)\) by
\(p(x)\): \(a(x) = p(x) q(x) + r(x)\) with the degree of \(r(x)\) less
or equal \(n-1\). Therefore, \(a(c) = p(c) q(c) + r(c) = r(c)\) because
- by the very definition - \(p(c) = 0\). This shows that every element
of F(c) can be written in the form of

\[
a_0 + a_1 c + a_2 c^2 + \cdots + a_{n-1} c^{n-1}
\]

which is a linear combination of \(1, c, c^2, \ldots, c^{n-1}\);
i.e.~these elements form a basis for \(F(c)\). To prove that
\(1, c, c^2, \ldots, c^{n-1}\) are linearly independent, suppose that
\(a_0 + a_1 c + a_2 c^2 + \cdots + a_{n-1} c^{n-1} = 0\). If the \(a_i\)
coefficients were not all zero, c would be the root of a polynomial of
degree n-1 or less, which is impossible because the minimum polynomial
of c over F has degree n. Thus the \(a_i\) must be all zero and
therefore the \(1, c, c^2, \ldots, c^{n-1}\) are independent.

\subsubsection{Example}\label{example}

Consider \(\mathbb{Q}(\sqrt{2})\) with \(\sqrt{2}\) not being a root of
any monic polynomial of degree 1 over \(\mathbb{Q}\). This would be the
polynomial \(x-\sqrt{2}\) and is not in \(\mathbb{Q}[x]\). But,
\(\sqrt{2}\) is the root of \(x^2-2\) which is therefore the minimal
polynomial over \(\mathbb{Q}\) and therefore
\([\mathbb{Q}(\sqrt{2}) : \mathbb{Q}] = 2\).

The elements \(1, \sqrt{2}\) form a basis for \(\mathbb{Q}(\sqrt{2})\);
i.e.~we can write every element of this field as
\(a + b\sqrt{2}, ab,b \in \mathbb{Q}\).

\subsection{Connection between Field Extensions and
Degree}\label{connection-between-field-extensions-and-degree}

If the field E is a finite extension of K and K is a finite extension of
F, then E is also a finite extension of F and we have

\[
[E:F] = [E:K] [K:F]
\]

If \(a_1, a_2, \ldots, a_m\) is a basis of the vector space K over F and
\(b_1, b_2, \ldots, b_n\) is a basis of the vector space E over K, then
the set of mn products \(\{a_i b_j\}\) is a basis for the vector space E
over F.

If c is algebraic over F, then we say that F(c) is constructed from F by
\emph{adjoining} c to F. If both c and d are algebraic over F, then we
can first adjoin F with c, obtain F(c), and then adjoin d to F(c) to
obtain F(c,d). The order of the adjoin operations does not matter. The
resulting F(c,d) then contains all roots c and d.

\subsubsection{Example}\label{example-1}

If we expand the example of \(\mathbb{Q}(\sqrt{2})\) with \(\sqrt{3}\),
we observe that \(\sqrt{3}\) cannot be expressed in
\(\mathbb{Q}(\sqrt{2})\): \(\sqrt{3} = a + b\sqrt{2}\) can not be
fulfilled with \(a,b \in \mathbb{Q}\). However, \(\sqrt{3}\) is a root
of \(x^2-3\) which is therefore a minimum polynomial in
\(\mathbb{Q}(\sqrt{2})\). Therefore \(\mathbb{Q}(\sqrt{2}, \sqrt{3})\)
is therefore degree 2 over \(\mathbb{Q}(\sqrt{2})\) and therefore
\(\mathbb{Q}(\sqrt{2}, \sqrt{3})\) is of degree 4 over \(\mathbb{Q}\).
The elements \(\{1, \sqrt{2}\}\) are a basis for
\(\mathbb{Q}(\sqrt{2})\) and the elements \(\{1, \sqrt{3}\}\) are a
basis for \(\mathbb{Q}(\sqrt{2}, \sqrt{3})\). Therefore,
\(\{1, \sqrt{2}, \sqrt{3}, \sqrt{2 \cdot 3} = \sqrt{6}\}\) is a basis
for \(\mathbb{Q}(\sqrt{2}, \sqrt{3})\).

If we consider \(x^3-2\), then \(\sqrt[3]{2}\) is the root of the
minimial polynomial \(x^3-2 = 0\). The field extension therefore has
degree 3 and \(\{1, \sqrt[3]{2}, \sqrt[3]{2}^2\}\) forms a basis.

Things get a bit more tricky if we consider the roots of polynomials
over non-\(\mathbb{Q}\) or non-\(\mathbb{Z}\) fields. For example,
consider \(a(x) = x^2+x+1\) over \(\mathbb{Z}_2\). We have
\(a(0) = 1, a(1)=1\) and therefore no roots in \(\mathbb{Z}_2\). If we
denote a root of \(a(x)\) by \(\alpha\), we have \(\alpha^2+\alpha+1=0\)
and we can construct a field extension using the basis
\(\{1, \alpha\}\). Addition and multiplication tables therefore have the
following form

\[
\begin{array}{c|cccc}
+  &       0        & 1          & \alpha     & 1+\alpha \\
\hline
0 &        0        & 1          & \alpha     & 1+\alpha \\
1 &        1        & 0          & 1 + \alpha & \alpha   \\
\alpha &   \alpha   & 1 + \alpha & 0          & 1        \\
1+\alpha & 1+\alpha & \alpha     & 1          & 0        \\
\end{array}
\]

wherer we have used the fact that \(1 + 1 = 0\) and
\(\alpha + \alpha = \alpha (1+1) = 0\). The multiplication table looks
as follows

\[
\begin{array}{c|cccc}
\times  &    0        & 1        & \alpha     & 1+\alpha \\
\hline
0 &        0        & 0        & 0          & 0 \\
1 &        0        & 1        & \alpha     & 1 + \alpha   \\
\alpha &   0        & \alpha   & 1 + \alpha & 1        \\ 
1+\alpha & 0        &1+\alpha &  1  &        \alpha      \\
\end{array}
\]

where we used the fact that \(\alpha^2 + \alpha +1 =0\). From this we
deduce \(\alpha^2=-1-\alpha = 1 + \alpha\) because \(-1 \mod 2 =1\) and
also
\((1+\alpha)^2 = 1 + 2\alpha + \alpha^2 = 1 + \alpha^2 = 1 + 1 + \alpha = \alpha\).
