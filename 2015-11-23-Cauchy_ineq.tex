\DiaryEntry{Cauchy Inequality - Application}{2015-11-23}{Maths}

\subsubsection{Inner Product}

We can interpret the \(a_k\) as elements of a vector. In this case, all inequalities extend to the vector case; for example, the Cauchy
inequality becomes

\[
<\mathbf{a},\mathbf{b}> \leq |\mathbf{a}|^2 |\mathbf{b}|^2
\]

where \(\mathbf{a}\) denotes a vector and \(<\mathbf{a},\mathbf{b}>\)
denotes the inner product between two vectors \(\mathbf{a}\) and
\(\mathbf{b}\)

The second inequality becomes

\[
<\mathbf{a},\mathbf{b}> \leq \frac{1}{2} |\mathbf{a}|^2 + \frac{1}{2} |\mathbf{b}|^2
\]

\subsubsection{Bound on \(\sum 1/k\)}

Set the sequence $b_n=1$ (for all $n$), and we obtain the following inequality

\[ \sum_k a_k \leq \sqrt{N} \sqrt{\sum_k a_k^2} \]

As a simple example consider $a\_n = \frac{1}{n}$ and we have

\[ \sum_k \frac{1}{k} \leq \sqrt{N} \sqrt{\frac{\pi^2}{6}} \]

Note that the RHS divergers, but this does \textbf{not} imply that also the LHS diverges. There could well be another expression
$X$ which is smaller than the RHS, but which converges. Existence of this $X$ would then allow to deduce the convergence of the LHS.

\subsubsection{Splitting}

If we have a sum $\sum_k a_k $ we can split the summand $a_k$ as follows: $a\_k = |a_k|^{1/3} |a_k|^{2/3}$. Applying Cauchy's
inequality to this sum, we have

\[
\sum_k a_k = \sum_k |a_k|^{1/3} |a_k|^{2/3} \leq \left( \sum_k |a_k|^{2/3} \right)^{1/2} \left( \sum_k |a_k|^{4/3} \right)^{1/2}
\]
