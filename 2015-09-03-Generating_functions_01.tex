\DiaryEntry{Generating Functions, Part 1}{2015-09-09}{GenFuncs}


Based on Concrete Mathematics.

\subsection{Definition}\label{definition}

The generating function of a time-discrete sequence $\{g\_n\}, n=0\ldots{}\infty$ has the following form

\[ G(z) = z^0 g_0 + z^1 g_1 + z^2 g_2 + \cdots = \sum_{n \geq 0} g_n z^n\]

\subsubsection{Application to Recurrence Relations}

The most prominent usage of generating functions is to find closed-form expressions for recurrence relations.

\begin{itemize}
\item
  First write down the sequence \({g_n}, n=0...\infty\); i.e.~express
  \$g\_n \$ in terms of other elements of the sequence. This expression
  shall be valid for \emph{all} n, assuming that \(g_n=0\) for
  \(n < 0\).
\item
  Multiply both sides of the equation with z\^{}n and sum over all n. On
  the left side of the equation this gives \(G(z)\); the right side
  becomes ``something else''.
\item
  Find a closed form expression for \(G(z)\) and expand this expression
  into a power series. Read off the coefficients of \(z^n\).
\end{itemize}

\subsection{Simple Examples}

\subsubsection{\texorpdfstring{\(g_n=\alpha^n\)}{g\_n=\textbackslash{}alpha\^{}n}}

We have a recurrence relation \(g_n=\alpha g_{n-1}\) with the inital
condition \(g_0=1\).

We have

\[\sum_n g_n z^n = \sum_n \alpha g_{n-1} z^n + 1\]

which we can simplify to

\[G(z) = \alpha \sum_n g_{n} z^{n+1} + 1 = \alpha z \sum_n g_n z^n +1 = \alpha z G(z) + 1\]

This yields the following expression for the generating function

\[ G(z) = \frac{1}{1-\alpha z} \]

and corresponds to the sequence $g_n = \alpha^n$.

\subsubsection{\texorpdfstring{\(g_n=-\alpha^n\)}{g\_n=-\textbackslash{}alpha\^{}n}}\label{g_n-alphan}

We have (with the initial condition \(g_0=1\))

\[\sum_n g_n z^n = -\alpha \sum_n g_{n-1} z^n + 1\]

which yields for the generating function $G(z) = \frac{1}{z+\alpha}$. We can rewrite it as follows:

\[G(z) = \frac{1}{\alpha} \frac{1}{1 + x/\alpha} = \frac{1}{\alpha} \frac{1}{1 - x/(-\alpha)} \]

This function corresponds to the sequence \(g_n = (-\alpha)^n\).

\subsection{Fibonacci Numbers Example}\label{fibonacci-numbers-example}

Define \(g_n\) as

\[ g_n = g_{n-1} + g_{n-2} + [n=1] \]

where the expression \([n=1]\) is one when n=1 and zero otherwise. This
is an expression for \(g_n\) valid for all n (positive and negative).

Executing steps \#2, we obtain

\[ \sum_n g_n z^n = \sum_n g_{n-1} z^n + \sum_n g_{n-2} z^n  + \sum_n [n=1] z^n \]

Keeping in mind that this epxression is valid for \emph{all} n, we can
shift the summation indices in order to obtain

\[ \sum_n g_n z^n = \sum_n g_{n} z^{n+1} + \sum_n g_{n} z^{n+2}  + \sum_n [n=1] z^n \]

The last sum yields \(z\) and we finally obtain

\[ \sum_n g_n z^n = z \sum_n g_{n} z^n + z^2 \sum_n g_{n} z^n + z \]

This form allows a closed-form expression for the generating function as

\[ G(z) = z G(z) + z^2 G(z) + z \]

and we arrive at

\[ G(z) = \frac{z}{1-z-z^2} \]

We can make a partial fraction expansion to get a closed-form expression
or we throw the expression at \texttt{sympy}, Wolfram Alpha etc to get a
series expansion.

With sympy, this works as follows:

\begin{verbatim}
import sympy as sym

x, n = sym.symbols("x n")
F = x / (1-x-x**2)
res = sym.series(F, x, 0, 10)

sym.pprint(res)
\end{verbatim}

and results in

\begin{verbatim}
     2      3      4      5      6       7       8       9    / 10\
x + x  + 2*x  + 3*x  + 5*x  + 8*x  + 13*x  + 21*x  + 34*x  + O\x  /
\end{verbatim}

The coefficients are the Fibonacci numbers with
\(g_0=0, g_1=1, g_2=1, g_3=2 \ldots\).

To see the effect of the inital condition, we can start with \(g_0=1\);
in this case we have

\[ g_n = g_{n-1} + g_{n-2} + [n=0] \]

and

\[ \sum_n g_n z^n = z \sum_n g_{n} z^n + z^2 \sum_n g_{n} z^n + 1 \]

Therefore, the generating function becomes $G(z) = \frac{1}{1-z-z^2}$ and a series expansion yields

\begin{verbatim}
           2      3      4      5       6       7       8       9    / 10\
1 + x + 2*x  + 3*x  + 5*x  + 8*x  + 13*x  + 21*x  + 34*x  + 55*x  + O\x  /
\end{verbatim}

These are the Fibonacci numbers again, but shifted by to the right by one.
