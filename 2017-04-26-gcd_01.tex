\DiaryEntry{GCD}{2017-04-26}{Algebra}

\subsection{Basics}

\begin{definition}
The greatest common divisor (GCD) of two numbers is defined the largest integer that divides both of the numbers. If the greatest common divisor is 1, this means that there are no prime factors in common. We say the numbers are coprime in this case.
\end{definition}

In GAP, we can calculate the gcd of two numbers as follows

\begin{verbatim}
gap> GcdInt(3,4);
1
gap> GcdInt(3,6);
3
gap> GcdInt(8,12);
4
\end{verbatim}

\begin{theorem}
The Greatest Common Divisor Theorem. Given two positive integers $x$ and $y$, the greatest common divisor of $x$ and $y$ is the smallest positive integer which can be expressed in the form

\bee
ux + vy
\eee

with $u$ and $v$ being integers.
\end{theorem}

We can do that with GAP as follows

\begin{verbatim}
gap> Gcdex(8,12);
rec( coeff1 := -1, coeff2 := 1, coeff3 := 3, coeff4 := -2, gcd := 4 )
\end{verbatim}

We have $-1 \times 8 + 1 \times 12 = 4$ as in the theorem above; however, we get some more information in that $(-1 + 3) \times 8 + (1 - 2) \times 12 = 4$.

\paragraph{Proof.} Denote the set of positive numbers which can be expressed in the form $ux + vy$ as $\Ac$ and denote the smallest such number as $n$. Since $\gcd(x,y)$ is a factor of both $x$ and $y$, it must also be a factor of $n$. Next, consider $k \equiv x (\mod n)$ which must fulfill $0 \leq k < n$ and therefore $k = x + nr$ for some number $r$. 

We can now insert the expression for $n$ and obtain $k = x + (ux + vy)r = (1+ru)x + (rv)y$ for some numbers $u, v$. This shows that $k \in \Ac$. 

We assumed $n$ to be the smallest number in $\Ac$, $k$ cannot be equivalent $\mod n$ to any number less than $n$, other than $0$. Therefore, $x \equiv 0 \mod n$ and $n$ is a divisor of $x$. Similar reasoning yields that also $n$ is a divisor of $y$. Thus, $n$ is a common divisor of $x,y$ abd since $\gcd(x,y)$ is also a divisor of $n$, $n = \gcd(x,y)$. \qed

\subsection{Connections to Groups}

In previous posts, we considered finite groups (consisting of the elements $1 \ldots n$) with group operation to be multiplication modulo $n$. One interesting question is whether a group element has an inverse or not. The following theorem answers this question.

\begin{theorem}
For $n$ being a positive integer, then a group element $x$ ($0 \leq x < n$) has a multiplicative inverse modulo-n, iff $x$ is coprime to $n$.
\end{theorem}

\paragraph{Proof.} If $x$ and $n$ are not coprime, they have a common factor which we call $p$. In order for $x$ to have a mutliplicative inverse, there must be a $y$ such that

\bee
x y \equiv 1 (\mod n)
\eee

This means $xy = 1 + wn$ for some integer $w$. But since $xy$ is a multiple of $p$, $xy$ can not also be a multiple of $1 + wn$.
