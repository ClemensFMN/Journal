\DiaryEntry{Kalman Filter, 1}{2019-06-06}{Stochastic}

\subsection{Introduction}

The entry \ref{2017-02-01:entry} considered the posterior distribution of Gaussian RVs. In particular, a random vector $\xbf$ has distribution

\bee
p(\xbf) = \Nc(\xbf|\hat\xbf, \Sigmabf)
\eee

The random vector $\ybf$ is observed according to

\bee
\ybf = \Gbf \xbf + \vbf, \quad \vbf \sim \Nc(\zerobf, \Rbf)
\eee

and therefore has the following conditional distribution

\bee
p(\ybf | \xbf) = \Nc(\ybf | \Gbf \xbf, \Rbf)
\eee

The posterior distribution of $\xbf | \ybf$ is then given according to

\bee
p(\xbf | \ybf) = \Nc(\xbf | \hat\xbf + \Sigmabf \Gbf^T (\Gbf \Sigmabf \Gbf^T + \Rbf )^{-1} (\ybf - \Gbf \hat\xbf), \Sigmabf - \Sigmabf \Gbf^T (\Gbf \Sigmabf \Gbf^T + \Rbf)^{-1} \Gbf \Sigmabf
\eee

The posterior mean is a mixture of the prior $\hat\xbf$ and the scaled difference $(\ybf - \Gbf \hat\xbf)$. In the extreme case of perfect prior knowledge $\Sigmabf \rightarrow \zerobf$, we have

\bee
\lim_{\Sigmabf \rightarrow \zerobf} \hat\xbf + \Sigmabf \Gbf^T (\Gbf \Sigmabf \Gbf^T + \Rbf )^{-1} (\ybf - \Gbf \hat\xbf) = \hat\xbf
\eee

i.e. the prior information alone is considered. In the other extreme case of erfect observation $\Rbf \rightarrow \zerobf$, we have

\bee
\lim_{\Rbf \rightarrow \zerobf} \hat\xbf + \Sigmabf \Gbf^T (\Gbf \Sigmabf \Gbf^T + \Rbf )^{-1} (\ybf - \Gbf \hat\xbf) = \hat\xbf + \Sigmabf \Gbf^T (\Gbf \Sigmabf \Gbf^T)^{-1} (\ybf - \Gbf \hat\xbf) = \hat\xbf + \Gbf^{-1} (\ybf - \Gbf \hat\xbf) = \Gbf^{-1} \ybf
\eee

i.e. the observation alone is considered.


\subsection{Kalman Filter}

The Kalman filter solves the problem of estimating a sequence of hidden RVs, $\{ \xbf_k \}$ from a sequence of (noisy) observations, $\{ \ybf_k \}$. The sequence of $\{ \xbf_k \}$ is described by a state space equation,

\bee
\xbf_{k+1} = \Abf \xbf_k + \wbf_{k+1}, \quad \wbf_k \sim \Nc(\zerobf, \Qbf)
\eee

and the "model noise" $\wbf_k$ being independent for different $k$. At time $k=1$, the system is in an initial state, given by the prior $\xbf_1 \sim \Nc(\hat \xbf_1, \Sigmabf_1)$. The observations are given by

\bee
\ybf_k = \Gbf \xbf_k + \vbf, \quad \vbf \sim \Nc(\zerobf, \Rbf)
\eee

with the observation noise $\vbf$ being independent for different $k$.

Upon observation of $\ybf_1$, the Kalman filter treats its knowledge of $\hat \xbf_1$ and $\Sigmabf_1$ as prior and calculates the posterior density which is a Gaussian with mean $\hat\xbf_1^F$ and covariance matrix $\Sigmabf_1^F$. This is called the \emph{filtering step} of the filter. The posterior density is then used in the \emph{prediction step} to form an estimate for $\xbf_2$ in the form of the mean $\hat\xbf_2^P$ and covariance matrix $\Sigmabf_2^P$.

After observing $\ybf_2$, the filter uses the predictions $\hat\xbf_2^P$ and $\Sigmabf_2^P$ as prior and performs a filtering and prediction step. The filter continues this process.



%%% Local Variables:
%%% mode: latex
%%% TeX-master: "journal"
%%% End:
