\DiaryEntry{Rings - Integral Domains}{2016-07-21}{Algebra}

Integral domains are a commutative ring with unity (i.e.~multiplication
is commutative and there is an identity element for multiplication) in
which the cancellation property holds: In case \(a \neq 0\) and
\(ab = ac\), we can deduce that \(b=c\). The cancellation property is
equivalent to that there are no divisors of zero: If \(ab=0\), then
\(a=0\) or \(b=0\) (or both are zero).

Most prominent example are the integers \(\mathbb{Z}\).

Note that \(\mathbb{Z}/6\mathbb{Z}\) is \textbf{not} an integral domain
as it does not have the cancellation property; e.g. \(2 \times 3=0\). It
is, however, a commutative ring with unity.

A nilpotent element r of a ring R is an element for which
\(e^n=0, n \in \mathbb{N}\) holds. If a ring R has a nilpotent element
\(r \neq 0\), then this element is a divisor of zero. We have \(r^n=0\)
and can rewrite this as \(r r^{n-1}=0\)\$, hence r is a divisor of zero.

\(\mathbb{Z}/16\mathbb{Z}\), \(2^4=0\) and therefore 2 is a divisor of
zero in this ring. Note that a ring may have divisors of zero without
having nilpotent elements (e.g. \(\mathbb{Z}/6\mathbb{Z}\)).

A subring of a field is an integral domain. Examples: Any subring of
\(\mathbb{R}\) or \(\mathbb{C}\) is an integral domain. Thus for example
\(\mathbb{Z}[\sqrt{2}]\) is an integral domain.

\subsubsection{Examples}\label{examples}

Besides the obvious \(\mathbb{Z}\) and \(\mathbb{Z}/p \mathbb{Z}\) being
integral domains, also \(\mathbb{Z}[x]\) is an integral domain. Since
this integral domain is not finite, it is also not a field; i.e.~not
every element has an inverse.

To show that \(\mathbb{Z}[x]\) is an integral domain, we note that the
sum, difference, and product of two elements \(a, b \in \mathbb{Z}[x]\)
is also in \(\mathbb{Z}[x]\). To show that the cancellation property
holds, we define the conjugate of \(a=a_1+a_2x\) as
\(\bar{a} = a_1-a_2x\) and the function \(n(a) = a \bar{a}\). Note that
this function takes elements from \(\mathbb{Z}[x]\) and outputs values
in \(\mathbb{Z}\).

We have \(n(a) = a_1^2 - x^2 a_2^2\) and this is only zero when
\(a_1=a_2=0\); i.e. \(a=0\) (because x is irrational / imaginary). Next
note that \(n(ab) = a \bar{a} b\bar{b} = a\bar{a}b\bar{b} = n(a)n(b)\)
because of the commutative law for multiplication.

Assume that \(ab=0\) and therefore \(n(ab)=0\), from which
\(n(a)n(b)=0\) follows. We therefore have \(n(a)=0\) or \(n(b)=0\) and
from this follows that either \(a=0\) or \(b=0\).

Taken from
\href{www.cut-the-knot.org/arithmetic/int_domain.shtml}{here}.

\subsection{Divisibility of 1 / Unit}\label{divisibility-of-1-unit}

A unit in a commutative ring R is an element that divides 1:
\(uu^{-1}=1\). If every element in R has a unit and R is commutative,
then R is a field. However, there are cases where not every element is a
unit.

\subsubsection{Examples}\label{examples-1}

Taken from
\href{http://www.cut-the-knot.org/arithmetic/int_domain2.shtml}{here}.

I think this stuff holds only for \(\mathbb{Z}[x]\).

If \(u\) is a unit; i.e. \(uu^{-1}=1\), then \(n(u)=\pm 1\).

If \(n(a)=\pm 1\), then \(a\) is a unity. We have \(n(a)=a\bar{a}=1\)
and this shows that \(a\) times ``something'' is 1. Furthermore, if
\(u\) is a unit, then \(u^n, n \in \mathbb{Z}\) is also a unit (We have
\(n(u^2) = n(uu) = n(u)n(u) = 1 1 = 1\).).

In \(\mathbb{Z}[\sqrt{3}]\), the element \(2+\sqrt{3}\) is a unity. We
have

\[
\frac{1}{2+\sqrt{3}} = \frac{2 - \sqrt{3}}{4-3} = 2 - \sqrt{3}
\]

Furthermore, we have \(n(2+\sqrt{3}) = 4 - 3 = 1\). Finally, by squaring
\(2 + \sqrt{3}\) we obtain another unity:
\((2 + \sqrt{3})^2 = 7 + 2\sqrt{3}\).

\subsection{Characteristic of an Integral
Domain}\label{characteristic-of-an-integral-domain}

The (additive) characteristic of a ring with unity is the least positive
integer \(n\) such that \(na =0\). If \(n\) does not exist, then the
ring has characteristic \(0\).

Therefore \(\mathbb{Z}\) has characteristic 0. The ring
\(\mathbb{Z}/5\mathbb{Z}\) in the previous example has characteristic
\(5\), as \(1+1+1+1+1=0, 2+2+2+2+2=0, \ldots\).

\subsubsection{Further Results}\label{further-results}

All non-zero elements in an integral domain have the same (additive)
characteristic: With the characteristic \(n^\star\), we observe that
\(n^\star a = a +a + \cdots + a = 1a + 1a + \cdots + 1a = (1+1+\cdots+1)a = (n^\star 1)a\).
Since we are in an integral domain, \((n^\star 1)a=0\) if
\(n^\star 1=0\) (We consider only nonzero elements, \(a \neq 0\) and in
an integral domain a product is zero, if one or two factors is zero).
Therefore \(n^\star a=0\) and \(a\) has characteristic \(n^\star\).

In an integral domain with non-zero characteristic, the characteristic
is a prime number. If it were a composite number \(mn\), we would have
\(mn = (m1)(n1) = (mn)1 = 0\) and therefore wither \(m1\) or \(n1\)
would be zero. But the characteristic is the \textbf{least} number for
which \(mn1=0\) and this is a contradiction. Therefore, the
characteristic cannot be composite and must be prime.

In an integral domain with characteristic \(p\),
\((a+b)^p = a^p + b^p\). In every commutative ring, the binomial formula
holds, therefore we can expand the expression:

\[
(a+b)^p = a^p + {p \choose 1} a^{p-1} b + \cdots + {p \choose p-1} a b^{p-1} + b^p
\]

If \(p\) is prime and \(0 < k < p\), then \({p \choose k}\) is a
multiple of \(p\) and is therefore zero.

Finally, we have that every finite integral domain is a field. The only
difference between an integral domain and a field is the invertibility
of its elements; in a finite integral domain every element is invertible
so it also a field. The proof is actually quite nice: Asume the integral
domain has the following elements \(0, 1, a_1, a_2, \ldots,a_n\), that
iss \(n+2\) elements. Next note that the products
\(a_i 0, a_i 1, a_i a_1, \ldots, a_i a_n\) are all distinct because if
\(a_i x = a_i y\), then \(x=y\) would follow by the cancellation
property. Therefore, we also have \(n+2\) different products. Since we
have \(n+2\) different domain elements, every domain element is equal to
one of these products. So, \(a_i x = 1\) for some \(x\), there (every)
\(a_i\) is invertible. Note that the proof does not depend on which of
the product actualy equals \(1\); it is enough that the products are
distinct.
