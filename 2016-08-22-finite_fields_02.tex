\DiaryEntry{Finite Fields, II}{2016-08-22}{Algebra}

\subsection{General}\label{general}

All GF(\(p^n\)) elements takes coefficients from \(\mathbb{Z}_p\); an
element from GF(\(p^n\)) has therefore the form

\[
\sum_{k=0}^{n-1} c_k \alpha^k, c_k \in \mathbb{Z}_p
\]

We can add and multiply these elements - however, it can happen (in case
of multiplication) that the resulting element degree is larger than
\(n\). This is nothing to worry, but we seek a method of bringing these
elements into a space with degree less than \(n\) anyway. The idea is to
use an irreducible polynomial of degree n, denoted as \(p(x)\). Because
it is irreducible, it has a root \(\alpha\) outside \(\mathbb{Z}_p\):

\[
p(\alpha) = 0
\]

This expression allows us to reduce expressions with higher degree than
\(n\): Assume that we have an GF(\(p^n\)) element \(a(\alpha)\) with
degree larger than \(n\), then we can write \[
a(\alpha) = b(\alpha) p(\alpha) + c(\alpha)
\]

where \(b(\alpha), c(\alpha)\) are uniquely defined. But we have defined
\(p(\alpha)=0\), therefore \(b(\alpha) p(\alpha) = 0\) as well. So
\(a(\alpha)\) and \(c(\alpha)\) are equivalent modulo-p(x); instead of
working with polynomial degrees larger than n we can consider working
with \(c(\alpha)\) instead.

It is important to note that the difference between different galois
fields is the irreducible polynomial;the elements always look the same,
only the polynomials \(p(x)\) differ. This is analoguous to
\(\mathbb{Z}_p\): Here we reduce large numbers (resulting from addition
and multiplication) by taking their modulo expression. The whole
construction works only if \(p\) is prime in case of \(\mathbb{Z}_p\)
(otherwise there would be no multiplicative inverse element). In a
similar spirit, the polynomial \(p(x)\) must be irreducible for the
resulting structure to be a galois field.

\subsection{Cyclic Multiplicative
Group}\label{cyclic-multiplicative-group}

Every field F contains a multiplicative group denoted by \(F^\star\). In
the examples from the previous post, this group is F without the element
0.

There is a general theorem (without proof): If G is a finite subgroup of
\(F^\star\) (the multiplicative group of nonzero elements of a field F),
then G is cyclic.

We can specialise this to: The multiplicative group of all nonzero
elements of a finite field is cyclic.

As an example, take GF(2\^{}3) with the irreducible polynomial
\(\alpha^3 + \alpha + 1\). Starting with \(\alpha\), we have the
following table:

\[
\begin{array}{cc}
\alpha^0 & 1 \\
\alpha^1 & \alpha \\
\alpha^2 & \alpha^2 \\
\alpha^3 & \alpha +1 \\
\alpha^4 & \alpha^2 + \alpha \\
\alpha^5 & \alpha^2 + \alpha + 1 \\
\alpha^6 & \alpha^2 + 1 \\
\alpha^7 & 1 \\
... & ...
\end{array}
\]

So, the element \(x\) is a generator for the field GF(2\^{}3).
