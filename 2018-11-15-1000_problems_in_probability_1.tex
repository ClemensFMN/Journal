\DiaryEntry{1000 Problems in Probability - 1}{2018-11-15}{Stochastic}

\paragraph{Section 1.3, Problem 1.} We have two events $\Ac, \Bc$ with probabilities $P(\Ac) = 3/4, P(\Bc) = 1/3$. We seek bounds for $P(\Ac \cap \Bc)$ and $P(\Ac \cup \Bc)$.

We first note that $P(\Ac \cap \Bc) = P(\Ac) + P(\Bc) - P(\Ac \cup \Bc)$. If we upper-bound the last term with $1$, we obtain the lower bound

\bee
P(\Ac \cap \Bc) \geq P(\Ac) + P(\Bc) - 1 = \frac{1}{12}
\eee

For the ``other direction'', assume that one set is a subset of the other (e.g. $\Bc$ is a subset of $\Ac$). Then we can lower-bound $P(\Ac \cap \Bc)$ as

\bee
P(\Ac \cap \Bc) \leq \min\{P(\Ac), P(\Bc) \} = \frac{1}{3}
\eee

Combining the two bounds yields

\bee
P(\Ac) + P(\Bc) - 1 \leq P(\Ac \cap \Bc) \leq \min\{P(\Ac), P(\Bc) \}
\eee

An upper bound for $P(\Ac \cup \Bc)$ is obtained by noting that $P(\Ac \cup \Bc) = P(\Ac) + P(\Bc) - P(\Ac \cap \Bc) \leq P(\Ac) + P(\Bc)$. This bound is obtained, when $\Ac$ and $\Bc$ do not overlap. A lower bound is $P(\Ac \cup \Bc) \geq \max\{ P(\Ac), P(\Bc) \}$, obtained when one set is a subset of the other. Combining, we obtain

\bee
\max\{ P(\Ac), P(\Bc) \} \leq P(\Ac \cup \Bc) \leq P(\Ac) + P(\Bc)
\eee

\paragraph{Section 1.7, Problem 1.} Denote a path between $A$ and $B$ via $A \sim B$ and no path via $A \not\sim B$. We have $P(A \not\sim B) = P(B \not\sim C) = p^2$. The required probability is then

\begin{align*}
P(A \sim B | A \not\sim C) = &\frac{P(A \sim B \cap A \not\sim C)}{P(A \not\sim C)} = \frac{P(A \sim B \cap B \not\sim C)}{1 - P(A \sim C)} = \frac{P(A \sim B) P(B \not\sim C)}{1 - P(A \sim B \cap B \sim C)} \\ = & \frac{P(A \sim B) P(B \not\sim C)}{1 - P(A \sim B) P(B \sim C)} = \frac{(1-p^2)p^2}{1-(1-p^2)^2}
\end{align*}

\paragraph{Section 1.8, Problem 1.} Two (fair) dice with probability of one side (per dice) is $1/6$. Probability that $6$ turns up exactely once (a) is given by enumerating all ``good'' cases: $1-6, 2-6, 3-6, 4-6, 5-6, 6-1, 6-2, 6-3, 6-4, 6-5$ and the probability is then $P = 10/36 = 5/18$. For the probability that both numbers are odd (b), note that this happens in $50\%$ of the cases; i.e. $P = 1/2$. The ``good'' cases for throwing a sum of $4$ (c) are $1-3, 2-2, 3-1$ and the probability is $P = 3/36 = 1/12$. Finally, for the ``good'' cases that the sum is divisible by $3$ (d), the relevant sums are $3,6,9,12$. ``Good'' cases yielding a sum of $3$ are $1-2, 2-1$, ``good'' cases for a sum of $6$ are $1-5, 2-4, 3-3, 4-2, 5-1$, ``good'' cases for a sum of $9$ are $3-6, 4-5, 5-4, 6-3$, and finally the ``good'' case for a sum of $12$ is $6-6$. The corresponding probability is then $P = (2 + 5 + 4 + 1)/36 = 12/36 = 1/3$.

\paragraph{Section 1.8, Problem 2.} A fair coin is thrown $n$ times. The probability that - after the $n$-th throw:

\begin{itemize}

\item a head is thrown is $P( n-1 \times T, 1 \times H) = \left(\frac{1}{2}\right)^{n-1} \frac{1}{2} = 2^{-n}$

\item $n/2$ heads and tails have been thrown (assume $n$ even), is $P = {n \choose n/2} 2^{-n/2} 2^{-n/2} = {n \choose n/2}2^{-n}$.

\item exactely two heads have been thrown is $P = {n \choose 2} 2^{-n}$

\item at least two heads have been thrown is $P = 1 - 2^{-n} - {n \choose 1} 2^{-n}$.  
  
\end{itemize}


%%% Local Variables:
%%% mode: latex
%%% TeX-master: "journal"
%%% End:
