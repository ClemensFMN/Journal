\DiaryEntry{Differentiation of Implicit Functions}{2018-06-08}{Maths}

An implicit function is given in the following form

\bee
f(x) + g(y) + h(x,y) = 0
\eee

A simple example is the equation for the circle, $x^2 + y^2 - 1 = 0$. We want to calculate $y' = \frac{dy}{dx}$.

One solution is to rearrange the equation in an explicit expression $y u(x)$ and differentiate afterwards. This can be tedious (as in the circle equation), and in some cases not even possible because there is no closed-form expression for $y$. We therefore need a different method and that's differentiating the implicit function. We take the derivative wrt to $x$; in the terms $g(y)$ and $h(x,y)$ we need to apply the chain rule and arrive at the following expression

\bee
\frac{df(x)}{dx} + \frac{dg(y)}{dy} \frac{dy}{dx} + \frac{\partial h(x,y)}{\partial x} + \frac{\partial h(x,y)}{\partial y}\frac{dy}{dx} = 0
\eee

From this expression it may be possible to obtain a closed-form expression for $\frac{dy}{dx}$.

Continuing with the example of the circle, we have

\bee
2x + 2y \frac{dy}{dx} = 0 \rightarrow \frac{dy}{dx} = -\frac{x}{y}
\eee

We can consider the following special cases: At $x=0$, the derivative is zero, the tangent is a horizontal line and that's correct. At $y=0$, the derivative becomes infinity, the tangent is a vertical line and that's correct, too. For both $x,y > 0$ and $x,y<0$, the derivative is negative, and if $x>0, y<0$ and $x<0, y>0$, the derivative is positive - this matches the circle as well.

\paragraph{Other Examples.} Consider $y(x-1) - x^3 = 0$. We differentiate the implicit function and arrive at

\bee
y + (x-1) \frac{dy}{dx} - 3x^2 = 0 \rightarrow \frac{dy}{dx} = \frac{3x^2 - y}{x-1}
\eee

Note that the first two terms correspond to $\frac{\partial h(x,y)}{\partial x} + \frac{\partial h(x,y)}{\partial y}\frac{dy}{dx}$. Without knowing $y$ that would be the end of the story. However, this is an example which allows for an explicit expression as $y = \frac{x^3}{x-1}$. We can substitute this into the above result and obtain

\bee
\frac{dy}{dx} = \frac{3x^2 - y}{x-1} = \frac{3x^2 - \frac{x^3}{x-1}}{x-1} = \frac{3x^2 (x-1) - x^3}{(x-1)^2}
\eee

We can also directly differentiate $y = \frac{x^3}{x-1}$ and arrive at

\bee
\frac{dy}{dx} = \frac{3x^2(x-1)-x^3}{(x-1)^2} \qed
\eee

\paragraph{Khan Academy - Problems.} Let's start with

\bee
2x^3 - 5xy - y^2 = 3
\eee

Differentiating both sides wrt to $x$ yields

\bee
6x^2 - 5y - 5x \frac{dy}{dx} - 2y^2 \frac{dy}{dx} = 0 \rightarrow 6x^2 - 5y = (5x+2y) \frac{dy}{dx}
\eee

and from this we obtain

\bee
\frac{dy}{dx} = \frac{6x^2-5y}{5x+2y} \qed
\eee

Consider 

\bee
x^3y - 2x^2 + y^4 = 8
\eee

and differentiate to arrive at

\bee
3x^2y + x^3 \frac{dy}{dx} - 4x + 4y^3 \frac{dy}{dx} = 0 \rightarrow \frac{dy}{dx} = \frac{4x-3x^2y}{x^3 + 4y^3} \qed
\eee