\DiaryEntry{Continued Fractions}{2017-08-24}{Maths}

\subsection{Motivation}

Let's find a continued fraction for $\frac{649}{200}$: We first split into an integer and a fraction (smaller than $1$):

\bee
\frac{649}{200} = 3 + \frac{49}{200}= 3 + \frac{1}{ \frac{200}{49} }
\eee

where in the last step we "smuggled" in an additional reciprocal. We can now play the same game with $\frac{200}{49} = 4 + \frac{4}{49}$ and continue $\frac{49}{4} = 12 + \frac{1}{4}$. The reciprocal of the last fraction is an integer: $\frac{4}{1}= 4$, therefore the fractional part is zero. This is the signal that we are done. Putting everything together, we arrive at

\bee
\frac{649}{200} = 3 + \frac{1}{ 4 + \frac{4}{49} } = 3 + \frac{1}{ 4 + \frac{1}{ 12  + \frac{1}{4}} }
\eee

There is a special notation for continued fraction, we write it as $[3;4,12,4]$.

\subsection{Procedure}

We can turn the above example into a procedure which turns a number into a continued fraction as follows:

\begin{enumerate}
	\item Write down the integer part of the number.
	\item Subtract the integer part from the number.
	\item If the difference is zero, then stop; otherwise take the reciprocal of the difference and repeat.
\end{enumerate}

Above procedure is not restricted to numbers represented as fractions (like the first example); instead we can use it with normal numbers as well.

Continuing the example from above, we have $\frac{649}{200} = 3.245$. Taking the integer part yields $3$, the fractional part is $0.245$ and its reciprocal is approximately $4.081...$. Again subtracting the integer part $4$ yields a factional part of approximately $0.081..$ and we can continue. The process stops when the fractional part becomes zero (due to rounding errors, this might not be as obvious as in case of fractions).

\subsection{Properties}

\begin{itemize}

	\item Every fraction can be represented as a continued fraction which eventually stops; i.e. a finite continued fraction.

	\item An irrational number is represented by infinite continued fraction.

	\item An irrational number can be approximated by truncating its continued fraction. TODO: Add further information

\end{itemize}


As an example consider $\pi = 3.14 \ldots$: Subtract the integer value yields $0.14$, taking the reciprocal yields $7....$ and so on. The process never stops ($\pi$ is irrational, after all), and the first few values of the continued fraction are

\bee
\pi \approx [3;7,15,1,292,\ldots]
\eee




