\DiaryEntry{Groupd - Symmetric Groups}{2016-10-27}{Algebra}

\subsection{Cycles}\label{cycles}

An important observation is that every permutation can be described as
product of disjoint cycles. Disjoint cycles are commutative; i.e.
\(\alpha \beta = \beta \alpha\) if \(\alpha\) and \(\beta\) are disjoint
cycles.

Every cycle of length \textgreater{} 2 can be written as product of
length-2 cycles (transpositions). We have

\begin{equation}
\label{eq:exp}
(a_1 a_2 \ldots a_k) = (a_1 a_k) \ldots (a_1 a_3)(a_1 a_2)
\end{equation}

This decomposition is not necessarily unique.

As every element of the symmetric group can be written as group of
disjoint cycles and every cycle can be written as product of
transpositions, every element of the symmetric group can be written as
product of transpositions.

\subsubsection{Inverse}\label{inverse}

The inverse of a permutation is again a permutation (otherwise we won't
have a group). If a permutation is given in term of its cycles, the
inverse is obtained by reversing the elements in each cycle; e.g.

\[
\alpha = (123)(45) \rightarrow \alpha^{-1} = (321)(54)
\]

\subsection{Transpositions}\label{transpositions}

A transposition of two arbitrary elements (a, b) can also be rewritten
as

\begin{equation}
\label{eq:transp}
(a b) = (1 a)(1 b)(1 a)
\end{equation}

We have \((1 a)(1 b)(1 a)a = (1 a)(1 b) 1 = (1 a)b = b\) and
\((1 a)(1 b)(1 a)b = (1 a)(1 b)b = (1 a)1 = a\) and finally
\((1 a)(1 b)(1 a)1 = (1 a)(1 b)a = (1 a)a = 1\).

The inverse element is \((ba)\) and can be expressed as follows

\[
(ab)^{-1} = (ba) = (1a)(1b)(1a) = (ab)
\]

By using this decomposition, we can generate any transposition and
therefore any element of \(S_n\). Therefore, the transpositions
\((1 2),(1 3), \ldots, (1 n)\) generate \(S_n\).

A given element of \(S-n\) can be written as product of transpositions
in many different ways. However, the number of transpositions which
occur is either always even or always odd.

\subsubsection{Further Identities}\label{further-identities}

We have

\[
(abc) = (ac)(ab)
\]

and we can specialize \eqref{eq:exp} to \(n=3\) with \(a_1 = 1\) to

\begin{equation}
\label{eq:exp3}
(1ba) = (1a)(1b)
\end{equation}

\subsubsection{Sign of Permutations}\label{sign-of-permutations}

We define a polynomial as

\[
P(x_1,x_2,\ldots,x_n) = \prod_{1 \leq i,j \leq n, i < j} (x_i - x_j)
\]

For a permutation \(\alpha \in S_n\), we define

\[
\alpha P = \prod_{1 \leq i,j \leq n, i < j} (x_{\alpha(i)} - x_{\alpha(j)})
\]

Example: Consider \(n = 3\) and \(\alpha = (132)\). Then we have

\[
P(x_1, x_2, x_3) = (x_1 - x_2)(x_1 - x_3)(x_2 - x_3)
\]

and (since the permutation does the following transformation:
\(1 \rightarrow 3, 2 \rightarrow 1, 3 \rightarrow 2\)):

\[
\alpha P(x_1, x_2, x_3) = (x_3 - x_1)(x_3 - x_2)(x_1 - x_2) = + P
\]

In general, \(\alpha P\) just permutes the terms of P and changes the
sign of some of them. Therefore, \(\alpha P = \pm P\); and this defines
the sign of the permutation: If \(\alpha P = P\), the sign is positive,
if \(\alpha P = -P\), the sign is negative.

Consider two permutations \(\alpha, \beta \in S_n\). Then the sign of
their product \(\alpha \beta\) is the product of the signs of \(\alpha\)
and \(\beta\).

The sign of the transposition \((12)\) is always \(-1\):
\(P(x_1, x_2) = x_1 - x_2\) and \(\alpha = (12)\), then
\(\alpha P = x_2 - x_1 = -P\). From \eqref{eq:transp}, the general
transposition \((ab)\) can be written as product of three terms,
\((a b) = (1 a)(1 b)(1 a)\) and therefore the sign of a general
transposition \((ab)\) is negative.

If a permutation contains an even number of transpositions, its sign is
+1 and it is called an even permutation; if a permutation contains an
odd number of transpositions, its sign is -1 and it is called an odd
permutation.

Since \((a_1 a_2 \ldots a_k) = (a_1 a_k) \ldots (a_1 a_3)(a_1 a_2)\), a
cycle has a positive sign if its length is odd.

\subsection{Alternating Subgroup}\label{alternating-subgroup}

The even permutations of \(S_n\) form a subgroup \(A_n\) with \(n!/2\)
elements which is called the alternating subgroup.

Proof: If \(\alpha, \beta \in A_n\), then each permutation is composed
of an even number of transpositions. The combination \(\alpha \beta\) is
then the combination of two even numbers of transpositions and therefore
also even. Writing the transposition product of \(\alpha\) in reverse
order yields \(\alpha^{-1}\) which is therefore also even. Finally, the
identity element can be obtained by combining the same transpositions;
i.e. \(e = (ab)(ab)\). Each transposition has odd sign, therefore the
sign of the prodcut of two transpositions (and therefore the identity
element) is even. \(\square\)

For \(n \geq 3\), the 3-cycles generate \(A_n\).

Proof: A 3-cycle is an even permutation: \((abc) = (ac)(ab)\), every
transposition is negative and the product of two negative signs is
positive. Both transpositions can be written in the form \((1x)\) to
obtain \((abc) = (ac)(ab) = (1a)(1c)(1b)(1a)(1b)(1a)\). Now we can
collect consecutive transposition pairs into 3-cycles (cf.
\eqref{eq:exp3}) and obtain \((abc) = (1ca)(1ab)(1ab)\) \(\square\).

\subsubsection{\texorpdfstring{Example
\(A_4\)}{Example A\_4}}\label{example-a_4}

The \(4! / 2 = 12\) elements of \(A_4\) are as follows (I have no idea
how to obtain this list :-( ):

\[
\begin{array}{cccc}
e, & (12)(34), &  (13)(24), &(14)(23) \\
(123), & (124), & (134), & (234) \\
(132), & (142), & (143), & (243)
\end{array}
\]

The product of transpositions in the first row can be rewritten as
product of 3-cycles; e.g.

\[
(12)(34) = (12)(13)(14)(13) = (132)(134)
\]

The remaining 6 odd permutations of \(S_4\) are given by

\[
\begin{array}{cccc}
(12), (13), & (14), (23), & (24), (34) \\
(1234), & (1243), & (1324)  \\
(1432), & (1342), & (1423)
\end{array}
\]

These must contain all transpositions (as they all have negative sign)
and then ``other'' permutations. The ``others'' are single cycles with
even length (have negative sign), or a product of several cycles with an
odd number of cycles having negative sign; i.e.~even length (so that the
overall sign is negative). It seem that for \(S_4\), transpositions and
length-4 cycles are enough to generate all odd permutations. Maybe this
is different for other symmetric groups.

\subsection{Using GAP}\label{using-gap}

Another tool required for the job\ldots{}

Anyway, with \href{\%7Bfilename\%7D/files/symmetric_group.g}{this
script} we can create the group \(S_4\) and list its elements using gap
as follows

\begin{verbatim}
s4:=SymmetricGroup(4);
Elements(s4);

[ (), (3,4), (2,3), (2,3,4), (2,4,3), (2,4), (1,2), (1,2)(3,4), (1,2,3), (1,2,3,4), (1,2,4,3), 
(1,2,4), (1,3,2), (1,3,4,2), (1,3), (1,3,4), (1,3)(2,4), (1,3,2,4), (1,4,3,2), (1,4,2), 
(1,4,3), (1,4), (1,4,2,3), (1,4)(2,3) ]
\end{verbatim}

From this we can deduce that \(S_4\) has the following elements:

\begin{itemize}

\item
  6 elements containing one transposition and 2 fixed points (e.g.
  \((1,2)\))
\item
  3 double transpositions (e.g. \((1,2)(3,4)\))
\item
  8 elements containing one 3-cycle and one fixed point (e.g.
  \((1,2,3)\))
\item
  6 elements containing one 4-cycle and no fixed point (e.g.
  \((1,2,3,4)\))
\end{itemize}

We can create the alternating subgroup \(A_4\) as follows

\begin{verbatim}
a4:=DerivedSubgroup(s4);
Elements(a4);

[ (), (2,3,4), (2,4,3), (1,2)(3,4), (1,2,3), (1,2,4), (1,3,2), (1,3,4), (1,3)(2,4), (1,4,2), 
(1,4,3), (1,4)(2,3) ]
\end{verbatim}

which matches the list from above.

\subsubsection{Multiplication Table of $A_4$}

The following multiplication table for \(A_4\) is created by GAP. Be
careful, the order is ``colum \(\times\) row''; e.g.~applying
\((2,4,3)\) before \((1,2,3)\) is written as \((1,2,3)(2,4,3)\) and
equals \((1,2,4)\) and \textbf{not} \((1,4,3)\).

\[
\begin{array}{c|cccccccccccc}
\star &  () & (2,3,4) & (2,4,3) & (1,2)(3,4) & (1,2,3) & (1,2,4) & (1,3,2) & (1,3,4) & (1,3)(2,4) & (1,4,2) & (1,4,3) & (1,4)(2,3) \\
\hline
() & () & (2,3,4)& (2,4,3)& (1,2)(3,4)& (1,2,3)& (1,2,4)& (1,3,2)& (1,3,4)& (1,3)(2,4)& (1,4,2)& 
(1,4,3)& (1,4)(2,3)\\
(2,3,4) & (2,3,4)& (2,4,3)& ()& (1,2,4)& (1,2)(3,4)& (1,2,3)& (1,3,4)& (1,3)(2,4)& (1,3,2)& (1,4)(2,3)& 
(1,4,2)& (1,4,3) \\
(2,4,3) & (2,4,3)& ()& (2,3,4)& (1,2,3)& (1,2,4)& (1,2)(3,4)& (1,3)(2,4)& (1,3,2)& (1,3,4)& (1,4,3)& (1,4)
(2,3)& (1,4,2)\\
(1,2)(3,4) & (1,2)(3,4)& (1,3,2)& (1,4,2)& ()& (1,3,4)& (1,4,3)& (2,3,4)& (1,2,3)& (1,4)(2,3)& (2,4,3)& 
(1,2,4)& (1,3)(2,4)\\
(1,2,3) & (1,2,3)& (1,3)(2,4)& (1,4,3)& (2,4,3)& (1,3,2)& (1,4)(2,3)& ()& (1,2,4)& (1,4,2)& (2,3,4)& (1,2)
(3,4)& (1,3,4)\\ 
(1,2,4) & (1,2,4)& (1,3,4)& (1,4)(2,3)& (2,3,4)& (1,3)(2,4)& (1,4,2)& (2,4,3)& (1,2)(3,4)& (1,4,3)& ()& 
(1,2,3)& (1,3,2)\\
(1,3,2) & (1,3,2)& (1,4,2)& (1,2)(3,4)& (1,4,3)& ()& (1,3,4)& (1,2,3)& (1,4)(2,3)& (2,3,4)& (1,3)(2,4)& 
(2,4,3)& (1,2,4)\\ 
(1,3,4) & (1,3,4)& (1,4)(2,3)& (1,2,4)& (1,4,2)& (2,3,4)& (1,3)(2,4)& (1,2)(3,4)& (1,4,3)& (2,4,3)& 
(1,3,2)& ()& (1,2,3)\\
(1,3)(2,4) & (1,3)(2,4)& (1,4,3)& (1,2,3)& (1,4)(2,3)& (2,4,3)& (1,3,2)& (1,2,4)& (1,4,2)& ()& (1,3,4)& 
(2,3,4)& (1,2)(3,4)\\
(1,4,2) & (1,4,2)& (1,2)(3,4)& (1,3,2)& (1,3,4)& (1,4,3)& ()& (1,4)(2,3)& (2,3,4)& (1,2,3)& (1,2,4)& (1,3)
(2,4)& (2,4,3)\\
(1,4,3)& (1,4,3)& (1,2,3)& (1,3)(2,4)& (1,3,2)& (1,4)(2,3)& (2,4,3)& (1,4,2)& ()& (1,2,4)& (1,2)(3,4)& 
(1,3,4)& (2,3,4)\\
(1,4)(2,3)& (1,4)(2,3)& (1,2,4)& (1,3,4)& (1,3)(2,4)& (1,4,2)& (2,3,4)& (1,4,3)& (2,4,3)& (1,2)(3,4)& 
(1,2,3)& (1,3,2)& ()
\end{array}
\]

\subsubsection{\texorpdfstring{Subgroups of
\(S_4\)}{Subgroups of S\_4}}\label{subgroups-of-s_4}

Can be listed like this

\begin{verbatim}
AllSubgroups(s4);

[ Group(()), Group([ (1,2)(3,4) ]), Group([ (1,3)(2,4) ]), Group([ (1,4)(2,3) ]), 
Group([ (3,4) ]), Group([ (2,3) ]), Group([ (2,4) ]), Group([ (1,2) ]), Group([ (1,3) ]), 
Group([ (1,4) ]), Group([ (2,4,3) ]), Group([ (1,3,2) ]), Group([ (1,4,2) ]), 
Group([ (1,4,3) ]), Group([ (1,4)(2,3), (1,3)(2,4) ]), Group([ (3,4), (1,2)(3,4) ]), 
Group([ (1,4), (1,4)(2,3) ]), Group([ (2,4), (1,3)(2,4) ]), Group([ (1,3,2,4), (1,2)(3,4) ]), 
Group([ (1,4,3,2), (1,3)(2,4) ]), Group([ (1,2,4,3), (1,4)(2,3) ]), Group([ (3,4), (2,4,3) ]), 
Group([ (1,4), (1,4,3) ]), Group([ (2,3), (1,3,2) ]), Group([ (1,2), (1,4,2) ]), Group([ (1,4)(2,3), (1,3)(2,4),
(3,4) ]), Group([ (1,2)(3,4), (1,3)(2,4), (1,4) ]), Group([ (1,2)(3,4), (1,4)(2,3), (2,4) ]), Group([ (1,4)(2,3),
(1,3)(2,4), (2,4,3) ]), Group([ (1,4)(2,3), (1,3)(2,4), (2,4,3), (3,4) ]) ]
\end{verbatim}

This may not be so meaningful; listing the elements of each group brings
more information. We can do this with this script:

\begin{verbatim}
sbgps:=AllSubgroups(s4);
for e1 in sbgps do
    Print(e1, " -> ", Elements(e1), StructureDescription(e1), "\n");
od;
\end{verbatim}

and obtain the following (comments and structure added manually).
According to the Lagrange Theorem, the subgroup order must divide 24
(the order of \(S_4\)), therefore, there must be subgroups of order
\(1,2,3,4,6,8,12,24\).

\paragraph{Trivial subgroup}\label{trivial-subgroup}

\begin{verbatim}
Group( () ) -> [ () ]
\end{verbatim}

\paragraph{Subgroups with 2 elements}\label{subgroups-with-2-elements}

These groups are isomorphic to the cyclic group \(\mathbb{Z}_2\).
Principle: Every transposition is its own inverse.

\begin{verbatim}
Group( [ (1,2)(3,4) ] ) -> [ (), (1,2)(3,4) ]
Group( [ (1,3)(2,4) ] ) -> [ (), (1,3)(2,4) ]
Group( [ (1,4)(2,3) ] ) -> [ (), (1,4)(2,3) ]
Group( [ (3,4) ] ) -> [ (), (3,4) ]
Group( [ (2,3) ] ) -> [ (), (2,3) ]
Group( [ (2,4) ] ) -> [ (), (2,4) ]
Group( [ (1,2) ] ) -> [ (), (1,2) ]
Group( [ (1,3) ] ) -> [ (), (1,3) ]
Group( [ (1,4) ] ) -> [ (), (1,4) ]
\end{verbatim}

\paragraph{Subgroups with 3 elements}\label{subgroups-with-3-elements}

These groups are isomorphic to the cyclic group \(\mathbb{Z}_3\).
Principle: \((2,3,4)^2 = (2,4,3)\) and \((2,3,4)^3 = ()\).

\begin{verbatim}
Group( [ (2,4,3) ] ) -> [ (), (2,3,4), (2,4,3) ]
Group( [ (1,3,2) ] ) -> [ (), (1,2,3), (1,3,2) ]
Group( [ (1,4,2) ] ) -> [ (), (1,2,4), (1,4,2) ]
Group( [ (1,4,3) ] ) -> [ (), (1,3,4), (1,4,3) ]
\end{verbatim}

\paragraph{Subgroups with 4 elements}\label{subgroups-with-4-elements}

The first 4 subgroups are isomorphic to
\(\mathbb{Z}_2 \times \mathbb{Z}_2\) and this is equivalent to the
Klein-4 group.

\begin{verbatim}
Group( [ (1,4)(2,3), (1,3)(2,4) ] ) -> [ (), (1,2)(3,4), (1,3)(2,4), (1,4)(2,3) ]
Group( [ (3,4), (1,2)(3,4) ] ) -> [ (), (3,4), (1,2), (1,2)(3,4) ]
Group( [ (1,4), (1,4)(2,3) ] ) -> [ (), (2,3), (1,4), (1,4)(2,3) ]
Group( [ (2,4), (1,3)(2,4) ] ) -> [ (), (2,4), (1,3), (1,3)(2,4) ]
\end{verbatim}

The next 3 subgroups are isomorphic to \(\mathbb{Z}_4\).

\begin{verbatim}
Group( [ (1,3,2,4), (1,2)(3,4) ] ) -> [ (), (1,2)(3,4), (1,3,2,4), (1,4,2,3) ]
Group( [ (1,4,3,2), (1,3)(2,4) ] ) -> [ (), (1,2,3,4), (1,3)(2,4), (1,4,3,2) ]
Group( [ (1,2,4,3), (1,4)(2,3) ] ) -> [ (), (1,2,4,3), (1,3,4,2), (1,4)(2,3) ]
\end{verbatim}

\paragraph{Subgroups with 6 elements}\label{subgroups-with-6-elements}

These subgroups are isomorphic to \(S_3\).

\begin{verbatim}
Group( [ (3,4), (2,4,3) ] ) -> [ (), (3,4), (2,3), (2,3,4), (2,4,3), (2,4) ]
Group( [ (1,4), (1,4,3) ] ) -> [ (), (3,4), (1,3), (1,3,4), (1,4,3), (1,4) ]
Group( [ (2,3), (1,3,2) ] ) -> [ (), (2,3), (1,2), (1,2,3), (1,3,2), (1,3) ]
Group( [ (1,2), (1,4,2) ] ) -> [ (), (2,4), (1,2), (1,2,4), (1,4,2), (1,4) ]
\end{verbatim}

\paragraph{Subgroups with 8 elements}\label{subgroups-with-8-elements}

These subgroups are isomorphic to \(D_8\), the diheadral group of order
8.

\begin{verbatim}
Group( [ (1,4)(2,3), (1,3)(2,4), (3,4) ] ) -> [ (), (3,4), (1,2), (1,2)(3,4), (1,3)(2,4), (1,3,2,4), (1,4,2,3), (1,4)(2,3) ]
Group( [ (1,2)(3,4), (1,3)(2,4), (1,4) ] ) -> [ (), (2,3), (1,2)(3,4), (1,2,4,3), (1,3,4,2), (1,3)(2,4), (1,4), (1,4)(2,3) ]
Group( [ (1,2)(3,4), (1,4)(2,3), (2,4) ] ) -> [ (), (2,4), (1,2)(3,4), (1,2,3,4), (1,3), (1,3)(2,4), (1,4,3,2), (1,4)(2,3) ]
\end{verbatim}

\paragraph{Subgroup with 12 elements}\label{subgroup-with-12-elements}

This is the alternating group \(A_4\).

\begin{verbatim}
Group( [ (1,4)(2,3), (1,3)(2,4), (2,4,3) ] ) -> [ (), (2,3,4), (2,4,3), (1,2)(3,4), (1,2,3), (1,2,4), (1,3,2), (1,3,4), (1,3)(2,4), (1,4,2), (1,4,3), (1,4)(2,3) ]
\end{verbatim}

\paragraph{Subgroup with 24 elements}\label{subgroup-with-24-elements}

Finally, the symmetric group \(S_4\) itself.

\begin{verbatim}
Group( [ (1,4)(2,3), (1,3)(2,4), (2,4,3), (3,4) ] ) -> [ (), (3,4), (2,3), (2,3,4), (2,4,3), (2,4), (1,2), (1,2)(3,4), (1,2,3), (1,2,3,4), (1,2,4,3), (1,2,4), (1,3,2), (1,3,4,2), (1,3), (1,3,4), (1,3)(2,4), (1,3,2,4), (1,4,3,2), (1,4,2), (1,4,3), (1,4), (1,4,2,3), (1,4)(2,3) ]
\end{verbatim}
