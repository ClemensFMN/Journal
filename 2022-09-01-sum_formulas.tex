\DiaryEntry{Sum Formulas}{2022-09-01}{Maths}

We want to solve sum problems (i.e. finding a closed-form expression for the sum). We do that by assuming that the sum formula is a polynomial function and determine the polynom coefficients.

As a simple example, we start with the sum

\bee
S_n = \sum_{k=0}^n k
\eee

and we make the (educated guess) that the sum can be expressed as $S_n = a_0 + a_1 n + a_2 n^2$. First we calculate the sum for three values of $n$,

\begin{align*}
  S_0 &= 0 \\
  S_1 &= 0 + 1 = 1 \\
  S_2 &= 0 + 1 + 2 = 3
\end{align*}

For these three values, our polynomial Ansatz yields

\begin{align*}
  S_0 &= a_0 \\
  S_1 &= a_0 + a_1 \\
  S_2 &= a_0 + 2a_1 + 4a_2
\end{align*}

We can equate these and solve for the three unknowns $a_0, a_1, a_2$,

\bee
a_0 = 0, a_1 = a_2 = \frac{1}{2}
\eee

Therefore we reobtain the well-known sum formula

\be\label{2022-09-01:eq1}
S_n = \frac{1}{2}n + \frac{1}{2}n^2 = \frac{1}{2}n(n+1) \qed
\ee

As next example, consider the sum of squares,

\bee
S_n = \sum_{k=0}^n k^2
\eee

Calculating the sum for the first four values of $n$ yields,

\begin{align*}
  S_0 &= 0 \\
  S_1 &= 1 \\
  S_2 &= 5 \\
  S_3 &= 14 \\
\end{align*}

The Ansatz for the sum formula is $S_n = a_0 + a_1 n + a_2 n^2 + a_3 n^3$ and we get the following expressions

\begin{align}\label{2022-09-01:eq2}
  S_0 &= a_0 \\
  S_1 &= a_0 + a_1 + a_2 + a_3 \nonumber \\
  S_2 &= a_0 + 2a_1 + 4a_2 + 8a_3 \nonumber \\
  S_3 &= a_0 + 3a_1 + 9a_2 + 27a_3 \nonumber
\end{align}

Solving the system of expressions for the polynom coefficients (using Maxima) yields

\bee
a_0=0, a_1 = \frac{1}{6}, a_2 = \frac{1}{2}, a_3 = \frac{1}{3}
\eee

We obtain the following expression for the sum,

\be\label{2022-09-01:eq2}
S_n = \frac{1}{6}n + \frac{1}{2}n^2 + \frac{1}{3}n^3
\ee

We can factor this expression to get the well-known sum formula,

\bee
S_n = \frac{n(n+1)(2n+1)}{6} \qed
\eee

Let's try something slightly different,

\bee
S_n = \sum_{k=0}^n k(k+1)
\eee

The sum values are

\begin{align*}
  S_0 &= 0 \\
  S_1 &= 2 \\
  S_2 &= 8 \\
  S_3 &= 20 \\
\end{align*}

Our Ansatz stays the same, $S_n = a_0 + a_1 n + a_2 n^2 + a_3 n^3$; solving the system of equations in \eqref{2022-09-01:eq2} with a different RHS yields the following coefficients $a_i$,

\bee
a_0=\frac{2}{3}, a_1 = \frac{2}{3}, a_2 = 1, a_3 = \frac{1}{3}
\eee

and we obtain the following expression for the sum,

\bee
S_n = \frac{2}{3}n + n^2 + \frac{1}{3}n^3
\eee

Of course we could have obtained this result by combining \eqref{2022-09-01:eq1} and \eqref{2022-09-01:eq2},

\bee
\sum_{k=0}^n k(k+1) = \sum_{k=0}^n k^2 + k = \frac{1}{6}n + \frac{1}{2}n^2 + \frac{1}{3}n^3 + \frac{1}{2}n + \frac{1}{2}n^2 = \frac{2}{3}n + n^2 + \frac{1}{3}n^3 \qed
\eee

The Maxima expressions we used are as follows,

\begin{verbatim}
(%i1)	N:3;
(%o1)	3
(%i2)	rhs:makelist(sum(k*(k+1),k,1,n),n,0,N);
(%o2)	[0,2,8,20]
(%i3)	coeffs:makelist(concat(a,i),i,0,N);
(%o3)	[a0,a1,a2,a3]
(%i4)	seq:sum(coeffs[i+1]*n^i, i, 0, N);
(%o4)	a3*n^3+a2*n^2+a1*n+a0
(%i5)	lhs:makelist(subst(n=i,seq),i,0,N);
(%o5)	[a0,a3+a2+a1+a0,8*a3+4*a2+2*a1+a0,27*a3+9*a2+3*a1+a0]
(%i6)	eqs:makelist(lhs[i]=rhs[i], i, 1, N+1);
(%o6)	[a0=0,a3+a2+a1+a0=2,8*a3+4*a2+2*a1+a0=8,27*a3+9*a2+3*a1+a0=20]
(%i8)	sol:solve(eqs, coeffs);
(%o8)	[[a0=0,a1=2/3,a2=1,a3=1/3]]
(%i17)	sum(rhs(sol[1][i+1])*n^i, i, 0, N);
(%o17)	n^3/3+n^2+(2*n)/3
\end{verbatim}


%%% Local Variables:
%%% mode: latex
%%% TeX-master: "journal"
%%% End:
