\DiaryEntry{Inside Interesting Integrals, 7 (Section 2.5)}{2018-03-26}{Integrals}

\subsection{Challenge problem C2.2}

The problem asks for the integral

\bee
\int_0^1 \frac{dx}{x^3+1}
\eee

We first make a partial fraction expansion as follows

\begin{align*}
\int_0^1 \frac{dx}{x^3+1} &= \int_0^1 \frac{1}{3}\frac{1}{x+1} - \frac{1}{3}\frac{x-2}{x^2-x+1} dx \\ &= \int_0^1 \frac{1}{3}\frac{1}{x+1} - \int_0^1 \frac{1}{3} \left[ \frac{1}{2} \frac{2x-1}{x^2-x+1} - \frac{3}{2} \frac{1}{x^2-x+1} \right] dx = I_1 + I_2 + I_3
\end{align*}

where in the second step we split the terms and sneaked in factors so that the enumerator is the derivative of the denominator: $(x^2 - x+1)' = 2x-1$. IN the following, we solve one of the three integrals.

\paragraph{Integral $I_1$:} We set $u=x+1$ and obtain

\bee
\int_0^1 \frac{1}{3}\frac{1}{u} du = \frac{1}{3} \left. \ln (x+1) \right|_0^1 = \frac{1}{3} \ln 2
\eee

\paragraph{Integral $I_2$:} Here we make the substitution $u = x^2-x+1, du/dx = 2x-1, du/(2x-1) = dx$ and arrive at

\bee
\int \frac{2x-1}{x^2-x+1} dx = \int \frac{2x-1}{u} \frac{du}{2x-1} = \ln u = \left. \ln(x^2-x+1) \right|_0^1 = \ln 1 - \ln 1 = 0
\eee

Therefore $I-2 = 0$.

\paragraph{Integral $I_3$:} This integral is solved by completing the square:

\bee
\int \frac{dx}{x^2-x+1} = \int \frac{dx}{x^2-x+\frac{1}{4} + \frac{3}{4}} = \int \frac{dx}{\left(x-\frac{1}{2}\right)^2 + \frac{3}{4}} = \int \frac{dx}{u^2 + \frac{3}{4}}
\eee

where we have set $u=x-1/2$ (and $du/dx=1$). This integral can be solved and we obtain

\begin{align*}
I_3 = \frac{1}{2} \frac{2 \arctan \frac{2u}{\sqrt{3}}}{\sqrt{3}} = \left. \frac{ \arctan \frac{2(x-1/2)}{\sqrt{3}}}{\sqrt{3}} \right|_0^1 & = \frac{\arctan 1/\sqrt{3}}{\sqrt{3}} - \frac{\arctan \left(-1/\sqrt{3}\right) }{\sqrt{3}} \\ & = 2 \arctan \left(1/\sqrt{3}\right)/\sqrt{3} = \frac{\pi}{\sqrt{27}}
\end{align*}

Combining everything together, we arrive at

\bee
\boxed{
\int_0^1 \frac{dx}{x^3+1} = \frac{1}{3} \ln 2 + \frac{\pi}{\sqrt{27}}
}
\eee

We can do this in Maxima as well

\begin{verbatim}
(%i1)	integrate(1/(x^3+1), x, 0, 1);
(%o1)	(6*log(2)+sqrt(3)*%pi)/18+%pi/(2*3^(3/2))
(%i2)	float((6*log(2)+sqrt(3)*%pi)/18+%pi/(2*3^(3/2))), numer;
(%o2)	0.8356488482647211
(%i3)	partfrac(1/(x^3+1), x);
(%o3)	1/(3*(x+1))-(x-2)/(3*(x^2-x+1))
(%i4)	atan(1/sqrt(3));
(%o4)	%pi/6
(%i5)	float(log(2)/3+%pi/3^(3/2)), numer;
(%o5)	0.8356488482647211
\end{verbatim}


\subsection{Challenge problem  C2.3}

Start instead with p. 58:

\bee
\int_0^\infty \frac{dx}{x^4 + 2x^2 \cos(2\alpha) + 1}
\eee

\todo{Calculate the integral}

\subsubsection{Partial Integration}

We start with the differentiation rule for the product of two functions:

\bee
\left(u(x)v(x)\right)' = u'(x)v(x) + u(x)v'(x)
\eee

If we integrate both sides and rearrange terms, we arrive at

\be
\label{2018-03-26:eq1}
\int u'(x)v(x) dx  = u(x)v(x) - \int u(x)v'(x) dx
\ee

Hidden in the simple-looking formula is the fact that we need to integrate $u'$ in order to arrive at $u$ on the right side. The "trick" is to choose $u$ and $v$ on the LHS in such a way, that (i) integrating $u'$ is possible and (ii) that the integral $\int u(x)v'(x) dx$ is simpler than the original integral.

We can demonstrate this with a very simple example; namely $\int x^3 dx$. We choose $u'=x \rightarrow u=x^2 / 2$ and $v=x^2 \rightarrow v'=2x$. Inserting into above expression, we obtain

\bee
\int x^3 dx = \frac{x^2}{2} x^2 - \int \frac{x^2}{2} 2x dx = \frac{x^4}{2} - \int x^3 dx = \frac{x^4}{2} - \frac{x^4}{4} = \frac{x^4}{4} \qed
\eee

We can do the same trick also with definite integrals like this

\be
\label{2018-03-26:eq2}
\int_A^B u'(x)v(x) dx  = u(x)v(x) \bigg|_A^B - \int_A^B u(x)v'(x) dx
\ee

In the first term on the RHS $u(x)v(x) \bigg|_A^B$, we can exchange the calculation at $A$ and $B$ with multiplication; i.e. we can also write

\bee
\int_A^B u'(x)v(x) dx  = u(x) \bigg|_A^B v(x) \bigg|_A^B - \int_A^B u(x)v'(x) dx
\eee

meaning that we only need to know the definite integral $\int_A^B u'(x)dx$. However, \emph{I think} this is not possible in the second term: Here we need to know $u(x)$, multiply this expression with $v(x)$ and perform the integration \emph{afterwards}. In other words,

\bee
\int_A^B u(x)v'(x) dx \neq \int_A^B \left( \int_A^B u(x)dx \right) v'(x) dx = \int_A^B u(x)dx \cdot \int_A^B v'(x) dx
\eee

where we have realized that $\int_A^B u(x)dx$ is a constant factor and have it moved in front of the outer integral. This makes it pretty obvious why the two expressions cannot be equal.

We can illustrate this with a simple example

\bee
\int_0^1 x \cdot x^2 dx = \frac{x^3}{3}\bigg|_0^1 = \frac{1}{3}
\eee

whereas

\bee
\int_0^1 \left( \int_0^1 x dx \right) x^2 dx = \int_0^1 \left( \frac{x^2}{2} \bigg|_0^1 \right) x^2 dx = \frac{1}{2} \cdot \int_0^1  x^2 dx = \frac{1}{2} \frac{1}{3} = \frac{1}{6} \qed
\eee

We can demonstrate \eqref{2018-03-26:eq1} with a simple example: $\int_0^1 x^3 dx = 1/4$. Setting $u'=x \rightarrow u=x^2 / 2$ and $\int_0^1 u' dx = 1/2$ and $v=x^2 \rightarrow v'=2x$ we have

\bee
\int_0^1 x^3 dx = \frac{1}{2} \cdot x^2 \bigg|_0^1 - \int_0^1 \frac{x^2}{2} 2x dx = \frac{1}{2} - \frac{x^4}{4}\bigg|_0^1 = \frac{1}{2} - \frac{1}{4} = \frac{1}{4} \qed
\eee

\subsubsection{Partial Integration - Examples}

A very simple integral is

\bee
\int x \sin(x) dx
\eee

Here we choose $u$ and $v$ so, that $v'$ is simpler: $u'=\sin(x) \rightarrow u = -\cos(x)$ and $v=x \rightarrow v'=1$. Then we have

\bee
\int x \sin(x) dx = -x \cos(x) + \int \cos(x) dx = \sin(x) - x \cos(x) \qed
\eee

In a similar spirit, we obtain

\bee
\int x e^x dx = xe^x - \int e^x dx = e^x(x-1) \qed
\eee

with $u'=e^x \rightarrow u=e^x$ and $v=x \rightarrow v'=1$. 

A slightly more complicated example is

\bee
I = \int \frac{\ln(x)}{x^2}dx
\eee

Setting $u'=1/x^2 \rightarrow u=-1/x$ and $v=\ln(x) \rightarrow v'=1/x$. Then we have

\bee
I = -\frac{\ln(x)}{x} - \int \left(-\frac{1}{x}\right)\frac{1}{x}dx = -\frac{\ln(x)}{x} + \int \frac{1}{x^2}dx = -\frac{\ln(x)}{x} - -\frac{1}{x} = -\frac{\ln(x)+1}{x} \qed
\eee

Now back to the challenge problem C2.3. We want to show

\bee
\int_0^\infty \frac{dx}{(1+x^4)^{m+1}} = \frac{4m-1}{4m}\int_0^\infty \frac{dx}{(1+x^4)^m}
\eee
