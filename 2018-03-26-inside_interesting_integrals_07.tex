\DiaryEntry{Inside Interesting Integrals, 7 (Section 2.5)}{2018-03-26}{Integrals}

Challenge problem C2.2 asks for the integral

\bee
\int_0^1 \frac{dx}{x^3+1}
\eee

We first make a partial fraction expansion as follows

\begin{align*}
\int_0^1 \frac{dx}{x^3+1} &= \int_0^1 \frac{1}{3}\frac{1}{x+1} - \frac{1}{3}\frac{x-2}{x^2-x+1} dx \\ &= \int_0^1 \frac{1}{3}\frac{1}{x+1} - \int_0^1 \frac{1}{3} \left[ \frac{1}{2} \frac{2x-1}{x^2-x+1} - \frac{3}{2} \frac{1}{x^2-x+1} \right] dx = I_1 + I_2 + I_3
\end{align*}

where in the second step we split the terms and sneaked in factors so that the enumerator is the derivative of the denominator: $(x^2 - x+1)' = 2x-1$. IN the following, we solve one of the three integrals.

\paragraph{Integral $I_1$:} We set $u=x+1$ and obtain

\bee
\int_0^1 \frac{1}{3}\frac{1}{u} du = \frac{1}{3} \left. \ln (x+1) \right|_0^1 = \frac{1}{3} \ln 2
\eee

\paragraph{Integral $I_2$:} Here we make the substitution $u = x^2-x+1, du/dx = 2x-1, du/(2x-1) = dx$ and arrive at

\bee
\int \frac{2x-1}{x^2-x+1} dx = \int \frac{2x-1}{u} \frac{du}{2x-1} = \ln u = \left. \ln(x^2-x+1) \right|_0^1 = \ln 1 - \ln 1 = 0
\eee

Therefore $I-2 = 0$.

\paragraph{Integral $I_3$:} This integral is solved by completing the square:

\bee
\int \frac{dx}{x^2-x+1} = \int \frac{dx}{x^2-x+\frac{1}{4} + \frac{3}{4}} = \int \frac{dx}{\left(x-\frac{1}{2}\right)^2 + \frac{3}{4}} = \int \frac{dx}{u^2 + \frac{3}{4}}
\eee

where we have set $u=x-1/2$ (and $du/dx=1$). This integral can be solved and we obtain

\begin{align*}
I_3 = \frac{1}{2} \frac{2 \arctan \frac{2u}{\sqrt{3}}}{\sqrt{3}} = \left. \frac{ \arctan \frac{2(x-1/2)}{\sqrt{3}}}{\sqrt{3}} \right|_0^1 & = \frac{\arctan 1/\sqrt{3}}{\sqrt{3}} - \frac{\arctan \left(-1/\sqrt{3}\right) }{\sqrt{3}} \\ & = 2 \arctan \left(1/\sqrt{3}\right)/\sqrt{3} = \frac{\pi}{\sqrt{27}}
\end{align*}

Combining everything together, we arrive at

\bee
\boxed{
\int_0^1 \frac{dx}{x^3+1} = \frac{1}{3} \ln 2 + \frac{\pi}{\sqrt{27}}
}
\eee

We can do this in Maxima as well

\begin{verbatim}
(%i1)	integrate(1/(x^3+1), x, 0, 1);
(%o1)	(6*log(2)+sqrt(3)*%pi)/18+%pi/(2*3^(3/2))
(%i2)	float((6*log(2)+sqrt(3)*%pi)/18+%pi/(2*3^(3/2))), numer;
(%o2)	0.8356488482647211
(%i3)	partfrac(1/(x^3+1), x);
(%o3)	1/(3*(x+1))-(x-2)/(3*(x^2-x+1))
(%i4)	atan(1/sqrt(3));
(%o4)	%pi/6
(%i5)	float(log(2)/3+%pi/3^(3/2)), numer;
(%o5)	0.8356488482647211
\end{verbatim}