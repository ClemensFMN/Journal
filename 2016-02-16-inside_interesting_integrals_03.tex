\DiaryEntry{Inside Interesting Integrals, 3 (Section 2.1)}{2016-02-16}{Integrals}

These integrals use substitution in a ``backwards'' manners. See below,
what this means.

\subsubsection{Integral 2.1.a}

We have the integral

\[
I = \int \frac{dx}{(x+a)\sqrt{x-1}}
\]

We set \(t^2 = x-1\) from which we have \(t=\sqrt{x-1}\) and
\(x=1+t^2\). Therefore \(\frac{dx}{dt} = 2t = 2\sqrt{x-1}\). Note that
here we have expressed the derivative not in terms of \(t\) but in terms
of \(x\). It is this ``trick'' which allows to solve the integral. We
have

\[
I = \int \frac{1}{(1+t^2+a)\sqrt{x-1}}2\sqrt{x-1}dt = 2 \int \frac{dt}{t^2+a+1} = 2 \frac{\arctan {\sqrt{\frac{x-1}{a+1}}}}{\sqrt{a+1}}
\]

The ``trick'' causes the \(\sqrt{x-1}\) terms to cancel and brings the
integral into a standard form.

If we want to calculate the definite integral \(\int_1^\infty\) we
obtain

\[
\int_1^\infty \frac{dx}{(x+a)\sqrt{x-1}} = 2 \frac{\arctan \infty}{\sqrt{a+1}} = \frac{\pi}{\sqrt{a+1}}
\]

\subsubsection{Integral 2.1.d}

In a similar spirit, we can solve the integral

\[
I = \frac{dx}{1+e^{ax}}
\]

We substitute \(u = e^{ax}\) and obtain
\(\frac{du}{dx} = a e^{ax} = au\). The last step is again expressing the
derivative not in terms of \(x\) but in terms of \(u\). Therefore,
\(dx = \frac{du}{au}\) and we have

\[
I = \int \frac{1}{1+u} \frac{du}{au} = \frac{1}{a} \int \frac{du}{u(1+u)} = \frac{1}{a} \int \frac{1}{u} - \frac{1}{1+u} du
\]

where we made a partial fraction expansion in the last step. The
integral can now be solved according to

\[
I = \frac{1}{a} \left( \ln u - \ln (1+u) \right) = \frac{1}{a} \ln \frac{e^{ax}}{1+e^{ax}}
\]

Calculating the definite integral yields

\[
I = \int_0^\infty \frac{dx}{1+e^{ax}} = \frac{1}{a} \left( \ln 1 - \ln 1/2 \right) = -\frac{\ln 1/2}{a} = \frac{\ln 2}{a}
\]


%%% Local Variables:
%%% mode: latex
%%% TeX-master: "journal"
%%% End:
