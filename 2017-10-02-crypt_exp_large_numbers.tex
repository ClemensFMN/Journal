\DiaryEntry{Cryptography and Exponentation of Large Numbers}{2017-02-10}{Crypto}


For many cryptographic algorithms, exponentiation of large numbers is needed. The simplest approach is to continuously multiply:

\bee
x \xrightarrow{\text{mul}} x^2 \xrightarrow{\text{mul}} x^3 \xrightarrow{\text{mul}} x^4 \cdots
\eee

The number of required multiplications equals the exponent and this growth can be prohibitive for large exponents and / or numbers. A more suitable approach is the \emph{square-and-multiply algorithm}.

\subsection{Square-and-Multiply Algorithm}

As a simple example, consider calculation of $x^8$:

\bee
x \xrightarrow{\text{squ}} x^2 \xrightarrow{\text{squ}} x^4 \xrightarrow{\text{squ}} x^8
\eee

which requires just $3$ squaring operations instead of $7$ multiplications. Since squaring has approximately the same complexity as multiplication, we have saved a large amount of effort. This scheme works only for exponents being powers of two; however, the method can be generalized for arbitrary exponents as follows.

As an example consider $x^{26}$ which can be calculated using a mixture of squaring and multiplication

\bee
x \xrightarrow{\text{squ}} x^2 \xrightarrow{\text{mul}} x^3 \xrightarrow{\text{squ}} x^6 \xrightarrow{\text{squ}} x^{12} \xrightarrow{\text{mul}} x^{13} \xrightarrow{\text{squ}} x^{26}
\eee

Writing the exponent in binary (MSB-first), we can see what is going on: $26 = 11010_2 = b_4 b_3 b_2 b_1 b_0$.

\begin{itemize}
	\item In the first step, we process the MSB $b_4 = 1$: $x^1 = x^{1_2}$.
	\item In the second step, we process $b_3 = 1$: $(x^1)^2 = x^2 = x^{10_2}$ and $x^2 x = x^3 = x^{10_2} x^{1_2} = x^{11_2}$.
	\item In the third step, we process $b_2 = 0$: $(x^3)^2 = x^6 = (x^{11_2})^2 = x^{110_2}$.
	\item In step 4, we process $b_1 = 1$: $(x^6)^2 = x^{12} = x^{(110_2)2} = x^{1100_2}$ and $x^{12} x = x^{13} = x^{1100_2}x = x^{1101_2}$.
	\item Finally, we have $b_0 = 0$ and $x^{13}2 = x^{26} = x^{1101_2}2 = x^{11010_2}$.
\end{itemize}

Squaring corresponds to shifting the (binary) exponent one position to the left (and inserting a zero); multiplication adds a binary one to the exponent.

From this we can see that we scan the exponent bitwise from MSB to LSB and start with $r = x$. When the "current" bit is zero, we square $r$, when the "current" bit is one, we square and multiply $r$ with $x$.

More formally, we have the following algorithm:

\begin{itemize}
	\item Input: $x$ and exponent $H$ in binary form: $H = b_l 2^l + b_{l-1} 2^{l-1} + \cdots b_1 2 + b_0$.
	\item Output: $x^H$
	\item Set $r = x$
	\item For $i=l-1$ down-to $i=0$
	\begin{itemize}
		\item $r = r^2$
		\item If $b_i=1$, then $r = r x$
	\end{itemize}
	\item Output $r$
\end{itemize}

Assuming that squaring and multiplication have the same complexity and that the number of ones in an exponent with $N$ is $N/2$ on average, the complexity of this algorithm is $N + N/2 = \frac{3}{2}N$ as compared to $2^N$ multiplications in the "standard approach".

\subsection{Example}

As a last example, let's calculate $x^{13} = x^{1101_2}$. Applying the \emph{square-and-multiply algorithm}, we obtain

\begin{align*}
1: & x \xrightarrow{\text{squ}} x^2 \xrightarrow{\text{mul}} x^3 \\
0: & x^3 \xrightarrow{\text{squ}} x^6 \\
1: & x^6 \xrightarrow{\text{squ}} x^{12} \xrightarrow{\text{mul}} x^{13}
\end{align*}



