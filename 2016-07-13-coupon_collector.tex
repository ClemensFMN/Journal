\DiaryEntry{Coupon Collector Problem}{2016-07-13}{Stochastic}


A source sends out an infinite sequence of coupons and there are \(N\)
different coupons. The same coupon will be sent several times.

We are interested in obtaining a set of \(K\) \emph{different} coupons
(\(K \leq N\)) - What is the expected number of coupons to receive
\(\mathbb{E}(D(N,K))\) in order to have \(K\) different ones? This is
actually a more general question than the coupon collector's problem
where only the expected number of coupons till \(K=N\) different ones
have been observed.

For the first coupon, any coupon will be a new one and therefore
\(\mathbb{E}(D(N,1)) = 1\). For the second new coupon, the process is
started anew: The source sends coupons; with a probability of
\((N-1)/N\) we will receive a new one and with a probability of \(1/N\)
we will receive the one we already have. This is a Bernoulli process and
the expected number of coupons required before a new one is observed is
one over the probability; i.e. \(N/(N-1)\).

In a similar we can argue for the third coupon: The probability to
observe a new one is \((N-2)/N\) and the expected number of coupons to
wait is \(N/(N-2)\).

Therefore in the general case we have

\bee
\mathbb{E}(D(N,K)) = \sum_{k=0}^{K-1} \frac{N}{N-k} = N \sum_{k=0}^{K-1} \frac{1}{N-k}
\eee

For increasing \(K\), more terms in the sum are added and these terms
are increasing - the more different coupons we want to obtain, the
longer we need to wait.
