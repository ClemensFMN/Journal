\DiaryEntry{Number-theoretic Functions}{2020-12-22}{Number Theory}

For a positive integer $n$, we define $\tau(n)$ to be the number of positive divisors of $n$ and $\sigma(n)$ to be the sum of these divisors.

As an example, consider $n = 15$. Its divisors are $1, 3, 5, 15$, and so we have $\tau(15) = 4$ and $\sigma(15) = 1 + 3 + 5 + 15 = 24$.

In case of $n$ being prime, $\tau(p) = 2$ and $\sigma(p) = 1+p$ as a prime has only two divisors ($1$ and itself).

We also define some notation: The sum

\bee
\sum_{d \mid n} f(n)
\eee

denotes the sum of all values $f(d)$ for all $d$ which are divisors of $n$. With this definition, we can write our two functions as follows,

\bee
\tau(n) = \sum_{d \mid n} 1, \qquad \sigma(n) = \sum_{d \mid n} d
\eee

If we know the prime factorization of a number $n$ as

\bee
n = p_1^{k_1} p_2^{k_2} \cdots p_r^{k_r}
\eee

then the positive divisors of $n$ are those integers $d$ of the form

\bee
d = p_1^{a_1} p_2^{a_2} \cdots p_r^{a_r}, \quad 0 \leq a_i \leq k_i, \quad i = 1, 2, \ldots r
\eee

The proof is omitted (not sure what needs to be proven here anyway).


%%% Local Variables:
%%% mode: latex
%%% TeX-master: "journal"
%%% End:
