\DiaryEntry{Number-theoretic Functions}{2020-12-22}{Number Theory}

For a positive integer $n$, we define $\tau(n)$ to be the number of positive divisors of $n$ and $\sigma(n)$ to be the sum of these divisors.

As an example, consider $n = 15$. Its divisors are $1, 3, 5, 15$, and so we have $\tau(15) = 4$ and $\sigma(15) = 1 + 3 + 5 + 15 = 24$.

In case of $n$ being prime, $\tau(p) = 2$ and $\sigma(p) = 1+p$ as a prime has only two divisors ($1$ and itself).

We also define some notation: The sum

\bee
\sum_{d \mid n} f(n)
\eee

denotes the sum of all values $f(d)$ for all $d$ which are divisors of $n$. With this definition, we can write our two functions as follows,

\bee
\tau(n) = \sum_{d \mid n} 1, \qquad \sigma(n) = \sum_{d \mid n} d
\eee

If we know the prime factorization of a number $n$ as

\bee
n = p_1^{k_1} p_2^{k_2} \cdots p_r^{k_r}
\eee

then the positive divisors of $n$ are those integers $d$ of the form

\bee
d = p_1^{a_1} p_2^{a_2} \cdots p_r^{a_r}, \quad 0 \leq a_i \leq k_i, \quad i = 1, 2, \ldots r
\eee

The proof is omitted (not sure what needs to be proven here anyway).

We can use this theorem to obtain a different presentation of $\sigma(n)$ and $\tau(n)$.

\begin{theorem}
    If the prime factorization of $n$ is given by $n = p_1^{k_1} p_2^{k_2} \cdots p_r^{k_r}$, then

    \bee
    \tau(n) = (k_1 + 1)(k_2 + 1) \cdots (k_r + 1)
    \eee
    
    and
    
    \bee
    \sigma(n) = \frac{p_1^{k_1+1} - 1}{p_1 - 1} \frac{p_2^{k_2+1} - 1}{p_2 - 1} \cdots \frac{p_r^{k_r+1} - 1}{p_r - 1}
    \eee
\end{theorem}

Looking at the prime factorization $n = p_1^{k_1} p_2^{k_2} \cdots p_r^{k_r}$, we can count the number of factors as follows: Every factor can be chosen indepedently; there are $k_1 + 1$ possible exponents for $p_1$, $k_2+1$ possible factors for $p_2$ and so on. Therefore, we have $(k_1 + 1)(k_2 + 1) \cdots (k_r + 1)$ different divisors.

The sum of divisors can be written as 

\bee
\sigma(n) = \sum_{a_1 = 1}^{k_1} \sum_{a_2 = 1}^{k_2} \cdots \sum_{a_r = 1}^{k_r} p_1^{a_1} p_2^{a_2} \cdots p_r^{a_r}
\eee

The sums can be calculated indepedently; i.e. we have

\bee
\sigma(n) = \sum_{a_1 = 1}^{k_1} p_1^{a_1} \sum_{a_2 = 1}^{k_2} p_2^{a_2} \cdots \sum_{a_r = 1}^{k_r} p_r^{a_r}
\eee

Each sum is a finite geometric series and we obtain

\bee
\sigma(n) = \frac{p_1^{k_1+1} - 1}{p_1 - 1} \frac{p_2^{k_2+1} - 1}{p_2 - 1} \cdots \frac{p_r^{k_r+1} - 1}{p_r - 1} \qed
\eee

\paragraph{GAP / Example.} GAP provides the functions as \verb|Tau(n)| and \verb|Sigma(n)|.

As an example, consider $n = 360 = 2^3 \cdot 3^2 \cdot 5$. We therefore have $\tau(360) = 4 \cdot 3 \cdot 2 = 24$ and $\sigma(360) = \frac{2^4-1}{2-1} \frac{3^3-1}{3-1} \frac{5^2-1}{5-1} = 1170$. This can be checked in GAP via

\begin{verbatim}
    gap> Tau(360);
    24
    gap> Sigma(360);
    1170
\end{verbatim}

The two functions defined above fulfill the \emph{multiplicative property} as follows: A (number-theoretic) function $f$ is said to be multiplicative if

\bee
f(mn) = f(m)f(n), \quad \gcd(m,n) = 1
\eee

Multiplicative functions have the important property that the function is fully defined when their values at prime powers are known. If $n$ is any integer with factorization $n = p_1^{k_1} \cdot p_r^{k_r}$, then the factors are relatively prime and a multiplicative function simplifies to

\bee
f(n) = f(p_1^{k_1} \cdot p_r^{k_r}) = f(p_1^{k_1}) \cdots f(p_r^{k_r})
\eee

If $f$ is a multiplicative function that does not vanish identically, then there exists an integer $n$ for which $f(n) \neq 0$. We can rewrite this as

\bee
f(n) = f(n \cdot 1) = f(n) \cdot f(1)
\eee

With our assumption that $f(n) \neq 0$, we can cancel it on both sides of the equation and we therefore have $f(1) = 1$ for \emph{any} multiplicative function not identically zero.

Our functions $\tau(n)$ and $\sigma(n)$ fulfill the multiplicative property which we prove as follows.

\begin{theorem}
    The functions $\tau(n)$ and $\sigma(n)$ have the multiplicative property.
\end{theorem}

Let $m,n$ be relatively prime integers. The result is (trivially) true if either $m$ or $n$ is equal to $1$, therefore we assume $m,n > 1$. Define the prime factorizations of $m,n$ as

\bee
m = p_1^{k_1} \cdots p_r^{k_r}, \qquad n = q_1^{j_1} \cdots q_r^{j_r}
\eee

Since the numbers were assumed relatively prime, the primes $p_i$ and $q_j$ are all different. The product $mn$ is therefore

\bee
mn = p_1^{k_1} \cdots p_r^{k_r} q_1^{j_1} \cdots q_s^{j_s}
\eee

and cannot be simplified further. Therefore,

\bee
\tau(n) = \left[ (k_1+1) \cdots (k_r+1) \right] \left[ (j_1+1) \cdots (j_s+1) \right] = \tau(m) \tau(n)
\eee

In a similar fashion, we have

\bee
\sigma(mn) = \left[ \frac{p_1^{k_1+1} - 1}{p_1 - 1} \frac{p_r^{k_r+1} - 1}{p_r - 1} \right] \left[ \frac{q_1^{j_1+1} - 1}{q_1 - 1} \frac{q_s^{j_s+1} - 1}{p_s - 1} \right] = \sigma(m) \sigma(n) \qed
\eee

Before we continue, we have the following lemma.

\begin{theorem}
    If $\gcd(m,n) = 1$, then the set of positive divisors of $mn$ consists of all products $d_1 d_2$ where $d_1 | m, d_2 | n$ and $\gcd(d_1, d_2) = 1$; furthermore, these products are all distinct.
\end{theorem}

Assuming $m,n > 1$, we write for the prime factorizations $m = p_1^{k_1} \cdots p_r^{k_r}, n = q_1^{j_1} \cdots q_r^{j_r}$. With the same arguments as above, the primes $p_i$ and $q_j$ are all distinct and the prime factorization of $mn$ is therefore

\bee
mn = p_1^{k_1} \cdots p_r^{k_r} q_1^{j_1} \cdots q_s^{j_s}
\eee

Therefore, any positive divisor $d$ of $mn$ is uniquely representable in the form

\bee
d = p_1^{a_1} \cdots p_r^{a_r} q_1^{b_1} \cdots q_s^{b_s}, \quad 0 \leq a_i \leq k_i, 0 \leq b_i \leq j_i
\eee

We can split the product into two factors $d_1 d_2 = d$, where $d_1 = p_1^{a_1} \cdots p_r^{a_r}$ divides $m$ and $d_2 = q_1^{b_1} \cdots q_s^{b_s}$ divides $n$. All primes $p_i$ and $q_j$ are different, therefore $\gcd(d_1, d_2) = 1$. \qed

\paragraph{Example.} As an example, consider $m = 10, n = 21$. Their gcd is $1$ and their divisors are obtained via GAP as follows.

\begin{verbatim}
gap> GcdInt(10,21);
1gap> d1:=DivisorsInt(10);
[ 1, 2, 5, 10 ]
gap> d2:=DivisorsInt(21);
[ 1, 3, 7, 21 ]
\end{verbatim}

According to the theorem, the divisors of $mn = 210$ are therefore the products $d_1 d_2$ and are given as follows.

\begin{verbatim}
gap> DivisorsInt(210);
[ 1, 2, 3, 5, 6, 7, 10, 14, 15, 21, 30, 35, 42, 70, 105, 210 ]
gap> c:=Cartesian(d1, d2);
[ [ 1, 1 ], [ 1, 3 ], [ 1, 7 ], [ 1, 21 ], [ 2, 1 ], [ 2, 3 ], 
[ 2, 7 ], [ 2, 21 ], [ 5, 1 ], [ 5, 3 ], [ 5, 7 ], [ 5, 21 ], 
[ 10, 1 ], [ 10, 3 ], [ 10, 7 ], [ 10, 21 ] ]
gap> Apply(c, i -> i[1]*i[2]);
gap> c
> ;
[ 1, 3, 7, 21, 2, 6, 14, 42, 5, 15, 35, 105, 10, 30, 70, 210 ]
gap> Sort(c);
gap> c;
[ 1, 2, 3, 5, 6, 7, 10, 14, 15, 21, 30, 35, 42, 70, 105, 210 ]
\end{verbatim}

We see that the divisors of $210$ equal the products of $d_1 d_2$ whcih was stated in the theorem. Note the difference between a \emph{factor} and a \emph{divisor}; factors (usually) denote only prime factors whereas divisors $d$ are all numbers 


%%% Local Variables:
%%% mode: latex
%%% TeX-master: "journal"
%%% End:
