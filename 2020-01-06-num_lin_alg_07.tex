\DiaryEntry{Linear Algebra - QR Decomposition, Givens Rotations}{2020-01-06}{Linear Algebra}

The Householder transform zeros out all matrix elements but one in a row. On contrast, the Givens rotation zeros out one matrix element. 

Let's start with a simple two-dimensional case. We have a vector

\bee
\xbf = \begin{pmatrix} x_1 \\ x_2 \end{pmatrix}
\eee

and a rotation matrix

\bee
\Rbf = \begin{pmatrix} \cos \phi & -\sin\phi \\ \sin\phi & \cos\phi \end{pmatrix}
\eee

We can choose the rotation angle $\phi$ in such a way that the y-component becomes zero. This is achieved by choosing

\bee
\phi = \arctan \left( - \frac{x_2}{x_1} \right)
\eee

As an example, consider the following Julia session where we zero the second element of $\xbf$:

\begin{verbatim}
x=[2;5]
2-element Array{Int64,1}:
 2
 5
phi=atan(-5/2)
-1.1902899496825317
R=[cos(phi) -sin(phi); sin(phi) cos(phi)]
2×2 Array{Float64,2}:
  0.371391  0.928477
 -0.928477  0.371391
R*x
2-element Array{Float64,1}:
 5.385164807134504    
 2.220446049250313e-16
\end{verbatim}

This method can be (i) extended to higher-dimensions and (ii) several rotation matrices can be (multiplicatively) combined to zero out more than one matrix element.


\bee
\Abf = \begin{pmatrix} 1 & 4 & 6 \\ -2 & 5 & 10 \\ 4 & 2 & -5 \end{pmatrix}
\eee

We start with zeroing out the element $-2$ using the rotation matrix

\bee
\Rbf_1 = \begin{pmatrix} \cos \phi_1 & -\sin\phi_1 & 0 \\ \sin\phi_1 & \cos\phi_1 & 0 \\ 0 & 0 & 1 \end{pmatrix}
\eee

%%% Local Variables:
%%% mode: latex
%%% TeX-master: "journal"
%%% End:
