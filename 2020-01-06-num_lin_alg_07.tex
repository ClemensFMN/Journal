\DiaryEntry{Linear Algebra - QR Decomposition, Givens Rotations}{2020-01-06}{Linear Algebra}

The Householder transform zeros out all matrix elements but one in a row. On contrast, the Givens rotation zeros out one matrix element. 

Let's start with a simple two-dimensional case. We have a vector

\bee
\xbf = \begin{pmatrix} x_1 \\ x_2 \end{pmatrix}
\eee

and a rotation matrix

\bee
\Rbf = \begin{pmatrix} \cos \phi & -\sin\phi \\ \sin\phi & \cos\phi \end{pmatrix}
\eee

We can choose the rotation angle $\phi$ in such a way that the y-component becomes zero. This is achieved by choosing

\bee
\phi = \arctan \left( - \frac{x_2}{x_1} \right)
\eee

As an example, consider the following Julia session where we zero the second element of $\xbf$:

\begin{verbatim}
x=[2;5]
2-element Array{Int64,1}:
 2
 5
phi=atan(-5/2)
-1.1902899496825317
R=[cos(phi) -sin(phi); sin(phi) cos(phi)]
2×2 Array{Float64,2}:
  0.371391  0.928477
 -0.928477  0.371391
R*x
2-element Array{Float64,1}:
 5.385164807134504    
 2.220446049250313e-16
\end{verbatim}

This method can be (i) extended to higher-dimensions and (ii) several rotation matrices can be (multiplicatively) combined to zero out more than one matrix element. We will show this with a $3 \times 3$ example with matrix $\Abf$ given by

\bee
\Abf = \begin{pmatrix} 1 & 4 & 6 \\ -2 & 5 & 10 \\ 4 & 2 & -5 \end{pmatrix}
\eee

We have three directions to rotate vectors; we can rotate in the $x-y$ plane (leaving $z$ unchanged) and given by the following matrix

\bee
\Rbf_1 = \begin{pmatrix} \cos \phi_1 & -\sin\phi_1 & 0 \\ \sin\phi_1 & \cos\phi_1 & 0 \\ 0 & 0 & 1 \end{pmatrix}
\eee

The next option is to rotate in the $x-z$ plane which is given by the matrix

\bee
\Rbf_2 = \begin{pmatrix} \cos \phi_2 & 0 & -\sin\phi_2 \\ 0 & 1 & 0 \\ \sin\phi_2 & 0 & \cos\phi_2 \end{pmatrix}
\eee

and finally, we can rotate in the $y-z$ plane which is given by the matrix

\bee
\Rbf_3 = \begin{pmatrix} 1 & 0 & 0 \\ 0 & cos \phi_3 & -\sin\phi_3 \\ 0 & \sin\phi_3 & \cos\phi_3 \\ 0 & 0 & 1 \end{pmatrix}
\eee

We can choose the three rotation angles in such a way that the three elements on the subdiagonal of $\Abf$ become zero. For the example matrix from above, this is done in the following Julia session (code is \href{https://github.com/ClemensFMN/JuliaStuff/blob/master/qr_givens.jl}{here}).

\begin{verbatim}
A=[1 4 6;-2 5 10;4 2 -5]
3×3 Array{Int64,2}:
  1  4   6
 -2  5  10
  4  2  -5
\end{verbatim}

We start by zeroing out the element $-2$ by a rotation in the $x-y$ plane as follows

\begin{verbatim}
phi1 = atan(-A[2,1]/A[1,1])
  1.1071487177940904
R1 = [cos(phi1) -sin(phi1) 0; sin(phi1) cos(phi1) 0; 0 0 1]
3×3 Array{Float64,2}:
 0.447214  -0.894427  0.0
 0.894427   0.447214  0.0
 0.0        0.0       1.0
B = R1*A
3×3 Array{Float64,2}:
  2.23607      -2.68328  -6.26099
 -2.22045e-16   5.81378   9.8387 
  4.0           2.0      -5.0    
\end{verbatim}

With the next rotation, we zero out the element $4$.

\begin{verbatim}
phi2 = atan(-B[3,1]/B[1,1])
  -1.0610566479633896
R2 = [cos(phi2) 0 -sin(phi2); 0 1 0; sin(phi2) 0 cos(phi2)]
3×3 Array{Float64,2}:
  0.48795   0.0  0.872872
  0.0       1.0  0.0     
 -0.872872  0.0  0.48795 
C=R2*R1*A
3×3 Array{Float64,2}:
  4.58258      0.436436  -7.41941
 -2.22045e-16  5.81378    9.8387 
  4.44089e-16  3.31806    3.02529
\end{verbatim}

Now we have zeroed the two elements in the first column. Last element is $3.318$ wich can be zeroed as follows

\begin{verbatim}
phi3 = atan(-C[3,2]/C[2,2])
-0.5186145980395779
R3 = [1 0 0; 0 cos(phi3) -sin(phi3); 0 sin(phi3) cos(phi3)]
3×3 Array{Float64,2}:
 1.0   0.0       0.0     
 0.0   0.868507  0.495677
 0.0  -0.495677  0.868507
R3*R2*R1*A
3×3 Array{Float64,2}:
 4.58258      0.436436     -7.41941
 2.22045e-16  6.69399      10.0445 
 6.66134e-16  9.99201e-16  -2.24934
\end{verbatim}

We are done and have the following connections to the QR decomposition: $\Rbf_3 \Rbf_2 \Rbf_1 = \Qbf^T$ and $\Rbf_3 \Rbf_2 \Rbf_1 A = \Rbf$. We can compare with Julia's QR function

\begin{verbatim}
(R3*R2*R1)'
3×3 Adjoint{Float64,Array{Float64,2}}:
  0.218218  0.583323  -0.782378
 -0.436436  0.775393   0.456387
  0.872872  0.241866   0.423788
R3*R2*R1*A
3×3 Array{Float64,2}:
 4.58258      0.436436     -7.41941
 2.22045e-16  6.69399      10.0445 
 6.66134e-16  9.99201e-16  -2.24934
using LinearAlgebra
qr(A)
LinearAlgebra.QRCompactWY{Float64,Array{Float64,2}}
Q factor:
3×3 LinearAlgebra.QRCompactWYQ{Float64,Array{Float64,2}}:
 -0.218218  -0.583323  -0.782378
  0.436436  -0.775393   0.456387
 -0.872872  -0.241866   0.423788
R factor:
3×3 Array{Float64,2}:
 -4.58258  -0.436436    7.41941
  0.0      -6.69399   -10.0445 
  0.0       0.0        -2.24934
\end{verbatim}

The results are the same apart from some signs.

Note that the order in which elements are zeroed out \emph{is} important; for example, a rotation sequence $\Rbf_3, \Rbf_1, \Rbf_2$ does \emph{not} work; however, a sequence $\Rbf_2, \Rbf_1, \Rbf_3$ works.

%%% Local Variables:
%%% mode: latex
%%% TeX-master: "journal"
%%% End:
