\DiaryEntry{Hamilton Cycles and Euler Circuits}{2020-07-21}{Graphs}

\subsection{Hamilton Cycles}

A cycle containing all vertices in a graph is a \emph{Hamilton cycle} of the graph; a graph which contains a Hamilton cyacle is said to be \emph{Hamiltonian}. A path containing all vertices in a graph is a \emph{Hamilton path}.

The problem of finding a Hamilton cycle or path can be solved by an exhaustive search across all paths in the graph and there is no known algorithm which is fundamentally simpler; the problem is NP-hard.

Hamilton cycles are related to the \emph{travelling salesman problem}: A salesman shall make a tour and visit $n$ cities, returning to the starting city at the end of the tour. The distances (and therefore travel costs) between the cities are known and the tour shall be the shortest one. This problem is also NP-hard.

If the distances between all cities are the same, then the least expensive tour is any permutation of $n-1$ cities (the $n$-th city being the start / end point of the tour).

An additional twist is the condition that the salesman must not take a road (edge) again while there are other untravelled roads (edges). In order to solve this, we have to decompose the complete graph $K_n$ into a union of some edge-disjoint Hamilton cycles. \todo{continue?}


\subsection{Euler Circuits}

A graph is \emph{Eulerian} if there exists a circuit that visits every edge exactely once. Existence of an Euler circuit / trail is defined by a simple condition.

\begin{theorem} A non-trivial connected graph has an Eulerian circuit iff each vertex has even degree. A connected graph has an Euler trail from vertex $u$ to vertex $v \neq u$ iff $u$ and $v$ are the only vertices of odd degree.
\end{theorem}

Note that in case of a circuit, the first and last vertex are the same, whereas in case of a trail, the first and last vertex need not be the same.

The following Figure shows three example graphs. Starting with the graph on the left, we see that it has an Euler trail from vertex $1$ to vertex $4$ via $1-2-3-4-5-1-4$. All vertices on the path have even degree, only vertices $1$ and $4$ have odd degree. Therefore, no other Euler trail is possible. If we remove vertex $5$ (shown in the middle of the Figure), all vertices have even degree. This allows for an Eulerian circuit (and of course, Eulerian trails between all vertices are possible).

\begin{figure}[H]
\centering
\includegraphics[scale=0.4]{images/graphs_04_10.png}
\end{figure}


\begin{proof}
TBD
\end{proof}


%%% Local Variables:
%%% mode: latex
%%% TeX-master: "journal"
%%% End:
