\DiaryEntry{Number-theoretic Functions, Additional Stuff}{2021-01-20}{Number Theory}

This entry collects additional properties of the number-theoretic functions discussed before. As the previous entries it is based on \cite{Burton2011}; this entry is based particularly on the exercises in Sections 6.1 and 6.2.


\paragraph{Section 6.1, Exercise 8.} We have

\bee
\sum_{d | n} \frac{1}{d} = \frac{\sigma(n)}{n}.
\eee

We can prove this as follows. First we write down the LHS as

\bee
\sum_{d | n} \frac{1}{d} = \frac{1}{d_1} + \frac{1}{d_2} + \cdots + \frac{1}{d_r}
\eee

where the $d_i$ are the $r$ divisors of $n$. Note that the largest divisor (assume this to be $d_r$) equals $n$. We then can rewrite this as

\bee
\frac{1}{d_1} + \frac{1}{d_2} + \cdots + \frac{1}{d_r} = \frac{\frac{n}{d_1} + \frac{n}{d_2} + \cdots + \frac{n}{n}}{n}
\eee

The expressions $n/d_i$ are all integers (by the very definition of divisor) and themsevles are divisors. Consider the example $n = 14$ with divisors $1, 2, 7, 14$. The $n / d_i$ are the integers $14, 7, 2, 1$. We therefore have

\bee
\frac{\frac{n}{d_1} + \frac{n}{d_2} + \cdots + \frac{n}{n}}{n} = \frac{d_r + d_{r-1} + \cdots + d_1}{n} = \frac{\sigma(n)}{n} \qed
\eee

\paragraph{Section 6.1, Exercise 10a.} If $n$ has a prime factorization of $n = p_1^{k_1} \cdots p_r^{k_r}$, then

\bee
1 \geq \frac{n}{\sigma(n)} > \left(1 - \frac{1}{p_1}\right) \cdots \left(1 - \frac{1}{p_r}\right)
\eee

The first inequality can be shown as follows: $\sigma(n)$ is minimal, when $n$ is prime; in this case $\sigma(n) = 1 + n$. Inserting this into $\frac{n}{\sigma(n)}$ yields

\bee
\frac{n}{\sigma(n)} = \frac{n}{1+n} < 1 \qed
\eee

The second inequality is obtained by using the closed-form expression of $\sigma(n)$,

\bee
\sigma(n) = \prod_i \frac{p_i^{k_i+1} -1}{p_i - 1}
\eee

and inserting it into above to obtain

\be\label{2021-01-20:eq1}
\frac{n}{\sigma(n)} = \frac{p_1^{k_1} \cdots p_r^{k_r}}{\prod_i \frac{p_i^{k_i+1} -1}{p_i - 1}}
\ee

Taking out one factor involving $p_i$ we have

\bee
\frac{p_i^{k_i}}{\frac{p_i^{k_i+1} -1}{p_i - 1}} = \frac{(p_i - 1)p_i^{k_1}}{p_i^{k_1+1}-1} > \frac{(p_i - 1)p_i^{k_1}}{p_i^{k_1+1}} = \frac{p_i-1}{p_i} = 1 - \frac{1}{p_i}
\eee

This can be done for every factor in \eqref{2021-01-20:eq1} and we obtain

\bee
\frac{n}{\sigma(n)} > \left(1 - \frac{1}{p_1}\right) \cdots \left(1 - \frac{1}{p_r}\right) \qed
\eee


\paragraph{Section 6.1, Exercise 10b.} Exercise $8$ showed that 

\bee
\sum_{d | n} \frac{1}{d} = \frac{\sigma(n)}{n}.
\eee

Considering $n!$ instead of $n$, we have

\bee
\sum_{d | n!} \frac{1}{d} = \frac{\sigma(n!)}{n!}
\eee

Since $n! = 1 \cdot 2 \cdot 3 \cdots n$, it has \emph{at least} the divisors $1, 2, 3, \ldots, n$. The other divisors are made up of any combination of the integer factors $2, 3, 4, \ldots n$; e.g. $2 \cdot 4 \cdot 5 = 40$ is also a divisor of $5! = 120$.



We can insert this into the LHS and consider that we do not consider all divisors, therefore we ``just'' have a bound as

\bee
\frac{1}{1} + \frac{1}{2} + \frac{1}{3} + \cdots + \frac{1}{n} \leq \frac{\sigma(n!)}{n!} \qed
\eee





%%% Local Variables:
%%% mode: latex
%%% TeX-master: "journal"
%%% End:
