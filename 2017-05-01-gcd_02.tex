\DiaryEntry{GCD - Algorithms}{2017-05-01}{Algebra}


\subsection{Euclidean Algorithm}

\subsubsection{Basics}

\begin{theorem}
Assume that $a \geq b$, then we have the fact that $\gcd(a,b) = \gcd(a - b,b)$.
\end{theorem}

\paragraph{Proof.} Let $g = \gcd(a,b)$, then $a = a' g$ and $b = b' g$, with $a', b'$ being relatively prime (i.e. their gcd is $1$). Now, $a - b = g(a' - b')$ and therefore shares $g$ as common factor with $a = a'g$. In other words, $a' - b'$ and $a'$ are relatively prime. If this were not the case, we could write $a - b = g \tilde{g}(a'' - b'')$ and $a = a'' \tilde{g}g$ but this contradicts the fact that $g = \gcd(a,b)$. \qed

By repeatedly subtracting $b$ from $a$ (until a value smaller than $b$ is left), we arrive at the following identity

\be\label{gcd_02_1}
\gcd(a,b) = \gcd(a \bmod b, b)
\ee

\subsubsection{Algorithm.}

The principle behind the algorithm is to use \eqref{gcd_02_1} to subsequently reduce the values of $a$ and $b$, until one of them becomes zero. That's the end of the algorithm (we have $\gcd(x,0) = x$).

The algorithm starts with $a$ and $b$ and finds the greatest common divisor $\gcd(a,b)$. The algorithm proceeds in a series of steps and the integer $k$ counts the steps of the algorithm. The algorithms starts with $k=0$.

Each step takes two integers $r_{k-1}, r_{k-2}$ and calculates $q_k, r_k$ as follows

\bee
r_{k-2} = q_k r_{k-1} + r_k, \qquad r_k < r_{k-1}
\eee
%
In other words, multiples of the smaller number $r_{k-1}$ are subtracted from the larger number $r_{k-2}$ until the remainder $r_k$ is smaller than $r_{k-1}$.
%
Initially ($k=0$), $r_{-2} = a, r_{-1} = b$. For subsequent steps, we have

\begin{align*}
  a &= q_0 b + r_0 \\
  b &= q_1 r_0 + r_1 \\
  r_0 &= q_2 r_1 + r_2 \\
  r_1 &= q_3 r_2 + r_3 \\
  &\cdots  
\end{align*}

The algorithm ends, when a remainder $r_N$ becomes zero. Then $\gcd(a,b) = r_{N-1}$.

\subsubsection{Example}

As a first example, let us calculate $\gcd(1071, 462)$:

\begin{align*}
  1071 &= 2 \times 462 + 147 \\
  462 &= 3 \times 147 + 21 \\
  147 &= 7 \times 21 + 0
\end{align*}

With this the algorithm stops (the remainder is $0$) and the gcd becomes the previous remainder; i.e. $\gcd(1071, 462) = 21$.

As second example, consider $\gcd(1071, 463)$:

\begin{align*}
  1071 &= 2 \times 463 + 145 \\
  463 &= 3 \times 145 + 28 \\
  145 &= 5 \times 28 + 5 \\
  28 &= 5 \times 5 + 3 \\
  5 &= 1 \times 3 + 2 \\
  3 &= 1 \times 2 + 1 \\
  2 &= 1 \times 1 + 0
\end{align*}

Therefore, we have $\gcd(1071, 463) = 1$.
