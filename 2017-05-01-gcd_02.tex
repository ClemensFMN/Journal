\DiaryEntry{GCD - Algorithms}{2017-05-01}{Algebra}


\Subsection{Euclidean Algorithm}

Start with two numbers, $a$ and $b$ and find the greatest common divisor. The algorithm proceeds in a series of steps and the integer $k$ counts the steps of the algorithm. The algorithms starts with $k=0$.

Each step takes two integers $r_{k-1}, r_{k-2}$ and calculates $q_k, r_k$ as follows

\bee
r_{k-2} = q_k r_{k-1} + r_k, \qquad r_k < r_{k-1}
\eee
%
In other words, multiples of the smaller number $r_{k−1}$ are subtracted from the larger number $r_{k−2}$ until the remainder $r_k$ is smaller than $r_{k−1}$.
%
Initially ($k=0$), $r_{-2} = a, r_{-1} = b$. For subsequent steps, we have

\begin{align*}
  a &= q_0 b + r_0 \\
  b &= q_1 r_0 + r_1 \\
  r_0 &= q_2 r_1 + r_2 \\
  r_1 &= q_3 r_2 + r_3
  &\cdots  
\end{align*}

The algorithm ends, when a remainder $r_N$ becomes zero. Then $\gcd(a,b) = r_{N-1}$.
