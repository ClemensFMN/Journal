\DiaryEntry{Euclidean Distance Matrices}{2017-11-20}{Maths}

Based on Ivan Dokmanić, Reza Parhizkar, Juri Ranieri and Martin Vetterli, "Euclidean Distance Matrices - Essential Theory, Algorithms and Applications", also located \href{files/1502.07541.pdf}{here}. The first entry for this paper is here \ref{2016-04-21:entry}.

This entry discusses some issues that arose while implementing the "classical MDS" algorithm from the paper.


\subsection{The effect of scaling on angles}

Consider two vectors $v_1, v_2$ which are taken as

\bee
v_1 = [1 0]^T, v_2=[\cos \alpha \sin \alpha]^T
\eee

with $\alpha \in \mR$. The inner product of two vectors is defined as

\bee
v_1^T v_2 = ||v_1|| \cdot ||v_2|| \cos \phi
\eee

where $\phi$ is the angle between the two vectors. The inner product of the two vectors from above is

\bee
v_1^T v_2 = \cos \alpha
\eee

since they both are unit-length vectors. Now assume that the vectors are scaled by a matrix S, $S v_1$, $Sv_2$, with $S$ given by

\bee
S = \begin{bmatrix}
	s_x       & 0 \\
	0         & s_y
\end{bmatrix}
\eee

and consider the effect the scaling has on the inner product. We have

\bee
(Sv_1)^T (S v_2) = v_1^T S^T S v_2 = v_1^T 
\begin{bmatrix}
	s_x^2       & 0 \\
	0           & s_y^2
\end{bmatrix}
v_2 = s_x^2 \cos \alpha
\eee

The lengths of the vectors $v_1$ and $v_2$ are

\bee
||v_1|| = s_x , \qquad ||v_2|| = \sqrt{s_x^2 \cos^2 \alpha + s_y^2 \sin^2 \alpha}
\eee

Therefore

\bee
s_x^2 \cos \alpha = s_x \sqrt{s_x^2 \cos^2 \alpha + s_y^2 \sin^2 \alpha} \cos \alpha'
\eee

where $\alpha'$ denotes the angle between the scaled vectors $S_v1$ and $Sv_2$. Solving for $\cos \alpha'$ yields

\bee
\cos \alpha' = \frac{s_x \cos \alpha}{\sqrt{s_x^2 \cos^2 \alpha + s_y^2 \sin^2 \alpha}}
\eee

In general, the original angle $\alpha$ and the angle of the scaled vectors $\alpha'$ will differ; only in the special case of $s_x=s_y$, the two will be the same:

\bee
\cos \alpha' = \frac{s_x \cos \alpha}{\sqrt{s_x^2 \cos^2 \alpha + s_x^2 \sin^2 \alpha}} = \frac{\cos \alpha}{\sqrt{\cos^2 \alpha + \sin^2 \alpha}} = \cos \alpha
\eee


\subsection{Rotating the Result of Classical MDS}

The restoration of the points based on their distance is not unique; see Section II-A, especially equation 7, which shows that rotations and translation do not change euclidean distance matrix.

The eigenvalue decomposition in my implementation  \href{https://github.com/ClemensFMN/JuliaStuff/blob/master/articles/edm_01.jl}{here} does not shift, but rotate the point set. In order to revert the rotation, we multiply the result with a rotation matrix

\bee
R(\phi) = \begin{bmatrix}
	cos \phi -\sin \phi \\
	\sin \phi \cos \phi
\end{bmatrix}
\eee

where we need to determine the angle. When our points contains a unit vector in e.g. x-direction, we can proceed as follows: Assume that the classical MDS performs a rotation with $-\phi$, so that the unit vector $e_x$ becomes

\bee
R(-\phi) e_x = \begin{bmatrix}
	cos \phi \sin \phi \\
	-\sin \phi \cos \phi
\end{bmatrix} [1 0]^T = [\cos \phi -\sin \phi]^T
\eee

From this we can either (i) obtain $\phi$ via $\arccos$ or $\arcsin$ and use this to construct a rotation matrix $R(\phi)$ or (ii) directly use the values of $\sin \phi, \cos \phi$ to construct the rotation matrix. In any case, we use the constructed rotation matrix to revert the MDS-rotation and obtain the original point set.

%%% Local Variables:
%%% mode: latex
%%% TeX-master: "journal"
%%% End:
