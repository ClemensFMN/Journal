\DiaryEntry{Hypergeometric Functions}{2024-01-12}{Maths}

\subsection{Introduction}

We first define the rising factorial or Pochhammer symbol, $(q)_n$ for non-negative $n$ as

\bee
(q)_n = \begin{cases} 1 & n=0 \\
q(q+1)\cdots (q+n-1) &  n > 0 \end{cases}
\eee

For $q$ being an integer, an equivalent definition using factorials is as follows,

\begin{equation}
\label{2024-01-12:eq1}
(m)_n = \frac{(m+n-1)!}{(m-1)!}
\end{equation}

Some stuff for rising factorials can also be found in entry \ref{2015-10-29:entry} (but with slightly different notation). For our purposes here, it is important to note that $q$ can be a real / complex number. In case of $q$ being a negative integer, the Pochhammer symbol becomes zero; for example we have

\begin{align*}
(-2)_0 &= 1 \\
(-2)_1 &= -2 \\
(-2)_2 &= (-2)(-1) = 2 \\
(-2)_3 &= (-2)(-1)0 = 0
\end{align*}
  
For $m$ being a positive integer, we can write this as

\bee
(-m)_{n} = 0, \quad n \geq m+1
\eee

We can calculate values of the Pochhammer symbol using Python's sympy as follows.

\begin{verbatim}
from sympy import rf, Poly
from sympy.abc import x
rf(x, 5)
>> x*(x + 1)*(x + 2)*(x + 3)*(x + 4)
rf(-2,0)
>> 1
rf(-2,1)
>> -2
rf(-2,2)
>> 2
rf(-2,3)
>> 0
\end{verbatim}

When $m$ is an integer, the Pochhammer symbols have some special values (using \eqref{2024-01-12:eq1}),

\bee
(1)_m = \frac{(m+1-1)!}{(0)!} = m!, \quad (2)_m = \frac{(2+m-1)!}{1!} = (m+1)!
\eee

We can also express the binomial coefficients by means of Pochhammer symbols according to

\be
\label{2024-01-12:eq2}
\frac{(m)_n}{n!} = {m+n-1 \choose n}
\ee

Finally, we note that

\bee
(a)_{n+1} = a (a+1)_n
\eee

because

\bee
(a)_{n+1} = a(a+1) \cdots (a+n+1) = a \left[ (a+1) \cdots (a+n+1) \right] = a (a+1)_n \qed
\eee


With that in place, the hypergeometric function $_2 F_1(a,b;c;z)$ is defined as

\bee
_2 F_1(a,b;c;z) = \sum_{n=0}^\infty \frac{(a)_n (b)_n}{(c)_n} \frac{z^n}{n!} = 1 + \frac{ab}{c} \frac{z}{1!} + \frac{a(a+1)b(b+1)}{c(c+1)} \frac{z^2}{2!} + \cdots
\eee

Based on the discussion above about $(q)_n$ becoming zero, we note that the hypergeometric function reduces to a polynomial for either $a$ or $b$ being a negative integer,

\begin{align*}
_2 F_1(-m,b;c;z) &= \sum_{n=0}^m \frac{(a)_n (b)_n}{(c)_n} \frac{z^n}{n!} \\
_2 F_1(a,-m;c;z) &= \sum_{n=0}^m \frac{(a)_n (b)_n}{(c)_n} \frac{z^n}{n!}
\end{align*}

Note also that $_2 F_1(a,b;c;z)$ is undefined (or infinite) if $c$ equals a non-positive integer.

\subsection{Application / Special Cases}

The most simple case of the hypergeometric function is with $a = 1, b = c$ which yields the geometric series,

\bee
_2 F_1(1, b; b; z) = \sum_{n=0}^\infty \frac{(1)_n (b)_n}{(b)_n} \frac{z^n}{n!} = \sum_{n=0}^\infty z^n = 1 + z + z^2 + \cdots
\eee

A bit more advanced is the expansion of $(1 + x)^a$ which can be shown to be

\bee
(1 + x)^a = _2 F_1(-a,b; b; -z)
\eee

We have (using \eqref{2024-01-12:eq1})

\bee
_2 F_1(-a,b; b; -z) = \sum_{n=0}^\infty \frac{(-a)_n (b)_n}{(b)_n} \frac{(-z)^n}{n!} = \sum_{n=0}^\infty \frac{(-a)_n }{(b)_n (-z)^n}{n!} = \sum_{n=0}^\infty {-a+n-1 \choose n} (-z)^n
\eee

Interestingly, the last expression matches the series expansion of $(1+x)^a$,

\bee
(1+x)^a = \sum_n {a \choose n} x^n
\eee

Many of the common mathematical functions can be expressed in terms of the hypergeometric function, or as limiting cases of it. As an example consider the function

\bee
\frac{\ln(x+1)}{x}
\eee

which has the following series expansion

\bee
\frac{\ln(x+1)}{x} = 1 - \frac{x}{2} + \frac{x^2}{3} - \frac{x^3}{4} + \frac{x^4}{5} \mp \cdots
\eee

We can make an educated guess (or check Wikipedia) how we can express this series by means of the hypergeometric function,

\begin{align*}
_2F_1(1,1;2;-z) &= \sum_{m=0}^\infty \frac{(1)_m (1)_m}{(2)_m} \frac{(-z)^m}{m!} \\
&= \sum_{m=0}^\infty \frac{m! m!}{(m+1)!} \frac{(-z)^m}{m!} \\
&= \sum_{m=0}^\infty \frac{m!}{(m+1)!} (-z)^m \\
&= \sum_{m=0}^\infty \frac{(-z)^m}{m+1}
\end{align*}

Therefore, we have

\bee
\frac{\ln(x+1)}{x} = _2F_1(1,1;2;-z)
\eee

\subsection{Differentiation Formulas, Differential Equation}

We start with the following identity,

\begin{equation*}
    (a)_{n+1} = a (a+1)_n
\end{equation*}

This allows us to arrive with the following identity,

\begin{equation*}
    \frac{d}{dz} _2 F_1(a,b;c;z) = \frac{ab}{c} _2 F_1(a+1,b+1;c+1;z)
\end{equation*}

The hypergeometric function is a solution of the hypergeometric differential equation,

\begin{equation*}
    z(1-z) \frac{d^2 w}{dz^2} + \left[c - (a+b+1)\right] \frac{dw}{dz} - abw = 0
\end{equation*}




%%% Local Variables:
%%% mode: latex
%%% TeX-master: "journal"
%%% End:
