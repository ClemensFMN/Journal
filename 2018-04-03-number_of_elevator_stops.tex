\DiaryEntry{Expected Number of Elevator Stops}{2018-04-04}{maths}

Based on "Problems in Applied Mathematics - Selections from SIAM Review", problem 72-20.

$P$ people enter an elevator on the ground floor. The elevator has $N$ stops and assume that the probability of a person leaving the elevator is the same (for all floors and all people). Determine the expected number of stops until the elevator is emptied.

Define the RV $x_i$ to have value $1$ if the $i$-th floor is a stop, and $0$ otherwise. Since this is a binary variable, we have $E(x_i) = P(x_i)$. The probability that no one stops at floor $i$, $P(x_i=0)$, is the probability that all $P$ persons chose to exit the elevator at a different floor. One person does this with probability $1-1/N$; since they are independent, we have

\bee
P(x_i=0)= (1-1/N)^P
\eee

and from this follows $P(x_i=1)= 1-(1-1/N)^P$. The expected number of stops is given by

\bee
E \left( \sum_{i=0}^N P(x_i=1) \right) = N \left[ 1-(1-1/N)^P \right]
\eee

If $N$ goes to infinity (big house), but $\lim_{n \rightarrow \infty} P/N = A$, then

\bee
E \left( \sum_{i=0}^N P(x_i=1) \right) = N \left[ 1-e^{-A} \right]
\eee

The calculation depends on the independence of the persons and floors but does not depend on the uniform probability. If person $k$ stops at floor $i$ with probability $p_{ik}$, then the expected number of stops is

\bee
\sum_{i=1}^N \left[ 1-\prod_{k=1}^P (1-p_{ik}) \right]
\eee

If all persons have the same preference regarding the stops; i.e. $p_{ik} = p_i$, then the expected number of stops is

\bee
\sum_{i=1}^N \left[ 1- (1-p_{i})^P \right]
\eee

\subsection{Results}

Assume $P=10$ and $N = 10$.

\begin{itemize}
	\item Scenario 1: $p_1 = \cdots = p_{10} = 0.1$ (uniform distribution). This yields an expected number of steps as $6.513$. If we increase the number of people to $P=100$, the expected number of steps becomes $10$.
	
	\item Scenario 2: $p_1 = p_2 = p3 = p_7 = \cdots= p_{10} = 0, p_4 = p_5 = p_6 = 1/3$. This yields an expected number of steps as $2.948$, and for $P=100$ the expected number of steps is $3$.
\end{itemize}


