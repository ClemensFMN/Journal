\DiaryEntry{Student Distribution}{2019-07-04}{Stochastic}

\subsection{Estimating Mean and Variance}

We have $N$ observations $x_1 \cdots x_N$ of a Gaussian RV, $X \sim \Nc(\mu, \sigma^2)$ and want to obtain estimates for mean and variance from the samples.

The \emph{sample mean} $\hat \mu$ is given by

\be\label{2019-07-04:eq1}
\hat \mu = \frac{1}{N} \sum_{i=1}^N x_i
\ee

Its expected value is

\bee
E\{\hat \mu\} = E\left\{ \frac{1}{N} \sum_{i=1}^N x_i \right\} = \mu
\eee

which means the estimator is \emph{consistent}. \qed

An intuitive estimator for the variance would be

\bee
\hat \sigma^2 = \frac{1}{N} \sum_{i=1}^N (x_i - \hat\mu)^2
\eee

however it turns out that this estimator is \emph{not} consistent; i.e.

\bee
E\{\hat\sigma^2\} = E\left\{ \frac{1}{N} \sum_{i=1}^N (x_i - \hat\mu)^2 \right\} = \frac{1}{N} \sum_{i=1}^N E\left\{ (x_i - \hat\mu)^2 \right\} = \frac{1}{N} \sum_{i=1}^N E\left\{ x_i^2 - 2 x_i \hat\mu + \hat\mu^2 \right\}
\eee

Let's tackle the terms one-by-one: We have

\bee
E\left\{ x_i^2 \right\} = \mu^2 + \sigma^2
\eee

The next one is a bit more complicated,

\bee
E\left\{ x_i \hat\mu \right\} = E\left\{ x_i \frac{1}{N} \sum_{j=1}^N x_i \right\} = \frac{1}{N} \left( (N-1) \mu^2 + \mu^2 + \sigma^2 \right) = \mu^2 + \frac{\sigma^2}{N}
\eee

Here we have used the fact that 

\bee
E\{x_i x_j\} = \begin{cases} \mu^2 \quad i \neq j \\ \mu^2 + \sigma^2 \quad i=j \\  \end{cases}
\eee

The two RVs $x_i, x_j$ can be interpreted as mixed DC/AC signals and $E\{x_i x_j\}$ measures the power: If $i \neq j$, then just the power from the DC part (i.e. $\mu$) is relevant as the AC parts cancel each other out (the RVs are independent). If $i = j$, then the full DC and AC power is observed. The first case appears $N-1$-times in the sum above; the latter case only once.

Finally, we have

\bee
E\left\{ \hat\mu^2 \right\} = E\left\{ \frac{1}{N} \sum_i x_i \sum_j x_j \right\} = \frac{1}{N^2} \left( N (\mu^2 + \sigma^2) + (N^2-N) \mu^2 \right) = \frac{\sigma^2 + N \mu^2}{N}
\eee

Putting it all together, we obtain

\begin{align*}
E\{\hat\sigma^2\} &= \frac{1}{N} \sum_{i=1}^N \left( \mu^2 + \sigma^2 - 2 \left( \mu^2 + \frac{\sigma^2}{N} \right) + \frac{\sigma^2 + N \mu^2}{N} \right) = \frac{1}{N} \sum_{i=1}^N \sigma^2 \left(1 - \frac{1}{N} \right) \\ &= \sigma^2 \left(1 - \frac{1}{N} \right) = \sigma^2 \frac{N-1}{N}
\end{align*}

The bad news is that the estimator is \emph{not} consistent; the good news is that we can derive a consistent estimator $\hat S$ easily as follows. Let's define this new estimator as

\bee
\hat S = A \hat \sigma^2
\eee

and we therefore have $E\{\hat S\} = A \sigma^2 \frac{N-1}{N}$. We ask for a value $A$ that $\hat S$ becomes consistent; i.e.

\bee
A \sigma^2 \frac{N-1}{N} = \sigma^2 \rightarrow A = \frac{N}{N-1}
\eee

and therefore the consistent estimator becomes

\be\label{2019-07-04:eq2}
\hat S = \frac{1}{N-1} \sum_{i=1}^N (x_i - \hat\mu)^2 \qed
\ee

A Julia simulation is done \href{https://github.com/ClemensFMN/JuliaStuff/blob/master/stochastic/student_0.jl}{here}, showing the difference between the biased and unbiased estimator.


\subsection{Student Distribution}

If we shift and normalize $\hat \mu$, it becomes Gaussian according to

\bee
\frac{\hat \mu - \mu}{\sigma / \sqrt{N}} \sim \Nc(0, 1)
\eee

If we replace the true variance with the sample mean, we obtain

\bee
t_{n-1} = \frac{\hat \mu - \mu}{\sqrt{\hat S / N} }
\eee

which has a Student distribution with $N-1$ degrees of freedom. The pdf is given by

\bee
f_n(x) = \frac{\Gamma\left(\frac{n+1}{2}\right)} {\sqrt{n\pi}~\Gamma\left(\frac{n}{2}\right)}\left(1+\frac{x^{2}}{n}\right)^{-\frac{n+1}{2}}
\eee

The following Figure shows the pdf: It can be seen that the Student distribution is "fatter" than the Normal distribution; i.e. it falls off less quickly. This effect is stronger for smaller values of $N$.

\begin{figure}[hbt!]
\centering
\includegraphics[scale=0.65]{images/Student_2.png}
\end{figure}


A simulation result for $N=3$ is shown in the following Figure. The blue dots correspond to the histogram of the simulation, the red curve is the Student distribution and the green curve curve is a $\Nc(0,1)$ Normal distribution.

\begin{figure}[hbt!]
\centering
\includegraphics[scale=0.65]{images/Student_1.png}
\end{figure}

Using the fact from above that 

\bee
t_{n-1} = \frac{\hat \mu - \mu}{\sqrt{\hat S / N} }
\eee

has Student distribution, we can give confidence intervals for the sample mean $\hat \mu$. The distance between mean $\mu$ and sample mean $\hat \mu$ is therefore distributed according to $t_{n-1} \sqrt{\hat S / N}$ and we have

\be\label{2019-07-04:eq3}
\hat \mu - t^\star \sqrt{\hat S / N} \leq \mu \leq \hat \mu + t^\star \sqrt{\hat S / N}
\ee

where $t^\star$ is chosen so that a percentage of $p$P (a given confidence interval) is in the interval $[-t^\star; t^\star]$. We define $F_N(t)$ as the cdf of the Student distribution with $N$ degrees of freedom; i.e. $P(T<t) = F(t)$. Since the Student distribution is symmetric, we have $p = F(t^\star) - F(-t^\star)$. Furthermore, if $p$ is concentrated in $[-t^\star; t^\star]$, we must have $\frac{1-p}{2}$ to "the left" and to "the right"; i.e.

\bee
F(t^\star) = \frac{1-p}{2} + p = \frac{p+1}{2}
\eee

From this, $t^\star$ can be obtained and therefore the confidence interval \eqref{2019-07-04:eq3}. The cool thing is that we observe $N$ RVs from which we \emph{only} know that they are Gaussian, both mean and variance are unknown. Equations \eqref{2019-07-04:eq1}, \eqref{2019-07-04:eq2} provide estimates for mean and variance, and \eqref{2019-07-04:eq3} gives the confidence interval for the mean estimate. Sidenote: A confidence interval for the variance would also be interesting...

Simulation is a bit tricky as the confidence interval depends on the sample mean (it is around it). What we did instead was to generate (lots of) $N$ normal samples, calculate the confidence interval, and check how often the true mean falls within the confidence interval. Simulation is \href{https://github.com/ClemensFMN/JuliaStuff/blob/master/stochastic/student_3.jl}.



%%% Local Variables:
%%% mode: latex
%%% TeX-master: "journal"
%%% End:
