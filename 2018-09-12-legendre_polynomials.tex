\DiaryEntry{Legendre Polynomials - Orthogonality}{2018-09-12}{Maths}

Legendre polynomials $P_m(x)$ are solution to the differential equation

\bee
\frac{d}{dx} \left[ (1-x^2) \frac{dP_n(x)}{dx}  \right] + n(n+1)P_n(x) = 0
\eee

and can be defined by the recurrence relation

\bee
(n+1)P_{n+1}(x) = (2n+1)xP_n (x) - nP_{n-1}(x)
\eee

They are orthogonal according to

\bee
\langle P_m, P_n \rangle = \int_{-1}^1 P_m(x) P_n(x) dx = \frac{2}{2n+1} \delta_{m,n}
\eee


Based on \href{https://math.stackexchange.com/questions/2499216/legendre-polynomial-orthogonality-integral?noredirect=1&lq=1}{this} article, we want to prove this relation.

\paragraph{The case $m \neq n$.} We start by writing out the defining ODE for both $m, n$ (and leaving out the variable $x$ in the Legendre Polynomials):

\begin{align*}
  &\frac{d}{dx} \left[ (1-x^2) \frac{dP_m}{dx}  \right] + m(m+1)P_m = 0 \\
  &\frac{d}{dx} \left[ (1-x^2) \frac{dP_n}{dx}  \right] + n(n+1)P_n = 0
\end{align*}

Multiplying both equations with $P_n$ and $P_m$, respectively and subtracting them, yields the following expression

\be\label{20180912:eq1}
P_n \frac{d}{dx} \left[ (1-x^2) \frac{dP_m}{dx}  \right] - P_m \frac{d}{dx} \left[ (1-x^2) \frac{dP_n}{dx}  \right] + [ m(m+1) - n(n+1)] P_m P_n = 0
\ee

Now comes a clever ``trick'', by noting that (product rule of differentiation)

\bee
\frac{d}{dx} \left[ P_n (1-x^2) \frac{dP_m}{dx} \right] = \frac{P_n}{dx} (1-x^2) \frac{dP_m}{dx} + P_n \frac{d}{dx} \left[ (1-x^2) \frac{dP_m}{dx} \right]
\eee

which we can rearrange to obtain

\bee
P_n \frac{d}{dx} \left[ (1-x^2) \frac{dP_m}{dx} \right] = \frac{d}{dx} \left[ P_n (1-x^2) \frac{dP_m}{dx} \right] - \frac{P_n}{dx} (1-x^2) \frac{dP_m}{dx}
\eee

and this is exactely the kind of expression which appears twice in \eqref{20180912:eq1},

\begin{align*}
  & \frac{d}{dx} \left[ P_n (1-x^2) \frac{dP_m}{dx} \right] - \frac{P_n}{dx} (1-x^2) \frac{dP_m}{dx} - \\
  - & \frac{d}{dx} \left[ P_m (1-x^2) \frac{dP_n}{dx} \right] + \frac{P_m}{dx} (1-x^2) \frac{dP_n}{dx} + \\
  + & (m^2 + m - n^2 - n) P_m P_n = 0
\end{align*}

This expression can be simplified to

\bee
\frac{d}{dx} \left[ P_n (1-x^2) \frac{dP_m}{dx} \right] -  \frac{d}{dx} \left[ P_m (1-x^2) \frac{dP_n}{dx} \right] + (m^2 + m - n^2 - n) P_m P_n = 0
\eee

Now we integrate the expression wrt to $x$,

\bee
\int_{-1}^1 \frac{d}{dx} \left[ P_n (1-x^2) \frac{dP_m}{dx} \right] -  \frac{d}{dx} \left[ P_m (1-x^2) \frac{dP_n}{dx} \right] + (m^2 + m - n^2 - n) P_m P_n dx = 0
\eee

The integral of the first two simply inverses the $\frac{d}{dx}$ and we obtain

\bee
\left. \left[ P_n (1-x^2) \frac{dP_m}{dx} \right]\right|_{-1}^1 - \left. \left[ P_m (1-x^2) \frac{dP_n}{dx} \right] \right|_{-1}^1 + (m^2 + m - n^2 - n) \int_{-1}^1 P_m P_n dx = 0
\eee

The term $1-x^2$ causes the first two expressions to become zero and we have proven the $m \neq n$ part,

\bee
\int_{-1}^1 P_m P_n dx = 0 \, , \quad m \neq n \qed
\eee


\paragraph{The case $m = n$.} Let us define

\bee
A_n = \int_{-1}^1 P_n^2 dx
\eee

Next we use the recurrence relation for $n \rightarrow n-1$; i.e.

\bee
n P_{n} = (2n-1)xP_{n-1} - (n-1) P_{n-2}
\eee

and from there

\bee
P_n = \frac{2n-1}{n} x P_{n-1} - \frac{n-1}{n} P_{n-2}
\eee

Using this once in our definition of $A_n$, we obtain

\begin{align*}
  A_n & = \int_{-1}^1 P_n  \frac{2n-1}{n} x P_{n-1} - \frac{n-1}{n} P_{n-2} dx \\
  & = \int_{-1}^1 \frac{2n-1}{n} x P_n P_{n-1} - \frac{n-1}{n} P_n P_{n-2} dx
\end{align*}

The integral of the second term becomes zero because of orthogonality and we are left with

\be\label{20180912:eq2}
A_n = \frac{2n-1}{n} \int_{-1}^1 x P_n P_{n-1} dx
\ee

Now let's use the original recurrence relation to express $x P_n$,

\bee
xP_n = \frac{n+1}{2n+1} P_{n+1} + \frac{n}{2n+1} P_{n-1}
\eee

and substitute this back into \eqref{20180912:eq2},

\bee
A_n = \frac{2n-1}{n} \left( \int_{-1}^1 \frac{n+1}{2n+1} P_{n+1} P_{n-1} + \frac{n}{2n+1} P_{n-1} P_{n-1} dx \right)
\eee

The integral of the first integrand vanishes because of orthogonality and we are left with

\bee
A_n = \frac{2n-1}{2n+1} \int_{-1}^1 P_{n-1}^2 dx
\eee

and we can build on this as

\begin{align*}
  A_n &= \frac{2n-1}{2n+1} A_{n-1} \\
      &= \frac{2n-1}{2n+1} \frac{2n-3}{2n-1} A_{n-2} = \cdots \\
      &= \frac{2n-1}{2n+1} \frac{2n-3}{2n-1} \cdots \frac{3}{5} \frac{1}{3} A_0
\end{align*}

For $n=0$, we have $P_0(x) = 1$ and therefore

\bee
A_0 = \int_{-1}^1 1 dx = 2
\eee

This finally yields

\bee
\int_{-1}^1 P_n^2 = \frac{2}{2n+1} \qed
\eee




%%% Local Variables:
%%% mode: latex
%%% TeX-master: "journal"
%%% End:
