\DiaryEntry{Rings - Examples}{2016-07-20}{Algebra}

\subsection{Example $\mZ / 5\mZ$}

As an example consider the ring \(\mathbb{Z}/5\mathbb{Z}\) with the following table for addition (mod-5)

\[
\begin{array}{c|ccccc}
+  & 0 & 1 & 2 & 3 & 4 \\
\hline
0  & 0 & 1 & 2 & 3 & 4 \\
1  & 1 & 2 & 3 & 4 & 0 \\
2  & 2 & 3 & 4 & 0 & 1 \\
3  & 3 & 4 & 0 & 1 & 2 \\
4  & 4 & 0 & 1 & 2 & 3 \\
\end{array}
\]

Every row contains exactely one zero; in other words, every element has exactely one additive inverse and the set with addition forms an abelian group.

The multiplication (mod-5) table looks like this

\[
\begin{array}{c|ccccc}
\times  & 0 & 1 & 2 & 3 & 4 \\
\hline
      0 & 0 & 0 & 0 & 0 & 0 \\
      1 & 0 & 1 & 2 & 3 & 4 \\
      2 & 0 & 2 & 4 & 1 & 3 \\
      3 & 0 & 3 & 1 & 4 & 2 \\
      4 & 0 & 4 & 3 & 2 & 1 \\
\end{array}
\]

Every row (except zero) contains exactely one element 1; this implies that every element apart from zero has exactely one multiplicative inverse.

The set is actually not only a ring but a field as every element has a multiplicative inverse.

\subsection{Example $\mZ/6\mZ$}

As an example of a ring consider the field \(\mathbb{Z}/6\mathbb{Z}\) with the following table for addition

\[
\begin{array}{c|cccccc}
+  & 0 & 1 & 2 & 3 & 4 & 5 \\
\hline
0  & 0 & 1 & 2 & 3 & 4 & 5 \\
1  & 1 & 2 & 3 & 4 & 5 & 0 \\
2  & 2 & 3 & 4 & 5 & 0 & 1 \\
3  & 3 & 4 & 5 & 0 & 1 & 2 \\
4  & 4 & 5 & 0 & 1 & 2 & 3 \\
5  & 5 & 0 & 1 & 2 & 3 & 4 \\
\end{array}
\]

Every element has an additive inverse;i.e. the set forms a group with respect to addition. The multiplication table has the following form

\[
\begin{array}{c|cccccc}
\times  & 0 & 1 & 2 & 3 & 4 & 5 \\
\hline
     0  & 0 & 0 & 0 & 0 & 0 & 0 \\
     1  & 0 & 1 & 2 & 3 & 4 & 5 \\
     2  & 0 & 2 & 4 & 0 & 2 & 4 \\
     3  & 0 & 3 & 0 & 3 & 0 & 3 \\
     4  & 0 & 4 & 2 & 0 & 4 & 2 \\
     5  & 0 & 5 & 4 & 3 & 2 & 1 \\
\end{array}
\]

The rows with element 2, 3, and 4 have more than one zero; i.e.~the element 2 has two multiplicative inverses, namely 3 and 5, because \(2\times3=2\times5=0\). It is exactely the rows with value \(k\) for
which \(\gcd(k,6) \neq 1\) for which no unique multiplicative inverse exists.

However, $\mZ/6\mZ$ is a ring; the set with multiplication need not be a group (and therefore have aninverse).
