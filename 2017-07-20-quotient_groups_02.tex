\DiaryEntry{Quotient Groups, II (Examples)}{2017-07-20}{Algebra}

\subsection{Modulo Operation}

The modulo operation is a homomorphism: Consider $\phi(x) = x \bmod n$ (with fixed $n$ and we have $\phi(a+b) = a+b \bmod n = a \bmod n + b \bmod n = \phi(a) + \phi(b)$. \qed

The kernel of this mapping is the set of all integers which are divisible by $n$ (or in other words, the multiples of $n$): $\text{ker}(\phi) = \{x | x \bmod n = 0\} = \{ xn | x \in \mZ\}$.i The fibers of the mapping $\phi$ over $a$ are all numbers in $\mZ$ which have remainder $a$ after division by $n$. For the representative of such a fiber we will write $\bar{a}$.

Consider now the group $G = \mZ$ with group operation addition (denoted as $+_g$) under this mapping which produces a subgroup $H$. The group operation of $H$ (denoted as $+_H$) for two elements $a, b$ can be obtained as follows: We take any element $a'$ of $G$ which is contained in the fiber over $a$ and add any element $b'$ of the fiber over $b$ to obtain $a' +_G b' = c'$. Then we find the representative of the element $c'$ which we denote as $c$. We therefore have $a +_H b = c$.
Note that for the final result (i.e. $c$) it does \emph{not} matter which elements $a'$ and $b'$ we choose; if we take two different elements $a'', b''$, the result of the group operation will be different $a'' +_G b'' = c'' \neq c'$, but the resulting representative $c$ will be the same.





\subsection{Vector Addition in $\mR^2$}

Take $G = \mR^2$ with group operation vector addition (the sum of two vectors is again a vector, the zero vector is the identity element and every element has an inverse, therefore we have a group). Let $H = \mR$ with group operation addition. Define $\phi([x,y]) = x$; i.e. the function projects (the two-dimensional) vectors onto the x-Axis. This is a homomorphism, as

\bee
\phi([x_1,y_1] + [x_2,y_2]) = \phi([x_1+x_2,y_1+y_2]) = x_1 + x_2 = \phi([x_1,y_1]) +  \phi([x_2,y_2]) 
\eee

The kernel of the function is

\bee
\text{ker}(\phi) = \{[x,y | \phi([x,y]) = 0]\} = \{[x,y] | x=0\}
\eee

which is the y-Axis. The kernel is a subgroup of $G$ (the sum of two vectors with zero x-component is again such a vector, there is an identity element, and there exists an inverse element for every subgroup element)and the fiber of $\phi$ over $a \in mR$ is the translate of the y-Axis over $a$; i.e. the set $\{[x,y] | x=a\}$. This is also the left (and right) coset with representative $[a,0]$ wich we write as $\bar{[a,0]}$.

The group operation on $H$ is either described using the map $\phi$: The sum of the line $x=a$ and the line $x=b$ is the line $x=a+b$ or via coset representatives: The sum of $\bar{[a,0]}$ and $\bar{[b,0]}$ is $\bar{[a+b,0]}$.

\subsection{Cyclic Subgroups $\mZ_n$ (Obsolete)}

Let $G = \mZ$ and $H = \mZ_n$; i.e. the cyclic group of order $n$. The group operation in $G$ is addition and in $H$ multiplication (modulo-$n$).

Define a function $\phi: \mZ \rightarrow \mZ_n$ by $\phi(a) = x^a$. We have $\phi(a + b) = x^{a+b} = x^a x^b = \phi(a) \phi(b)$ which shows that $\phi$ is a homomorphism.

The kernel of $\phi$ are therefore all numbers $m$ for which $m \equiv 0 \bmod n$; i.e. all multiples of $n$. The fiber of $\phi$ over $x^a$ are all $m$ for which $m \equiv a \bmod n$. For the representative of the fiber over $a$ we write $\bar{a}$.

The group operation for the fibers is $\bar{a} \bar{b} = \bar{a+b}$.

\subsection{Homomorphisms}

It is actually interesting how many functions are homomorphisms and how difficult it is to find a function which is \emph{not} a homomorphism:

\begin{itemize}

\item Every linear function is a homomorphism: $\phi(x+y) = \phi(x) + \phi(y)$. However, there are functions which are not linear and homomorphic nevertheless.

\item Squaring $\phi(x) = x^2$ with multiplication as group operation is a homomorphism $\phi(xy) = (xy)^2 = x^2 y^2$. This works for other exponents as well. As an example consider $\mZ_9^\star = \{1,2,4,5,7,8\}$. The mapping $\phi(x) = x^2$ yields the following: $1\rightarrow 1, 2\rightarrow 4, 4 \rightarrow 7, 5 \rightarrow 7, 7 \rightarrow 4, 8 \rightarrow 1$. The kernel is therefore $N = \text{ker}(\phi) = \{1,8\}$ and we have the following cosets: $1N = 8N = \{1,8\}, 2N = 7N = {2,7}, 4N = 5N = {4,5}$. The resulting quotient group has the following operation table

\bee
\begin{array}{c|ccc}
\star & 1 & 2 & 4 \\
\hline 
1     & 1 & 2 & 4 \\
2     & 2 & 4 & 1 \\
4     & 4 & 1 & 2
\end{array}
\eee

As an example consider $2 \star 4$. We can either use any element from the corresponding cosets, multiply them and return the corresponding coset's representative as result: $2 \cdot 4 \equiv 8 \bmod 9$ or $2 \cdot 5 \equiv 1 \bmod 9$, all of them produce an element of the coset $1N = 8N$. Or we multiply the representatives and obtain the result directly $2 \cdot 4 \equiv 8 \bmod 9$ which is the representative of $1N = 8N$.

\item Squaring with addition as group operation does not produce a homomorphism $\phi(x+y) = (x+y)^2 \neq x^2 + y^2 = \phi(x) + \phi(y)$.



\end{itemize}
