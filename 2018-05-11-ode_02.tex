\DiaryEntry{First Order Differential Equations (II)}{2018-05-15}{ODE}

\subsubsection{In-homogenuous ODEs}

The solution to the generic in-homogeneous ODE

\bee
y'(t) - a(t)y(t) = g(t), \quad y(0) = y_0
\eee

is given by

\bee
y(t) = y_0 \exp \left( \int_0^t a(s) ds \right) + \int_0^t g(s) \left( \exp \int_s^t a(r)dr \right) ds
\eee

The first integral is the solution to the homogeneous ODE $y'(t) - a(t)y(t) = 0$ (see also last integral in the previous post). The second integral considers the effect of the RHS $g(t)$: Like an initial condition, $g(t)$ is multiplied with $\exp \int_s^t a(r)dr$; but $g(t)$ acts "longer" on the solution, therefore we integrate over all contributions over time (up to $t$).

\paragraph{Constant Input.} We start with the most simple example

\bee
y'(t) - a y(t) = g, \quad y(0) = y_0
\eee

Inserting this into the solution, we obtain

\begin{align*}
y(t) & = y_0 \exp (at) + \int_{s=0}^t g \exp \left[ a(t-s)\right] ds \\
&= y_0 \exp(at) +gexp(at) \int_{s=0}^t g \exp \left[ -as \right] ds \\
&= e^{at} (y_0 + g/a) - g/a
\end{align*}

From this we see that the constant input contributes in two ways to $y(t)$: (i) It acts like an (additional) initial value and correspondingly follows $e^{at}$, and (ii), affects the steady state $\lim_{t \rightarrow \infty} y(t)$, after all transients have decayed. We can see the latter effect by assuming a constant $y(t)$, therefore $y' = 0$: We have $-ay(t) = g \rightarrow y(t) = -g/a$. Assuming $a < 0$ (otherwise there is no steady solution!), $y(t) = g/a$ for $t \rightarrow \infty$. The constant $a$ can be interpreted as "damping factor"; the larger $a$ is, the smaller the contribution from $g(t)$.

\paragraph{Sinus Input.} We next consider a sinusoidal input function,

\bee
y'(t) - ay(t) = \sin(t)
\eee

The solution from above yields

\begin{align*}
y(t) &= y_0 \exp \left( \int_0^t a ds \right) + \int_0^t \sin(s) \left( \exp \int_s^t a dr \right) ds = y_0 \exp \left( at \right) + \int_0^t \sin(s) \exp \left( a(t-s) \right) ds \\
&= y_0 \exp \left( at \right) + \exp(at) \int_0^t \sin(s) \exp (-as) ds
\end{align*}

The integral can be calculated analytically and the whole expression can be simplified to

\bee
y(t) = y_0 \exp \left( at \right) - \frac{a \sin(t)+\cos(t) - e^{a t}}{a^2+1} =  \frac{\left[ y_0(a^2+1) + 1 \right] e^{a t} - a \sin(t)-  \cos(t) }{a^2+1}
\eee

We again see two contributions: An effect of the initial conditions (which decay for $a < 0$) and a steady-state solution which is controlled by $g(t)$. As in the previous example, the factor $a$ controls how strongly $y(t)$ is affected by $g(t)$.


A solution by Maxima can be obtained as follows

\begin{verbatim}
(%i19)	de:'diff(u,t)-a*u-sin(t);
(de)	'diff(u,t,1)-a*u-sin(t)
(%i20)	gsoln:ode2(de,u,t);
(gsoln)	u=%e^(a*t)*((%e^(-a*t)*(-a*sin(t)-cos(t)))/(a^2+1)+%c)
(%i27)	ic1(gsoln, t=0, u=y0), ratsimp;
(%o27)	u=((a^2+1)*%e^(a*t)*y0-a*sin(t)-cos(t)+%e^(a*t))/(a^2+1)
\end{verbatim}

\todo{plot}

\paragraph{Sinus Input with varying frequency.} We can extend the previous example to

\bee
y'(t) - ay(t) = \sin(\omega t)
\eee

\todo{continue, plot}