\DiaryEntry{Dice Problems, I}{2016-05-11}{Stochastic}

This is based on problems 1, 2, and 3
\href{http://www.madandmoonly.com/doctormatt/mathematics/dice1.pdf}{here};
locally stored \href{\%7Bfilename\%7D/files/dice1.pdf}{here}.

\subsection{Problem 1; Roll until a 6 comes up}

On average, how many times must a 6-sided die be rolled until a 6 turns
up?

One way of answering this question is to observe that the distribution
of throwing a 6 follows a
\href{\%7Bfilename\%7D2015-08-18-Geometric_Distribution.markdown}{Geometric
Distribution} and use the expectation of the distribution.

Another way, which is followed in the linked document, is to directly
set up an equation for the expectation E. When rolling a die for the
first time, there is a chance of 1/6 of rolling a 6. In this case, we
are done and the expected value \(E=1\). In all other cases (we throw
1\ldots{}5) with probability 5/6, we need to continue with another dice
throw. In other words, we start all over again. In this case, the
expected value is \(E+1\) (whatever value E is).

So we have

\[
E = \frac{1}{6} \times 1 + \frac{5}{6} \times (E+1)
\]

from which follows \(E=6\).

\subsection{Problem 2, Sequence of two sixes}

On average, how many times must a 6-sided die be rolled until a 6 turns
up twice in a row?

The expected number of rolls to come up with the first 6 is 6 (see
above). Once we have rolled a first 6, there are two options: (i) We
roll another 6 (with probability 1/6) and are done; (ii) we roll
something else (with probability 5/6) and need to start all over again.
In this case, the expected value is \(E+1\).

Therefore, we have

\[
E = 6 + \frac{1}{6} \times 1 + \frac{5}{6} \times (E+1)
\]

from which follows \(E=42\).

\subsection{Problem 3, Sequence of six and five}

On average, how many times must a 6-sided die be rolled until the
sequence 65 appears (i.e., a 6 followed by a 5)?

This one is a bit tricky / unintuitive.

We can express this with two equations and two expectations: Let \(E\)
denote the expected number of rolls till we have a 6-5; let \(E_6\) the
expected number of rolls till we have a 6-5, \textbf{after} having
thrown a 6.

For expressing \(E_6\), we note that we have \textbf{three} options
after we roll a 6: (i) we roll a 5 and are finished, (ii) we roll a 6
(and need to wait for a wait for a subsequent 5), (iii) we roll
something else (1\ldots{}4) and start all over again.

Therefore,

\[
E_6 = \frac{1}{6} \times 1 + \frac{1}{6} \times (E_6 + 1) + \frac{4}{6} (E+1)
\]

For expressing \(E\), we note that there are two options: (i) After we
have rolled a 6, we roll a 5 and are finished. This happens with
probability 1/6; (ii) in all othe cases, we need to start all over again
and the expected value is \(E+1\).

Therefore,

\[
E = \frac{1}{6} \times (E_6 + 1) + \frac{5}{6} \times (E+1)
\]

From these two equations we obtain \(E = 36\). Interestingly, this is
smaller than the expected number of throws till two consecutive sixes
come up.

\subsubsection{Further Thoughts}

The probability for length-2 sequences is independent how they are
stopped (5-6 or 6-6). In all cases, there are 36 possibilities (1-1, 1-2
\ldots{} 6-6) and one of them (either 5-6 or 6-6) stops. Therefore, the
probabilities for a length-2 sequence is \(1/36\).

The probabilitiy for length-3 sequences is similar. For stopping with
5-6, the first element must be from the set \(\{1,2,3,4,6\}\) and will
appear with probability \(5/6\). The 5-6 happens with probability
\(1/36\) and in total we are at \(5/216\). The same holds true for
stopping on a 6-6; the first element must be from the set
\(\{1,2,3,4,5\}\); but all probabilities are the same.

The probability for length-4 sequences are more involved. Let's start
with a stop sequence 5-6 first. We simply write down all length-4
combinations with the last 2 elements being 5-6.

\bee
\begin{array}{cc}
& 1 1 5 6 \\
& 1 2 5 6 \\
& \cdots \\
& 1 6 5 6 \\
& \cdots \\
& 5 5 5 6 \\
& \color{red}{5 6 5 6} \\
& 6 1 5 6 \\
& \cdots \\
& 6 6 5 6
\end{array}
\eee

The sequence in red is \textbf{not allowed} as it is actually a length-2
sequence. That means, we have \(35\) ``good'' cases out of \(6^4\) and
the probability of a length-4 sequence with stop sequence 5-6 is
\(35 / 6^4\).

If we have a stop sequence of 6-6, more sequences are not allowed; in
the block starting with 1, the sequence 1-6-6-6 is not allowed. This
holds for every block (i.e.~2-6-6-6, 3-6-6-6, 4-6-6-6, 5-6-6-6, and
6-6-6-6); therefore only 30 sequences remain and the probability of a
length-4 sequence with stop sequence 6-6 is \(30 / 6^4\).

This is somewhat counterintuitive; at first sight, sequences should have
the same probability irrespective of the stop sequence. However, with
some stop sequences, there are more ways that a sequence stops earlier;
these do not contribute to the good cases and therefore the probability
is less.

I am sure there is some expression for sequences with arbitrary lengths
and different stop sequences\ldots{}
