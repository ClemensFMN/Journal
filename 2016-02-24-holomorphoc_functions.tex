\DiaryEntry{Holomorphic FUnctions}{2016-02-24}{Maths}

We deal with complex functions, that is function mapping from the
(complex) \(z\) plane to the complex \(w\) plane:
\(w=u+jv = f(z) = f(x+jy)\). Both the real and imaginary part of \(w\)
are functions of both \(x\) and \(y\).

For the function to be holomorphic, means that it is complex
differentiable, that is, the expression

\[
f'(z_0) = \lim_{h \rightarrow 0} \frac{f(z_0+h) - f(z_0)}{h}
\]

exists and is \textbf{independent} of the ``direction'' of \(h\). I.e.,
it does not matter in which direction \(z\) approaches the limit value
\(z_0\). If the limit exists, we say that \(f(z)\) is complex
differentiable at \(z_0\). If this holds true for all \(z\) in a certain
region \(\mathcal{A}\), then \(f(z)\) is complex differentiable in
\(\mathcal{A}\).

Holomorphicity is a strong condition on a function \(f(z)\). For a
function to be holomorphic, its real and imaginary parts need to fulfill
the Cauchy- Riemann equations. This can be proven / illustrated as
follows.

\subsubsection{Cauchy-Riemann Equations}

Consider a small change of the variable \(z=x+jy\) by \(dx+jdy\). It has
the following effect on the value \(w = u(x,y)+jv(x,y)\):


\begin{align*}
du &= \frac{\partial u(x,y)}{\partial x} dx + \frac{\partial u(x,y)}{\partial y} dy \\
dv &= \frac{\partial v(x,y)}{\partial x} dx + \frac{\partial v(x,y)}{\partial y} dy \\
\end{align*}


If we collect \(dz = (dx, dy)^T\) in a vector and \(dw = (du, dv)^T\) in
a vector, we see that the two vectors are related by the Jacobi matrix
\(dw = J dz\) with the Jacobi matrix defined as follows

\[J = \left(
\begin{array}{ccc}
\frac{\partial u(x,y)}{\partial x} & \frac{\partial u(x,y)}{\partial y} \\
\frac{\partial v(x,y)}{\partial x} & \frac{\partial v(x,y)}{\partial y} \\
\end{array}\right)
\]

This matrix describes the linearized effect of the function \(f(z)\) on
the vector \(dz = (dx, dy)^T\). The matrix will be different for every
point \(z\). The matrix describes a general linear transformation -
``anything'' can happen to the vector \(dz = (dx, dy)^T\).

For a function to be holomorphic, the matrix needs to describe a
combination of rotation and scaling. This means that any vector
\((dx, dy)^T\) at position \(z_0\) is rotated+scaled by the same amount.
Note that vectors at a different position \(z_1\) can be scaled+rotated
by a different amount (but again, this amount needs to be same for every
vector at the new position).

A matrix combining rotation and scaling has the following form (rotation
by \(\phi\), scaling by \(r\)):

\[
J = r \left(
\begin{array}{ccc}
\cos \phi & - \sin \phi \\
\sin \phi & \cos \phi \\
\end{array}\right)
\]

From this we see that \(J_{1,1} = J_{2,2}\) and \(J_{1,2} = -J_{2,1}\)
(where \(J_{n,m}\) denotes the \(n,m\)-entry of the matrix \(J\)). This
implies the following conditions for the partial derivatives of
\(u(x,y)\) and \(v(x,y)\), the Cauchy-Riemann equations:

\[
\frac{\partial u(x,y)}{\partial x} = \frac{\partial v(x,y)}{\partial y} \mbox{ and } \frac{\partial u(x,y)}{\partial y} = - \frac{\partial v(x,y)}{\partial x}
\]

\subsubsection{Example 1}

The function \(w = f(z) = z^2\) is holomorphic, because we have

\[
w = z^2 = (x+jy)^2 = x^2-y^2+j2xy = u + jv
\]

We have \(\frac{\partial u(x,y)}{\partial x}=2x\) which equals
\(\frac{\partial v(x,y)}{\partial y} = 2x\) and
\(\frac{\partial u(x,y)}{\partial y} = -2y\) which equals
\(- \frac{\partial v(x,y)}{\partial x} = -2y\).

We want to calculate the complex derivative; i.e.

\[
f'(z_0) = \lim_{h \rightarrow 0} \frac{(z_0+h)^2 - z_0^2}{h} = \lim_{h \rightarrow 0} \frac{(x+jy+h)^2 - (x+jy)^2}{h}
\]

First we consider a real value of \(h\), and we have

\[
f'(z_0) = \lim_{h \rightarrow 0} \frac{(x+h+jy)^2 - (x+jy)^2}{h} = \lim_{h \rightarrow 0} \frac{(x+jy)^2 + 2h(x+jy) +h^2 - (x+jy)^2}{h} = 2(x+jy) = 2z_0
\]

which is in line with the result that \((z^2)' = 2z\).

As second case, consider an imaginary value of \(h\), and obtain

\[
f'(z_0) = \lim_{h \rightarrow 0} \frac{(x+jy+jh)^2 - (x+jy)^2}{jh} = \lim_{h \rightarrow 0} \frac{(x+jy)^2 + 2(x+jy)jh - h^2 - (x+jy)^2}{h} = 2(x+jy) = 2z_0
\]

wich equals the case of a real \(h\). I don't think that this actually
provies that the derivative is independent of the choice of \(h\)
(i.e.~any complex \(h\)), but I think we better stop here\ldots{}

\subsubsection{Example 2}

As a counter example consider the function \(f(z) = \bar{z}\). For a
real variable \(x\), we have \(\bar{x} = x\) and \(\bar{jx} = -jx\).

If we want to calculate the derivative of \(f(z)\) along the real axis
we have

\[
f'(z_0) = \lim_{h \rightarrow 0} \frac{\bar{z_0+h} - \bar{z_0}}{h} = \lim_{h \rightarrow 0} \frac{\bar{z_0}+ h - \bar{z_0}}{h} = 1
\]

For the calculation along the imaginary axis, we have

\[
f'(z_0) = \lim_{h \rightarrow 0} \frac{\bar{z_0+jh} - \bar{z_0}}{jh} = \lim_{h \rightarrow 0} \frac{\bar{z_0} - jh - \bar{z_0}}{jh} = -1
\]

The derivative depends on the direction it is being calculated and
therefore the function \(f(z) = \bar{z}\) is \textbf{not} holomorphic.
