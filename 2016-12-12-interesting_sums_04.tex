\section{2016-12-12, Interesting Sums, Part 4}


Based on
\href{http://www.maa.org/programs/maa-awards/writing-awards/lester-r-ford-awards/another-way-to-sum-a-series-generating-functions-euler-and-the-dialog-function}{this
article}.

\subsection{Simple}\label{simple}

Let's start with a simple sum

\[
S = \sum_{k=1}^\infty (-1)^{k-1} \frac{1}{k} = 1 - \frac{1}{2} + \frac{1}{3} - \frac{1}{4} + \cdots
\]

Instead consider the power series

\[
f(z) = \sum \frac{1}{k}z^k = z + \frac{z^2}{2} + \frac{z^3}{3} + \frac{z^4}{4} + \cdots
\]

and differentiate it with respect to \(z\) to obtain

\[
f'(z) = \sum z^{k-1} = 1 + z + z^2 + \cdots
\]

For \(|z| < 1\), we can write the sum as closed form expression

\[
f'(z) = \frac{1}{1-z}
\]

and by integrating it, we obtain \(f(z) = - \ln(1-z) + C\), where \(C\)
is the integration constant. From the definition of \(f(z)\), we know
that \(f(0) = 0\) and therefore \(C = 0\). So we have

\[
f(z) = \sum \frac{1}{k}z^k = z + \frac{z^2}{2} + \frac{z^3}{3} + \frac{z^4}{4} + \cdots = -\ln(1-z)
\]

In order to calculate the sum \(S\) from the beginning, we set \(z=-1\)
and obtain

\[
f(-1) = -1 + \frac{1}{2} - \frac{1}{3} \pm \cdots = - S = -\ln(1-z)
\]

Therefore (there is a sign error in the paper)

\[
S = \sum_{k=1}^\infty (-1)^{k-1} \frac{1}{k} = -f(1-(-1)) = ln(2) \approx 0.6931
\]

We can extend this a bit; e.g.

\[
f(1/3) = \sum \frac{1}{k} \left( \frac{1}{3} \right)^k = -\ln(1-1/3) = -\ln(2/3) = \ln(3/2) \approx 0.4055
\]

By the way, this is wrong in the linked article.

\subsubsection{Partial Sum}\label{partial-sum}

Now we are interested in considerung the effect of truncating the
summation; i.e.~consider only the first N terms

\[
S_N = \sum_{k=1}^N (-1)^{k-1} \frac{1}{k}
\]

To calculate the error we make, we consider the following power series

\[
\bar{f}(z) = \sum_{k=N+1}^\infty \frac{1}{k}z^k
\]

with derivative

\[
\bar{f'}(z) = \sum_{k=N+1}^\infty z^{k-1} = z^N + z^{N+1} + \cdots = \frac{1}{1-z} - \frac{1-z^{N-1}}{1-z} = \frac{z^{N-1}}{1-z}
\]

We (Wolfram Alpha) can integrate this to obtain

\[
\bar{f}(z) = \frac{z^{N+1} {}_2 F_1(1;N+1;N+2;z)}{N+1}
\]

where \({}_2 F_1\) is the hypergeometric function. Crazily enough, this
is correct :-) Calculating the truncation error in Julia yields

\begin{verbatim}
k=1:9
sum((-1).^(k-1)./k) - log(2)
0.052487740074975364
\end{verbatim}

and the truncation error is calculated in Python / SciPy

\begin{verbatim}
import scipy.special as sp
(-1)**10 * sp.hyp2f1(1,10,11,-1) / 10
0.05248774007497533
\end{verbatim}

\subsection{Advanced}\label{advanced}

Now consider something similar,

\[
S = \sum \frac{1}{k^2}
\]

with a related power series and derivative

\[
g(z) = \sum \frac{1}{k^2}z^k, \qquad g'(z) = \sum \frac{1}{k}z^{k-1}
\]

We have \(g'(z) = f(z) / z\) and therefore

\[
g'(z) = -\frac{\ln(1-z)}{z}
\]

We want to integrate this to obtain \(g(z)\): We know that \(g(0) = 0\)
and therefore

\[
\sum \frac{1}{k^2} = g(1) = - \int_0^1 \frac{\ln(1-z)}{z} dz
\]

and discover that this is the \emph{dilogarithm} function defined as

\[
\mathbb{Li}_2(z) = - \int_0^z \frac{\ln(1-t)}{t} dt
\]

So, \(\sum \frac{1}{k^2} = \mathbb{Li}_2(1)\). In order to find this
value, we first observe the identitiy

\[
\mathbb{Li}_2(-1/z) + \mathbb{Li}_2(-z) + \frac{1}{2}(\ln z)^2 = C
\]

with \(C\) being a constant (This can be shown by calculating the
derivative on both sides via the chain rule). Taking \(z=1\) leads to
\(C = 2\mathbb{Li}_2(-1) = 2(-1+1/4-1/9 +- \cdots)\).

Next we split the sum \(\sum 1/k^2\) into a sum of the even terms
\(E = \sum 1/(2k)^2 = 1/4 \mathbb{Li}_2(1)\) and a sum of the odd terms
\(O = 3/4 \mathbb{Li}_2(1)\). Therefore, the alternating sum
\((-1+1/4-1/9+- \cdots) = E - O = -1/2 \mathbb{Li}_2(1)\) and
\(C = -\mathbb{Li}_2(1)\).

We can therefore simplify the identity to

\[
\mathbb{Li}_2(-1/z) + \mathbb{Li}_2(-z) + \frac{1}{2}(\ln z)^2 = -\mathbb{Li}_2(1)
\]

and setting \(z=-1\), we obtain

\[
2\mathbb{Li}_2(1) + \frac{1}{2} [\ln(-1)]^2 = -\mathbb{Li}_2(1)
\]

and

\[
\mathbb{Li}_2(1) = - \frac{1}{6} [\ln(-1)]^2
\]

To continue, we can pull off the following trick (which is far from
being rigorous): We have \(e^{i\pi} = -1\) and therefore
\(i\pi = \ln(-1)\). Squaring this yields \(\ln^2(-1) = -\pi^2\) and
therefore

\[
\mathbb{Li}_2(1) = - \frac{\pi^2}{6}
\]
