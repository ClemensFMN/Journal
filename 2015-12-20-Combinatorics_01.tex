\DiaryEntry{Combinatorics, 1}{2015-12-20}{Combinatorics}

\subsection{Consecutive Elements in Permutations}

Given a set \(\mathcal{S}_n = \{1,2,3,\ldots,n\}\) and consider
permutations of this set (there are \(n!\) of them). How many
permutations are there, with \(1\) directly followed by a \(2\), up to
\(k\)?

For example, have \(n=4\) and \(k=2\), then a ``valid'' permutation is
\((3,1,2,4)\).

Consider a permutation where the sequence \((1,2,\ldots,k)\) is at the
beginning:

\[
(1,2,\ldots,k,x,\ldots,x)
\]

where \(x\) denotes any number from \(k+1\) to \(n\). These are \(n-k\)
numbers and there are \((n-k)!\) possibilities to place them.

Next we shift the ``\((1,2,\ldots,k)\) group'' one place to the right

\[
(x, 1,2,\ldots,k,x,\ldots,x)
\]

Again, there are \((n-k)!\) possibilities to place the \(x\) values.

We note that there are \(n-k+1\) ways to place the ``\((1,2,\ldots,k)\)
group'' (from the very left to the very right); thus in total there are
\((n-k+1)(n-k)!\) permutations.

\subsubsection{Extreme Cases}

For \(k=n-1\) we obtain for the number of permutations
\([n-(n-1)+1][n-(n-1)]! = 2+1! = 2\). There are only two possibilities
of the form ``\((1,2,\ldots,k,n)\) and''\((n,1,2,\ldots,k)\).

For \(k=1\) we have \((n-1+1)(n-1)! = n(n-1)! = n!\): We are free to
place all elements in any order - therefore we get the numbe of
permutations of the original set \(\mathcal{S}_n\).

\subsubsection{Example for \(n=4\)}

Below the ``good'' cases are shown. The values 3 and 4 can be placed
onto the x's; for each row, there are \(2!=2\) ways to place them
(\((3,4)\) and \((4,3)\)). Therefore in total there are
\(3 \times 2 = 6\) possible placements.


\begin{align*}
&(1,2,x,x)\\
&(x,1,2,x)\\
&(x,x,1,2)
\end{align*}


\subsection{Number of Words}

The number of words with length \(k\) in an alphabet with \(n\) elements
is \(n^k\). This is easy. Now add the constraint, that two consecutive
letters must be different.

The first letter can be chosen independently, so we have \(n\)
possibilities. The second letter must be different from the first,
therefore there are only \(n-1\) possibilities left. This argument holds
for all letters except the first; therefore, there are \(n(n-1)^{k-1}\)
possibilities.

\subsection{Set-Partitioning into 2 Subsets}

The set \(\mathcal{S}\) has \(n\) elements. In how many ways can the set
be partitioned into two subsets \(\mathcal{A}\) and \(\mathcal{B}\)
which are not necessarily disjoint and whose union is \(\mathcal{S}\):
\(\mathcal{A} \cup \mathcal{B} = \mathcal{S}\).

For every element \(x \in \mathcal{S}\), there are three cases:

\begin{itemize}
\item
  \(x \in \mathcal{A}, x \notin \mathcal{B}\)
\item
  \(x \notin \mathcal{A}, x \in \mathcal{B}\)
\item
  \(x \in \mathcal{A}, x \in \mathcal{B}\)
\end{itemize}

Therefore, there are \(3^n\) ways to choose \(\mathcal{A}\) and
\(\mathcal{B}\).

We can arrive at this result via induction as well. Start with \(n=1\);
i.e. \(\mathcal{S} = \\{1\\}\). Then we can place (i) 1 into
\(\mathcal{A}\) and leave \(\mathcal{B}\) empty, (ii) place 1 into
\(\mathcal{B}\) and leave \(\mathcal{A}\) empty or (iii) put 1 into both
\(\mathcal{A}, \mathcal{B}\).

So, for \(n=1\), we have \(3\) possibilities. If we add a second element to \(\mathcal{S}\), we use the same argument for placing the second element \textbf{independently} form the first one. Therefore, there are \(3^n\) ways to select the subsets.

Note that the Stirling Numbers of the second kind \({n \brace k}\) count the number of \(n\) elements into \(k\) \textbf{non-empty} partitions. For \(k=2\), we have the special relation \({n \brace 2} = 2^{n-1}-1\).

We can prove this as follows: A binary string \((b_1, \ldots,b_n)\)
represents a partition. If \(b_i = 0\), then \(i \in \mathcal{A}\), if \(b_i = 1\), then \(i \in \mathcal{B}\). The all-zero and the all-one string are not allowed as the would leave one of the two sets empty. Furthermore, we double count; i.e.~we treat
\(\mathcal{A} = \{1,2\}, \mathcal{B} = \{3\}\) and \(\mathcal{A} = \{3\}, \mathcal{B} = \{1,2\}\) as separate partitions (which they are not). Therefore, we have \({n \brace 2} = \frac{2^{n}-2}{2} = 2^{n-1}-1\).
