\DiaryEntry{Representing Integers as Sums of Squares}{2023-04-24}{Number Theory}

We deal with the problem of representing integers as sums of squares. For example, we have

\begin{align*}
    1 &= 1^2 \\
    2 &= 1^2 + 1^2 \\
    3 &= 1^2 + 1^2 + 1^2 \\
    4 &= 2^2 \\
    5 &= 1^2 + 2^2
\end{align*}

The question is if there is an upper limit on the number of squares to express every integer. A famous theorem of Lagrange, proved in 1770, asserts that four squares are sufficient; that is, every positive integer is realizable as the sum of four squared integers, some of which may be $0 = 0^2$.

\subsection{Sum of two Sqaures}

We start with a simple theorem.

\begin{theorem}
    If $m$ and $n$ are each the sum of two squares, then so is their product $mn$.
\end{theorem}

We can prove this by performing the calculation: Let $m = a^2 + b^2$ and $n = c^2 + d^2$, then we have

\bee
mn = (a^2 + b^2)(c^2 + d^2) = a^2c^2 + b^2c^2 + a^2d^2 + b^2d^2 = (ac + bd)^2 + (ad - bc)^2 \qed
\eee

It is clear, that not every prime can be written as sum of two squares; eg. $3 = a^2 + b^2$ does not have an integer solution. In particular, we have the following theorem.

\begin{theorem}\label{2023_04_24:th1}
    No prime of the form $4k+3$ is a sum of two squares.
\end{theorem}

We first note that $a \equiv 0, 1, 2, 3 \mod 4$ for any integer $a$. Therefore, $a^2 \equiv 0, 1 \mod 4$ (as $2^2 \equiv 0 \mod 4$ and $3^2 \equiv 1 \mod 4$). So, for two integers $a, b$, we have

\bee
a^2 + b^2 \equiv 0, 1, 2 \mod 4
\eee

with equivalence $\mod-2$ when both $a \equiv b \equiv 1 \mod 4$. But our prime has the form $4k+3$, therefore $4k+3 \equiv 3 \mod 4$ and therefore such a prime cannot be expressed as sum of two squares. \qed

Fun side-note: The primeness of $4k+3$ is actually not relevant here, so we have the stronger statement that no integer of the form $4k+3$ is a sum of two squares.

On the other hand, the proof shows that a prime can be expressed as sum of two squares if it can be expressed in the form $4k+1$. The first few primes of the form $4k+1$ and their expression as squares are as follows,

\bee
1 = 1^2 + 0^2, 5 = 1^2 + 2^2, 13 = 2^2 + 3^2, 17 = 1^2 + 4^2, 29 = 2^2 + 5^2, \cdots
\eee

The book presents another proof. It requires the pigeon principle.

\begin{theorem}
    Pigeonhole principle. If $n$ objects are placed in $m$ pigeonholes and if $n > m$, then some pigeonhole will contain at least two objects.
\end{theorem}

We next have the following theorem.

\begin{theorem}\label{2023_04_24:th2}
    Let $p$ be a prime and let $\gcd(a,p)=1$. Then the congruence

    \bee
    ax \equiv y \mod p
    \eee

    has a solution $x_0, y_0$ where

    \bee
    0 < | \sqrt{x_0} < \sqrt{p} \quad \text{and} \quad 0 < | \sqrt{y_0} < \sqrt{p}
    \eee

\end{theorem}

Be careful, as the theorem does \emph{not} mean that we are allowed to fix eg $y_0$ and there will be a corresponding $x_0$ such that $0 < | \sqrt{x_0} < \sqrt{p}$ and $0 < | \sqrt{y_0} < \sqrt{p}$ for every choice of $y_0$; instead, the theorem guarantees there is a pair $(x_0, y_0)$ which fulfills the conditions. As an example, $p=7, a=3$ and we have

\bee
3x \equiv y \mod 7
\eee

This can be solved by many $(x,y)$ pairs, but $x_0 = -2$ yields $y = 3 x_0 = -6 \equiv 1 \mod 7$ is a valid solution.

Proof: Let $k = \lfloor \sqrt{p} \rfloor + 1$ and consider the set of integers

\bee
S = \{ax - y | 0 \leq x \leq k-1, 0 \leq y \leq k-1\}
\eee

The expression $ax-y$ takes on $k^2$ different values ($k$ values for $x$ and $k$ valuues for $y$). From the $k$-definition we have $k^2 > p$, and by the pigeonhole principle we have that at least two members of $S$ must be congruent modulo $p$; let's call them $ax_1 - y_1$ and $ax_2 - y_2$, with $x_1 \neq x_2$ and $y_1 \neq y_2$. Then we have

\bee
a(x_1 - x_2) \equiv y_1 - y_2 \mod p.
\eee

Setting $x_0 = x_1 - x_2$ and $y_0 = y_1 - y_2$, it follows that $(x_0, y_0)$ provide a solution to $ax \equiv y \mod p$. If either $x_0 = 0$ or $y_0 = 0$, then by $\gcd(a,p) = 1$ it follows that the other must also be zero, contrary to the assumption. Therefore, $0 < |x_0| \leq k-1 < \sqrt{p}$ and $0 < |y_0| \leq k-1 < \sqrt{p}$. \qed


We can now state and prove the following theorem by Fermat.

\begin{theorem}
An odd prime $p$ is expressible as a sum of two squares iff $p \equiv 1 \mod 4$.
\end{theorem}

Proof: In theorem \eqref{2023_04_24:th1} we showed that $a^2 \equiv 0, 1 \mod 4$ and therefore $a^2 + b^2 \equiv 0, 1, 2 \mod 4$ - \qed

The book gives an additional proof: Suppose that $p$ can be written as sum of two squares, $p = a^2 + b^2$. Because $p$ is a prime, we have $p \nmid a$ and $p \nmid b$ - if $p | a$, then $p | b^2$ and therefore $p|b$, leading to the contradiction $p^2 | p$.

Therefore, there exists an integer $c$ for which $bc equiv 1 \mod p$. Taking the relation $(ac)^2 +(bc)^2 = pc^2$ mod-$p$, we obtain

\bee
(ac)^2 \equiv -1 \mod p
\eee

which makes $-1$ a quadratic residue of $p$. From one of the previous entries (actually, the stuff below \ref{2023-02-13:th3}), we have $(-1/p) = 1$ only when $p \equiv 1 \mod 4$. So that's the "only if" part.

For the converse, assume that $p \equiv 1 \mod 4$. Because $-1$ is a quadratic residue of $p$, we can find an integer $a$ such that $a^2 \equiv -1 \mod p$, namely $a = [(p-1)/2]!$ (according to Wilson's theorem). With $\gcd(a,p) = 1$ the congruence $ax \equiv y \mod p$ has a solution $(x_0, y_0)$ for which the conclusion of theorem \ref{2023_04_24:th2} holds; Therefore, 

\bee
-x_0^2 \equiv a^2 x_0^2 \equiv (a x_0)^2 \equiv y_0^2 \mod p
\eee

or $x_0^2 + y_0^2 \equiv 0 \mod p$. This is equivalent to $x_0^2 + y_0^2 = kp$ for some integer $k \geq 1$. From the bounds on $x_0$ and $y_0$, $0 < | \sqrt{x_0} < \sqrt{p}$ and $0 < | \sqrt{y_0} < \sqrt{p}$, we obtain $0 < x_0^2 + y_0^2 < 2p$ which implies $k=1$. Therefore $x_0^2 + y_0^2 = p$ and we are finished. \qed

If we count $a^2$ and $(-a)^2$ as the same, we can strengthen the statement to: Any prime of the form $4k+1$ can be expressed uniquely as the sum of two squares. Proof is omitted.

As an example, consider $p = 13 = 4 \cdot 2 + 1$. We first need to choose an integer $a$ such that $a^2 \equiv -1 \mod p$: This is achieved by choosing $a = [(p-1)/2]! = 6! = 720$. So the congruence to solve becomes

\bee
720 x \equiv y \mod 13 \rightarrow 5 x \equiv y \mod 13
\eee

To solve it we consider the set

\bee
S = \{ 5x - y | 0 \leq x, y < 4\}
\eee

which are the integers $S = \{0, 5, 10, 15, -1, 4, 9, 14, -2, 3, 8, 13, -3, 2, 7, 12\}$ which, modulo-$13$ becomes $\{0, 5, 10, 12, 12, 4, 9, 1, 11, 3, 8, 0, 10, 2, 7, 12\}$. We choose

\bee
5 \cdot 1 - 3 \equiv 2 \equiv 5 \cdot 3 - 0 \mod 13
\eee

or

\bee
5 (1-3) \equiv 3 \mod 13
\eee

Therefore, we take $x_0 = -2$ and $y_0 = 3$ to obtain

\bee
13 = x_0^2 + y_0^2 = 2^2 + 3^2
\eee


However, it's not only primes $p$ of the form $p \equiv 1 \mod 4$ which can be epxressed as sum of two squares; it's also integers; eg

\bee
10 = 1^2 + 3^2
\eee

Which integers can be expressed as sum of two squares is given by the following theorem.

\begin{theorem}
Let the positive integer $n$ be written as $n = N^2 m$, where $m$ is squarefree. Then $n$ can be represented as the sum of two squares if and only if $m$ contains no prime factor of the form $4k + 3$.
\end{theorem}

We can express $10 = 1^2 \cdot 10$ and the prime factors of $m = 10$ are $10 = 2 \cdot 5$. Neither has the form $4k+3$ and therefore $10$ can be represented as the sum of two squares.

Proof is omitted.

We can extend this to the following result.

\begin{theorem}
A positive integer $n$ is representable as the sum of two squares if and only if each of its prime factors of the form $4k + 3$ occurs to an even power.
\end{theorem}

As an example, the number $459$ cannot be written as the sum of two squares, because $459 = 3^33 \cdot 17$, with the prime $3$ occurring to an odd exponent. On the other hand, $153 = 3^2 \cdot 17$ admits representation as sum of two squares, $153 = 12^2 + 3^2$ (the proof of the theorem explains how the representation can be achieved).

As a last topic considering two squares, we can ask for representation as \emph{difference} of two squares - the answer is given in the following theorem.

\begin{theorem}
A positive integer $n$ can be represented as the difference of two squares iff $n$ is not of the form $4k+2$.
\end{theorem}

Proof: We start by nothing that $a^2 \equiv 0, 1 \mod 4$ for all integers $a$ from which follows that

\bee
a^2 - b^2 \equiv 0, 1, 3 \mod 4
\eee

Thus, if $n \equiv 2 \mod 4$, we cannot have $n = a^2 - b^2$ for any choice of $a$ and $b$.

On the other hand, if $n$ is not of the form $4k+2$, we have $n \equiv 0, 1, 3 \mod 4$. If $n \equiv 1, 3 \mod 4$, then $n+1$ and $n-1$ are both even integers; therefore $n$ can be written as

\bee
n = \left( \frac{n+1}{2} \right)^2 - \left( \frac{n-1}{2} \right)^2
\eee

which is a difference of squares. If $n \equiv 0 \mod 4$, then we have

\bee
n = \left( \frac{n}{4} + 1 \right)^2 - \left( \frac{n}{4} - 1 \right)^2
\eee

As a Corollary, we note that an odd prime is the difference of two successive squares; eg $11 - 6^2 - 5^2, 17 = 9^2 - 8^2, 29 = 15^2 - 14^2$.


\subsection{Sums of more than two Sqaures}

The more summands we allow, the fewer numbers will be which cannot be expressed as sum of such squares. Eg we cannot express $14$ as sum of two sqaures, but $14 = 3^2 + 2^2 + 1^2$. However, there are still numbers which are not expressible as the sum of two sqaures.

\begin{theorem}
	No positive integer of the form $4^n (8m + 7)$ can be represented as the sum of three squares.
\end{theorem}


Proof is omitted.

For the final results, we need Euler's Lemma.

\begin{theorem}
	If the integers $m$ and $n$ are each the sum of four squares, then $mn$ is likewise so representable.
\end{theorem}

The proof relies on expressing $m = a_1^2 + \cdots a_4^2$ (same for $n$) and writing out $mn$.

Using this and some other stuff introduced in the book, we have the following theorem.

\begin{theorem}
	Any prime p can be written as the sum of four squares.
\end{theorem}

Proof is omitted; however, we can express any number $n$ as product of $r$ primes. From above, we have that every prime can be expressed as sum of four squares. Euler's identity permits us to express the product of any two primes as a sum of four squares. This, by induction, extends to any finite number of prime factors, so that applying the identity $r - 1$ times, we obtain that we can express any positive integer as sum of four squares. This is the famous Lagrange theorem.

\begin{theorem}
Any positive integer $n$ can be written as the sum of four squares, some of which may be zero.
\end{theorem}




%%% Local Variables:
%%% mode: latex
%%% TeX-master: "journal"
%%% End:
