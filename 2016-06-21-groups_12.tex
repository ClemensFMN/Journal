\DiaryEntry{Groups - Matrices}{2016-06-21}{Algebra}

We consider \(n \times n\) matrices with real coefficients.

\subsection{Determinants}\label{determinants}

The determinant of a matrix has several properties:

\begin{itemize}
\item
  For two matrices \(A, B\) we have $\det AB = A \det B$; i.e.~the
  determinant is a homomorphism into the mutliplicative group of real
  numbers.
\item
  If \(A\) is invertible, we have \(\det A^{-1} = 1/ \det A\).
\item
  Transposition does not change the determinant.
\item
  If the matrix \(A\) is interpretaed as linear transformation (of a
  vector \(a\)), then the determinant \(\det A\) is the change in volume
  induced by the transformation. In case of \(2 \times 2\) matrices,
  \(A\) multiplies areas by a factor of \(\det A\).
\end{itemize}

\subsection{Group Structure}\label{group-structure}

The set of all invertible \(n \times n\) matrices forms the
\textbf{general linear group} denoted \(GL_n(\mathbb{R})\). The group
operation is matrix multiplication. The group conditions are fulfilled
as the product of two invertible matrices is invertible again and there
exists an inverse element.

The \textbf{special linear group} \(SL_n(\mathbb{R})\) s a subgroup of
the general linear group and contains all matrices with determinant one.
Multiplying two matrices from this group yields a matrix whose
determinant is again one (see first determinant property above). As
geometric interpretation, this group contains all transformation which
leave the area constant.

The \textbf{orthogonal group} \(O(n)\) is the group of matrices where
\(A^{-1} = A^T\). This is a subgroup, because
\((AB)^{-1} = A^{-1} B^{-1} = A^T B^T = (AB)^T\). Orthogonal matrices
preserve inner products; i.e.
\((Ax)^T (Ay) = x^T A^T A y = x^T A^{-1} A y = x^T y\) and have a
determinant of \(\pm 1\): \(\det A A^T = det I = 1\) and
\(det A^T = det A\). Orthogonal matrices contain rotations (these
matrices have determinant one) and reflections (these matrices have
determinant -1). In other words, we have that \(A^{-1}A = A^T A = I\);
i.e.~the column vecotrs of an orthogonal matrix all have length one and
are orthogonal to each other; they form an orthonormal set.

The \textbf{special orthogonal group} \(SO(n)\) is the intersection of
\(O(N)\) and \(SL_n(\mathbb{R})\) which are all orthogonal matrices with
determinant one; i.e.~reflections.
