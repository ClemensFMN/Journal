\DiaryEntry{Binomial Expansion}{2016-08-01}{Algebra}

Based on
\href{http://math.stackexchange.com/questions/1869954/expansion-of-1-sqrt2n}{this
question}.

Consider the expression \((1+\sqrt{2})^n, n \in \mathbb{N}\) and expand
it via the binomial formula:

\[
(1+\sqrt{2})^n = \sum_{k=0}^n {n \choose k} (\sqrt{2})^k = a_n + b_n \sqrt{2}
\]

Note that the terms with k being even become integers: \({n \choose k}\)
is integer and \((\sqrt{2})^k\) is an integer (namely \(2^{k/2}\)) for k
even. The \(a_n, b_n\) collect all these term.

We can do the same for the expression

\[
(1-\sqrt{2})^n = \sum_{k=0}^n {n \choose k} (-1)^k (\sqrt{2})^k
\]

In addition to the observation above, we note that
\((-1)^k (\sqrt{2})^k = 2^{k/2}\) for k even and
\((-1)^k (\sqrt{2})^k = -(\sqrt{2})^k\) for k odd. Therefore, we can
write

\[
(1-\sqrt{2})^n = a_n - b_n \sqrt{2}
\]

Collecting things, we have


\begin{align*}
(1-\sqrt{2})^n &= a_n - b_n \sqrt{2} \\
(1+\sqrt{2})^n &= a_n + b_n \sqrt{2}
\end{align*}


This can be extended according to


\begin{align*}
(a-b\sqrt{2})^n &= a_n - b_n \sqrt{2} \\
(a+b\sqrt{2})^n &= a_n + b_n \sqrt{2}
\end{align*}


which is a rather nice result: If we consider \(A = a+b\sqrt{2}\) as an
element of \(\mathbb{Z}[\sqrt{2}\), then \(A^n\) is also in this ring;
i.e.~taking it to the n-th power does \emph{not} introduce any new
irrational numbers.

We can multiply the left parts and the right parts with each other to
obtain

\[
a_n^2 - 2b_n^2 = (1-2)^n = (-1)^n
\]

So we see that the \emph{absolute} difference between \(a_n^2\) and
\(2b_n^2\) is one. If we arrange this to get an expression for
\(b_n \sqrt{2} = \sqrt{a_n^2-(-1)^n}\) and insert this into
\(a_n + b_n \sqrt{2}\) we obtain

\[
(1+\sqrt{2})^n = a_n + b_n \sqrt{2} = \sqrt{a_n^2} + \sqrt{a_n^2 - (-1)^n}
\]

which is (more or less) the result from the post.
