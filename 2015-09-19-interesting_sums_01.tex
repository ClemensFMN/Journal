\DiaryEntry{Interesting Sums, 1}{2015-09-19}{Sums}


\subsubsection{Geometric Series}

We want to calculate the sum

\[ S_N = \sum_{k=1}^N a s^k \]

Consider the following expression

\[ S_N + a s^{N+1} = a s^0 + \sum_{k=0}^N a s^{k+1} = a s^0 + s \sum_{k=0}^N a s^k = a s^0 + s S_N \]

From this we obtain \textbackslash{}( S\_N (1 - s) = a - a s\^{}\{N+1\}
\textbackslash{}) and can finally express the sum as

\[ S_N = a \frac{1 - s^{N+1} }{1 - s} \]

If $|s| < 1$, then the sum converges for $N \rightarrow \infty$ to $S\_\infty=\frac{a}{1-s}$.

\subsubsection{Calculating $\sum k a^k$}

Next we want to tackle the sum

\[ S_N = \sum_{k=1}^N k a^k = a + 2a^2 + 3a^3 + \cdots + Na^N\]

We write down $a S\_N$ as follows

\[ a S_N = \sum_{k=1}^N k a^{k+1} = \sum_{k=1}^{N+1} (k-1) a^{k} = \sum_{k=1}^{N} (k-1) a^{k} + N a^{N+1} = a^2 + 2 a^3 + 3 a^4 + \cdots + N a^{N+1}\]

From this we see that the difference $S\_N - a S\_N$ has an easy form:

\[ S_N - a S_N = \sum_{k=1}^N k a^k - \left( \sum_{k=1}^{N} (k-1) a^{k} + N a^{N+1} \right) = \sum_{k=1}^N a^k - N a^{N+1} \]

The sum is the one we calculated above, so we have

\[ S_N - a S_N = \frac{a - a^{N+1} }{1 - a} - N a^{N+1} = \frac{a - a^{N+1} - N a^{N+1}(1-a)}{1-a} = \frac{a - a^{N+1} - N a^{N+1} + N a^{N+2}}{1-a} \]

Solving the equation for $S_N$, we get

\[S_N = \frac{a - a^{N+1}(N+1) + N a^{N+2}}{(1-a)^2} \]

If $|a| < 1$, then the sum converges for $N \rightarrow \infty$ to $S\_\infty = \frac{a}{(1-a)^2}$.

There is a different way (see blog entry on 2015-09-07 "Generating Functions 03") which may be easier: Start with

\[ \sum_{k=1}^N a^k = \frac{1 - a^{N+1} }{1 - a} \]

Taking the derivative on both sides and multiplying with $a$, we obtain

\[ \sum_{k=1}^N k a^k = a \frac{d}{da} \frac{1 - a^{N+1} }{1 - a} \]

We can check all 3 results with Python / Julia as follows. Start with Julia and calculate the ``ground truth''

\begin{verbatim}
n = 1:10
a=0.4

sum(n.*a.^n)
1.110295552
\end{verbatim}

In Python we can check with Sympy

\begin{verbatim}
import sympy as sym

x, n, a = sym.symbols("x n a")
f = (x**(n+1)-1)/(x-1)
g = sym.diff(f, x)
res = g*x
sym.pprint(res.subs(x,0.4).subs(n,10))
1.110295552
\end{verbatim}

which is the same as the expression calculated above. Finally

\begin{verbatim}
print( (a - (N+1)*a**(N+1) + N*a**(N+2))/(1-a)**2)
1.110295552
\end{verbatim}

which shows the correctness of the approach taken above.
